\documentclass[12pt]{article}

\usepackage{graphics}
\usepackage{epsfig}
\usepackage{times}
\usepackage{amsmath,amssymb,amsfonts}
\usepackage[table,dvipsnames]{xcolor}
\usepackage{tcolorbox}
\usepackage{hyperref}

\usepackage{tabularx}
\usepackage{amsthm}
\usepackage{eqparbox}
\usepackage{relsize}
\usepackage{color}
\usepackage{listings}
\usepackage[export]{adjustbox}
\usepackage{multirow}
\usepackage{pifont}
\usepackage{soul}
\usepackage{booktabs,rotating}
\usepackage{array}
\usepackage{graphicx}
\usepackage{adjustbox}
\usepackage{caption}
\usepackage{subcaption}
\usepackage[normalem]{ulem}
\usepackage{xcolor}
\usepackage{xcolor}
\usepackage{colortbl}
\usepackage{listings}
\definecolor{mycolor}{RGB}{211, 211, 211} % Define the color 'mycolor' as blue
\definecolor{codegreen}{rgb}{0,0.6,0}
\definecolor{codegray}{rgb}{0.5,0.5,0.5}
\definecolor{codepurple}{rgb}{0.58,0,0.82}
\definecolor{backcolour}{rgb}{0.95,0.95,0.92}
\newtheorem{definition}{Definition}
\write18{}

\lstdefinestyle{XMLStyle}{
  language=XML,
  keywordstyle=\color{blue},
  numberstyle=\tiny\color{codegray},
  stringstyle=\color{codepurple},
  morekeywords={E,S,M,T,Failure,line},
  basicstyle=\tiny\ttfamily,
  columns=fullflexible,
  keepspaces=true,
  frame=none,
  breaklines=true,
  aboveskip=10pt,
  belowskip=10pt,
  numbers=none,
  xleftmargin=10.5em
}
\newcommand{\FRAME}{{\sc Frame}\xspace}
\newcommand{\MotorEaseB}{{\bfseries\scshape MotorEase}\xspace}
\newcommand{\MotorEase}{{\sc MotorEase}\xspace}
\newcommand{\MotorCheck}{{\sc MotorCheck}\xspace}
\newcommand{\Miracle}{{\sc Miracle}\xspace}
\newcommand{\AidUI}{{\sc AidUI}\xspace}
\newcommand{\AidUIs}{{\sc AidUI's~}\xspace}
 
% \usepackage{comment}
% \excludecomment{figure}
% \let\endfigure\relax

% %%% Remove the next two lines if you want the figures and tables at their place    
% \usepackage[nolists,nomarkers]{endfloat}
% \renewcommand{\processdelayedfloats}{}

%%% custom commands

\newcommand{\code}[1] {{\smaller\texttt{#1}}}

\usepackage{tikz}

\newcommand*\emptycirc[1][black]{\tikz\draw[#1] (0,0) circle (1ex);} 
\newcommand*\halfcirc[1][black]{%
    \begin{tikzpicture}
        \draw[fill=#1] (0,0)-- (90:1ex) arc (90:270:1ex) -- cycle[#1] ;
        \draw[#1] (0,0) circle (1ex);
    \end{tikzpicture}
  }
\newcommand*\fullcirc[1][black]{\tikz\fill[#1] (0,0) circle (1ex);} 

\newcommand{\SAT}{{SAT}}
\newcommand{\PAT}{{PAT}}
\newcommand{\NLC}{\textsf{NLC}}
\newcommand{\FC}{\textsf{FC}}

\newcommand*\circled[1]{\tikz[baseline=(char.base)]{\small{\textbf{
			\node[shape=circle,fill=mycolor,draw=black, inner sep=0.75pt] (char) {\textcolor{black}{#1}};}}}}


\newcommand{\subsubsubsec}[1]{\paragraph{#1}\mbox{}\\}

\definecolor{cOrange}{rgb}{0.93, 0.35, 0.0}
\definecolor{cPurple}{rgb}{0.55, 0.0, 0.55}




% <http://psl.cs.columbia.edu/phdczar/proposal.html>:
%
% The standard departmental thesis proposal format is the following:
%        30 pages
%        12 point type
%        1 inch margins all around = 6.5   inch column
%        (Total:  30 * 6.5   = 195 page-inches)
%
% For letter-size paper: 8.5 in x 11 in
% Latex Origin is 1''/1'', so measurements are relative to this.

\topmargin      0.0in
\headheight     0.0in
\headsep        0.0in
\oddsidemargin  0.0in
\evensidemargin 0.0in
\textheight     9.0in
\textwidth      6.5in
\linespread {1}



\usepackage{booktabs}
\usepackage{xspace}
\usepackage{enumitem}
\usepackage[table]{xcolor}  
\newcommand{\sysName}{FlakeFlagger\xspace}
\newcommand{\vocabName}{vocabulary-based approach\xspace}
\usepackage{flushend}

\newcommand{\testName}[1]{%
  \begingroup
  \ttfamily
  \begingroup\lccode`~=`/\lowercase{\endgroup\def~}{/\discretionary{}{}{}}%
  \begingroup\lccode`~=`[\lowercase{\endgroup\def~}{[\discretionary{}{}{}}%
  \begingroup\lccode`~=`.\lowercase{\endgroup\def~}{.\discretionary{}{}{}}%
  \catcode`/=\active\catcode`[=\active\catcode`.=\active
  \scantokens{#1\noexpand}%
  \endgroup
}
\definecolor{c1}{HTML}{AD302E}
\definecolor{c2}{HTML}{E9923E}
\definecolor{c3}{HTML}{F8CC47}
\definecolor{c4}{HTML}{428F4D}
\newcommand{\rowHighlight}{\rowcolor{gray!15}}


% Terms used :
\newcommand{\stack}{stacktrace lines\xspace}
\newcommand{\exception}{Exception\xspace}

\newcommand{\failure}{failure message and stacktraces\xspace}
\newcommand{\failures}{failure messages and stacktraces\xspace}


% Table 1 


% Names
\newcommand{\syntax}{text-based matching\xspace}
\newcommand{\classifier}{Failure Log Classifier\xspace}
\newcommand{\tfidf}{{TF-IDF}\xspace}
\newcommand{\flaky}{\emph{OnlyFlaky}\xspace}
\newcommand{\nonflaky}{\emph{Only Non-Flaky}\xspace}
\newcommand{\both}{\emph{Both}\xspace}


% Project names
\newcommand{\alluxio}{\emph{Alluxio}\xspace}
\newcommand{\okhttp}{\emph{Okhttp}\xspace}
\newcommand{\hbase}{\emph{Hbase}\xspace}
\newcommand{\ambari}{\emph{Ambari}\xspace}
\newcommand{\hector}{\emph{Hector}\xspace}
\newcommand{\activiti}{\emph{Activiti}\xspace}
\newcommand{\httpcore}{\emph{Httpcore}\xspace}
\newcommand{\websocket}{\emph{Java-websocket}\xspace}
\newcommand{\logback}{\emph{logback}\xspace}
\newcommand{\wildfly}{\emph{Wildfly}\xspace}
\newcommand{\http}{\emph{Http-request}\xspace}
\newcommand{\spring}{\emph{Spring-boot}\xspace}
\newcommand{\undertow}{\emph{Undertow}\xspace}
\newcommand{\elastic}{\emph{Elastic-job-lite}\xspace}
\newcommand{\orbit}{\emph{Orbit}\xspace}
\newcommand{\exec}{\emph{Commons-exec}\xspace}





\newcommand{\Space}[1]{}
% list of general macros .. ~ removed just to manage when numbers between parenthesis .. ~ will be followed each command instead. 

\newcommand{\numruns}{10,000} % number of runs ..
\newcommand{\numtests}{22,244} % number of all tests ..
\newcommand{\numflakyruns}{811} % number of observed flaky tests .. 
\newcommand{\numprojects}{24} % number of projects in rerun .. 

% after inspection phase
\newcommand{\numtestsinspected}{22,244} % number of tests ( more than 10 flaky + no missing values) .. 
\newcommand{\numflakyinspected}{808}%number of actual flaky tests ..
\newcommand{\numflakypredict}{599}% number of predicted flaky tests ..
\newcommand{\bestfscore}{67\%} % best F1 score
\newcommand{\misstests}{298} % Tests ( flaky + non flaky) with missing values
\newcommand{\missflaky}{three} % flaky with missing values .. 

% number of projects in/out classification and others .. 
\newcommand{\projectsin}{14}% projects have more than or(=) 10 flaky tests .. 
\newcommand{\projectsout}{10}% projects have less than 10 flaky tests ..
\newcommand{\projectsmaxflaky}{163} % a project has maximum number of flaky tests ..
\newcommand{\projectsminflaky}{16} % a project has minimum number of flaky tests ..
\newcommand{\projectsoneflaky}{4} % projects which have only one flaky test ..
\newcommand{\projectshundredsflaky}{4} %  projects which have more than or(=) 100 flaky test ..

% list of flaky tests by projects
\newcommand{\springbootFlaky}{163}
\newcommand{\hbaseFlaky}{145}
\newcommand{\alluxioFlaky}{116}
\newcommand{\okhttpFlaky}{100}
\newcommand{\ambariFlaky}{52}
\newcommand{\hectorFlaky}{33}
\newcommand{\activitiFlaky}{32}
\newcommand{\javawebsocketFlaky}{23}
\newcommand{\wildflyFlaky}{23}
\newcommand{\logbackFlaky}{22}
\newcommand{\httpcoreFlaky}{22}
\newcommand{\incubatordubboFlaky}{19}
\newcommand{\httprequestFlaky}{18}
\newcommand{\wrojFlaky}{16}
\newcommand{\orbitFlaky}{7}
\newcommand{\undertowFlaky}{7}
\newcommand{\achillesFlaky}{4}
\newcommand{\elasticjobliteFlaky}{3}
\newcommand{\zxingFlaky}{2}
\newcommand{\handlebarsFlaky}{1}
\newcommand{\ninjaFlaky}{1}
\newcommand{\assertjcoreFlaky}{1}
\newcommand{\commonexecFlaky}{1}


% FP results .. 
\newcommand{\flaggerfp}{406} % total FP in FlakeFlagger .. 
\newcommand{\msrfp}{4,683} % total FP in MSR .. 
\newcommand{\mergefp}{314} % total FP in combined models .. 


% related Works stats
\newcommand{\idflakiesTotalFlaky}{422} % total flaky tests in idflakies .. 
\newcommand{\idflakiesTotalprojects}{82} % total projects in idflakies ..
\newcommand{\idflakiesNODFlaky}{191} % total NOD flaky tests in idflakies ..
\newcommand{\idflakiesCommonProjects}{3} % Common project with rerun (consider SHA)
\newcommand{\idflakiesRerunFlaky}{70} % flaky tests based on common projects
\newcommand{\idflakiesCommonFlaky}{28} % common flaky tests based on rerun
\newcommand{\idflakiesMissedFlaky}{42} % number of flaky tests that not observed by rerun

\newcommand{\deflakerTotalFlaky}{96} % total flaky tests in deflaker ..
\newcommand{\deflakerTotalProjects}{26} % total projects in deflaker ..
\newcommand{\deflakerCommonProjects}{12} % Common project with rerun (consider SHA)
\newcommand{\deflakerRerunFlaky}{20} % flaky tests based on common projects
\newcommand{\deflakerCommonFlaky}{10} % common flaky tests based on rerun
\newcommand{\deflakerMissedFlaky}{10} % number of flaky tests that not observed by rerun


% FLAST part ..
\newcommand{\FLASTdefalker}{57}
\newcommand{\FLASTidflakies}{258}
\newcommand{\FLASTsmells}{3,424}

% Percentages lists ... 
\newcommand{\missingtestsrate}{1.3} % ratio of tests with missing values to total tests.. 
\newcommand{\flakytestsrate}{3.6} % ratio of flaky tests to total tests..
\newcommand{\highestflakyrate}{20} % ratio of highest project (spring-boot) to total number of flaky tests ..
\newcommand{\redbarsratio}{8} % how many projects where red bar appears as the majority ..
\newcommand{\alluxiofailrate}{90}% ratio of flaky test failure in alluxio project .. 
\newcommand{\NumFailingRunsTen}{33}% 268 flaky tests fail <=10 times .. 
\newcommand{\NumFailingRunsHundred}{22}% 175 flaky tests fail (10:100] times .. 
\newcommand{\NumFailingRunsThousand}{19}% 157 flaky tests fail (100:1000] times .. 
\newcommand{\NumFailingRunsOthers}{26}% 210 flaky tests fail >1000 times .. 


% missclassified result .. 
\newcommand{\missclassifiedrate}{26\%}% percentage of FN to flaky tests
\newcommand{\FNokhttp}{63}% FN in okhttp
\newcommand{\FNfourprojects}{74}% FN in activiti, logback, httpcore and wro4j
\newcommand{\Flakyfourprojects}{92}% flaky tests in activiti, logback, httpcore and wro4j


% smells part .. 
\newcommand{\smellsflaky}{80\%}% percentage of flaky tests which have at least one smell
\newcommand{\smellsnonflaky}{85\%}% percentage of non flaky tests which have at least one smell


\usepackage{textcomp}

{\makeatletter
 \gdef\jonmark{%
   \expandafter\ifx\csname @mpargs\endcsname\relax % in minipage?
     \expandafter\ifx\csname @captype\endcsname\relax % in figure/caption?
       \marginpar{\textcolor{blue}{jon~}}% not in a caption or minipage, can use marginpar
     \else
       \textcolor{blue}{jon~}% notice trailing space
     \fi
   \else
     \textcolor{blue}{jon~}% notice trailing space
   \fi}
 \gdef\jon{\@ifnextchar[\jon@lab\jon@nolab}
 \long\gdef\jon@lab[#1]#2{{\bf [\jonmark \textcolor{blue}{#2} ---{\sc #1}]}}
 \long\gdef\jon@nolab#1{{\bf [\jonmark \textcolor{blue}{#1}]}}
  % This turns them off:
%  \long\gdef\jon@lab[#1]#2{}\long\gdef\jon@nolab#1{}%
}

{\makeatletter
 \gdef\checkmark{%
  \expandafter\ifx\csname @mpargs\endcsname\relax % in minipage?
     \expandafter\ifx\csname @captype\endcsname\relax % in figure/caption?
      \marginpar{\textcolor{red}{CHECK~}}% not in a caption or minipage, can use marginpar
     \else
      \textcolor{red}{CHECK~}% notice trailing space
     \fi
  \else
     \textcolor{red}{CHECK~}% notice trailing space
  \fi}
 \gdef\check{\@ifnextchar[\check@lab\check@nolab}
 \long\gdef\check@lab[#1]#2{{\bf [\checkmark \textcolor{red}{#2} ---{\sc #1}]}}
 \long\gdef\check@nolab#1{{\bf [\checkmark \textcolor{red}{#1}]}}
  % This turns them off:
%  \long\gdef\check@lab[#1]#2{}\long\gdef\check@nolab#1{}%
}

{\makeatletter
 \gdef\michaelmark{%
   \expandafter\ifx\csname @mpargs\endcsname\relax % in minipage?
     \expandafter\ifx\csname @captype\endcsname\relax % in figure/caption?
       \marginpar{\textcolor{blue}{michael~}}% not in a caption or minipage, can use marginpar
     \else
       \textcolor{blue}{michael~}% notice trailing space
     \fi
     \textcolor{blue}{michael~}% notice trailing space
   \fi}
 \gdef\michael{\@ifnextchar[\michael@lab\michael@nolab}
 \long\gdef\michael@lab[#1]#2{{\bf [\michaelmark \textcolor{blue}{#2} ---{\sc #1}]}}
 \long\gdef\michael@nolab#1{{\bf [\michaelmark \textcolor{blue}{#1}]}}
  % This turns them off:
%  \long\gdef\michael@lab[#1]#2{}\long\gdef\michael@nolab#1{}%
}


\definecolor{paulcolor}{rgb}{0.44, 0.26, 0.08}
{\makeatletter
 \gdef\paulmark{%
   \expandafter\ifx\csname @mpargs\endcsname\relax % in minipage?
     \expandafter\ifx\csname @captype\endcsname\relax % in figure/caption?
       \marginpar{\textcolor{paulcolor}{paul~}}% not in a caption or minipage, can use marginpar
     \else
       \textcolor{paulcolor}{paul~}% notice trailing space
     \fi
   \else
     \textcolor{paulcolor}{paul~}% notice trailing space
   \fi}
 \gdef\paul{\@ifnextchar[\paul@lab\paul@nolab}
 \long\gdef\paul@lab[#1]#2{{\bf [\paulmark \textcolor{paulcolor}{#2} ---{\sc #1}]}}
 \long\gdef\paul@nolab#1{{\bf [\paulmark \textcolor{paulcolor}{#1}]}}
  % This turns them off:
%  \long\gdef\paul@lab[#1]#2{}\long\gdef\paul@nolab#1{}%
}

\definecolor{abdulcolor}{rgb}{0.0, 0.37, 0.23}
{\makeatletter
 \gdef\abdulmark{%
   \expandafter\ifx\csname @mpargs\endcsname\relax % in minipage?
     \expandafter\ifx\csname @captype\endcsname\relax % in figure/caption?
       \marginpar{\textcolor{abdulcolor}{abdul~}}% not in a caption or minipage, can use marginpar
     \else
       \textcolor{abdulcolor}{abdul~}% notice trailing space
     \fi
   \else
     \textcolor{abdulcolor}{abdul~}% notice trailing space
   \fi}
 \gdef\abdul{\@ifnextchar[\abdul@lab\abdul@nolab}
 \long\gdef\abdul@lab[#1]#2{{\bf [\abdulmark \textcolor{abdulcolor}{#2} ---{\sc #1}]}}
 \long\gdef\abdul@nolab#1{{\bf [\abdulmark \textcolor{abdulcolor}{#1}]}}
  % This turns them off:
%  \long\gdef\abdul@lab[#1]#2{}\long\gdef\abdul@nolab#1{}%
}


\definecolor{dkgreen}{rgb}{0,0.6,0}
\definecolor{gray}{rgb}{0.5,0.5,0.5}
\definecolor{verylightgray}{rgb}{0.94,0.94,0.94}

\definecolor{mauve}{rgb}{0.58,0,0.82}
\definecolor{lightblue}{rgb}{0.61,0.78,0.91}
\definecolor{lightpurple}{rgb}{0.85,0.83,0.91}

\lstdefinelanguage{HTML5}{
        language=html,
        sensitive=true, 
        alsoletter={<>=-},
        otherkeywords={
        % HTML tags
        <html>, <head>, <title>, </title>, <meta, />, </head>, <body>,
        <canvas, \/canvas>, <script>, </script>, </body>, </html>, <!, html>, <style>, </style>, ><
        },  
        ndkeywords={
        % General
        =,
        % HTML attributes
        charset=, id=, width=, height=,
        % CSS properties
        border:, transform:, -moz-transform:, transition-duration:, transition-property:, transition-timing-function:
        },  
        morecomment=[s]{<!--}{-->},
        tag=[s]
}


\lstdefinestyle{inlinecode}{basicstyle=\ttfamily,language=Java, backgroundcolor = \color{lightgray}, breaklines=true}
\lstdefinestyle{inlinesql}{basicstyle=\ttfamily,language=SQL, backgroundcolor = \color{lightgrey}, showspaces=false,                
        showstringspaces=false,
        showtabs=false, breaklines=true}
\lstdefinestyle{inlinehtml}{basicstyle=\ttfamily,language=Java, backgroundcolor = \color{lightgrey}, 
        showtabs=false, breaklines=true} %I actually DONT think we should have code colors in this so this is set to "java" to not highlight HTML tags.
\lstset{
	language=Java,
        basicstyle=\small,
	breaklines=true,
	showspaces=false,
	showstringspaces=false,
	commentstyle=\color{dkgreen},
	stringstyle=\color{mauve},
	abovecaptionskip=2pt,
	captionpos=b,
	numbers=left,
	frame=single,
	xleftmargin=10pt,
	xrightmargin=5pt,
	numbersep=2pt,
	framexleftmargin=7pt,
	framexrightmargin=0pt,
	numberstyle=\color{gray},
	keywordstyle=\color{blue},
	tabsize=2,
	keepspaces=true,
	moredelim=[is][\underbar]{ZZ}{ZZ},
    escapeinside={/*@}{@*/},
	upquote=true
}

\lstdefinestyle{smalljava}{
	language=Java,
        basicstyle=\footnotesize,
	breaklines=true,
	showspaces=false,
	showstringspaces=false,
	commentstyle=\color{dkgreen},
	stringstyle=\color{mauve},
	abovecaptionskip=2pt,
	captionpos=b,
	numbers=none,
	frame=single,
	xleftmargin=4pt,
	xrightmargin=4pt,
	numbersep=0pt,
	framexleftmargin=0pt,
	framexrightmargin=0pt,
	numberstyle=\color{gray},
	keywordstyle=\color{blue},
	tabsize=2,
	keepspaces=true,
	moredelim=[is][\underbar]{ZZ}{ZZ},
    escapeinside={/*@}{@*/},
	upquote=true
}
\lstdefinestyle{smalljavanumbered}{
	language=Java,
        basicstyle=\footnotesize,
	breaklines=true,
	showspaces=false,
	showstringspaces=false,
	commentstyle=\color{dkgreen},
	stringstyle=\color{mauve},
	abovecaptionskip=2pt,
	captionpos=b,
	numbers=left,
	frame=single,
	xleftmargin=4pt,
	xrightmargin=4pt,
	numbersep=0pt,
	framexleftmargin=3pt,
	framexrightmargin=0pt,
	numberstyle=\color{gray},
	keywordstyle=\color{blue},
	tabsize=2,
	keepspaces=true,
	moredelim=[is][\underbar]{ZZ}{ZZ},
    escapeinside={/*@}{@*/},
	upquote=true
}


% \newcommand{\code}{\lstinline[style=inlinecode]}
\newcommand{\inlinesql}{\lstinline[style=inlinesql]}
\newcommand{\inlinehtml}{\lstinline[style=inlinehtml]}

\begin{document}
\pagestyle{plain}
\pagenumbering{roman}
% \maketitle

\begin{titlepage}
   \begin{center}
       \vspace*{1cm}

       \textbf{\LARGE Engineering Accessible Software}

       \vspace{2.5cm}
        \textit{Thesis proposal} 
        
       \vspace{2.5cm}

       {\bf Arun Krishna Vajjala}  \\
Department of Computer Science \\
George Mason University\\
Fairfax, VA 22030\\
 akrishn@gmu.edu \\

\vfill
            
\textbf{Committee}\\
Kevin Moran, University of Central Florida (Chair)\\
Brittany Johnson-Matthews, George Mason University \\
Andrian Marcus, George Mason University \\ 
Thomas LaToza, George Mason University \\
Vivian Motti, George Mason University \\
    
\vspace{0.8cm}
 
    
\date{\today}
\end{center}
\end{titlepage}

\pagebreak

\begin{abstract}
Software has become an integral part of daily life, spanning web applications, smartphone apps, and desktop systems. As reliance on technology grows, ensuring accessibility for users of all abilities is important. This is not only an ethical motivation but also a governmental requirement. Consequently, there's been a notable increase in efforts to make software more accessible, driving the need for innovative tools to assist developers in this endeavor.

This thesis explores advancements in software engineering research, particularly in the area of accessibility testing and developer tools. Leveraging state-of-the-art techniques like computer vision and machine learning, novel approaches are being developed to automatically detect accessibility issues within software interfaces. 

The core challenge lies in developing models capable of understanding UI designs while capturing their semantic meanings accurately. Addressing this challenge, the thesis proposes the development of tools to assist developers in enhancing software accessibility and tackling future problems requiring a deeper understanding of UI semantics.

Specifically, the thesis proposes a computer vision-based tool to identify motor-impairment accessibility issues in Android mobile apps and demonstrates its efficacy against state-of-the-art baselines. Additionally, a screen understanding tool leveraging computer vision techniques is proposed to aid in screen recognition and retrieval tasks, notably improving search efficiency.

The thesis outlines the background, main contributions, and future directions of the proposed research. It highlights the importance of addressing software accessibility, explores new techniques for detecting accessibility issues and failures, and lays out a research plan for further investigation and development in this domain.


\end{abstract}

\pagebreak
\tableofcontents
\pagebreak
\listoffigures
\listoftables
\newpage
\cleardoublepage
\pagenumbering{arabic}


% input each section here .. 

\section{Introduction}
\label{sec:introduction}

Software has become an integral part of everyone's lives and its impact continues to grow. Software takes many forms such as web applications, smartphone applications, and desktop or operating system applications, etc \cite{7}. Our everyday lives depend on the use of critical software applications which allow us to bank, invest, get news, and communicate with others. Touch screen devices such as smartphones and tablets provide a quick and easy means of access to important information and functions within our daily lives. The abundance of critical software introduces the responsibility to ensure people of all abilities and skill level are able to use and access their information. Software is ideally designed as all encompassing, where any user can use it as intended under their own abilities, but that is far from the current situation of software.

Software accessibility has become more important as more users are dependent on smartphones and computers. People of different abilities have found it difficult to use software the way it is currently developed and designed \cite{11}. According to the world health organization (WHO), 15\% of people have some disability \cite{28}, making software accessibility more important to ensure all users are able to use applications as intended.Though software engineers and companies are ethically motivated to create more accessible software, the United States Government is also making efforts to require public websites and services to be accessible \cite{23}. The government in conjunction with the American with Disabilities Act (ADA) introduced legislation which "prohibits discrimination on the basis of disability in the activities of public accommodations" \cite{23,16}. This has lead to a 180\% increase in more accessible software as of 2018 \cite{27}. This change in policy increases a need for more accessible software and tools that will help developers make their applications more accessible.

This effort to make more accessible software, however, has found its limitations. Google and Apple have the largest distribution for applications for the market \cite{17}. Google's Google Play and Apple’s App Store make it convenient for users to both download and create their own applications \cite{17}. Though, these companies have worked to make their own devices more accessible, most of the applications on their app stores are not controlled and developed by them, therefore being largely inaccessible \cite{16}. Large corporations have made their guidelines available to developers to follow \cite{25,26}, but prior work has suggested that, although there is an abundance of accessibility research and guidelines, there has not been much research or work done to educate the large community of developers on accessibility related issues \cite{16, 15}. The current state of software accessibility tools are generally ignored by developers because of a lack of concise warnings and difficulty of use \cite{9,16}.As a result, 95\% of the android applications randomly mined had elements that violated android GUI accessibility guidelines making the applications less accessible \cite{15}. This raises the need for better accessibility focused developer tools so that developers can make more accessible applications. 

Software engineering research is constantly innovating how software is made. Software testing and developer tools are constantly evolving and are making their way into the accessibility space. Software testing has been around for decades, but has recently grown into a more complex field with the use of computer vision and machine learning techniques to automatically generate and run tests. Moran et al. \cite{42}, for example, created a testing framework that took in application screenshots and automatically reported GUI design violations set by Android, therefore, letting developers know that their applications are not in accordance with the guidelines that are suggested. This exciting new way of testing could provide an ample amount of ways to test for accessibility guideline violations without developers needing to explicitly check for violations on their own. An extremely important part of software engineering research is the constant need for new, robust developer tools to assist developers in the process of designing and creating software. These tools allow developers to have quick access to information and functions that make the development process more efficient. Zhang et al. \cite{15} created a tool for developers to automatically label the elements in their code that helps screen readers identify each element on the screen. Simple, but intelligent tools such as the ones Zhang et al. \cite{15} proposed would drastically improve how accessible software is made by providing an effortless way for developers to quickly incorporate accessibility features into their software. 

Given the current state of HCI, accessibility, and software engineering research, we can work towards the intersection of these areas and develop exciting new ways to make accessible software. Using data, accessibility research, and software engineering tools, we can address the lack of software accessibility at its core by providing intelligent tools for developers to create accessible software in the developmental and design stages. 


The thesis proposal is organized as follows: Section~\ref{sec:background} provides an introduction of works towards the problem of test flakiness. The main contributions of the thesis are discussed in Section~\ref{sec:thesis}. Section~\ref{sec:detectFlakyTests} provides a summary of the findings related to the detection of flaky tests. Section~\ref{sec:livingTestFlakiness} emphasizes current research on how to identify flaky \emph{failures} and explores techniques for their detection. The main key points for my future work, which form the remainder of my PhD, are discussed in Section~\ref{sec:categotize}. Lastly, the current plan for the remaining phase of my PhD is outlined in Section~\ref{sec:researchPlan}.

\section{Background}
\label{sec:background}

This thesis introduces two ideas within software engineering which intersect at the need for developer focused tools for software accessibility and the comprehension of the UI screen to facilitate advanced accessibility improvement techniques. To gain a deeper understanding of this intersected research area, it is important to understand the motivation behind each research area and how they benefit each other. 

\subsection{Software Accessibility}

Research in software accessibility relies on the knowledge of various demographics of the disabled community. Most research is focused on solving a problem within a single population of disabled people, for example, a study could test to see if Android applications are accessible to Motor-impaired users \cite{Alshayban20}, while another could develop a tool to label UI elements in order to help screen readers read out UI elements to visually impaired people \cite{Salehnamadi21}. This section will go through how people with different disabilities use touch screen devices. 

\noindent \textbf{\textit{Visually Impaired Users}}

Visually impaired users use screen readers about 90.5\% of the time \cite{Salehnamadi21}. Low vision (LV) users, or users who can see, but not to the extent that legally blind people can, do not always depend on screen readers. They depend on being able to see contrast in UI design and fonts, making it so that elements and text on the screen are clearly visible and large enough for them to read \cite{IOSDesign}. LV users do not only depend on being able to see and navigate the screen, various keystrokes and actions are required in order to interact with touch screen devices. 

\begin{figure}
    \centering
    \includegraphics[width=0.5\textwidth]{imgs/hits.jpg}
    \caption{Polygons representing clusters of hit points for LV users \cite{4}}
    \label{fig:HitPoint}
\end{figure}

An example of this is shown in Figure \ref{fig:HitPoint}. LV users have difficulty seeing the screen and resort to guessing where their taps land. The image shows clustered taps in an attempt to type out sentences on a touch device. It is clear that LV users are not properly able to tap where they intend to. Speech to text \cite{AppleAccess,GoogleAccess} or braille inputs are a way for LV users to effectively input information into the device \cite{Azenkot12}. Keystrokes are identified by gesture recognizers built within touch screen devices \cite{Peng19}. These recognizers are built with the able-bodied user in mind, so companies like Apple and Google allow users to modify the sensitivity and duration of the touch inputs \cite{Peng19,AppleAccess,GoogleAccess}. These accessibility measures help LV users use their devices more efficiently.\\ 

\noindent \textbf{\textit{Deaf and Hard of Hearing Users}}
 
Deaf and Hard of Hearing (DHH) users, unlike LV users, are comfortably able to physically use their touchscreen devices. DHH users need live captioning or American-Sign-Language (ASL) interpretations of any audio outputted by the device \cite{Berke17}. There are currently tools that provide live captioning \cite{Stream22} and research that is being done in converting audio to ASL depictions for users \cite{Bragg21}. DHH users also rely on text-to-speech or limited verbal communication to provide audio inputs into their devices \cite{Fok18}.\\

\noindent \textbf{\textit{Motor-impaired Users}}
\begin{figure}[h]
    \centering
    \includegraphics[width=0.6\textwidth]{imgs/switchInterface.jpg}
    \caption{Switch Interface}
    \label{fig:SwitchInterface}
\end{figure}

Motor-impaired users use devices in two main ways: Switch input and touch input \cite{Zhang13}. Switch based input uses an external hardware input device. Switches are common in users with limited to no motor control who still retain their cognitive functions \cite{Zhang13}. Current touch devices offer scanning methods that scan a curser across the screen and highlights them as they pass \cite{AppleAccess,GoogleAccess}. The users then can click the switch and select the application that is highlighted and continue using the device. An example of this switch interface is shows in Figure \ref{fig:SwitchInterface}. The most commonly used switch is a single input switch, which is then connected to an adapter to create signals for the device, and then to the device, in this case, an iPad. The iPad then can scan and highlight rows and columns for the user to tap on the switch and select. This gets very tedious as average words per minute typed using tradition scanning based techniques is 3, which is significantly lower than the average normal users, which is, 39 \cite{MacKenzie11}. The other form of input, touch input, is the same as a normal user, but motor-impaired users generally have a tremor or lack of speed in gestures made on the screen. Slightly motor-impaired users also find themselves using voice commands to navigate applications and send messages since keyboards can be frustrating to use with smaller keys \cite{Zhang13}. Work has been done to improve gesture recognition. Google and Apple offer sensitivity and touch settings that allow users to better physically interact with their devices \cite{Peng19,AppleAccess,GoogleAccess}.\\ 

\noindent \textbf{\textit{Neurodivergent and Cognitively Impaired Users}}

This demographic of users is primarily classified as individuals with cerebral differences such as Attention-deficit/hyperactivity disorder (ADHD), Autism Spectrum Disorder (ASD), dyslexia, and memory loss. It is estimated that 7\% of the population identifies as neurodivergent-encompassing cognitive and learning disabilities \cite{Race21}. These users, unlike the users presented previously, are severely underrepresented in design of the web. The Web Content Accessibility Guidelines (WCAG) \cite{WebGuide} requires accessibility measures for LV, DHH, and motor-impaired users, but does not require the implementation of neurodivergent supporting guidelines. Software and tools that neurodivergent users rely on tend to be low-sensory to avoid sensory overload. Users who are distracted easily or have highly receptive cognitive functionality tend to need software where there is a clear focal point to each functionality in the application without the need of unnecessary animations and sounds \cite{Rudy21}. The current set of applications provides many complex avenues of accessing information, but can be challenging to neurodivergent users. An application like Twitter, for example, has an abundance of scrolling animation and task bars that can be used to navigate the app. The animations, videos playing, and options of navigation can overload users since there is no focal point to the screen and what the user should be looking at. 

\subsection{Developer Tools for Accessibility}

The current landscape of developer tools is limited. Studies have shown that current developers do not utilize tools or ignore them because they introduce warnings, some of which are completely wrong \cite{Alshayban20}. In this survey we found very few developer-specific tools, but found many studies that looked at software guidelines for developers. Guidelines are set for developers on all platforms to make their websites and applications more accessible \cite{AppleAccess,GoogleAccess,Bajammal21,Vendome19}. GUI guidelines require developers to abide by certain standards and practices in order to help disabled users navigate and use the application as intended. Guidelines specify certain requirements for elements such as size, placement, and color contrast to help users see tap and see them easier \cite{AppleAccess,GoogleAccess}.


\subsection{UI Understanding}

Much of the research in accessible GUI analysis has been done in UI enhancement and UI comprehension for use in accessibility studies, hence it is important to understand the current state of UI comprehension techniques. Accessibility based UI augmentation and comprehension is only as good as the current techniques in the field. 

Wu et al. \cite{Wu21} worked on a screen parser that is able to predict relationships within elements on the screen. They did this by training a Faster-RCNN using thousands of both IOS and Android screenshots. Then they  determined node correspondence and extracted hierarchical correspondence for each of the elements. Then they grouped the elements together into groups that were under a certain category. So interactive elements were one group while background and images would have been another group \cite{Wu21}. This work was done to better understand the relation between elements on the screen, but can also be used to label and identify screen elements and traverse them in a better order for users with disabilities. This approach was expensive and the UIElement detector they used was observed to slow other processes down.

Most accessibility software tools have been related to testing where researchers tend to find accessibility guideline violations. Moran et al. \cite{Moran18} worked to create an automated GUI checker that checks to see if GUIs were made to their intended design \cite{Moran18}. They then used a GUI comprehension technique and a set of design violations to work on a detector that would detect design violations. It gave promising results that correctly detected design violations in Android apps. This visual GUI testing (VGT) is a great way to start testing for more accessible GUIs. This testing method, however, does not allow GUI boxes to overlap and requires a straightforward GUI. This can hinder GUI mockups and novice developers from being able to test for violations. 

Like the project by Moran et al \cite{Moran18}, UIBert by Google \cite{bai2021uibert} works to identify and comprehend screen elements. It works by taking in UIs and parsing it to identify all of the elements. It then uses that information to predict which type of app it is. It classifies the icons and then uses those generalized icon predictions to make an app prediction. This is a way for UI comprehension to be useful for screen readers or voice assistants to be able to access a certain type of app, e.g. games or music \cite{bai2021uibert}.  

It is really important to understand GUIs when looking to help developers make more accessible software because most techniques from GUI comprehension are used to label and modify GUIs to make applications more accessible, so by looking at the current research in this field, we can see how GUI accessibility is dependent on advancements in GUI comprehension. 

\subsection{Novelty Statement}

I aim to leverage the work that has been done in the past and build upon it with my thesis. Many accessibility testing tools have been unable to address issues due to the previous state of data and technology. Given the progress being made, this thesis will differentiate itself by proposing machine learning and computer vision based approaches to address previously unsolved problems. 



























\newpage
\section{Thesis}
\label{sec:thesis}

% \abdul{The remaining: Maping the RQs with the correct subsection (e.g. 5.1.2). Will be finalized}

% \abdul{After last discussion with Jon and Paul on July 26, this section has been added.}
% \abdul{ I need to make the proposal clear before start }

% \begin{itemize}
%     % \item I need to check the structure of the CS proposal. 
%     \item Thesis: ML and data science can address better the problem of test flakiness in terms of detecting, living with, and categorizing test flakiness. 
% \begin{itemize}
%     \item{Detecting Flaky Tests}
%     Words, research questions
%     \item{Living with test flakiness}
%     Words (detecting failure tests isn't enough; developers keep flaky tests.  How does a developer deal with flaky tests in practice?  Answer:  Flaky failure detection), research questions
%     \item{Categorizing flaky failures}
%     Words, research questions
% \end{itemize}
 

%     % \item each chapter ( make the intro clear that map with the thesis goal then ends with a summary point.

%     % \item I need to make the high level clear before go to the details. (due to next week)
% \end{itemize}


\subsection{Problem Statement}

Test flakiness presents significant challenges in software testing and development. As previously discussed in Section~\ref{sec:background}, various existing works show variable success, starting from detecting flaky tests to fixing them.
Yet, despite these works, test flakiness continues to be a significant issue. While some solutions rely on rerun-based approaches that introduce overhead or depend on extra metrics like code coverage, others struggle to scale effectively across varied codebases. Most importantly, several of these methods often fall short to align with the practical needs of developers as they need to detect flaky tests without any overhead costs. Given the critical role of tests in software development and the presence of test flakiness problem, there is a need for comprehensive and efficient solutions to detect and deal with test flakiness.



% \jon{Small fix - might help to describe what the problems are that existing works aim to address. I think that there are probably two issues: 1) prior work doesn't ask all of the questions that we think are important, and 2) prior work may not answer those questions sufficiently adequately}





% Some of the works related to detecting flaky tests are basically rerun-based-approaches and 

% As previously outlined, test flakiness presents significant challenges in the field of software testing and development as can impact the progress of software releases and the overall reliability of the testing outcomes.
% Existing methods toward test flakiness show variable success, ranging from accurately labeling a test as flaky to the more complicated task of identifying and fixing the flakiness root cause.
% \jon{Small fix - might help to describe what the problems are that existing works aim to address. I think that there are probably two issues: 1) prior work doesn't ask all of the questions that we think are important, and 2) prior work may not answer those questions sufficiently adequately}
% Despite advancements in this field, the issue of test flakiness still demands further focus to aid developers to deal with test flakiness. The detection of test flakiness continues to be a challenge that requires additional effort to enhance the reliability of software testing activities.
% While there have been strides towards proposing tools and strategies to manage and mitigate test flakiness, there remains substantial room for innovation and improvement.


\subsection{Thesis Statement}

% \abdul{my thesis is that ML and DS .... } 
% \abdul{Missing at the end: I will evaluate the thesis by ... }

% \sout{The main objective of this thesis is to make meaningful contributions to the field of test flakiness and aid the research community. The thesis emphasizes on how machine learning and data science can address better the problem of test flakiness in terms of detecting, living with, and categorizing test flakiness. Specifically, the thesis aims to illustrate the impact of employing machine learning classifiers to assist developers in identifying test flakiness, and examine how data science can enhance decision-making processes related to test flakiness.}

The thesis statement is that machine learning and data science can address \emph{better} the problem of test flakiness in terms of detecting, living with, and categorizing test flakiness. To investigate this, I evaluate current methodologies addressing test flakiness based on the discussed problem statement. By understanding the needs from these evaluations, I discuss and propose solutions based on machine learning and data science to overcome the needs. I evaluate the thesis by discussing each proposed solution with research questions, which are detailed in following subsections.


\subsubsection{Detecting Flaky Tests}

I started my research in detecting flaky tests by conducting a rerun experiment to analyze this approach. During the experiment, I investigate the frequency of detecting flaky tests within a specific number of test suite runs and analyze the possibility of reproducing previously detected flaky tests by other studies.
Follow this experiment, I investigate using machine learning techniques to predict if a test is flaky or not learning from other flaky tests in the test suite. I propose \sysName, a machine learning classifier to make predictions for new tests by utilizing data of both flaky and non-flaky tests. In addition to the purpose of \sysName, it could be valuable as it allows developers to minimize the cost of re-running tests by focusing on tests that \sysName predicts as being more likely to become flaky. 
% Detecting flaky tests proactively, before they actually fail, enables developers to pay close attention to these tests and take appropriate actions.
To address the challenge of detecting flaky tests using my experiment of the re-run and \sysName, I am answering the following questions:

% I investigate the frequency of detecting flaky tests within a specific number of test suite runs and analyze the possibility of reproducing previously detected flaky tests under different rerun environment setups.
% The traditional method used to detect flaky tests is by running a test multiple times. However, I am well aware of the impracticality and cost associated with this technique. To start with, my research starts with assessing the reproducibility of flaky tests using this re-run approach. I investigate the frequency of detecting flaky tests within a specific number of test suite runs and analyze the possibility of reproducing previously detected flaky tests under different rerun environment setups.


\begin{description}
% \setlength{\itemindent}{3em}
\item[\textbf{RQ \ref{FlakeFlaggerRQ1}:}]
How many flaky tests can be found by rerunning tests given different rerun budgets?
\item[\textbf{RQ \ref{FlakeFlaggerRQ2}:}] 
How hard is it to reproduce a flaky test failure? 

\item[\textbf{RQ \ref{FlakeFlaggerRQ3}:}] How effective is \sysName at predicting flaky tests?
\item[\textbf{RQ \ref{FlakeFlaggerRQ4}:}] How helpful \sysName's features in distinguishing between flaky and non flaky tests?


\end{description}

The related findings of the detection of flaky tests are detailed in Section~\ref{sec:detectFlakyTests}. Specifically, the first two research questions are discussed in Section~\ref{sec:flakeFlaggerStudy}. The responses to research questions 3 and 4 are found in Section~\ref{sec:flakeFlaggerClassifier}. The answers to all these questions are primarily summarized from the paper ``FlakeFlagger: Predicting Flakiness Without Rerunning Tests"\cite{alshammari2021flakeflagger}.



% Living ... 
\subsubsection{Living with Test Flakiness}
Even with the detection of flaky tests, they continue to exist in test suites. Developers often keep these tests for various reasons, such as understanding these tests impacts or they may be relied upon to detect true (non-flaky) failures. 
% Generally, tests may produce both types of failures.
Hence, for a given failure from a known flaky test, how to determine if the failure is flaky or true failure. Recent studies show that developers can recognize a failure is flaky by examining the failure message and stacktraces as they could have encountered flaky failures with similar failure message and stacktraces\cite{gradlePreventingFlaky}. A recent study refer to the process of identifying two failures with matching failure messages and stacktraces as \emph{failure de-duplication}. Based on this approach, I am studying the failure logs as a source to compare a new failure with both flaky and true failures and using approaches based on \emph{failure de-duplication} to determine if the failure is flaky or not~\cite{Podgurski03Automated,Jiang17WhatCauses}. As it is possible not to have previous flaky failure to compare with, I have proposed machine learning classifiers to learn from already existed flaky and true failure from other flaky tests. This lead me to formulate the following research questions:


% I am studying the failure logs of flaky failures to uncover unique characteristics that set them apart from non-flaky failures.

% This situation highlights the significance of developers accepting the existence of flaky tests, but distinguishing failures that may be come from these flaky tests by identifying flaky and non-flaky failures, as tests may produce both types of failures.

% In response to the observation that developers often rely on their previous knowledge with flaky failures, my research focuses on determining if a given failure is flaky by comparing it with previous flaky failures. Furthermore, I am studying the failure logs of flaky failures to uncover unique characteristics that set them apart from non-flaky failures. This investigation involves a tool I am proposing, referred to as \syntax, designed to compare the failure logs of flaky failures with the non-flaky failures for each test being studied. If this approach is successful in highlighting differences, these unique elements could serve as distinguishers between the two types of failures. In addition to \syntax, it could be useful to create a classifier trained on both types of failure logs, which could then be utilized for predicting flaky failures. This line of investigation lead me to the formulation of the following research questions:

\begin{description}
  \item[\textbf{RQ \ref{matchingRQ1}:}] How often are flaky failures repetitive?
  \item[\textbf{RQ \ref{matchingRQ2}:}] With prior flaky and true failures, is it feasible to use the failure de-duplication to tell if a failure is flaky or true one?
  \item[\textbf{RQ \ref{matchingRQ3}:}] How far utilizing machine learning being helpful in finding the differences between flaky and true failures?  
 \end{description}


The research and its findings are detailed in Section~\ref{sec:livingTestFlakiness}. Specifically, Section~\ref{sec:approaches} introduces the proposed approaches for failure de-duplication. In Section~\ref{sec:matchingEvaluation}, the methodology for addressing the research questions is discussed, and the findings are presented after answering the questions, as outlined in Section~\ref{matchingResult}. This works is already submitted.



\subsubsection{Categorizing Flaky Failures}

Identifying the cause behind test flakiness can aid in assessing flaky failures effectively. In the remainder of my thesis, I am working on proposing a tool designed to categorize flaky failures based on failure logs by clustering failures where each cluster should represent one root cause.
% The evaluation of this tool should be conducted on well-understood causes for a collection of known flaky failures. 
% The goal is to group flaky failures, where each group represents a specific root cause. 
I will explore leveraging machine learning to learn from the features of collected failures and predict the root causes.
This portion of the thesis is currently in the discussion phase, with particular focus on the process of data collection and the definition of the research methodology. This work remains flexible and may be adjusted to reflect the current trends and concerns within the broader research community, but I am initially focus to answer these research questions. 

% My main objective is to evaluate if failure logs from flaky tests can be leveraged to categorize the root causes of test flakiness and to explore how machine learning might assist in this task. 
\begin{description}
    \item \textbf{RQ \ref{future1}}: Is it possible for a flaky test to be triggered by multiple flakiness root causes?
    \item \textbf{RQ  \ref{future2}}: Can failure logs associate flaky failures with their root causes using machine learning?
\end{description}

The discussion of these research questions, along with detailed insights on the topic of categorizing flaky failures, can be found in Section~\ref{sec:categotize}.




\section{Detecting Flaky Tests}
\label{sec:detectFlakyTests}

The detection of flaky tests is a crucial aspect of software testing and development. This process not only protect resources by preventing wasted efforts on resolving misleading failures but also influences software release timelines by providing clearer insights into test outcomes. When test failures are accurately identified as flaky, developers can make more efficient release decisions and avoiding unnecessary delays. My motivation to work in this field falls in the significance of flaky test detection in enhancing software development processes and quality. My goal is to analyze existing detection techniques, assess their strengths and limitations, and use these insights to propose more detection techniques. 

Running tests many times is the traditional way to find flaky tests. Unfortunately, there is little industrial and academic guidelines regarding how many times to rerun each test in order to check if there are flaky or not. Prior studies consider many different numbers to run each test such as 10 \cite{bell2018deflaker}, 16 \cite{lam2019idflakies}, 100 \cite{lam2019root} or, 4,000 times \cite{lam2020Understanding}.
As there are various non-deterministic reasons behind flaky tests, it is hard to claim that developers will observe test flakiness within a fixed number of runs. 
Running tests could not be a major problem if developers can ensure that within e.g. 5 runs flaky tests can be detected.
Another problem related to rerun is that it could be hard to reproduce the flaky failure detected in the original environment using another environment (e.g. developers who locally debug flaky tests which failed on a server). This problem of reproducing flaky tests is due the lack of knowledge about the non-deterministic source that cause flaky failures whether it is due to Java version, network speed, etc. 


Alternatively, I am proposing \sysName, a Machine Learning (ML) approach to identify which tests in a test suite are flaky, \emph{without} rerunning them many times. \sysName learns from existing flaky tests in order to predict unseen tests if they are flaky or not. \sysName can be used to search for flaky tests in a large test suite, where developers identify that a portion of the test suite is or is not flaky, and use \sysName to help label the rest of the tests as flaky or not. Also, \sysName could help in terms \emph{prioritize} which tests should be run first by reporting which tests are most likely to be flaky. 


By proactively identifying flaky tests, I may also help developers understand why these tests are flaky.
Prior work has suggested different properties of tests that might make them more likely to be flaky, and \sysName can report which of these features are present in each test~\cite{eck2019understanding,ahmad2021empirical}.
In practice, if a feature has a strong correlation with flakiness, developers might choose to focus on this feature in their future test maintenance and development activities. 



\subsection{RERUN: Empirical Study}
\label{sec:flakeFlaggerStudy}
%
% This section is a summary of the study conducted in my paper ``FlakeFlagger: Predicting Flakiness Without Rerunning Tests" published in ICSE2021 \cite{alshammari2021flakeflagger}.

% I am motivated to conduct this study in order to answer these main research questions:

% \begin{itemize}
% \setlength{\itemindent}{3em}
% \item[\textbf{RQ1:}]
% How many flaky tests can be found by rerunning tests given different rerun budgets?
% \item[\textbf{RQ2:}] 
% How hard is it to reproduce a flaky test failure? 

% \end{itemize}

% \subsubsection{Study Design}
To gather data on flaky tests, 24 Java projects were selected and run, some of which had been previously studied for test flakiness using different revisions \cite{bell2018deflaker} \cite{lam2019idflakies}. The entire test suites of these projects were run 10,000 times, which differed from the previous work \cite{bell2018deflaker} \cite{lam2019idflakies}. Only a single revision of each project was considered, which was either the most recent revision at the time of writing or the same revision studied in \cite{bell2018deflaker} or \cite{lam2019idflakies}.

The rerun scripts were used to break down large experiments into smaller units called "jobs," which were executed on virtual machines. A single job was the execution of a Java test suite on a specific revision of a project using the Maven build system. For each job, the Maven build log and XML reports for each test run were saved. This approach aimed to create a level of isolation between test runs and simulate how an actual integration server would compile, test, and run a project's test suite.


The method I used to detect flaky tests through rerun is not the only approach available. There are other ways to increase the likelihood of detecting flaky tests. For instance, some flaky tests can be affected by the order in which they are run, and running them in different orders may uncover additional flaky tests \cite{lam2019idflakies}. However, this may also introduce a bias towards certain categories of flaky tests. Another way to detect more flaky tests is to run the experiment on different platforms and devices. However, my goal was to align the rerun experiment with standard development practices.


\subsection{FlakeFlagger: Flaky Test Classifier}
\label{sec:flakeFlaggerClassifier}

This section focuses on the construction process of the \sysName, beginning with the feature collection, followed by the classification process, and finally detailing how all of these elements have been designed as illustrated in Figure~\ref{over_all_graph}.




% This section is a summary of the \emph{approach} section in my paper ``FlakeFlagger: Predicting Flakiness Without Rerunning Tests" published in ICSE2021 \cite{alshammari2021flakeflagger}.



\subsubsection{Features Collections}
\label{sec:detector}
Machine learning classifiers such as \sysName require a set of feature in order to learn and predict. I started with the prior work \cite{luo2014empirical,eck2019understanding,bell2018deflaker} to study which features is highly linked to flakiness. I aim to collect verity of features because of the fact that some flaky tests in different projects often have different root causes for their flakiness\cite{luo2014empirical}. Similarly, some features that are predictive for one project may not be as predictive for others, due to the inherent non-determinism in flaky tests. 
I intentionally collect some dynamic features, in addition to static features (e.g. presence of textual tokens in the body of each test method). This is important because some causes not in the test method itself, but instead, in the production code that is executed by that test \cite{eck2019understanding}.
Ahmed et al. \cite{ahmad2021empirical} categorized 23 developer-reported factors which affect test flakiness. 
These features are described by practitioners at a high level, and include test case complexity, hard-coded values and test smells.
Eck et al. \cite{eck2019understanding} interviewed 21\Space{ professional} developers about flaky tests and tabulated the frequency of different kinds of flaky tests as well as developers' fixes for those flaky tests. 


Inspired by previous studies on test flakiness, I developed a list of sixteen features, 
some of are based on general studies on the causes of flaky tests \cite{luo2014empirical,ahmad2021empirical}, while others are defined as bad practices in writing unit tests.
Hence, I considered all of the features described in the prior works, and then selected only those for which I could write automated detectors.
This ends up with implementing detectors for each of the features shown in Table \ref{table:Feature_desc}. This list of features is not intended to be complete: there may yet be other features that can be easily collected and will be useful for predicting test flakiness.

While some of the features can be detected by inspecting the test method statically (specifically, the conditional logic smell and test line of code), the rest of the features require more than static analysis.
A hybrid static/dynamic framework \emph{detector} was developed to collect the statement coverage of each test, and then statically analyze the covered code in order to collect these behavioral features.
The \emph{detector} also collect a variety of other features related to the statement coverage of each test, such as how many recently changed lines of code are covered.
The \emph{detector} is implemented as an extension to the Maven build system.


\begin{table*}[t]
\scriptsize
%\renewcommand{\arraystretch}{1.3}
    \caption[FlakeFlagger List of Collected Features.]{Complete list of features captured for test flakiness prediction. The Covered Lines Churn feature is represented in multiple forms based on the $h$ values (number of the past commits). In the evaluation, I considered $h=5, 10, 25, 50, 75, 100, 500$ and $10,000$}
    \vspace{-5pt}
    \label{table:Feature_desc}
    \begin{tabularx}{\textwidth}{l | l X}
    \toprule
    & \bfseries Feature & \bfseries Description\\
    \midrule
\parbox[t]{2mm}{\multirow{8}{*}{\rotatebox[origin=c]{90}{Test Smells}}}	&
	Indirect Testing  	&	 True if the test interacts with the object under test via an intermediary \cite{van2001refactoring}  \\
	&	Eager Testing 	&	 True if the test exercises more than one method of the tested object \cite{van2001refactoring} \\
	&	Test Run War  	&	 True if the test allocates a file or resource which might be used by other tests \cite{van2001refactoring} \\
	&	Conditional Logic 	&	 True if the test has a conditional if-statement within the test method body \cite{meszaros2007xunit} \\
	&	Fire and Forget  	&	 True if the test launches background threads or tasks. \cite{garousi2018smells} \\
	&	Mystery Guest 	&	 True if the test accesses external resources  \cite{van2001refactoring} \\
	&	Assertion Roulette  	&	 True if the test has multiple assertions \cite{van2001refactoring} \\
	&	Resources Optimism  	&	 True if the test accesses external resources without checking their availability \cite{van2001refactoring}\\ \hline
\parbox[t]{2mm}{\multirow{8}{*}{\rotatebox[origin=c]{90}{Numeric Features}}}	&	Test Lines of Code   	&	 Number of lines of code in the test method body \\
	&	Number  of Assertions  	&	 Number of assertions checked by the test \\
	&	Execution Time   	&	 Running time for the test execution \\
	&	Source Covered Lines  	&	 Number of lines covered by each test, counting only production code \\
	&	Covered Lines  	&	 Total number of lines of code covered by the test  \\
	&	Source Covered Classes  	&	 Total number of production classes covered by each test \\
	&	External Libraries  	&	 Number of external libraries used by the test \\
	&	Covered Lines Churn 	&	 $h$-index capturing churn of covered lines in past 5, 10, 25, 50, 75, 100, 500, and 10,000 commits. Each value $h$ indicates that at least $h$ lines were modified at least $h$ times in that period.\\
\bottomrule
    \end{tabularx}
    % \vspace{-14pt}
\end{table*}

% \begin{table*}[t]
% \begin{tabular}{l l l}
% & Indirect Test Smell & Testing blah\\
% \end{tabular}
% \end{table*}

\subsubsection{Classification Process}

FlakeFlagger takes a list of tests, where each of them is represented as a vector $\{x_1, x_2, x_3,\dots, x_n\}$  where each \emph{x} represents a feature value and \emph{n} correspond the total number of features. I applied data inspection and cleaning process to make that dataset more clear for the classification. Missing data can exist to the dataset due to the fact that some features collected by the detector can be incomplete e.g. due to crashes in the middle of the test execution. Some tests are not written in Java, and hence the feature detectors may not be applicable to them, and due to inheritance, some tests may not have source code in the project under test.

As in any classification problem, considering multiple supervised learning algorithms could be better. In \sysName, it is designed to use a set of models including Random Forest (\emph{RF}) and Decision Tree (\emph{DT}. I follow a feature selection process using \emph{information gain}, which computes the amount of information that a feature can provide for a classification \cite{lei2012feature}. Imbalanced datasets (where there is not an equal number of instances in each class --- flaky and non-flaky tests in my case) usually have very low information gain values. 


\subsubsection{Experimental Design}
\label{sec:Prediction_Design}
To evaluate \sysName, I used the same dataset described in Section \ref{sec:flakeFlaggerStudy}. I ran the detector described in Section \ref{sec:detector}, to collect the set of features shown in Table \ref{table:Feature_desc}. \sysName, similar to any machine learning classifiers, relies on two data sets: one to build the model (training) and another for testing. Because this is not already designed and I have only one dataset, I applied $k$-fold cross validation \cite{kohavi1995study} to evaluate the model. Following this practice, I split the data into $k$ parts, leave one part for testing and $k-1$ to train the classifier.
I then repeat this process with another $k-1$ parts, each time leaving one part for testing.
However, $k$-fold cross validation is most applicable to data that is evenly balanced, where the proportions of each class (flaky and not flaky) are similar. In fact, most tests are not flaky, which means imbalance data. To overcome this, I applied a sample technique SMOTE \cite{SMOTE} only on training dataset, to ensure a valid and fair result.



In my prediction evaluation, I label each prediction result as a True Positive (TP), False Negative (FN), False Positive (FP), or True Negative (TN) as follows:
TP - predicted flaky, known to be flaky; FP - predicted flaky, not known to be flaky; FN - predicted not flaky, known to be flaky; TN - predicted not flaky, not known to be flaky.
I also evaluate the models using F1-score, which is computed using the standard formula based on Recall and Precision. %, to evaluate how our approach works to detect flaky tests as $TP$s.
%  \abdul{I added the following sentences ... } JB: Looks good!
Lastly, I calculate the Area Under the Curve (AUC), a measure of how effective a model is at distinguishing classes.
% (in my case, flaky and not flaky).

In the evaluation, false positives represent the number of tests that might be considered as flaky by developers, resulting in excess effort spent re-running them to determine if they are flaky or not. I focus primarily on total positives, because I have confidence that the collected flaky tests are indeed flaky, but I cannot be confident in my classification of a test as not flaky. 
In other words, the oracle is a result of detecting flaky tests after \numruns~runs for each test, but this does not guarantee that the ``not flaky'' tests are really not flaky: they may just not have been observed to be flaky. 
This approach also allows us to confirm $FN$s are truly flaky tests because they fail at least once during rerun tests.
However, because of the inherent non-determinism in flaky tests, I cannot construct a reliable oracle to evaluate $TN$s and $FP$s, but report them as-is.



\begin{table*}[t]
    \caption[Flaky tests detected by re-running test suites 10,000 times]{Flaky tests detected by re-running test suites 10,000 times. \textnormal{I estimate the percentage of all flaky tests that would be detected if only 10, 100 or 1,000 reruns had been performed. Color bars are stacked bar charts showing the percentage of tests that failed with a given frequency. Columns \emph{DeFlaker} and \emph{IDFlakies} show the number of non-order dependent flaky tests found in total by those prior works, and the number of flaky tests shared by both datasets. Blank cells indicate that a different revision of the project was used due to historical compilation issues.}}
\label{table:rerun_result}
\vspace{-5pt}
\setlength{\tabcolsep}{1pt}
\setlength{\textwidth}{0.5pt}
\newcommand{\failureRateWidth}{2.5in}
\newcommand{\failureRateHeight}{1em}
\scriptsize
\centering
    \begin{tabular}{lrr|rr|rr|rrr|c}
    \toprule
    & & \textbf{Flaky by} & \multicolumn{2}{c}{\textbf{DeFlaker} \cite{bell2018deflaker}} & \multicolumn{2}{c}{\textbf{iDFlakies} \cite{lam2019idflakies}} & \multicolumn{3}{c}{\textbf{\% Flaky per run}} & \textbf{Distribution of Failure Frequencies, as \% of Tests Failing} \\
    
    \textbf{Project}&\textbf{Tests}& \textbf{Reruns}&\textbf{Shared} & \textbf{Total} & \textbf{Shared}&\textbf{Total} &\textbf{10}&\textbf{100}&\textbf{1,000}& \textbf{
  \textcolor{c1}{ (0,10]} \textcolor{c2}{ (10, 100]} \textcolor{c3}{ (100, 1,000]} \textcolor{c4}{ (1,000, 10,000]} runs of 10,000} \\
\midrule

spring-boot 	&	2,128	&	163			&	0	&	5	&	  	&	  	&	71\%	&	71\%	&	77\%	&	 \includegraphics[width=\failureRateWidth,height=\failureRateHeight]{Figures/barPlot/spring-boot.pdf} 		\\ 
\rowHighlight  hbase 	&	431	&	145			&	0	&	1	&	  	&	  	&	52\%	&	59\%	&	75\%	&	 \includegraphics[width=\failureRateWidth,height=\failureRateHeight]{Figures/barPlot/hbase.pdf} 		\\ 
  alluxio 	&	187	&	116			&	2	&	2	&	  	&	  	&	0\%	&	91\%	&	100\%	&	 \includegraphics[width=\failureRateWidth,height=\failureRateHeight]{Figures/barPlot/alluxio.pdf} 		\\ 
\rowHighlight  okhttp 	&	810	&	100			&	  	&	  	&	  	&	  	&	8\%	&	12\%	&	15\%	&	 \includegraphics[width=\failureRateWidth,height=\failureRateHeight]{Figures/barPlot/okhttp.pdf} 		\\ 
  ambari 	&	324	&	52			&	1	&	1	&	  	&	  	&	0\%	&	2\%	&	94\%	&	 \includegraphics[width=\failureRateWidth,height=\failureRateHeight]{Figures/barPlot/ambari.pdf} 		\\ 
\rowHighlight  hector 	&	142	&	33			&	1	&	1	&	  	&	  	&	3\%	&	3\%	&	100\%	&	 \includegraphics[width=\failureRateWidth,height=\failureRateHeight]{Figures/barPlot/hector.pdf} 		\\ 
  activiti 	&	2,044	&	32			&	  	&	  	&	  	&	  	&	0\%	&	3\%	&	44\%	&	 \includegraphics[width=\failureRateWidth,height=\failureRateHeight]{Figures/barPlot/activiti.pdf} 		\\ 
\rowHighlight  java-websocket 	&	145	&	23			&	  	&	  	&	22	&	52	&	0\%	&	26\%	&	87\%	&	 \includegraphics[width=\failureRateWidth,height=\failureRateHeight]{Figures/barPlot/java-websocket.pdf} 		\\ 
  wildfly 	&	1,238	&	23			&	  	&	  	&	  	&	  	&	0\%	&	0\%	&	4\%	&	 \includegraphics[width=\failureRateWidth,height=\failureRateHeight]{Figures/barPlot/wildfly.pdf} 		\\ 
\rowHighlight  httpcore 	&	712	&	22			&	1	&	1	&	  	&	  	&	0\%	&	9\%	&	9\%	&	 \includegraphics[width=\failureRateWidth,height=\failureRateHeight]{Figures/barPlot/httpcore.pdf} 		\\ 
  logback 	&	842	&	22			&	  	&	  	&	  	&	  	&	5\%	&	9\%	&	41\%	&	 \includegraphics[width=\failureRateWidth,height=\failureRateHeight]{Figures/barPlot/qos-ch-logback.pdf} 		\\ 
\rowHighlight  incubator-dubbo 	&	2,177	&	19			&	  	&	  	&	5	&	12	&	5\%	&	11\%	&	26\%	&	 \includegraphics[width=\failureRateWidth,height=\failureRateHeight]{Figures/barPlot/dubbo-test-integration.pdf} 		\\ 
  http-request 	&	163	&	18			&	  	&	  	&	  	&	  	&	0\%	&	83\%	&	83\%	&	 \includegraphics[width=\failureRateWidth,height=\failureRateHeight]{Figures/barPlot/http-request.pdf} 		\\ 
\rowHighlight  wro4j 	&	1,145	&	16			&	1	&	1	&	  	&	  	&	44\%	&	50\%	&	81\%	&	 \includegraphics[width=\failureRateWidth,height=\failureRateHeight]{Figures/barPlot/wro4j.pdf} 		\\ 
 % jackrabbit-oak 	&	3,994	&	15			&	1	&	1	&	  	&	  	&		&		&		&	 \includegraphics[width=\failureRateWidth,height=\failureRateHeight]{Figures/barPlot/jackrabbit-oak.pdf} 		\\ 
%  \multicolumn{8}{|l|}{ The next projects are \emph{NOT} part in \emph{ML} phase because each of them has less than 10 flaky tests} \\\hline																								
  orbit 	&	86	&	7			&	0	&	1	&	  	&	  	&	14\%	&	43\%	&	86\%	&	 \includegraphics[width=\failureRateWidth,height=\failureRateHeight]{Figures/barPlot/orbit.pdf} 		\\ 
\rowHighlight  undertow 	&	183	&	7			&	0	&	3	&	  	&	  	&	0\%	&	0\%	&	29\%	&	 \includegraphics[width=\failureRateWidth,height=\failureRateHeight]{Figures/barPlot/undertow.pdf} 		\\ 
  achilles 	&	1,317	&	4			&	  	&	  	&	  	&	  	&	0\%	&	25\%	&	75\%	&	 \includegraphics[width=\failureRateWidth,height=\failureRateHeight]{Figures/barPlot/achilles.pdf} 		\\ 
 \rowHighlight elastic-job-lite 	&	558	&	3			&	  	&	  	&	1	&	6	&	0\%	&	0\%	&	0\%	&	 \includegraphics[width=\failureRateWidth,height=\failureRateHeight]{Figures/barPlot/elastic-job-lite.pdf} 		\\ 
  zxing 	&	345	&	2			&	2	&	2	&	  	&	  	&	0\%	&	100\%	&	100\%	&	 \includegraphics[width=\failureRateWidth,height=\failureRateHeight]{Figures/barPlot/zxing.pdf} 		\\ 
 \rowHighlight assertj-core 	&	6,267	&	1			&	1	&	1	&	  	&	  	&	0\%	&	100\%	&	100\%	&	 \includegraphics[width=\failureRateWidth,height=\failureRateHeight]{Figures/barPlot/assertj-core.pdf} 		\\ 
  commons-exec 	&	55	&	1			&	  	&	  	&	  	&	  	&	0\%	&	0\%	&	100\%	&	 \includegraphics[width=\failureRateWidth,height=\failureRateHeight]{Figures/barPlot/commons-exec.pdf} \\ 		
																								
%  dropwizard 	&	428	&	1			&	1	&	1	&	  	&	  	&		&		&		&	 \includegraphics[width=\failureRateWidth,height=\failureRateHeight]{Figures/barPlot/dropwizard.pdf} 		\\ 
\rowHighlight handlebars.java 	&	428	&	1			&	  	&	  	&	  	&	  	&	0\%	&	100\%	&	100\%	&	 \includegraphics[width=\failureRateWidth,height=\failureRateHeight]{Figures/barPlot/handlebars_java.pdf} 		\\ 
  ninja 	&	306	&	1			&	1	&	1	&	  	&	  	&	0\%	&	100\%	&	100\%	&	 \includegraphics[width=\failureRateWidth,height=\failureRateHeight]{Figures/barPlot/ninja.pdf} 		\\ 
%\rowHighlight  togglz 	&	22	&	1			&	  	&	  	&	  	&	  	&		&		&		&	 \includegraphics[width=\failureRateWidth,height=\failureRateHeight]{Figures/barPlot/togglz.pdf} 		\\ 
 \rowHighlight jimfs 	&	212	&	0			&	  	&	  	&	  	&	  	&		&		&		&	 No flaky tests observed		\\ 
%\rowHighlight  oozie 	&	19	&	0			&	  	&	  	&	  	&	  	&		&		&		&	 N/A 		\\ 
%   hadoop 	&	  	&	  			&	  	&	  	&	  	&	  	&		&		&		&	 \includegraphics[width=\failureRateWidth,height=\failureRateHeight]{Figures/barPlot/hadoop.pdf} 		\\ 
%   incubator-dubbo 	&	  	&	  			&	  	&	  	&	  	&	  	&		&		&		&	 \includegraphics[width=\failureRateWidth,height=\failureRateHeight]{Figures/barPlot/incubator-dubbo.pdf} 		\\ 
%   oryx 	&	  	&	  			&	  	&	  	&	  	&	  	&		&		&		&	 \includegraphics[width=\failureRateWidth,height=\failureRateHeight]{Figures/barPlot/oryx.pdf} 		\\ 
%   tachyon 	&	  	&	  			&	2	&	2	&	  	&	  	&		&		&		&	 \includegraphics[width=\failureRateWidth,height=\failureRateHeight]{Figures/barPlot/tachyon.pdf} 		\\ 
%   \hline																								
   																								
     \midrule																								
\textbf{Total} 	&	22,245	&	811			&	10	&	20	&	28	&	70	&	26\%	&	45\%	&	67\%	&	 \includegraphics[width=\failureRateWidth,height=\failureRateHeight]{Figures/barPlot/total.pdf} 		\\ 
\bottomrule 
\end{tabular}

\vspace{-10pt}
\end{table*}





\subsection{Evaluation}
I evaluate my findings in detecting flaky tests by answering the following  main research questions:


\begin{description}
  \item[\textbf{RQ 4.3.1:}] How many flaky tests can be found by rerunning tests given different rerun budgets?
  \item[\textbf{RQ 4.3.2:}] How hard is it to reproduce a flaky test failure?

  \item[\textbf{RQ 4.3.3:}] How effective is \sysName at predicting flaky tests?
  \item[\textbf{RQ 4.3.4:}] How helpful is each feature in distinguishing between flaky and non flaky tests?
  
 \end{description}



By running each test 10,000 times, 811 flaky tests were detected: about \flakytestsrate\% of the total number of tests were flaky. The number of flaky tests in 24 projects are not equally distributed. For example, \projectsout~projects with less than 10 flaky tests, \projectsoneflaky~of which have only one, and one with none. On the other hand, there are \projectshundredsflaky~projects which have more than 100 flaky tests. \emph{Spring-boot} has \springbootFlaky~flaky tests, \highestflakyrate\% of the total observed flaky tests. Table \ref{table:rerun_result} summarizes these results. 

\subsubsection{How many flaky tests can be found by rerunning tests given different rerun budgets?} 
\label{FlakeFlaggerRQ1}

I calculated the probability that each flaky test would have been detected with fewer reruns. It is important to consider that it is not possible to state the probability due to the fact that there could be uncontrolled unaware conditions that cause the failure. As a result, only roughly a quarter of all of the flaky tests that I found in 10,000 runs would have been found with 10 reruns, roughly half with 100 reruns and roughly two thirds with 1,000 reruns.

Table \ref{table:rerun_result} categorizing each flaky test as failing either between 0 and 10 times, between 10 and 100 times, between 100 and 1,000 times, and finally over 1,000 times (out of the 10,000 runs).
These sets are represented by colors as shown in Table \ref{table:rerun_result}. The \emph{red} bar, which refers to tests that flake less than or equal 10 times, takes the majority in \redbarsratio~projects. 
In general, I found more than \NumFailingRunsTen\% of total flaky tests fail in less than or equal 10 times, \NumFailingRunsHundred\% of flaky tests fail more than 10 and less than or equal 100 out of 10,000.
This suggests that flaky tests datasets with limited runs still not ensure to detect \emph{most} flaky tests.
Furthermore, I acknowledge that even after 10,000 re-runs, it is still possible that all flaky tests in this dataset have been detected. 

\subsubsection{How hard is it to reproduce a flaky test failure?}
\label{FlakeFlaggerRQ2}

In the previous \textbf{RQ}, the rerun experiment aims to study the difficulty of identifying flaky tests by re-running them on the same platform. However, this does not capture the difficulty when a developer rerun the tests on different environment e.g. local machine. to meet this, I compared the set of flaky tests identified from the 10,000 reruns with those detected by prior researchers on the same versions of the same projects, but in different environments.
The columns \emph{DeFlaker} and \emph{iDFlakies} in Table \ref{table:rerun_result} show the total number of flaky tests that that paper reported on that version of that project, along with the number of those tests that were also found to be flaky based on the reruns.
A blank entry indicates that the revision of that project that I executed did not have any flaky tests reported by the prior work.
The DeFlaker dataset contains flaky tests from many revisions of each project, but I only studied a single revision of each project, and hence, it is possible that DeFlaker had not identified any flaky tests in that revision.
The iDFlakies dataset consists of both order dependent tests (which are detected by shuffling execution orders), and non-order dependent tests (which are flaky regardless of execution order).
Since I purposefully did not shuffle the execution order of the tests (as described above), I include only the non-order dependent tests from iDFlakies for comparison. 

Comparing to DeFlaker, I found \deflakerCommonFlaky~flaky tests out of the \deflakerRerunFlaky~tests identified as flaky by DeFlaker.
Unfortunately, the DeFlaker authors did not retain the build logs from their test runs, so I are unable to diagnose why those tests appeared as flaky to DeFlaker but not to my reruns.
Comparing to iDFlakies, I found \idflakiesCommonFlaky~flaky tests out of the \idflakiesRerunFlaky~non-order dependent tests that I reran.
In the case of iDFlakies, the authors \emph{did} retain the build logs that show how these tests failed, and I confirmed by hand that the tests that I missed in my rerun experiment truly were flaky, and could have been detected as flaky if I had rerun them more.
These results are indicative of the true non-determinism of flaky tests and the difficulties that developers face reproducing them: even with 10,000 reruns, I could not detect all flaky tests.





%  the new table 3 

\begin{table*}[t]
 \setlength{\tabcolsep}{1.6pt}

\caption[FlakeFlagger Prediction Result.]{Prediction performance for \sysName, the \vocabName, and the hybrid combination of both.
\textnormal{The hybrid approach builds a model with both \sysName's and the \vocabName's features. I show the number of True Positives, False Negatives, False Positives and True Negatives, Precision, Recall, and F1 scores per-project.
The AUC value is calculated after each fold where the reported value is the overall averages of AUC values after all folds. Projects with zero F1 values have very low numbers of flaky tests (less than 3 per project), and illustrate known limitations of \sysName. } }
%\caption{Prediction performance for \sysName, MSR and the combination between two approaches. 
% \textnormal{For \sysName, showing True Positives, False Negatives, False Positives, True Negatives, Precision, Recall and F1-score. For comparison approaches (FLAST \cite{flast} at $\sigma=0.5$ and $\sigma=0.95$ and random guessing with 50\% probability of flakiness or weighted by the distribution of flaky tests in the project), we show only Precision, Recall and F1-score. Highest F1-score in each row is shown in bold. Since each project uses a different model, we do \emph{not} include a ``total'' row.}
%}
\label{table:Full_result}
\vspace{-5pt}
\resizebox{\textwidth}{!}{
% \scriptsize
\begin{tabular}{l rr| rrrrrrr | rrrrrrr | rrrrrrr}

\toprule
% & \multicolumn{4}{c|}{} & \multicolumn{4}{c|}{\textbf{MSR tool *}} & \multicolumn{4}{c}{\textbf{Both}}\\
% \cmidrule(lr){2-5} \cmidrule(lr){6-10}
& & \textbf{Flaky by} & \multicolumn{7}{c|}{\textbf{\sysName}} & \multicolumn{7}{c|}{\textbf{Vocabulary-Based Approach \cite{pintovocabulary}}} & \multicolumn{7}{c}{\textbf{Combined Approach}}  \\
\cmidrule(lr){4-10} \cmidrule(lr){11-17}  \cmidrule(lr){18-24}
% \cmidrule(lr){2-14} \cmidrule(lr){10-13} \cmidrule(lr){14-17} \cmidrule(lr){18-21} 


\textbf{Project} & \textbf{Tests} & \textbf{Reruns} &\textbf{TP} & \textbf{FN} & \textbf{FP} & \textbf{TN} & \textbf{Pr} & \textbf{R} & \textbf{F} & \textbf{TP} & \textbf{FN} & \textbf{FP} & \textbf{TN} & \textbf{Pr} & \textbf{R} & \textbf{F} &  \textbf{TP} & \textbf{FN} & \textbf{FP} & \textbf{TN} & \textbf{Pr} & \textbf{R} & \textbf{F}\\
\midrule
spring-boot& 2,108 & 160 & 139 & 21 & 15 & 1,933 & 90\% & 87\% & 89\% & 134 & 26 & 703 & 1,245 & 16\% & 84\% & 27\% & 143 & 17 & 18 & 1,930 & 89\% & 89\% & 89\% \\
\rowHighlight hbase &431 & 145& 129 & 16 & 32 & 254 & 80\% & 89\% & 84\% & 89 & 56 & 152 & 134 & 37\% & 61\% & 46\% & 130 & 15 & 33 & 253 & 80\% & 90\% & 84\% \\
alluxio & 187 & 116& 116 & 0 & 0 & 71 & 100\% & 100\% & 100\% & 108 & 8 & 11 & 60 & 91\% & 93\% & 92\% & 116 & 0 & 0 & 71 & 100\% & 100\% & 100\% \\
\rowHighlight okhttp & 810 &100 & 52 & 48 & 159 & 551 & 25\% & 52\% & 33\% & 79 & 21 & 444 & 266 & 15\% & 79\% & 25\% & 46 & 54 & 104 & 606 & 31\% & 46\% & 37\% \\
ambari & 324 & 52& 47 & 5 & 3 & 269 & 94\% & 90\% & 92\% & 36 & 16 & 121 & 151 & 23\% & 69\% & 34\% & 47 & 5 & 3 & 269 & 94\% & 90\% & 92\% \\
\rowHighlight hector & 142 & 33 & 30 & 3 & 8 & 101 & 79\% & 91\% & 85\% & 13 & 20 & 23 & 86 & 36\% & 39\% & 38\% & 25 & 8 & 11 & 98 & 69\% & 76\% & 72\% \\
activiti & 2,043 & 32 & 10 & 22 & 43 & 1,968 & 19\% & 31\% & 24\% & 12 & 20 & 531 & 1,480 & 2\% & 38\% & 4\% & 7 & 25 & 34 & 1,977 & 17\% & 22\% & 19\% \\
\rowHighlight java-websocket & 145 & 23& 19 & 4 & 1 & 121 & 95\% & 83\% & 88\% & 23 & 0 & 74 & 48 & 24\% & 100\% & 38\% & 19 & 4 & 4 & 118 & 83\% & 83\% & 83\% \\
wildfly &1,023 & 23 & 11 & 12 & 27 & 973 & 29\% & 48\% & 36\% & 20 & 3 & 554 & 446 & 3\% & 87\% & 7\% & 17 & 6 & 24 & 976 & 41\% & 74\% & 53\% \\
\rowHighlight httpcore & 712 & 22& 14 & 8 & 23 & 667 & 38\% & 64\% & 47\% & 16 & 6 & 375 & 315 & 4\% & 73\% & 8\% & 15 & 7 & 24 & 666 & 38\% & 68\% & 49\% \\
logback & 805 & 22& 3 & 19 & 17 & 766 & 15\% & 14\% & 14\% & 10 & 12 & 259 & 524 & 4\% & 45\% & 7\% & 5 & 17 & 11 & 772 & 31\% & 23\% & 26\% \\
\rowHighlight incubator-dubbo & 2,174 & 19& 8 & 11 & 35 & 2,120 & 19\% & 42\% & 26\% & 11 & 8 & 813 & 1,342 & 1\% & 58\% & 3\% & 13 & 6 & 23 & 2,132 & 36\% & 68\% & 47\% \\
http-request & 163 & 18& 12 & 6 & 6 & 139 & 67\% & 67\% & 67\% & 16 & 2 & 84 & 61 & 16\% & 89\% & 27\% & 12 & 6 & 6 & 139 & 67\% & 67\% & 67\% \\
\rowHighlight wro4j & 1,135 & 16 & 4 & 12 & 2 & 1,117 & 67\% & 25\% & 36\% & 2 & 14 & 101 & 1,018 & 2\% & 12\% & 3\% & 0 & 16 & 1 & 1,118 & 0\% & 0\% & 0\% \\
orbit & 86 & 7& 1 & 6 & 8 & 71 & 11\% & 14\% & 12\% & 6 & 1 & 32 & 47 & 16\% & 86\% & 27\% & 1 & 6 & 7 & 72 & 12\% & 14\% & 13\% \\
\rowHighlight undertow & 183 & 7& 2 & 5 & 8 & 168 & 20\% & 29\% & 24\% & 6 & 1 & 63 & 113 & 9\% & 86\% & 16\% & 3 & 4 & 8 & 168 & 27\% & 43\% & 33\% \\
achilles &1,317 & 4 & 2 & 2 & 3 & 1,310 & 40\% & 50\% & 44\% & 0 & 4 & 0 & 1,313 & 0\% & 0\% & 0\% & 0 & 4 & 0 & 1,313 & 0\% & 0\% & 0\% \\
\rowHighlight elastic-job-lite & 558 &3 & 0 & 3 & 0 & 555 & 0\% & 0\% & 0\% & 0 & 3 & 34 & 521 & 0\% & 0\% & 0\% & 1 & 2 & 0 & 555 & 100\% & 33\% & 50\% \\
zxing & 345 & 2 & 0 & 2 & 2 & 341 & 0\% & 0\% & 0\% & 1 & 1 & 144 & 199 & 1\% & 50\% & 1\% & 0 & 2 & 2 & 341 & 0\% & 0\% & 0\% \\
\rowHighlight assertj-core & 6,261 & 1& 0 & 1 & 5 & 6,255 & 0\% & 0\% & 0\% & 0 & 1 & 6 & 6,254 & 0\% & 0\% & 0\% & 0 & 1 & 0 & 6,260 & 0\% & 0\% & 0\% \\
commons-exec & 55 & 1& 0 & 1 & 1 & 53 & 0\% & 0\% & 0\% & 1 & 0 & 18 & 36 & 5\% & 100\% & 10\% & 0 & 1 & 1 & 53 & 0\% & 0\% & 0\% \\
\rowHighlight handlebars.java & 420 & 1& 0 & 1 & 5 & 414 & 0\% & 0\% & 0\% & 0 & 1 & 91 & 328 & 0\% & 0\% & 0\% & 0 & 1 & 0 & 419 & 0\% & 0\% & 0\% \\
ninja &307 &1 & 0 & 1 & 3 & 303 & 0\% & 0\% & 0\% & 0 & 1 & 50 & 256 & 0\% & 0\% & 0\% & 0 & 1 & 0 & 306 & 0\% & 0\% & 0\% \\

\midrule
\rowHighlight \textbf{Total} & 21,734 & 808 & 599 & 209 & 406 & 20,520 & 60\%& 74\%& 66\%&  583 & 225 & 4,683 & 16,243 & 11\%& 72\%& 19\%& 600 & 208 & 314 & 20,612& 66\%& 74\%& 68\% \\
\midrule
% \textbf{Precision} &&& \multicolumn{7}{c|}{60\%} & \multicolumn{7}{c|}{11\%} & \multicolumn{7}{c}{66\%} \\
% \textbf{Recall} &&& \multicolumn{7}{c|}{74\%} & \multicolumn{7}{c|}{72\%} & \multicolumn{7}{c}{74\%} \\
% \textbf{F1-score} &&& \multicolumn{7}{c|}{66\%} & \multicolumn{7}{c|}{19\%} & \multicolumn{7}{c}{70\%} \\
% \textbf{AUC} (Average per fold) &&& \multicolumn{7}{c|}{86\%} & \multicolumn{7}{c|}{75\%} & \multicolumn{7}{c}{68\%}\\ \bottomrule	
\end{tabular}}
\vspace{-15pt}
\end{table*}





\subsubsection{How effective is \sysName at predicting flaky tests?} 
\label{FlakeFlaggerRQ3}

I used the results from rerunning tests (Section \ref{sec:flakeFlaggerStudy}) as the oracle for \sysName classification process, and ran the feature detector once on each of the same tests in order to gather the data needed to build a model.
I considered several different approaches to process the data, and measure classifier performance with a confusion matrix, precision, recall, F1-score and AUC. Even I applied different classification algorithms and balance techniques, I found that the best prefroamcne was  random forest model built using the SMOTE technique for balancing the training data (and using unbalanced testing data). I compare the result of \sysName classifier with the one of the state-of-the-art flaky test classifier, a \vocabName proposed by Pinto et al.~\cite{pinto2020vocabulary} which extracts tokens from each test using a simple bag-of-words model. I considered a hybrid model that adds the token features to \sysName's features. I consider only projects that have at least 10 flaky tests to ensure I have enough flaky tests for training as shown in Table \ref{table:Full_result}. 

Overall, \sysName and the \vocabName  both detected a very similar number of flaky tests (599 and 583 respectively, out of a total of 808 flaky tests), but the two approaches varied in terms of precision --- \sysName had a far lower false positive rate with just \flaggerfp, compared to \msrfp~false positives from the \vocabName. Considering the initial use-case of a researcher or developer using \sysName to determine which tests to run time-intensive flaky test detectors on, using either \sysName or the \vocabName would result the same number of flaky tests eventually detected (that is, both have comparable recall).
However, if a developer uses both models to detect tests that are most likely to be flaky (which are false positive tests), \sysName reports fewer rate than \vocabName (406 vs 4,683).

\sysName's performance varied across projects: some projects (e.g., alluxio), had perfect precision and recall, while on others (e.g., okhttp and activiti) the approach was less successful. I investigated more about the results per projects and the performance could vary due to many reasons. First, the training and testing dataset sizes vary from one project to another. Because each project has its own environmental assumptions, development patterns, and other unique characteristics, it is really difficult to create a single general-purpose approach for flakiness classifications. Another reason for why performance varies across projects may be that not all flaky tests have been labeled correctly --- no rerun-based technique can guarantee to find all flaky tests (even after 10,000 reruns). The higher number of observed flaky tests in a single project does not guarantee that \sysName performs well.
Some flaky failures are due to rare dependency conflicts and network failures that are not captured well from the features described in Table \ref{table:Feature_desc}.
For example, okhttp has a high number of false positives and false negatives. With a further inspection on this particular project, there is a group of tests had all failed in the same way due to the same dependency problem in one single run.



\begin{table}[t]
\centering
  \setlength{\tabcolsep}{5.0pt}

\caption{\centering{Information gain (IG) for \sysName and the \vocabName.}}

\label{table:tokenbyig}
\vspace{-4pt}
%\resizebox{\t  extwidth}{!}{
\scriptsize
\begin{tabular}{lr|lr}

\toprule
\multicolumn{2}{c|}{\textbf{Vocabulary-Based Features}} & \multicolumn{2}{c}{\textbf{\sysName Features}}  \\
\cmidrule(lr){1-2} \cmidrule(lr){3-4}  


\textbf{Feature/Token} & \textbf{IG} & \textbf{Feature} & \textbf{IG} \\
\midrule

Test Lines of Code & 0.023 & Execution Time & 0.121 \\
\rowHighlight throws & 0.022 & Source Covered Lines & 0.067 \\
should & 0.020 & Source Covered Classes & 0.057 \\
\rowHighlight exception & 0.018 & Covered Lines & 0.034 \\
mtfs & 0.018 & Covered Changes (past 75 commits) & 0.029 \\
\rowHighlight runbuildfortask & 0.017 & Covered Changes (past 50 commits) & 0.028 \\
tfs & 0.017 & Covered Changes (past 100 commits) & 0.028 \\
\rowHighlight run & 0.016 & Covered Changes (past 500 commits) & 0.024 \\
transitive & 0.016 & Test Lines of Code & 0.023 \\
\rowHighlight ioexception & 0.015 & Covered Changes (past 10 commits) & 0.018 \\
tachyon & 0.014 & Covered Changes (past 1000 commits) & 0.015 \\
\rowHighlight fileid & 0.011 & Covered Changes (past 5 commits) & 0.011 \\
if & 0.011 & External Libraries & 0.011 \\
\rowHighlight actual & 0.010 & Covered Changes (past 25 commits) & 0.010 \\
someinfo & 0.010 & Fire and Forget & 0.007 \\
\rowHighlight testutils & 0.010 & Number of Assertions & 0.006 \\
writetype & 0.010 & Resources Optimism & 0.005 \\
\rowHighlight some & 0.009 & Mystery Guest & 0.003 \\
checkspring & 0.009 & Assertion Roulette & 0.002 \\
\rowHighlight testfile & 0.009 & Conditional Logic & 0.002 \\
createbytefile & 0.009 & Indirect Testing & 0.001 \\
\rowHighlight family & 0.009 & Test Run War & 0.001 \\
checkcommonslogging & 0.009 & Eager Testing & 0.000 \\

\bottomrule
\end{tabular}
\vspace{-10pt}
\end{table}


\subsubsection{How helpful is each feature in distinguishing between flaky and non flaky tests?}
\label{FlakeFlaggerRQ4}

I reported the the information gain of each feature in \sysName's model, and the top 23 features in the model built using the \vocabName to get more insight about the effectiveness of these features. As shown in Table \ref{table:tokenbyig}, I noticed that features that considered dynamic behavior from each test (e.g., execution time, covered lines, and coverage of recently changed lines) had a far greater information gain than the tokens that were statically extracted from the test method bodies. I found that the top eight \sysName features each had a higher information gain than the highest gain vocabulary feature. In the model built using the \vocabName \cite{pinto2020vocabulary}, the features with the highest information gain were: test lines of code, presence of the `throws' Java keyword, and several tokens like `should', `exception', and `mtfs', each with an information gain significantly lower than the top features in \sysName's model.

The majority of the flaky tests in the prior study with the `job' token came from a single project, ``oozie,'' which is \emph{not} in my evaluation. At the same time, the majority of non-flaky tests with the token `job' in the dataset were in the project ``elastic-job-lite,'' which was not included in the prior evaluation.
The co-occurrence of individual tokens with flaky tests can vary dramatically between projects. Terms that correlate with flakiness in one project can not be expected to correlate with flakiness in other projects --- this is also evident from the limited number of projects which contain each token. Note that this finding only underscores the need for a large, balanced dataset of flaky tests: the DeFlaker dataset that Pinto et al. used contained \emph{more} flaky tests than \sysName dataset (1,403 vs 810). However, a single project in that dataset (``oozie'') contributed more than half of those flaky tests (856), which can make it extremely difficult to draw conclusions that can generalize beyond a single project, or beyond the dataset.


\subsection{Summary}

The result from rerunning tests emphasizes the importance of finding creative and automated tools to detect flaky tests that do not rely on rerunning them, since rerunning tests can be impractical in the necessary amount of time needed, and still may not observe all flaky tests. The experiment shows how it is hard to identify a fixed number of runs to observe flaky tests. I know that even running tests 10,000 times will \emph{still} not guarantee that all flaky tests have been found, since I did not succeed in reproducing many flaky test failures observed in prior work.


The proposed machine learning classifier, \sysName, shows promising results in the field of flaky test detection. The aim of \sysName is not only to predict flaky tests but also to prioritize tests by considering tests that are most likely to be flaky first for further investigation. Additionally, the advantage of \sysName is its ability to be expanded to include additional features as demonstrated in the \emph{Combined Approach} column in Table \ref{table:classifier_table}. Utilizing machine learning methods like \sysName could help identify flaky tests with fewer resources. Thus, \sysName presents a valuable adding in the area of flaky test detection.
% \begin{table*}[t]
    \caption[Flaky tests detected by re-running test suites 10,000 times]{Flaky tests detected by re-running test suites 10,000 times. \textnormal{I estimate the percentage of all flaky tests that would be detected if only 10, 100 or 1,000 reruns had been performed. Color bars are stacked bar charts showing the percentage of tests that failed with a given frequency. Columns \emph{DeFlaker} and \emph{IDFlakies} show the number of non-order dependent flaky tests found in total by those prior works, and the number of flaky tests shared by both datasets. Blank cells indicate that a different revision of the project was used due to historical compilation issues.}}
\label{table:rerun_result}
\vspace{-5pt}
\setlength{\tabcolsep}{1pt}
\setlength{\textwidth}{0.5pt}
\newcommand{\failureRateWidth}{2.5in}
\newcommand{\failureRateHeight}{1em}
\scriptsize
\centering
    \begin{tabular}{lrr|rr|rr|rrr|c}
    \toprule
    & & \textbf{Flaky by} & \multicolumn{2}{c}{\textbf{DeFlaker} \cite{bell2018deflaker}} & \multicolumn{2}{c}{\textbf{iDFlakies} \cite{lam2019idflakies}} & \multicolumn{3}{c}{\textbf{\% Flaky per run}} & \textbf{Distribution of Failure Frequencies, as \% of Tests Failing} \\
    
    \textbf{Project}&\textbf{Tests}& \textbf{Reruns}&\textbf{Shared} & \textbf{Total} & \textbf{Shared}&\textbf{Total} &\textbf{10}&\textbf{100}&\textbf{1,000}& \textbf{
  \textcolor{c1}{ (0,10]} \textcolor{c2}{ (10, 100]} \textcolor{c3}{ (100, 1,000]} \textcolor{c4}{ (1,000, 10,000]} runs of 10,000} \\
\midrule

spring-boot 	&	2,128	&	163			&	0	&	5	&	  	&	  	&	71\%	&	71\%	&	77\%	&	 \includegraphics[width=\failureRateWidth,height=\failureRateHeight]{Figures/barPlot/spring-boot.pdf} 		\\ 
\rowHighlight  hbase 	&	431	&	145			&	0	&	1	&	  	&	  	&	52\%	&	59\%	&	75\%	&	 \includegraphics[width=\failureRateWidth,height=\failureRateHeight]{Figures/barPlot/hbase.pdf} 		\\ 
  alluxio 	&	187	&	116			&	2	&	2	&	  	&	  	&	0\%	&	91\%	&	100\%	&	 \includegraphics[width=\failureRateWidth,height=\failureRateHeight]{Figures/barPlot/alluxio.pdf} 		\\ 
\rowHighlight  okhttp 	&	810	&	100			&	  	&	  	&	  	&	  	&	8\%	&	12\%	&	15\%	&	 \includegraphics[width=\failureRateWidth,height=\failureRateHeight]{Figures/barPlot/okhttp.pdf} 		\\ 
  ambari 	&	324	&	52			&	1	&	1	&	  	&	  	&	0\%	&	2\%	&	94\%	&	 \includegraphics[width=\failureRateWidth,height=\failureRateHeight]{Figures/barPlot/ambari.pdf} 		\\ 
\rowHighlight  hector 	&	142	&	33			&	1	&	1	&	  	&	  	&	3\%	&	3\%	&	100\%	&	 \includegraphics[width=\failureRateWidth,height=\failureRateHeight]{Figures/barPlot/hector.pdf} 		\\ 
  activiti 	&	2,044	&	32			&	  	&	  	&	  	&	  	&	0\%	&	3\%	&	44\%	&	 \includegraphics[width=\failureRateWidth,height=\failureRateHeight]{Figures/barPlot/activiti.pdf} 		\\ 
\rowHighlight  java-websocket 	&	145	&	23			&	  	&	  	&	22	&	52	&	0\%	&	26\%	&	87\%	&	 \includegraphics[width=\failureRateWidth,height=\failureRateHeight]{Figures/barPlot/java-websocket.pdf} 		\\ 
  wildfly 	&	1,238	&	23			&	  	&	  	&	  	&	  	&	0\%	&	0\%	&	4\%	&	 \includegraphics[width=\failureRateWidth,height=\failureRateHeight]{Figures/barPlot/wildfly.pdf} 		\\ 
\rowHighlight  httpcore 	&	712	&	22			&	1	&	1	&	  	&	  	&	0\%	&	9\%	&	9\%	&	 \includegraphics[width=\failureRateWidth,height=\failureRateHeight]{Figures/barPlot/httpcore.pdf} 		\\ 
  logback 	&	842	&	22			&	  	&	  	&	  	&	  	&	5\%	&	9\%	&	41\%	&	 \includegraphics[width=\failureRateWidth,height=\failureRateHeight]{Figures/barPlot/qos-ch-logback.pdf} 		\\ 
\rowHighlight  incubator-dubbo 	&	2,177	&	19			&	  	&	  	&	5	&	12	&	5\%	&	11\%	&	26\%	&	 \includegraphics[width=\failureRateWidth,height=\failureRateHeight]{Figures/barPlot/dubbo-test-integration.pdf} 		\\ 
  http-request 	&	163	&	18			&	  	&	  	&	  	&	  	&	0\%	&	83\%	&	83\%	&	 \includegraphics[width=\failureRateWidth,height=\failureRateHeight]{Figures/barPlot/http-request.pdf} 		\\ 
\rowHighlight  wro4j 	&	1,145	&	16			&	1	&	1	&	  	&	  	&	44\%	&	50\%	&	81\%	&	 \includegraphics[width=\failureRateWidth,height=\failureRateHeight]{Figures/barPlot/wro4j.pdf} 		\\ 
 % jackrabbit-oak 	&	3,994	&	15			&	1	&	1	&	  	&	  	&		&		&		&	 \includegraphics[width=\failureRateWidth,height=\failureRateHeight]{Figures/barPlot/jackrabbit-oak.pdf} 		\\ 
%  \multicolumn{8}{|l|}{ The next projects are \emph{NOT} part in \emph{ML} phase because each of them has less than 10 flaky tests} \\\hline																								
  orbit 	&	86	&	7			&	0	&	1	&	  	&	  	&	14\%	&	43\%	&	86\%	&	 \includegraphics[width=\failureRateWidth,height=\failureRateHeight]{Figures/barPlot/orbit.pdf} 		\\ 
\rowHighlight  undertow 	&	183	&	7			&	0	&	3	&	  	&	  	&	0\%	&	0\%	&	29\%	&	 \includegraphics[width=\failureRateWidth,height=\failureRateHeight]{Figures/barPlot/undertow.pdf} 		\\ 
  achilles 	&	1,317	&	4			&	  	&	  	&	  	&	  	&	0\%	&	25\%	&	75\%	&	 \includegraphics[width=\failureRateWidth,height=\failureRateHeight]{Figures/barPlot/achilles.pdf} 		\\ 
 \rowHighlight elastic-job-lite 	&	558	&	3			&	  	&	  	&	1	&	6	&	0\%	&	0\%	&	0\%	&	 \includegraphics[width=\failureRateWidth,height=\failureRateHeight]{Figures/barPlot/elastic-job-lite.pdf} 		\\ 
  zxing 	&	345	&	2			&	2	&	2	&	  	&	  	&	0\%	&	100\%	&	100\%	&	 \includegraphics[width=\failureRateWidth,height=\failureRateHeight]{Figures/barPlot/zxing.pdf} 		\\ 
 \rowHighlight assertj-core 	&	6,267	&	1			&	1	&	1	&	  	&	  	&	0\%	&	100\%	&	100\%	&	 \includegraphics[width=\failureRateWidth,height=\failureRateHeight]{Figures/barPlot/assertj-core.pdf} 		\\ 
  commons-exec 	&	55	&	1			&	  	&	  	&	  	&	  	&	0\%	&	0\%	&	100\%	&	 \includegraphics[width=\failureRateWidth,height=\failureRateHeight]{Figures/barPlot/commons-exec.pdf} \\ 		
																								
%  dropwizard 	&	428	&	1			&	1	&	1	&	  	&	  	&		&		&		&	 \includegraphics[width=\failureRateWidth,height=\failureRateHeight]{Figures/barPlot/dropwizard.pdf} 		\\ 
\rowHighlight handlebars.java 	&	428	&	1			&	  	&	  	&	  	&	  	&	0\%	&	100\%	&	100\%	&	 \includegraphics[width=\failureRateWidth,height=\failureRateHeight]{Figures/barPlot/handlebars_java.pdf} 		\\ 
  ninja 	&	306	&	1			&	1	&	1	&	  	&	  	&	0\%	&	100\%	&	100\%	&	 \includegraphics[width=\failureRateWidth,height=\failureRateHeight]{Figures/barPlot/ninja.pdf} 		\\ 
%\rowHighlight  togglz 	&	22	&	1			&	  	&	  	&	  	&	  	&		&		&		&	 \includegraphics[width=\failureRateWidth,height=\failureRateHeight]{Figures/barPlot/togglz.pdf} 		\\ 
 \rowHighlight jimfs 	&	212	&	0			&	  	&	  	&	  	&	  	&		&		&		&	 No flaky tests observed		\\ 
%\rowHighlight  oozie 	&	19	&	0			&	  	&	  	&	  	&	  	&		&		&		&	 N/A 		\\ 
%   hadoop 	&	  	&	  			&	  	&	  	&	  	&	  	&		&		&		&	 \includegraphics[width=\failureRateWidth,height=\failureRateHeight]{Figures/barPlot/hadoop.pdf} 		\\ 
%   incubator-dubbo 	&	  	&	  			&	  	&	  	&	  	&	  	&		&		&		&	 \includegraphics[width=\failureRateWidth,height=\failureRateHeight]{Figures/barPlot/incubator-dubbo.pdf} 		\\ 
%   oryx 	&	  	&	  			&	  	&	  	&	  	&	  	&		&		&		&	 \includegraphics[width=\failureRateWidth,height=\failureRateHeight]{Figures/barPlot/oryx.pdf} 		\\ 
%   tachyon 	&	  	&	  			&	2	&	2	&	  	&	  	&		&		&		&	 \includegraphics[width=\failureRateWidth,height=\failureRateHeight]{Figures/barPlot/tachyon.pdf} 		\\ 
%   \hline																								
   																								
     \midrule																								
\textbf{Total} 	&	22,245	&	811			&	10	&	20	&	28	&	70	&	26\%	&	45\%	&	67\%	&	 \includegraphics[width=\failureRateWidth,height=\failureRateHeight]{Figures/barPlot/total.pdf} 		\\ 
\bottomrule 
\end{tabular}

\vspace{-10pt}
\end{table*}


% \section{FlakeFlagger: Flaky Test Classifier}
\label{sec:flakeFlaggerClassifier}

% \begin{table*}[t]
\scriptsize
%\renewcommand{\arraystretch}{1.3}
    \caption[FlakeFlagger List of Collected Features.]{Complete list of features captured for test flakiness prediction. The Covered Lines Churn feature is represented in multiple forms based on the $h$ values (number of the past commits). In the evaluation, I considered $h=5, 10, 25, 50, 75, 100, 500$ and $10,000$}
    \vspace{-5pt}
    \label{table:Feature_desc}
    \begin{tabularx}{\textwidth}{l | l X}
    \toprule
    & \bfseries Feature & \bfseries Description\\
    \midrule
\parbox[t]{2mm}{\multirow{8}{*}{\rotatebox[origin=c]{90}{Test Smells}}}	&
	Indirect Testing  	&	 True if the test interacts with the object under test via an intermediary \cite{van2001refactoring}  \\
	&	Eager Testing 	&	 True if the test exercises more than one method of the tested object \cite{van2001refactoring} \\
	&	Test Run War  	&	 True if the test allocates a file or resource which might be used by other tests \cite{van2001refactoring} \\
	&	Conditional Logic 	&	 True if the test has a conditional if-statement within the test method body \cite{meszaros2007xunit} \\
	&	Fire and Forget  	&	 True if the test launches background threads or tasks. \cite{garousi2018smells} \\
	&	Mystery Guest 	&	 True if the test accesses external resources  \cite{van2001refactoring} \\
	&	Assertion Roulette  	&	 True if the test has multiple assertions \cite{van2001refactoring} \\
	&	Resources Optimism  	&	 True if the test accesses external resources without checking their availability \cite{van2001refactoring}\\ \hline
\parbox[t]{2mm}{\multirow{8}{*}{\rotatebox[origin=c]{90}{Numeric Features}}}	&	Test Lines of Code   	&	 Number of lines of code in the test method body \\
	&	Number  of Assertions  	&	 Number of assertions checked by the test \\
	&	Execution Time   	&	 Running time for the test execution \\
	&	Source Covered Lines  	&	 Number of lines covered by each test, counting only production code \\
	&	Covered Lines  	&	 Total number of lines of code covered by the test  \\
	&	Source Covered Classes  	&	 Total number of production classes covered by each test \\
	&	External Libraries  	&	 Number of external libraries used by the test \\
	&	Covered Lines Churn 	&	 $h$-index capturing churn of covered lines in past 5, 10, 25, 50, 75, 100, 500, and 10,000 commits. Each value $h$ indicates that at least $h$ lines were modified at least $h$ times in that period.\\
\bottomrule
    \end{tabularx}
    % \vspace{-14pt}
\end{table*}

% \begin{table*}[t]
% \begin{tabular}{l l l}
% & Indirect Test Smell & Testing blah\\
% \end{tabular}
% \end{table*}

\begin{figure*}[t]
%\captionsetup{singlelinecheck = false, justification=justified}
  \includegraphics[width=0.9\textwidth]{Figures/overview.pdf}
  \centering
  \vspace{-5pt}
  \caption{Overview of \sysName's approach to predict likely flaky tests given a set of known flaky tests.}
  \vspace{-15pt}
  \label{over_all_graph}
\end{figure*}
\vspace{-5pt}

This section is a summary of the \emph{approach} section in my paper ``FlakeFlagger: Predicting Flakiness Without Rerunning Tests" published in ICSE2021 \cite{alshammari2021flakeflagger}.

\sysName is a Machine Learning (ML) approach to identify which tests in a test suite are flaky, \emph{without} rerunning them many times. \sysName learns from existing flaky tests in order to predict unseen tests if they are flaky or not. \sysName can be used to search for flaky tests in a large test suite, where developers identify that a portion of the test suite is or is not flaky, and use \sysName to help label the rest of the tests as flaky or not. Also, \sysName could help in terms \emph{prioritize} which tests should be run first by reporting which tests are most likely to be flaky. 


By proactively identifying flaky tests, we may also help developers understand why these tests are flaky.
Prior work has suggested different properties of tests that might make them more likely to be flaky, and \sysName can report which of these features are present in each test~\cite{eck2019understanding,ahmad2021empirical}.
In practice, if a feature has a strong correlation with flakiness, developers might choose to focus on this feature in their future test maintenance and development activities. 





\subsection{Features Collections}
\label{sec:detector}
Machine learning classifiers such as \sysName require a set of feature in order to learn and predict. I started with the prior work \cite{luo2014empirical,eck2019understanding,bell2018deflaker} to study which features is highly linked to flakiness. I aim to collect verity of features because of the fact that some flaky tests in different projects often have different root causes for their flakiness\cite{luo2014empirical}. Similarly, some features that are predictive for one project may not be as predictive for others, due to the inherent non-determinism in flaky tests. 
I intentionally collect some dynamic features, in addition to static features (e.g. presence of textual tokens in the body of each test method). This is important because some causes not in the test method itself, but instead, in the production code that is executed by that test \cite{eck2019understanding}.
Ahmed et al. \cite{ahmad2021empirical} categorized 23 developer-reported factors which affect test flakiness. 
These features are described by practitioners at a high level, and include test case complexity, hard-coded values and test smells.
Eck et al. \cite{eck2019understanding} interviewed 21\Space{ professional} developers about flaky tests and tabulated the frequency of different kinds of flaky tests as well as developers' fixes for those flaky tests. 


Inspired by previous studies on test flakiness, I developed a list of sixteen features, 
some of are based on general studies on the causes of flaky tests \cite{luo2014empirical,ahmad2021empirical}, while others are defined as bad practices in writing unit tests.
Hence, I considered all of the features described in the prior works, and then selected only those for which I could write automated detectors.
This ends up with implementing detectors for each of the features shown in Table \ref{table:Feature_desc}. This list of features is not intended to be complete: there may yet be other features that can be easily collected and will be useful for predicting test flakiness.

While some of the features can be detected by inspecting the test method statically (specifically, the conditional logic smell and test line of code), the rest of the features require more than static analysis.
A hybrid static/dynamic framework \emph{detector} was developed to collect the statement coverage of each test, and then statically analyze the covered code in order to collect these behavioral features.
The \emph{detector} also collect a variety of other features related to the statement coverage of each test, such as how many recently changed lines of code are covered.
The \emph{detector} is implemented as an extension to the Maven build system.



\subsection{Classification Process}

FlakeFlagger takes a list of tests, where each of them is represented as a vector $\{x_1, x_2, x_3,\dots, x_n\}$  where each \emph{x} represents a feature value and \emph{n} correspond the total number of features. I applied data inspection and cleaning process to make that dataset more clear for the classification. Missing data can exist to the dataset due to the fact that some features collected by the detector can be incomplete e.g. due to crashes in the middle of the test execution. Some tests are not written in Java, and hence the feature detectors may not be applicable to them, and due to inheritance, some tests may not have source code in the project under test.

As in any classification problem, considering multiple supervised learning algorithms could be better. In \sysName, it is designed to use a set of models including Random Forest (\emph{RF}) and Decision Tree (\emph{DT}. I follow a feature selection process using \emph{information gain}, which computes the amount of information that a feature can provide for a classification \cite{lei2012feature}. Imbalanced datasets (where there is not an equal number of instances in each class --- flaky and non-flaky tests in our case) usually have very low information gain values. 


\subsection{Experimental Design}
\label{sec:Prediction_Design}
To evaluate \sysName, I used the same dataset described in Section \ref{sec:flakeFlaggerStudy}. I ran the detector described in Section \ref{sec:detector}, to collect the set of features shown in Table \ref{table:Feature_desc}. \sysName, similar to any machine learning classifiers, relies on two data sets: one to build the model (training) and another for testing. Because this is not already designed and I have only one dataset, I applied $k$-fold cross validation  \cite{kohavi1995study,bengio2004no} to evaluate our model. Following this practice, we split the data into $k$ parts, leave one part for testing and $k-1$ to train the classifier.
We then repeat this process with another $k-1$ parts, each time leaving one part for testing.
However, $k$-fold cross validation is most applicable to data that is evenly balanced, where the proportions of each class (flaky and not flaky) are similar. In fact, most tests are not flaky, which means imbalance data. To overcome this, I applied a sample technique SMOTE \cite{chawla2002smote} only on training dataset, to ensure a valid and fair result.




%  the new table 3 

\begin{table*}[t]
 \setlength{\tabcolsep}{1.6pt}

\caption[FlakeFlagger Prediction Result.]{Prediction performance for \sysName, the \vocabName, and the hybrid combination of both.
\textnormal{The hybrid approach builds a model with both \sysName's and the \vocabName's features. I show the number of True Positives, False Negatives, False Positives and True Negatives, Precision, Recall, and F1 scores per-project.
The AUC value is calculated after each fold where the reported value is the overall averages of AUC values after all folds. Projects with zero F1 values have very low numbers of flaky tests (less than 3 per project), and illustrate known limitations of \sysName. } }
%\caption{Prediction performance for \sysName, MSR and the combination between two approaches. 
% \textnormal{For \sysName, showing True Positives, False Negatives, False Positives, True Negatives, Precision, Recall and F1-score. For comparison approaches (FLAST \cite{flast} at $\sigma=0.5$ and $\sigma=0.95$ and random guessing with 50\% probability of flakiness or weighted by the distribution of flaky tests in the project), we show only Precision, Recall and F1-score. Highest F1-score in each row is shown in bold. Since each project uses a different model, we do \emph{not} include a ``total'' row.}
%}
\label{table:Full_result}
\vspace{-5pt}
\resizebox{\textwidth}{!}{
% \scriptsize
\begin{tabular}{l rr| rrrrrrr | rrrrrrr | rrrrrrr}

\toprule
% & \multicolumn{4}{c|}{} & \multicolumn{4}{c|}{\textbf{MSR tool *}} & \multicolumn{4}{c}{\textbf{Both}}\\
% \cmidrule(lr){2-5} \cmidrule(lr){6-10}
& & \textbf{Flaky by} & \multicolumn{7}{c|}{\textbf{\sysName}} & \multicolumn{7}{c|}{\textbf{Vocabulary-Based Approach \cite{pintovocabulary}}} & \multicolumn{7}{c}{\textbf{Combined Approach}}  \\
\cmidrule(lr){4-10} \cmidrule(lr){11-17}  \cmidrule(lr){18-24}
% \cmidrule(lr){2-14} \cmidrule(lr){10-13} \cmidrule(lr){14-17} \cmidrule(lr){18-21} 


\textbf{Project} & \textbf{Tests} & \textbf{Reruns} &\textbf{TP} & \textbf{FN} & \textbf{FP} & \textbf{TN} & \textbf{Pr} & \textbf{R} & \textbf{F} & \textbf{TP} & \textbf{FN} & \textbf{FP} & \textbf{TN} & \textbf{Pr} & \textbf{R} & \textbf{F} &  \textbf{TP} & \textbf{FN} & \textbf{FP} & \textbf{TN} & \textbf{Pr} & \textbf{R} & \textbf{F}\\
\midrule
spring-boot& 2,108 & 160 & 139 & 21 & 15 & 1,933 & 90\% & 87\% & 89\% & 134 & 26 & 703 & 1,245 & 16\% & 84\% & 27\% & 143 & 17 & 18 & 1,930 & 89\% & 89\% & 89\% \\
\rowHighlight hbase &431 & 145& 129 & 16 & 32 & 254 & 80\% & 89\% & 84\% & 89 & 56 & 152 & 134 & 37\% & 61\% & 46\% & 130 & 15 & 33 & 253 & 80\% & 90\% & 84\% \\
alluxio & 187 & 116& 116 & 0 & 0 & 71 & 100\% & 100\% & 100\% & 108 & 8 & 11 & 60 & 91\% & 93\% & 92\% & 116 & 0 & 0 & 71 & 100\% & 100\% & 100\% \\
\rowHighlight okhttp & 810 &100 & 52 & 48 & 159 & 551 & 25\% & 52\% & 33\% & 79 & 21 & 444 & 266 & 15\% & 79\% & 25\% & 46 & 54 & 104 & 606 & 31\% & 46\% & 37\% \\
ambari & 324 & 52& 47 & 5 & 3 & 269 & 94\% & 90\% & 92\% & 36 & 16 & 121 & 151 & 23\% & 69\% & 34\% & 47 & 5 & 3 & 269 & 94\% & 90\% & 92\% \\
\rowHighlight hector & 142 & 33 & 30 & 3 & 8 & 101 & 79\% & 91\% & 85\% & 13 & 20 & 23 & 86 & 36\% & 39\% & 38\% & 25 & 8 & 11 & 98 & 69\% & 76\% & 72\% \\
activiti & 2,043 & 32 & 10 & 22 & 43 & 1,968 & 19\% & 31\% & 24\% & 12 & 20 & 531 & 1,480 & 2\% & 38\% & 4\% & 7 & 25 & 34 & 1,977 & 17\% & 22\% & 19\% \\
\rowHighlight java-websocket & 145 & 23& 19 & 4 & 1 & 121 & 95\% & 83\% & 88\% & 23 & 0 & 74 & 48 & 24\% & 100\% & 38\% & 19 & 4 & 4 & 118 & 83\% & 83\% & 83\% \\
wildfly &1,023 & 23 & 11 & 12 & 27 & 973 & 29\% & 48\% & 36\% & 20 & 3 & 554 & 446 & 3\% & 87\% & 7\% & 17 & 6 & 24 & 976 & 41\% & 74\% & 53\% \\
\rowHighlight httpcore & 712 & 22& 14 & 8 & 23 & 667 & 38\% & 64\% & 47\% & 16 & 6 & 375 & 315 & 4\% & 73\% & 8\% & 15 & 7 & 24 & 666 & 38\% & 68\% & 49\% \\
logback & 805 & 22& 3 & 19 & 17 & 766 & 15\% & 14\% & 14\% & 10 & 12 & 259 & 524 & 4\% & 45\% & 7\% & 5 & 17 & 11 & 772 & 31\% & 23\% & 26\% \\
\rowHighlight incubator-dubbo & 2,174 & 19& 8 & 11 & 35 & 2,120 & 19\% & 42\% & 26\% & 11 & 8 & 813 & 1,342 & 1\% & 58\% & 3\% & 13 & 6 & 23 & 2,132 & 36\% & 68\% & 47\% \\
http-request & 163 & 18& 12 & 6 & 6 & 139 & 67\% & 67\% & 67\% & 16 & 2 & 84 & 61 & 16\% & 89\% & 27\% & 12 & 6 & 6 & 139 & 67\% & 67\% & 67\% \\
\rowHighlight wro4j & 1,135 & 16 & 4 & 12 & 2 & 1,117 & 67\% & 25\% & 36\% & 2 & 14 & 101 & 1,018 & 2\% & 12\% & 3\% & 0 & 16 & 1 & 1,118 & 0\% & 0\% & 0\% \\
orbit & 86 & 7& 1 & 6 & 8 & 71 & 11\% & 14\% & 12\% & 6 & 1 & 32 & 47 & 16\% & 86\% & 27\% & 1 & 6 & 7 & 72 & 12\% & 14\% & 13\% \\
\rowHighlight undertow & 183 & 7& 2 & 5 & 8 & 168 & 20\% & 29\% & 24\% & 6 & 1 & 63 & 113 & 9\% & 86\% & 16\% & 3 & 4 & 8 & 168 & 27\% & 43\% & 33\% \\
achilles &1,317 & 4 & 2 & 2 & 3 & 1,310 & 40\% & 50\% & 44\% & 0 & 4 & 0 & 1,313 & 0\% & 0\% & 0\% & 0 & 4 & 0 & 1,313 & 0\% & 0\% & 0\% \\
\rowHighlight elastic-job-lite & 558 &3 & 0 & 3 & 0 & 555 & 0\% & 0\% & 0\% & 0 & 3 & 34 & 521 & 0\% & 0\% & 0\% & 1 & 2 & 0 & 555 & 100\% & 33\% & 50\% \\
zxing & 345 & 2 & 0 & 2 & 2 & 341 & 0\% & 0\% & 0\% & 1 & 1 & 144 & 199 & 1\% & 50\% & 1\% & 0 & 2 & 2 & 341 & 0\% & 0\% & 0\% \\
\rowHighlight assertj-core & 6,261 & 1& 0 & 1 & 5 & 6,255 & 0\% & 0\% & 0\% & 0 & 1 & 6 & 6,254 & 0\% & 0\% & 0\% & 0 & 1 & 0 & 6,260 & 0\% & 0\% & 0\% \\
commons-exec & 55 & 1& 0 & 1 & 1 & 53 & 0\% & 0\% & 0\% & 1 & 0 & 18 & 36 & 5\% & 100\% & 10\% & 0 & 1 & 1 & 53 & 0\% & 0\% & 0\% \\
\rowHighlight handlebars.java & 420 & 1& 0 & 1 & 5 & 414 & 0\% & 0\% & 0\% & 0 & 1 & 91 & 328 & 0\% & 0\% & 0\% & 0 & 1 & 0 & 419 & 0\% & 0\% & 0\% \\
ninja &307 &1 & 0 & 1 & 3 & 303 & 0\% & 0\% & 0\% & 0 & 1 & 50 & 256 & 0\% & 0\% & 0\% & 0 & 1 & 0 & 306 & 0\% & 0\% & 0\% \\

\midrule
\rowHighlight \textbf{Total} & 21,734 & 808 & 599 & 209 & 406 & 20,520 & 60\%& 74\%& 66\%&  583 & 225 & 4,683 & 16,243 & 11\%& 72\%& 19\%& 600 & 208 & 314 & 20,612& 66\%& 74\%& 68\% \\
\midrule
% \textbf{Precision} &&& \multicolumn{7}{c|}{60\%} & \multicolumn{7}{c|}{11\%} & \multicolumn{7}{c}{66\%} \\
% \textbf{Recall} &&& \multicolumn{7}{c|}{74\%} & \multicolumn{7}{c|}{72\%} & \multicolumn{7}{c}{74\%} \\
% \textbf{F1-score} &&& \multicolumn{7}{c|}{66\%} & \multicolumn{7}{c|}{19\%} & \multicolumn{7}{c}{70\%} \\
% \textbf{AUC} (Average per fold) &&& \multicolumn{7}{c|}{86\%} & \multicolumn{7}{c|}{75\%} & \multicolumn{7}{c}{68\%}\\ \bottomrule	
\end{tabular}}
\vspace{-15pt}
\end{table*}



In our prediction evaluation, I label each prediction result as a True Positive (TP), False Negative (FN), False Positive (FP), or True Negative (TN) as follows:
TP - predicted flaky, known to be flaky; FP - predicted flaky, not known to be flaky; FN - predicted not flaky, known to be flaky; TN - predicted not flaky, not known to be flaky.
I also evaluate our models using F1-score, which is computed using the standard formula based on Recall and Precision. %, to evaluate how our approach works to detect flaky tests as $TP$s.
%  \abdul{I added the following sentences ... } JB: Looks good!
Lastly, I calculate the Area Under the Curve (AUC), a measure of how effective a model is at distinguishing classes (in our case, flaky and not flaky).

In the evaluation, false positives represent the number of tests that might be considered as flaky by developers, resulting in excess effort spent re-running them to determine if they are flaky or not. I focus primarily on total positives, because I have confidence that the collected flaky tests are indeed flaky, but I cannot be confident in our classification of a test as not flaky. 
In other words, the oracle is a result of detecting flaky tests after \numruns~runs for each test, but this does not guarantee that the ``not flaky'' tests are really not flaky: they may just not have been observed to be flaky. 
This approach also allows us to confirm $FN$s are truly flaky tests because they fail at least once during rerun tests.
However, because of the inherent non-determinism in flaky tests, I cannot construct a reliable oracle to evaluate $TN$s and $FP$s, but report them as-is.



\begin{table*}[t]
\scriptsize
%\renewcommand{\arraystretch}{1.3}
    \caption[FlakeFlagger List of Collected Features.]{Complete list of features captured for test flakiness prediction. The Covered Lines Churn feature is represented in multiple forms based on the $h$ values (number of the past commits). In the evaluation, I considered $h=5, 10, 25, 50, 75, 100, 500$ and $10,000$}
    \vspace{-5pt}
    \label{table:Feature_desc}
    \begin{tabularx}{\textwidth}{l | l X}
    \toprule
    & \bfseries Feature & \bfseries Description\\
    \midrule
\parbox[t]{2mm}{\multirow{8}{*}{\rotatebox[origin=c]{90}{Test Smells}}}	&
	Indirect Testing  	&	 True if the test interacts with the object under test via an intermediary \cite{van2001refactoring}  \\
	&	Eager Testing 	&	 True if the test exercises more than one method of the tested object \cite{van2001refactoring} \\
	&	Test Run War  	&	 True if the test allocates a file or resource which might be used by other tests \cite{van2001refactoring} \\
	&	Conditional Logic 	&	 True if the test has a conditional if-statement within the test method body \cite{meszaros2007xunit} \\
	&	Fire and Forget  	&	 True if the test launches background threads or tasks. \cite{garousi2018smells} \\
	&	Mystery Guest 	&	 True if the test accesses external resources  \cite{van2001refactoring} \\
	&	Assertion Roulette  	&	 True if the test has multiple assertions \cite{van2001refactoring} \\
	&	Resources Optimism  	&	 True if the test accesses external resources without checking their availability \cite{van2001refactoring}\\ \hline
\parbox[t]{2mm}{\multirow{8}{*}{\rotatebox[origin=c]{90}{Numeric Features}}}	&	Test Lines of Code   	&	 Number of lines of code in the test method body \\
	&	Number  of Assertions  	&	 Number of assertions checked by the test \\
	&	Execution Time   	&	 Running time for the test execution \\
	&	Source Covered Lines  	&	 Number of lines covered by each test, counting only production code \\
	&	Covered Lines  	&	 Total number of lines of code covered by the test  \\
	&	Source Covered Classes  	&	 Total number of production classes covered by each test \\
	&	External Libraries  	&	 Number of external libraries used by the test \\
	&	Covered Lines Churn 	&	 $h$-index capturing churn of covered lines in past 5, 10, 25, 50, 75, 100, 500, and 10,000 commits. Each value $h$ indicates that at least $h$ lines were modified at least $h$ times in that period.\\
\bottomrule
    \end{tabularx}
    % \vspace{-14pt}
\end{table*}

% \begin{table*}[t]
% \begin{tabular}{l l l}
% & Indirect Test Smell & Testing blah\\
% \end{tabular}
% \end{table*}


\subsection{Evaluation}
I evaluate the FlakeFlagger classifier by answering the following two main research questions:


\begin{description}
  \item[\textbf{RQ1:}] How effective is \sysName at predicting flaky tests?
  \item[\textbf{RQ2:}] How helpful is each feature in distinguishing between flaky and non flaky tests?

 \end{description}

\textbf{RQ1.} I used the results from rerunning tests (Section \ref{sec:flakeFlaggerStudy}) as the oracle for \sysName classification process, and ran the feature detector once on each of the same tests in order to gather the data needed to build a model.
I considered several different approaches to process the data, and measure classifier performance with a confusion matrix, precision, recall, F1-score and AUC. Even I applied different classification algorithms and balance techniques, I found that the best prefroamcne was  random forest model built using the SMOTE technique for balancing the training data (and using unbalanced testing data). I compare the result of \sysName classifier with the one of the state-of-the-art flaky test classifier, a \vocabName proposed by Pinto et al.~\cite{pinto2020vocabulary} which extracts tokens from each test using a simple bag-of-words model. I considered a hybrid model that adds the token features to \sysName's features. I consider only projects that have at least 10 flaky tests to ensure as shown in Table \ref{table:Full_result}. 

Overall, \sysName and the \vocabName  both detected a very similar number of flaky tests (599 and 583 respectively, out of a total of 808 flaky tests), but the two approaches varied in terms of precision --- \sysName had a far lower false positive rate with just \flaggerfp, compared to \msrfp~false positives from the \vocabName. Considering the initial use-case of a researcher or developer using \sysName to determine which tests to run time-intensive flaky test detectors on, using either \sysName or the \vocabName would result the same number of flaky tests eventually detected (that is, both have comparable recall).
However, if a developer uses both models to detect tests that are most likely to be flaky (which are false positive tests), \sysName reports fewer rate than \vocabName (406 vs 4,683).

\sysName's performance varied across projects: some projects (e.g., alluxio), had perfect precision and recall, while on others (e.g., okhttp and activiti) the approach was less successful. I investigated more about the results per projects and the performance could vary due to many reasons. First, the training and testing dataset sizes vary from one project to another. Because each project has its own environmental assumptions, development patterns, and other unique characteristics, it is really difficult to create a single general-purpose approach for flakiness classifications. Another reason for why performance varies across projects may be that not all flaky tests have been labeled correctly --- no rerun-based technique can guarantee to find all flaky tests (even after 10,000 reruns). The higher number of observed flaky tests in a single project does not guarantee that \sysName performs well.
Some flaky failures are due to rare dependency conflicts and network failures that are not captured well from our features described in Table \ref{table:Feature_desc}.
For example, okhttp has a high number of false positives and false negatives. With a further inspection on this particular project, there is a group of tests had all failed in the same way due to the same dependency problem in one single run.



\begin{table}[t]
\centering
  \setlength{\tabcolsep}{5.0pt}

\caption{\centering{Information gain (IG) for \sysName and the \vocabName.}}

\label{table:tokenbyig}
\vspace{-4pt}
%\resizebox{\t  extwidth}{!}{
\scriptsize
\begin{tabular}{lr|lr}

\toprule
\multicolumn{2}{c|}{\textbf{Vocabulary-Based Features}} & \multicolumn{2}{c}{\textbf{\sysName Features}}  \\
\cmidrule(lr){1-2} \cmidrule(lr){3-4}  


\textbf{Feature/Token} & \textbf{IG} & \textbf{Feature} & \textbf{IG} \\
\midrule

Test Lines of Code & 0.023 & Execution Time & 0.121 \\
\rowHighlight throws & 0.022 & Source Covered Lines & 0.067 \\
should & 0.020 & Source Covered Classes & 0.057 \\
\rowHighlight exception & 0.018 & Covered Lines & 0.034 \\
mtfs & 0.018 & Covered Changes (past 75 commits) & 0.029 \\
\rowHighlight runbuildfortask & 0.017 & Covered Changes (past 50 commits) & 0.028 \\
tfs & 0.017 & Covered Changes (past 100 commits) & 0.028 \\
\rowHighlight run & 0.016 & Covered Changes (past 500 commits) & 0.024 \\
transitive & 0.016 & Test Lines of Code & 0.023 \\
\rowHighlight ioexception & 0.015 & Covered Changes (past 10 commits) & 0.018 \\
tachyon & 0.014 & Covered Changes (past 1000 commits) & 0.015 \\
\rowHighlight fileid & 0.011 & Covered Changes (past 5 commits) & 0.011 \\
if & 0.011 & External Libraries & 0.011 \\
\rowHighlight actual & 0.010 & Covered Changes (past 25 commits) & 0.010 \\
someinfo & 0.010 & Fire and Forget & 0.007 \\
\rowHighlight testutils & 0.010 & Number of Assertions & 0.006 \\
writetype & 0.010 & Resources Optimism & 0.005 \\
\rowHighlight some & 0.009 & Mystery Guest & 0.003 \\
checkspring & 0.009 & Assertion Roulette & 0.002 \\
\rowHighlight testfile & 0.009 & Conditional Logic & 0.002 \\
createbytefile & 0.009 & Indirect Testing & 0.001 \\
\rowHighlight family & 0.009 & Test Run War & 0.001 \\
checkcommonslogging & 0.009 & Eager Testing & 0.000 \\

\bottomrule
\end{tabular}
\vspace{-10pt}
\end{table}


\textbf{RQ2}. I reported the the information gain of each feature in \sysName's model, and the top 23 features in the model built using the \vocabName to get more insight about the effectivnes of these features. As shown in Table \ref{table:tokenbyig}, I noticed that features that considered dynamic behavior from each test (e.g., execution time, covered lines, and coverage of recently changed lines) had a far greater information gain than the tokens that were statically extracted from the test method bodies. I found that the top eight \sysName features each had a higher information gain than the highest gain vocabulary feature. In the model built using the \vocabName \cite{pinto2020vocabulary}, the features with the highest information gain were: test lines of code, presence of the `throws' Java keyword, and several tokens like `should', `exception', and `mtfs', each with an information gain significantly lower than the top features in \sysName's model.

The majority of the flaky tests in the prior study with the `job' token came from a single project, ``oozie,'' which is \emph{not} in our evaluation. At the same time, the majority of non-flaky tests with the token `job' in the dataset were in the project ``elastic-job-lite,'' which was not included in the prior evaluation.
The co-occurrence of individual tokens with flaky tests can vary dramatically between projects. Terms that correlate with flakiness in one project can not be expected to correlate with flakiness in other projects --- this is also evident from the limited number of projects which contain each token. Note that this finding only underscores the need for a large, balanced dataset of flaky tests: the DeFlaker dataset that Pinto et al. used contained \emph{more} flaky tests than \sysName dataset (1,403 vs 810). However, a single project in that dataset (``oozie'') contributed more than half of those flaky tests (856), which can make it extremely difficult to draw conclusions that can generalize beyond a single project, or beyond the dataset.


\section{Living with Test Flakiness}
\label{sec:livingTestFlakiness}

Flaky tests may exist in test suites even after being identified, as some may play a role in detecting true (non-flaky) failures. Hence, a significant concern is about distinguishing whether a particular failure is flaky or not. Despite the research on detecting flaky tests, the issue of identifying the flakiness of a failure itself remains unexplored. In light of this, I am currently investigating the feasibility of using failure logs for detecting flaky failures.

% Flaky tests continue to be a part of test suite, regardless of the work in their detection. A major concern is whether a specific failure is flaky, especially if flaky tests continue to exist in the test suite. In response, I am studying the possibility of detecting flaky failures. First, I investigate if a failure can be detected as a flaky based on historical flaky failures, as detailed in Section~\ref{sec:failureLogsStudy}. Second, I am analyzing the failures logs to identify the differences between flaky and non-flaky failures using their failure logs. Using data analysis and machine learning algorithms, these differences could serve as a basis for distinguishing between the two types of failures, so enabling flaky failure detection. This approach is further discussed in Section~\ref{sec:failureLogsApproach}.



\subsection{Matching Failures Logs}
\label{sec:approaches}

When tests fail, they produce output that can be useful for debugging the failure.
Failure logs provide a detailed understanding of the origin of the failure. Hence, developers typically debug logs to better understand the failure cause. 
In detecting test flakiness, a recent survey shows that some developers may manually debug failures logs to tell if a failure is flaky or not~\cite{habchi2022qualitative}. 
Developers can recognize a failure is flaky by examining the \failure as they could have encountered flaky failures with similar \failure~\cite{gradlePreventingFlaky}.

The \failure describe the cause of the failure and hence, it is reasonable to use them to judge the failure de-duplication.
If two failures have the same \failures, it intents to have the same cause.
However, even if a failure log matches an existing flaky failure log, it is likely also important to determine if the failure log also matches a true failure.
If the flaky failure should differ from the true failures, this raises a question: To what extent can a failure log be informative to find the differences between the two type of failures? I study if the proposed de-duplication based approaches could help developers to determine if a new failure is flaky or true failure


This failure output might include a specific failure message (that corresponds to a failed assertion), a stack trace, and/or other console output.
In this article, I focus specifically on the output that is common to the test suites of all projects that I have studied: stack traces.
It is worthwhile to consider the case of matching different failure logs from the same test, and also matching failure logs between different tests.
On the one hand, there might be the greatest confidence in matching failure logs from the same tests.
On the other hand, utilizing data from multiple tests can increase the chances that a matching failure is found.
In this section, I propose two approaches designed to find a failure de-duplication.
The first approach, named \syntax, that use the text of failure logs to find the similarity of given two failures.
The second approach is called the \classifier which adopts machine learning to predict if a failure is flaky or true failure.


% employs a straightforward \emph{Diff} strategy for comparison. This method is particularly suitable when examining failures within the same test. However, when a test lacks prior failure history, I use machine learning. By leveraging existing failure logs of other tests, I can predict how similar a new failure is to previous failures of other tests.

% Failure logs are rich in data. 
% They contain both the failure messages and the associated stacktrace lines, which are crucial for debugging. 
% Given this detailed information, I believe it is possible to identify whether two failures are different, often based on the difference in failure message or stacktrace lines. 
% Therefor, I represent each failure log by its failure message and stacktrace lines. 

\subsubsection{Text-Based Matching}
\label{syntax}
The \syntax is my application of classic failure de-duplication approaches~\cite{Podgurski03Automated,Jiang17WhatCauses}, where I de-duplicate failures by matching common stacktraces. %JB: I should come back and be more precise about how I do this and how it relates to prior work in de-duplication
This approach is also motivated by grey-literature suggestions that, ``sometimes it's obvious to engineers that a test is flaky just by looking at the exception type and message'' \cite{gradlePreventingFlaky}.
Intuitively, if an engineer has repeatedly seen the same flaky failure symptoms, they may be able to guess that a new failure is also flaky.
As a result, developers with different experience, can leverage this approach to compare failures.
This also speeds up the comparison process and enhances the reliability of comparison result. 

I implement text-based matching by creating a dataset of parsed failure logs for each test.
% Parsing the failure logs is an important step to represent a failure in a way where I can find the failure de-duplication.
Each failure log is represented by its \failure.
In terms of a failure message, it consists  \emph{exception type} (for example, AssertionFailedError) and everything follows this is treated as the \emph{exception message}. For the stacktraces part, it is a set of lines representing the calls before the exception occurs and during the parsing, I are considering the top lines pointing directly to the test name. These lines reflect the most recent operations preceding the exception and often provide more details about the root cause of the failure.

I implement a pipeline to parse each failure into an XML file, cataloging all failures linked to a specific test.
As shown in Listing~\ref{lst:flakyFailures}, each failure block in the XML corresponds to one failure, containing four key components: the test name (\textbf{T}), exception type (\textbf{E}), exception message (\textbf{M}), and stacktrace lines (\textbf{S}). Within the \textbf{S} tag, individual lines are listed under the \textbf{line} tags, considering their order in the original log.
If the test name is missing from the stacktrace e.g. fail in setup method, I consider the last line from the test class. For example, in Listing~\ref{lst:flakyFailures}, the last line is not starting with the test name (present in \textbf{T}) but starts with the test class name. To categorize these XML files per project, the \textbf{T} tag includes a \emph{project} attribute, referring the project name where the test belongs.
In this phase, I also filter out non-deterministic stack trace lines internal to the JVM (e.g. \texttt{GeneratedMethodAccessor\$XYZ} lines).



\vspace{5px}
\begin{lstlisting}[style=XMLStyle, caption=Two flaky failures reported in Alluxio project after parsing their failure logs,label=lst:flakyFailures]

<Failure>
    <T project="alluxio">tachyon.JournalTest.TableTest</T>
    <E>UnknownHostException</E>
    <M>ip-172-31-48-81: ip-172-31-48-81: Temporary failure in name resolution</M>
    <S><line>java.net.Inet6AddressImpl.lookupAllHostAddr(Native Method)</line>
    <line>java.net.InetAddress$2.lookupAllHostAddr(InetAddress.java:929)</line>
    <line>java.net.InetAddress.getAddressesFromNameService(InetAddress.java:1324)</line>
    <line>java.net.InetAddress.getLocalHost(InetAddress.java:1501)</line>
    <line>tachyon.LocalTachyonCluster.start(LocalTachyonCluster.java:104)</line>
    <line>tachyon.JournalTest.before(JournalTest.java:33</line></S>
</Failure>
    ...
<Failure>
    <T project="alluxio">tachyon.JournalTest.TableTest</T>
    <E>UnknownHostException</E>
    <M>ip-172-31-58-81: ip-172-31-58-81: Temporary failure in name resolution</M>
    <S><line>java.net.Inet6AddressImpl.lookupAllHostAddr(Native Method)</line>             
    <line>java.net.InetAddress$2.lookupAllHostAddr(InetAddress.java:929)</line>            
    <line>java.net.InetAddress.getAddressesFromNameService(InetAddress.java:1324)</line>            
    <line>java.net.InetAddress.getLocalHost(InetAddress.java:1501)</line>
    <line>tachyon.LocalTachyonCluster.start(LocalTachyonCluster.java:104)</line>
    <line>tachyon.JournalTest.before(JournalTest.java:33)</line></S>
</Failure>
\end{lstlisting}

\vspace{5px}


The \syntax relies on the text of the \failure. Within the discussed example in Listing~\ref{lst:flakyFailures}, I found that the failure message (\textbf{M}) could contain information such as timestamp and IP address that make that failure unique by its test.
For example, in Listing~\ref{lst:flakyFailures}, different details like an IP address within \textbf{M} can set two failures apart.
Hence, the \syntax does not rely on \textbf{M}, and consider only stacktraces (\textbf{S}) and exception type (\textbf{E}).
Given the challenges in capturing all potential cases where the failure message (\textbf{M}) could be identical, I avoid modifying these unique message details and discard the \textbf{M} during the comparison. 

The main use of the \syntax method is find a failure de-duplication. 
When given flaky and true failures, the \syntax should be able to tell if a new failure is a de-duplication of flaky failures, true failures, or both. 
As this approach is design to find failure de-duplication within the same test, it could be useful to applicable across different tests especially if the failure stacktraces does not cover the test body, similar to the example provided in Listing~\ref{lst:flakyFailures}.

% One approach involves comparing a new failure against a history of flaky failures to determine if it matches with any prior occurrences. While \syntax is primarily designed for within-test comparisons because stacktrace lines include test-specific lines, it can also be possible for comparisons across different tests by disregarding these test-related lines. 
% Another application of the \syntax method aligns directly with the core objective I have discussed. Specifically, the \syntax is used at identifying similarities between a given flaky failure and the true failures that occur within the same test, and this hopefully will answer whether a flaky failure logs differ from the true ones. 


\begin{table*}[t]
    \caption{A list of features used to train the \classifier}
\label{table:Features}
\vspace{-5pt}
% \setlength{\tabcolsep}{2.5pt}
\newcommand{\failureRateWidth}{2.5in}
\newcommand{\failureRateHeight}{4em}
\scriptsize
\centering
    \begin{tabular}{l|c|l}
    \toprule     
     \textbf{Feature Name}&\textbf{Type}&\textbf{Description}\\
        \midrule
        Exception Type & Str & The name of the exception e.g. UnknownHostException \\
        Test name in Stacktrace & Boolean & \textit{True} if one of Stacktrace lines starts with the test name else \textit{False} \\
        Test Class name in Stacktrace & Boolean & \textit{True} if one of Stacktrace lines contains the test class name else \textit{False} \\
        Other Tests in Stacktrace & Boolean & \textit{True} if one of Stacktrace lines starts with other tests names else \textit{False} \\
        JUnit in Stacktrace & Boolean & \textit{True} if one of Stacktrace lines starts with any Junit Lines else \textit{False} \\
        CUT in Stacktrace & Boolean & \textit{True} if one of Stacktrace lines contains any lines from Code Under Test else \textit{False} \\
\bottomrule 
\end{tabular}
\vspace{-10pt}
\end{table*}







\subsubsection{Failure Log Classifier}
\label{classifier}


There are cases where a newly written test introduces flakiness, or when there is no prior failures for reference.
Motivated by these scenarios, I have proposed the \classifier, which is trained on both flaky and true failures from \emph{all} tests in a test suite. Then the classifier would be able to predict if the new encounter failure is flaky or true failure.
For training the \classifier, I considered a series of features, which represent each failure log. These features, shown in Table~\ref{table:Features}, are designed for generality. For instance, I ask whether the stacktrace lines cover any line in the test suite rather than the test itself.
I chose the features based on the text of the failure logs. 
However, in order to collect these features, I analyze further as some features require knowledge of all test names in the test suite, test names throughout the entire project, and all source code file names of the code under test to facilitate determining the feature values for each test failure.  
Although other studies for predicting flaky failures use dynamic details~\cite{lampel2021life}, mygoal is to determine if relying on the information in failure logs can effectively predict flaky failures.

% This classifier is trained on a set of both flaky and true failures logs from different flaky tests within the same project. The objective is to enable the classifier to predict the nature of a new failure, specifically determining whether it is flaky or not. As this becomes a classification problem, the true failures serve as a secondary label, assisting the classifier in its predictions of whether a given failure is indeed flaky.

% In order to train the \classifier, I introduce a set of features that represent each failure log as shown in Table~\ref{table:Features}. I designed these features to be more generic. For instance, I ask whether the stacktrace lines cover any line in the test suite rather than the test itself.
% I chose the features based on the contents of the failure logs. Although other studies for predicting flaky failures use dynamic details~\cite{lampel2021life}, mygoal is to determine if relying on the information in failure logs, especially the exception and stacktrace lines, can effectively predict flaky failures.



The \classifier employed a simple \emph{Decision Tree} (\textbf{DT}) as the supervised learning algorithm~\cite{DT}. Based on the binary features used to train the classifier, decision tree provides a clear way to handle non-linear relationships. In addition to the \textbf{DT}, I use the naive bayes as well. 
In terms of the trained dataset, I consider using the cross validation to split the whole dataset (flaky and true failures) to training set and testing set, as there are no provided separate testing dataset. If the dataset ends unbalanced (The number of flaky failures is less than true failures, or vice versa, I consider applying SMOTE~\cite{SMOTE} to balance the trained data and utilized stratified cross-validation~\cite{crossValidation} to ensure that the testing-fold part has at least one flaky failure. 


For evaluating the \classifier, I will use standard classification evaluation metrics, including the confusion matrix, precision, recall, and F1-score. 
These metrics will be calculated on a per-project basis to ensure that the results are not influenced by the diverse nature of different projects, as they do not represent a single domain.

\subsubsection{TF-IDF}
\label{sec:tfidf}
% \jon{There needs to be some discussion of the TF-IDF approach here. Abdul, please add}
% \abdul{Done}

% What is the tf-idf ... 
Term Frequency-Inverse Document Frequency (\tfidf) is a commonly used numerical statistic that reflects how important a word is to a document in corpus~\cite{tfidf}.
Hence, I also consider applying \tfidf for determining whether a failure is flaky or not based on matching other failures. 
\tfidf has two components: Term Frequency (\emph{TF}) which represents the frequency of a term (word) in a document and if a term appears frequently in a document, its \emph{TF} will be high. Second, Inverse Document Frequency (\emph{IDF}) which measures the significance of the term in the entire corpus and if a term appears in many documents, its \emph{IDF} value will be low, reflecting its lower importance. The \tfidf value of a term in a document is the product of its \emph{TF} and \emph{IDF} values. Equation~\ref{TF} and~\ref{IDF} show the computation of \emph{TF} and \emph{IDF}, respectively. 

\begin{equation}
\label{TF}
\text{TF (\textit{t})} = \frac{\text{Number of times term \textit{t} in a document}}{\text{Total number of terms in the document}}
\end{equation}


\begin{equation}
\label{IDF}
\text{IDF (\textit{t})} = \log(\frac{\text{Total number of documents}}{\text{Number of documents where \textit{t} in it}})
\end{equation}


In the context of studying failure logs, I refer \emph{document} to a \emph{failure} and the \textit{t} to the token I extract from each \failure. 
For each failure in the generated XML file used in the \syntax, I tokenize each line of each stacktrace (including the exception type) by split the words using the \emph{dot} as separator (and removing the symbols such parentheses). 
For example, the last line in Listing~\ref{lst:flakyFailures} will be converted to the following set of tokens \textit{(tachyon, JournalTest, before, JournalTest, java, 33)}.
As mygoal was to evaluate the overall potential for this approach, I did not consider more advanced tokenization approaches~\cite{tfidf1}.
% Even some practices of applying \tfidf in order to enhance the \tfidf are widely discussed~\cite{tfidf1}, I consider all tokens as they are provided in the \failures~.

% As the \syntax is fully rely on the text of the logs and the \classifier learn from a set of features that are mainly from the \failures, I chose this approach as the failure log could be represented as a document and the usability of applying the \tfidf. I use \tfidf as an alternative approach and a baseline for the \classifier. 



\subsection{Evaluation Methodology}
\label{sec:matchingEvaluation}
The core contribution in this work is a rigorous empirical evaluation of the three flaky failure detection approaches described in the prior section.  

\subsubsection{Datasets}


% \abdul{This Subsection is fully modified by Jon on August 30,2023}
% In order to effectively evaluate the similarity of flaky and true failure logs, I need a dataset that contains a large number of both flaky and true failures for the same test.
In order to effectively evaluate the failure de-duplication, I need a dataset that contains a large number of both flaky and true failures for the same test.
% For example: the dataset that I used in Section~\ref{sec:study} contains only flaky failures.
The``FlakeFlagger'' dataset was built by executing the test suites of 26 open-source Java projects 10,000 times and recording their outputs, yielding a large dataset of flaky failures~\cite{alshammari2021flakeflagger}.
I choose the FlakeFlagger dataset, as it contains the complete failure logs for each flaky failure, as opposed to other flaky test datasets like DeFlaker's~\cite{bell2018deflaker} or iDFlakies~\cite{lam2019idflakies}.
Whereas a dataset of flaky failures can be mined by repeatedly running the same versions of the same tests, a dataset of true failures can only be mined from buggy code.
While datasets of true failures \emph{do} exist \cite{just2014defects4j,saha2018bugs,tomassi2019bugswarm,bears}, these datasets are typically intentionally constructed from tests that are \emph{not} flaky (to make studying the defects easier).
However, I are not aware of any accessible datasets that provide both flaky and true failures logs for the same set of tests.
Even if one were to mine failures of flaky tests, there would still be a tremendous dataset imbalance problem: there tend to be far more tests that only fail due to flakiness as opposed to those that might also reveal faults~\cite{haben2023importance}.

I propose a novel methodology for constructing a dataset for this experiment, based on mutation testing.
Mutation testing runs a program's test suite on generated mutants (variants of the program under test), and evaluates how many of those mutants are detected by a failing test.
Mutants have been shown to be an effective substitute for real faults in software testing~\cite{just2014mutants}.
Hence, for each of the flaky tests in the dataset, I use mutation testing to build a large dataset of failure logs for true failures.
To avoid contaminating the true failure dataset with flaky failures (caused by tests failing due to flakiness on the mutated code), I apply Shi et al.'s approach for filtering flaky mutants~\cite{shi2019mitigating}.

Hence, the dataset for the experiment consists of all of the flaky failures extracted from the FlakeFlagger dataset~\cite{alshammari2021flakeflagger}, supplemented by true failures generated by executing Shi et al.'s version of the popular PIT mutation testing tool~\cite{shi2019mitigating,coles2016pit}.
This modified version of PIT is configured such that each test-mutant failure is confirmed by re-running the test on that mutant, 20 times.
Each failure that is deterministically reproduced is included in the dataset of failures.
This confirmation step is necessary to filter out any flaky failures from the mutation dataset, and is used only for confirming that the failure is deterministic (I do not include each failure 20 times from each of the confirmation runs).
% \jon{I think that I need to have some discussion here of the tests that I couldn't collect mutation results for...}
Then from the collected failure logs of each killed mutant, I collect the \failures. I extend the XML file per test to include a list of killed mutants, each of them contains the \failure. 

In practice, flaky failures tend to be far more common than true failures.
Given that the \failure includes the name of each test, the performance of any failure classifier could be misrepresented by a dataset that contained a large proportion of tests that \emph{only} failed due to flakiness.
For example, in a 9-month period observing Google's Chromium CI, Haben et al. observed that 1,446 tests failed with only true (``fault-revealing'') failures, 22,477 failed with only flaky failures, and 897 failed showing both failures.
A predictor based on the historical flaky failure rate of a test would easily have quite high recall at predicting flakiness (e.g. having at most $897/22,477=4\%$ true failures incorrectly labeled as flaky). 
My goal is to evaluate the performance of approaches that rely primarily on the \failure, and \emph{not} just the historical flake rate of a test.

Hence, I include in the evaluation \emph{only} tests with at least one flaky and non-flaky failure, and report the number of true and flaky failures in the dataset for each project.
I were not able to successfully apply the PIT mutation testing tool to all of the projects despite significant efforts (one author expended at least 2 hours per-project to attempt to get it to work) --- and hence, I were unable to gather a resource of failures for all projects.
As a result, it is important to note that I do \emph{not} include all projects or tests from the FlakeFlagger dataset.
Whereas the FlakeFlagger dataset includes 811 flaky tests from 24 projects, I analyze only those tests for which I could collect a dataset of true failures: 543 flaky tests from 22 projects.

% \input{sections/Tables/Classifier_table}



\subsubsection{Research Questions}

Using this dataset of 543 tests with both flaky and true failures, I design an experiment to answer the following research questions:

\begin{description}
    \item \textbf{RQ1: How often are flaky failures repetitive?} 
    I discuss how frequently a flaky failure matches \emph{at least} one other flaky failure, examining other flaky failures of the same test or other tests within the same project. By doing this, I show the repetition of flaky failures and the efficacy of the failure de-duplication approach. %\jon{Does this still match what the tables are?}\abdul{With adding a new table, I think yes}
    % I aim to validate the approach of using prior knowledge to compare new failures to previously observed flaky failures. A crucial part of this study is to examine flaky failures that have occurred only once and to understand the characteristics of such failures. Generally, I want to validate that developers can confidently rely on past knowledge when addressing flaky failures.
    
    \item \textbf{RQ2: With prior flaky and true failures, is it feasible to use the failure de-duplication to tell if a failure is flaky or true one?}
    The main objective is to evaluate the effectiveness of using \syntax as an approach to find the differences between flaky and true failures. This helps practitioners and researchers if they can rely on the approach in detecting flaky failures. Since projects differ in their domain, root causes of flakiness, and the total number of flaky tests, I evaluate the approach on a project-by-project basis.
    
    \item\textbf{RQ3: How far utilizing machine learning being helpful in finding the differences between flaky and true failures?} I aim to demonstrate the efficacy of employing machine learning classifiers in predicting whether a failure is flaky or not based on specific features extracted from failure logs. I are looking if a classifier can leverage failures from other tests within the same project to enhance the learning process of the model to better predict failures, especially from the newly written tests. 
    % The \classifier can be used to prioritize the investigation to check if a failure is flaky by alternative approaches by considering first the false positives failures (failures tend to be true but the classifier label them as flaky)
    
\end{description}






\subsection{Result}
\label{matchingResult}
\subsubsection{RQ1: How often are flaky failures repetitive?}
\label{matchingRQ1}










\begin{table}[t]
  \setlength{\tabcolsep}{2.5pt}

% \jon{Resized table to try to fit bigger column headers. Not clear what unique means. Unique by stack trace?}
\caption[Repetitive Flaky Failures within and across tests per project]{Repetitive Flaky Failures within and across tests per project. \\
\textnormal{Failures column shows the number of flaky failures and the different failures (Set). The columns (1:n) and [1] refer to flaky failures that are and are not repetitives, respectively.  Per Test refers to matching the failures within the same test. Across Tests refers matching all flaky failures from all tests.}
}
\label{table:repetitive}
\vspace{-4pt}
%\resizebox{\textwidth}{!}{
% \scriptsize
\footnotesize
\centering
\begin{tabular}{l|r|rr|rr|rr}

\toprule
      & &\multicolumn{2}{c|}{\textbf{Failures}} & \multicolumn{2}{c|}{\textbf{Per Test}} & \multicolumn{2}{c}{\textbf{Across Tests}}\\
      

\textbf{Projects} & \textbf{Tests} & \textbf{Flaky}  & \textbf{Set}  & \textbf{[1]} & \textbf{(1:n)}  & \textbf{[1]} & \textbf{(1:n)} \\
\midrule

Alluxio-alluxio&114&16,858&310&11&16,847&5&16,853\\
\cellcolor{gray!6}{square-okhttp}&\cellcolor{gray!6}{100}&\cellcolor{gray!6}{28,264}&\cellcolor{gray!6}{121}&\cellcolor{gray!6}{40}&\cellcolor{gray!6}{28,224}&\cellcolor{gray!6}{17}&\cellcolor{gray!6}{28,247}\\
apache-hbase&62&19,822&100&14&19,808&5&19,817\\
\cellcolor{gray!6}{apache-ambari}&\cellcolor{gray!6}{51}&\cellcolor{gray!6}{4,063}&\cellcolor{gray!6}{54}&\cellcolor{gray!6}{0}&\cellcolor{gray!6}{4,063}&\cellcolor{gray!6}{0}&\cellcolor{gray!6}{4,063}\\
hector-client-hector&33&6,529&33&0&6,529&0&6,529\\
\cellcolor{gray!6}{activiti-activiti}&\cellcolor{gray!6}{31}&\cellcolor{gray!6}{1,378}&\cellcolor{gray!6}{32}&\cellcolor{gray!6}{13}&\cellcolor{gray!6}{1,365}&\cellcolor{gray!6}{6}&\cellcolor{gray!6}{1,372}\\
tootallnate-java-websocket&22&2,095&43&2&2,093&0&2,095\\
\cellcolor{gray!6}{apache-httpcore}&\cellcolor{gray!6}{22}&\cellcolor{gray!6}{354}&\cellcolor{gray!6}{22}&\cellcolor{gray!6}{9}&\cellcolor{gray!6}{345}&\cellcolor{gray!6}{2}&\cellcolor{gray!6}{352}\\
qos-ch-logback&20&438&21&8&430&4&434\\
\cellcolor{gray!6}{kevinsawicki-http-request}&\cellcolor{gray!6}{18}&\cellcolor{gray!6}{3,501}&\cellcolor{gray!6}{18}&\cellcolor{gray!6}{3}&\cellcolor{gray!6}{3,498}&\cellcolor{gray!6}{0}&\cellcolor{gray!6}{3,501}\\
wildfly-wildfly&18&50&18&12&38&4&46\\
\cellcolor{gray!6}{wro4j-wro4j}&\cellcolor{gray!6}{14}&\cellcolor{gray!6}{10,833}&\cellcolor{gray!6}{21}&\cellcolor{gray!6}{3}&\cellcolor{gray!6}{10,830}&\cellcolor{gray!6}{2}&\cellcolor{gray!6}{10,831}\\
spring-projects-spring-boot&12&14&13&12&2&5&9\\
\cellcolor{gray!6}{orbit-orbit}&\cellcolor{gray!6}{7}&\cellcolor{gray!6}{2,943}&\cellcolor{gray!6}{7}&\cellcolor{gray!6}{0}&\cellcolor{gray!6}{2,943}&\cellcolor{gray!6}{0}&\cellcolor{gray!6}{2,943}\\
undertow-io-undertow&7&92&12&3&89&1&91\\
\cellcolor{gray!6}{doanduyhai-Achilles}&\cellcolor{gray!6}{4}&\cellcolor{gray!6}{165}&\cellcolor{gray!6}{5}&\cellcolor{gray!6}{1}&\cellcolor{gray!6}{164}&\cellcolor{gray!6}{1}&\cellcolor{gray!6}{164}\\
elasticjob-elastic-job-lite&3&7&4&3&4&0&7\\
\cellcolor{gray!6}{assertj-core}&\cellcolor{gray!6}{1}&\cellcolor{gray!6}{974}&\cellcolor{gray!6}{1}&\cellcolor{gray!6}{0}&\cellcolor{gray!6}{974}&\cellcolor{gray!6}{0}&\cellcolor{gray!6}{974}\\
ninja-ninja&1&476&1&0&476&0&476\\
\cellcolor{gray!6}{handlebars.java}&\cellcolor{gray!6}{1}&\cellcolor{gray!6}{411}&\cellcolor{gray!6}{1}&\cellcolor{gray!6}{0}&\cellcolor{gray!6}{411}&\cellcolor{gray!6}{0}&\cellcolor{gray!6}{411}\\
apache-commons-exec&1&33&1&0&33&0&33\\
\cellcolor{gray!6}{zxing-zxing}&\cellcolor{gray!6}{1}&\cellcolor{gray!6}{322}&\cellcolor{gray!6}{1}&\cellcolor{gray!6}{0}&\cellcolor{gray!6}{322}&\cellcolor{gray!6}{0}&\cellcolor{gray!6}{322}\\
\midrule
Total&543&99,622&839&134&99,488&52&99,570\\



% --> Consider the test line in stacktraces 

% Alluxio-alluxio&114&16,858&310&11&16,847&11&16,847\\
% \cellcolor{gray!6}{square-okhttp}&\cellcolor{gray!6}{100}&\cellcolor{gray!6}{28,264}&\cellcolor{gray!6}{121}&\cellcolor{gray!6}{40}&\cellcolor{gray!6}{28,224}&\cellcolor{gray!6}{32}&\cellcolor{gray!6}{28,232}\\
% apache-hbase&62&19,822&100&14&19,808&12&19,810\\
% \cellcolor{gray!6}{apache-ambari}&\cellcolor{gray!6}{51}&\cellcolor{gray!6}{4,063}&\cellcolor{gray!6}{54}&\cellcolor{gray!6}{0}&\cellcolor{gray!6}{4,063}&\cellcolor{gray!6}{0}&\cellcolor{gray!6}{4,063}\\
% hector-client-hector&33&6,529&33&0&6,529&0&6,529\\
% \cellcolor{gray!6}{activiti-activiti}&\cellcolor{gray!6}{31}&\cellcolor{gray!6}{1,378}&\cellcolor{gray!6}{32}&\cellcolor{gray!6}{13}&\cellcolor{gray!6}{1,365}&\cellcolor{gray!6}{13}&\cellcolor{gray!6}{1,365}\\
% tootallnate-java-websocket&22&2,095&43&2&2,093&1&2,094\\
% \cellcolor{gray!6}{apache-httpcore}&\cellcolor{gray!6}{22}&\cellcolor{gray!6}{354}&\cellcolor{gray!6}{22}&\cellcolor{gray!6}{9}&\cellcolor{gray!6}{345}&\cellcolor{gray!6}{9}&\cellcolor{gray!6}{345}\\
% qos-ch-logback&20&438&21&8&430&8&430\\
% \cellcolor{gray!6}{kevinsawicki-http-request}&\cellcolor{gray!6}{18}&\cellcolor{gray!6}{3,501}&\cellcolor{gray!6}{18}&\cellcolor{gray!6}{3}&\cellcolor{gray!6}{3,498}&\cellcolor{gray!6}{3}&\cellcolor{gray!6}{3,498}\\
% wildfly-wildfly&18&50&18&12&38&12&38\\
% \cellcolor{gray!6}{wro4j-wro4j}&\cellcolor{gray!6}{14}&\cellcolor{gray!6}{10,833}&\cellcolor{gray!6}{21}&\cellcolor{gray!6}{3}&\cellcolor{gray!6}{10,830}&\cellcolor{gray!6}{3}&\cellcolor{gray!6}{10,830}\\
% spring-projects-spring-boot&12&14&13&12&2&12&2\\
% \cellcolor{gray!6}{orbit-orbit}&\cellcolor{gray!6}{7}&\cellcolor{gray!6}{2,943}&\cellcolor{gray!6}{7}&\cellcolor{gray!6}{0}&\cellcolor{gray!6}{2,943}&\cellcolor{gray!6}{0}&\cellcolor{gray!6}{2,943}\\
% undertow-io-undertow&7&92&12&3&89&3&89\\
% \cellcolor{gray!6}{doanduyhai-Achilles}&\cellcolor{gray!6}{4}&\cellcolor{gray!6}{165}&\cellcolor{gray!6}{5}&\cellcolor{gray!6}{1}&\cellcolor{gray!6}{164}&\cellcolor{gray!6}{1}&\cellcolor{gray!6}{164}\\
% elasticjob-elastic-job-lite&3&7&4&3&4&3&4\\
% \cellcolor{gray!6}{assertj-core}&\cellcolor{gray!6}{1}&\cellcolor{gray!6}{974}&\cellcolor{gray!6}{1}&\cellcolor{gray!6}{0}&\cellcolor{gray!6}{974}&\cellcolor{gray!6}{0}&\cellcolor{gray!6}{974}\\
% ninja-ninja&1&476&1&0&476&0&476\\
% \cellcolor{gray!6}{handlebars.java}&\cellcolor{gray!6}{1}&\cellcolor{gray!6}{411}&\cellcolor{gray!6}{1}&\cellcolor{gray!6}{0}&\cellcolor{gray!6}{411}&\cellcolor{gray!6}{0}&\cellcolor{gray!6}{411}\\
% apache-commons-exec&1&33&1&0&33&0&33\\
% \cellcolor{gray!6}{zxing-zxing}&\cellcolor{gray!6}{1}&\cellcolor{gray!6}{322}&\cellcolor{gray!6}{1}&\cellcolor{gray!6}{0}&\cellcolor{gray!6}{322}&\cellcolor{gray!6}{0}&\cellcolor{gray!6}{322}\\
% \midrule
% Total&543&99,622&839&134&99,488&123&99,499\\


\bottomrule
\end{tabular}
\vspace{-10pt}
\end{table}



To answer this question, I being with the XML files that summarise all of the failures of each test (described in Sction~\ref{sec:approaches}).
As the failures in each file correspond to either flaky or true failure, I count the number of flaky failures per test. I compute how many different flaky failures by their \failures using the \syntax approach as well as computing how many of flaky failure is repetitive (by the failure de-duplication with flaky failures within the same test or across all tests in the same project) and how many is not. 

Table~\ref{table:repetitive} summarizes the findings by considering the two cases: matching the flaky failures within the same test (shown in column \emph{Per Test}), and matching the flaky failures across all failures from all tests in the same project (shown in the column \emph{Across Tests}). 
% of the repetitiveness of flaky failures using the failure de-duplication approach on all flaky failures. 
By considering \alluxio as an example from Table~\ref{table:repetitive}: in the first case, there are 114 flaky tests and those tests cumulatively have 16,858 flaky failures in total (16,847 of them are repetitive and 11 are not). 
The 16,847 failures that were an exact match for at least one other failure represent just 310 unique failures.
% as shown Table~\ref{table:repetitive}.
In the second case, the number of failures that are \emph{not} repetitive dropped to 5 (16,853 repetitive failures).  
When comparing a new failure to flaky failures from different tests, the \syntax might produce mis-match results due to lines in the stacktrace pointing to the test. To mitigate this, I exclude such lines during this type of comparison, ensuring a more accurate match result.

While I found that each flaky test in \alluxio could have different flaky failures, on average, each flaky test only had just over two different failures, each of which recurred many times.
In most of the projects I studied in Table~\ref{table:repetitive}, there are a reasonable amount of repetitive flaky failures (by both considering the ratio of the number of flaky failures in column (\textbf{[1]}) to the total number of flaky failures or even to the set of flaky failures) as some projects (6 out of 22) have all flaky failures are repetitive.
Hence, I conclude that, overall, flaky failures are extremely repetitive.
While it is inappropriate to assume that each flaky test can only fail with a single set of symptoms, the number of unique failures is dwarfed by the frequency with which those failures recur.

I also carefully examine when flaky failures are not repetitive, and occur only once in the dataset.
Across all the studied projects, there are only 134 out of 99,622 flaky failures (also out of 839 sets of flaky failures) that have never matched other flaky failures within the same project.
Out of 134 that failed once, I found 95 of them are actually lack of the history of flaky failures (from tests that only failed once). Out of 22 projects, there are only two projects where the number of repetitive flaky failure is just equal or less than the number of non-repetitive flaky failures (\elastic and \spring), and all these failures are from tests that only fails once. 






\begin{table*}[t]
\caption[The Text-Based Matching between flaky and true failures]{The Text-Based Matching between flaky and true failures\\
\textnormal{The \textit{Total Tests and Failures} column provides the total flaky tests, the number of true failures across these tests, and the count of flaky failures.  
% The \textit{Flaky Failures} column shows the count of flaky failures that are \emph{not} repetitive (labeled as [1]) and the number of repetitive flaky failures (1:n).
The \textit{Set of Failures} column displays the different failures within both flaky and true failures.
% The \textit{Flaky VS True Using the Failure De-duplication Approach} column presents the matching results between flaky and true failures by showing the confusion matrix per failures and tests as well as the evaluation metrics. The cumulative number of tests in \emph{By Tests} column might exceed the total given in \textbf{Test} because a test might have multiple flaky failures in different categories.
The \textit{Confusion Matrix and Evaluation By Failures} column presents the matching results between flaky and true failures by showing the confusion matrix per failures and the evaluation metrics. The \# of Tests in TP and FN shows how many different tests in each one. The cumulative number of tests in \emph{TP} and \emph{FN} might exceed the total given in \emph{Test} because a test might have multiple flaky failures in different categories.
}
}



\vspace{-5pt}
\setlength{\tabcolsep}{4.0pt}
\newcommand{\failureRateWidth}{2.5in}
\newcommand{\failureRateHeight}{4em}
\scriptsize
\centering


    \begin{tabular}{l|rrr|rr|rrrrrrr|rr}
    \toprule
      % & \multicolumn{3}{c}{\textbf{}} & \multicolumn{2}{c}{\textbf{}} & \multicolumn{9}{c}{\textbf{Flaky VS True Using the Failure De-duplication Approach}} \\ 
      
      & \multicolumn{3}{c|}{\textbf{Total Tests and Failures}} & \multicolumn{2}{c|}{\textbf{Set of Failures}} & \multicolumn{7}{c}{\textbf{Confusion Matrix and Evaluation By Failures}} &  \multicolumn{2}{c}{\textbf{\# of Tests in}} \\
    \midrule
     \textbf{Project}&\textbf{Test}&\textbf{True}&\textbf{Flaky} &\textbf{True}&\textbf{Flaky} &\textbf{TP}&\textbf{FN}&\textbf{FP}&\textbf{TN}&\textbf{P} &\textbf{R} &\textbf{SP} & \textbf{TP}&\textbf{FN}\\
\midrule
Alluxio&114&32,608&16,858&6,491&310&9,615&7,243&1,694&30,914&85\%&57\%&94\%&114&102\\
\cellcolor{gray!6}{Okhttp}&\cellcolor{gray!6}{100}&\cellcolor{gray!6}{34,266}&\cellcolor{gray!6}{28,264}&\cellcolor{gray!6}{18,609}&\cellcolor{gray!6}{121}&\cellcolor{gray!6}{16,517}&\cellcolor{gray!6}{11,747}&\cellcolor{gray!6}{114}&\cellcolor{gray!6}{34,152}&\cellcolor{gray!6}{99\%}&\cellcolor{gray!6}{58\%}&\cellcolor{gray!6}{99\%}&\cellcolor{gray!6}{58}&\cellcolor{gray!6}{53}\\
Hbase&62&11,324&19,822&811&100&18,496&1,326&1,198&10,126&93\%&93\%&89\%&58&14\\
\cellcolor{gray!6}{Ambari}&\cellcolor{gray!6}{51}&\cellcolor{gray!6}{11,049}&\cellcolor{gray!6}{4,063}&\cellcolor{gray!6}{4,563}&\cellcolor{gray!6}{54}&\cellcolor{gray!6}{4,003}&\cellcolor{gray!6}{60}&\cellcolor{gray!6}{5}&\cellcolor{gray!6}{11,044}&\cellcolor{gray!6}{99\%}&\cellcolor{gray!6}{98\%}&\cellcolor{gray!6}{99\%}&\cellcolor{gray!6}{50}&\cellcolor{gray!6}{2}\\
Hector&33&3,604&6,529&1,769&33&1,382&5,147&12&3,592&99\%&21\%&99\%&32&1\\
\cellcolor{gray!6}{Activiti}&\cellcolor{gray!6}{31}&\cellcolor{gray!6}{46,100}&\cellcolor{gray!6}{1,378}&\cellcolor{gray!6}{16,018}&\cellcolor{gray!6}{32}&\cellcolor{gray!6}{932}&\cellcolor{gray!6}{446}&\cellcolor{gray!6}{2,609}&\cellcolor{gray!6}{43,491}&\cellcolor{gray!6}{26\%}&\cellcolor{gray!6}{67\%}&\cellcolor{gray!6}{94\%}&\cellcolor{gray!6}{1}&\cellcolor{gray!6}{30}\\
Java-websocket&22&1,299&2,095&330&43&591&1,504&816&483&42\%&28\%&37\%&19&22\\
\cellcolor{gray!6}{Httpcore}&\cellcolor{gray!6}{22}&\cellcolor{gray!6}{8,333}&\cellcolor{gray!6}{354}&\cellcolor{gray!6}{663}&\cellcolor{gray!6}{22}&\cellcolor{gray!6}{0}&\cellcolor{gray!6}{354}&\cellcolor{gray!6}{2,117}&\cellcolor{gray!6}{6,216}&\cellcolor{gray!6}{0\%}&\cellcolor{gray!6}{0\%}&\cellcolor{gray!6}{74\%}&\cellcolor{gray!6}{0}&\cellcolor{gray!6}{22}\\
Logback&20&2,614&438&903&21&56&382&368&2,246&13\%&12\%&85\%&3&17\\
\cellcolor{gray!6}{Wildfly}&\cellcolor{gray!6}{18}&\cellcolor{gray!6}{4,364}&\cellcolor{gray!6}{50}&\cellcolor{gray!6}{1,497}&\cellcolor{gray!6}{18}&\cellcolor{gray!6}{38}&\cellcolor{gray!6}{12}&\cellcolor{gray!6}{0}&\cellcolor{gray!6}{4,364}&\cellcolor{gray!6}{100\%}&\cellcolor{gray!6}{76\%}&\cellcolor{gray!6}{100\%}&\cellcolor{gray!6}{6}&\cellcolor{gray!6}{12}\\
Http-request&18&387&3,501&229&18&981&2,520&40&347&96\%&28\%&89\%&4&14\\
\cellcolor{gray!6}{Wro4j}&\cellcolor{gray!6}{14}&\cellcolor{gray!6}{540}&\cellcolor{gray!6}{10,833}&\cellcolor{gray!6}{90}&\cellcolor{gray!6}{21}&\cellcolor{gray!6}{800}&\cellcolor{gray!6}{10,033}&\cellcolor{gray!6}{29}&\cellcolor{gray!6}{511}&\cellcolor{gray!6}{96\%}&\cellcolor{gray!6}{7\%}&\cellcolor{gray!6}{94\%}&\cellcolor{gray!6}{9}&\cellcolor{gray!6}{11}\\
Spring-boot&12&2,150&14&244&13&2&12&0&2,150&100\%&14\%&100\%&1&12\\
\cellcolor{gray!6}{Undertow}&\cellcolor{gray!6}{7}&\cellcolor{gray!6}{2,304}&\cellcolor{gray!6}{92}&\cellcolor{gray!6}{236}&\cellcolor{gray!6}{12}&\cellcolor{gray!6}{8}&\cellcolor{gray!6}{84}&\cellcolor{gray!6}{940}&\cellcolor{gray!6}{1,364}&\cellcolor{gray!6}{0\%}&\cellcolor{gray!6}{8\%}&\cellcolor{gray!6}{59\%}&\cellcolor{gray!6}{2}&\cellcolor{gray!6}{6}\\
Orbit&7&822&2,943&302&7&87&2,856&57&765&60\%&2\%&93\%&2&5\\
\cellcolor{gray!6}{Achilles}&\cellcolor{gray!6}{4}&\cellcolor{gray!6}{442}&\cellcolor{gray!6}{165}&\cellcolor{gray!6}{245}&\cellcolor{gray!6}{5}&\cellcolor{gray!6}{120}&\cellcolor{gray!6}{45}&\cellcolor{gray!6}{46}&\cellcolor{gray!6}{396}&\cellcolor{gray!6}{72\%}&\cellcolor{gray!6}{72\%}&\cellcolor{gray!6}{89\%}&\cellcolor{gray!6}{1}&\cellcolor{gray!6}{3}\\
Elastic-job-lite&3&111&7&68&4&4&3&0&111&100\%&57\%&100\%&1&3\\
\cellcolor{gray!6}{Cmmons-exec}&\cellcolor{gray!6}{1}&\cellcolor{gray!6}{59}&\cellcolor{gray!6}{33}&\cellcolor{gray!6}{13}&\cellcolor{gray!6}{1}&\cellcolor{gray!6}{0}&\cellcolor{gray!6}{33}&\cellcolor{gray!6}{2}&\cellcolor{gray!6}{57}&\cellcolor{gray!6}{0\%}&\cellcolor{gray!6}{0\%}&\cellcolor{gray!6}{96\%}&\cellcolor{gray!6}{0}&\cellcolor{gray!6}{1}\\
assertj-core&1&17&974&9&1&974&0&0&17&100\%&100\%&100\%&1&0\\
\cellcolor{gray!6}{Handlebars.java}&\cellcolor{gray!6}{1}&\cellcolor{gray!6}{147}&\cellcolor{gray!6}{411}&\cellcolor{gray!6}{61}&\cellcolor{gray!6}{1}&\cellcolor{gray!6}{0}&\cellcolor{gray!6}{411}&\cellcolor{gray!6}{16}&\cellcolor{gray!6}{131}&\cellcolor{gray!6}{0\%}&\cellcolor{gray!6}{0\%}&\cellcolor{gray!6}{89\%}&\cellcolor{gray!6}{0}&\cellcolor{gray!6}{1}\\
Zxing&1&76&322&37&1&322&0&0&76&100\%&100\%&100\%&1&0\\
\cellcolor{gray!6}{Ninja}&\cellcolor{gray!6}{1}&\cellcolor{gray!6}{209}&\cellcolor{gray!6}{476}&\cellcolor{gray!6}{6}&\cellcolor{gray!6}{1}&\cellcolor{gray!6}{0}&\cellcolor{gray!6}{476}&\cellcolor{gray!6}{90}&\cellcolor{gray!6}{119}&\cellcolor{gray!6}{0\%}&\cellcolor{gray!6}{0\%}&\cellcolor{gray!6}{56\%}&\cellcolor{gray!6}{0}&\cellcolor{gray!6}{1}\\
\midrule
22 Projects Total &543&162,825&99,622&53,194&839&54,928&44,694&10,153&152,672&&&&363&332\\



\bottomrule
\end{tabular}

\label{nonunique}
\vspace{-10pt}
\end{table*}



While it is common for frequently failing flaky tests to exhibit repetitive flaky failures, this trend is not consistent across all projects. For example, within the project \hbase, there are 6 flaky tests that failed more than 100 times have at least one non-repetitive flaky failure. 

% this details are not in any table (it is just a deep analysis)
I investigated whether specific exception types were associated with these non-repetitive flaky failures. From the dataset I analyzed, among the top 10 most frequently occurring exceptions, two exceptions appeared more frequently in non-repetitive failures than repetitive flaky failures. Specifically, the \emph{RuntimeException} was observed 14 times out of its 23 non-repetitive cases, while the \emph{SocketException} was also observed 19 times out of a total of 31 non-repetitive cases. I found that every failures with the \emph{SocketException} was linked within the \okhttp project.


I observed that certain test suite runs, especially those with a higher number of failed tests, tend to exhibit repetitive flaky failures across most or all the failed tests. For instance, within the \ambari, 48 out of 51 flaky tests consistently failed together and 47 of these tests displayed the same \failures each time they failed, and none of their stacktrace lines contain the test names. 



\textbf{Summary}. Flaky failures are often repetitive. This can serve as an indicator for developers: previous flaky failures can be a reference to check if a newly encountered failure is familiar. However, there are \emph{few} cases where a failure is not similar with any previously observed flaky failures. In such situations, a deeper investigation is needed to detect its flakiness. A valuable step in this investigative process involves comparing the failure with flaky failures from other tests, especially when the failure's stacktrace lines do not reference the test itself.



% section named as resutl

\subsubsection{RQ2: With prior flaky and true failures, is it feasible to use the failure de-duplicaiton to tell if a failure is flaky or true one?}
\label{matchingRQ2}



I investigate if the \syntax can be used to determine if a failure is flaky or not based on the failure de-duplication. As I consider both flaky and true failures, I use the basic of confusion matrix as follow:

\begin{description}
    \item \textbf{TP}: Flaky failures that match at least one flaky failure and do not match any of the true failures. 
    \item \textbf{FN}: Flaky failures that match at least one true failure \textit{or} does not match with any of the flaky failures. 
    \item \textbf{FP}: True failures that match at least one flaky failure. 
    \item \textbf{TN}: True failures that do not match with any of flaky failure.
    
\end{description}

This evaluation methodology follows the running use-case, where newly observed test failures are either labeled as flaky (and ignored), or triaged to developers for further debugging and analysis.
%In the case of a failure that does not match with any of flaky failures, I knew these failures are flaky as reported by the studied dataset~\cite{alshammari2021flakeflagger}. However, I label it as a \textit{FN} to follow the assumption of the discussed use case, where any failure that is not confirmed as flaky is triaged to developers for debugging and analysis.
%In terms of the true failures, I confirm that all the reported true failures are de-duplicaiton with other true failures as I run each killed mutant (where I collect the true failure) 20 times and each time fails match the exact \failure.
%To avoid biasing the result by having 20 times of each true failure, I consider one failure per each killed mutant. 
I then evaluate the result of matching using the \emph{Precision} (\textbf{P}), \emph{Recall} (\textbf{R}),and \emph{Specificity} (\textbf{SP}) as follow:

\begin{equation}
\label{precision}
\text{Precision (\textbf{P})} = \frac{\text{TP}}{\text{TP + FP}}
\end{equation}

\begin{equation}
\label{recall}
\text{Recall (\textbf{R})} = \frac{\text{TP}}{\text{TP + FN}}
\end{equation}

\begin{equation}
\label{specificity}
\text{Specificity (\textbf{SP})} = \frac{\text{TN}}{\text{TN + FP}}
\end{equation}

I report, per each project, the confusion matrix as well as the scores of \textbf{P}, \textbf{R} and \textbf{SP}. To highlight if one result could be biased by the number of flaky tests as the number of times each test fails may differ, I show the number of flaky tests that forming each part of the confusion matrix result. This result is shown in the column \emph{Flaky VS True} in Table~\ref{nonunique}.

% Why recall, precision, and precision 
I choose these metrics to evaluate the result as well as to reflect the use-case when a developer encounter a failure and decide to compare it with the historical flaky and true failures.
Given a model where developers ignore test failures that are labeled as flaky, a safer approach would have a higher precision, as precision reports the frequency with which an approach falsely determines a test to be flaky.
Since I consider scenarios where developers may be most concerned that there are few false positives, I also report specificity, which evaluates the percentage of true failures correctly labeled. 
Lower recall scores indicate that an approach inadvertently labels more flaky failures as true failures --- indicating that a developer might spend more time debugging them.
%  \abdul{I need to show here why I use these evaluation and how it match with the use case}
% \jon{Agreed: I think that one sentence per metric like "In this use-case, precision means..."}
% In this use-case, precision means that the higher the scores a project has, the more confidently a developer can rely on the failure de-duplication for previous flaky failures.
% Specificity shows how confidently I can determine that a new failure, which matches previous true failures, is indeed an actual failure. 
% \abdul{How recall helps?}

%In certain situations, accessing the complete stacktraces can be challenging due to various constraints, making only the failure message (the exception type and message) accessible.





% This is similar to the Table~\ref{nonunique}, but for the failures that flakes once (which have no match with other flaky failures). For the columns \textit{Failures Flake =1}: it shows the number of failures that flakes once (\textbf{f}), follow by the total number of tests where these failures belong (\textbf{t}), followed by the number of tests where the whole test only flakes once (\textbf{t_{1}})

\begin{table*}[t]
\caption[Top 10 Most Occurrence Exception in Flaky and True Failures]{Top 10 Most Occurrence Exception in Flaky and True Failures \\ 
\textnormal{ The \textit{Exception Occurrence} column details the frequency of a specific exception, indicating in how many projects, tests, and failures this exception has been observed. The \textit{Match Result (with Stacktraces)} column displays the match distributions, considering stacktraces and the related test count while the, \textit{Match Result (without Stacktraces)} column indicates match results based on exception types, excluding stacktraces.}}

\vspace{-5pt}
\setlength{\tabcolsep}{2.5pt}
\newcommand{\failureRateWidth}{2.5in}
\newcommand{\failureRateHeight}{4em}
\scriptsize
\centering

% -- > Version 2: With Tests .. 

    \begin{tabular}{l|rrr|rr|rrrr|rrrr}
    \toprule
      & \multicolumn{5}{c|}{\textbf{Exception Occurrence}} & \multicolumn{4}{c|}{\textbf{Match Result by Failures}} & \multicolumn{4}{c}{\textbf{Match Result by Failures}} \\ 

      & \multicolumn{5}{c|}{\textbf{}} & \multicolumn{4}{c|}{\textbf{(with Stacktraces)}}  &\multicolumn{4}{c}{\textbf{(without Stacktraces)}} \\ 
     
     \textbf{Exception Name}&\textbf{Projects}&\textbf{Tests}&\textbf{Failures}&\textbf{True}&\textbf{Flaky}&\textbf{TP}& \textbf{FN}&\textbf{FP}& \textbf{TN}&\textbf{TP}& \textbf{FN}&\textbf{FP}& \textbf{TN}\\
        \midrule
AssertionError&21&407&51,453&20,507&30,946&6,120&24,826&4,850&15,657&64&30,550&13,968&6,539\\

\cellcolor{gray!6}{NullPointerException}&\cellcolor{gray!6}{22}&\cellcolor{gray!6}{498}&\cellcolor{gray!6}{49,906}&\cellcolor{gray!6}{41,709}&\cellcolor{gray!6}{8,197}&\cellcolor{gray!6}{1,644}&\cellcolor{gray!6}{6,553}&\cellcolor{gray!6}{449}&\cellcolor{gray!6}{41,260}&\cellcolor{gray!6}{34}&\cellcolor{gray!6}{8,163}&\cellcolor{gray!6}{7,913}&\cellcolor{gray!6}{33,796}\\
IOException&7&257&20,097&15,963&4,134&3,614&520&519&15,444&28&3,141&3,717&12,246\\
\cellcolor{gray!6}{RuntimeException}&\cellcolor{gray!6}{17}&\cellcolor{gray!6}{420}&\cellcolor{gray!6}{13,810}&\cellcolor{gray!6}{13,676}&\cellcolor{gray!6}{134}&\cellcolor{gray!6}{43}&\cellcolor{gray!6}{91}&\cellcolor{gray!6}{1,011}&\cellcolor{gray!6}{12,665}&\cellcolor{gray!6}{31}&\cellcolor{gray!6}{103}&\cellcolor{gray!6}{1,141}&\cellcolor{gray!6}{12,535}\\
NoServerForRegionException&1&35&11,686&169&11,517&11,512&5&0&169&1,921&9,596&75&94\\
\cellcolor{gray!6}{UnknownHostException}&\cellcolor{gray!6}{9}&\cellcolor{gray!6}{234}&\cellcolor{gray!6}{9,942}&\cellcolor{gray!6}{319}&\cellcolor{gray!6}{9,623}&\cellcolor{gray!6}{9,620}&\cellcolor{gray!6}{3}&\cellcolor{gray!6}{0}&\cellcolor{gray!6}{319}&\cellcolor{gray!6}{9,620}&\cellcolor{gray!6}{3}&\cellcolor{gray!6}{0}&\cellcolor{gray!6}{319}\\
ActivitiException&1&30&9,893&9,821&72&0&72&614&9,207&0&72&3,094&6,727\\
\cellcolor{gray!6}{IllegalArgumentException}&\cellcolor{gray!6}{17}&\cellcolor{gray!6}{401}&\cellcolor{gray!6}{9,052}&\cellcolor{gray!6}{9,049}&\cellcolor{gray!6}{3}&\cellcolor{gray!6}{0}&\cellcolor{gray!6}{3}&\cellcolor{gray!6}{190}&\cellcolor{gray!6}{8,859}&\cellcolor{gray!6}{0}&\cellcolor{gray!6}{3}&\cellcolor{gray!6}{212}&\cellcolor{gray!6}{8,837}\\
AssertionFailedError&7&98&8,832&7,054&1,778&66&1,712&1,648&5,406&66&1,712&4,150&2,904\\
\cellcolor{gray!6}{PersistenceException}&\cellcolor{gray!6}{2}&\cellcolor{gray!6}{30}&\cellcolor{gray!6}{8,581}&\cellcolor{gray!6}{8,580}&\cellcolor{gray!6}{1}&\cellcolor{gray!6}{0}&\cellcolor{gray!6}{1}&\cellcolor{gray!6}{164}&\cellcolor{gray!6}{8,416}&\cellcolor{gray!6}{0}&\cellcolor{gray!6}{1}&\cellcolor{gray!6}{398}&\cellcolor{gray!6}{8,182}\\



%     \begin{tabular}{l|rrr|rr|rrrr|rrrr|rrrr|rrrr}
%     \toprule
%       & \multicolumn{5}{c|}{\textbf{Exception Occurrence}} & \multicolumn{8}{c|}{\textbf{Match Result (with Stacktraces)}} & \multicolumn{8}{c}{\textbf{Match Result (without Stacktraces)}} \\ 

%       & \multicolumn{5}{c|}{\textbf{}} & \multicolumn{4}{c|}{\textbf{By Failures}} & \multicolumn{4}{c|}{\textbf{By Tests}} & \multicolumn{4}{c}{\textbf{By Tests}} & \multicolumn{4}{c}{\textbf{By Failures}} \\ 
     
%      \textbf{Exception Name}&\textbf{Projects}&\textbf{Tests}&\textbf{Failures}&\textbf{True}&\textbf{Flaky}&\textbf{TP}& \textbf{FN}&\textbf{FP}& \textbf{TN}&\textbf{TP}& \textbf{FN}&\textbf{FP}& \textbf{TN}&\textbf{TP}& \textbf{FN}&\textbf{FP}& \textbf{TN}&\textbf{TP}& \textbf{FN}&\textbf{FP}& \textbf{TN}\\
%         \midrule
% AssertionError&21&407&51,453&20,507&30,946&6,120&24,826&4,850&15,657&64&122&96&371&396&30,550&13,968&6,539&5&176&173&226\\
% \cellcolor{gray!6}{NullPointerException}&\cellcolor{gray!6}{22}&\cellcolor{gray!6}{498}&\cellcolor{gray!6}{49,906}&\cellcolor{gray!6}{41,709}&\cellcolor{gray!6}{8,197}&\cellcolor{gray!6}{1,644}&\cellcolor{gray!6}{6,553}&\cellcolor{gray!6}{449}&\cellcolor{gray!6}{41,260}&\cellcolor{gray!6}{32}&\cellcolor{gray!6}{100}&\cellcolor{gray!6}{99}&\cellcolor{gray!6}{489}&\cellcolor{gray!6}{34}&\cellcolor{gray!6}{8,163}&\cellcolor{gray!6}{7,913}&\cellcolor{gray!6}{33,796}&\cellcolor{gray!6}{9}&\cellcolor{gray!6}{120}&\cellcolor{gray!6}{120}&\cellcolor{gray!6}{369}\\
% IOException&7&257&20,097&15,963&4,134&3,614&520&519&15,444&28&19&13&234&993&3,141&3,717&12,246&19&27&22&212\\
% \cellcolor{gray!6}{RuntimeException}&\cellcolor{gray!6}{17}&\cellcolor{gray!6}{420}&\cellcolor{gray!6}{13,810}&\cellcolor{gray!6}{13,676}&\cellcolor{gray!6}{134}&\cellcolor{gray!6}{43}&\cellcolor{gray!6}{91}&\cellcolor{gray!6}{1,011}&\cellcolor{gray!6}{12,665}&\cellcolor{gray!6}{7}&\cellcolor{gray!6}{14}&\cellcolor{gray!6}{2}&\cellcolor{gray!6}{409}&\cellcolor{gray!6}{31}&\cellcolor{gray!6}{103}&\cellcolor{gray!6}{1,141}&\cellcolor{gray!6}{12,535}&\cellcolor{gray!6}{5}&\cellcolor{gray!6}{16}&\cellcolor{gray!6}{12}&\cellcolor{gray!6}{399}\\
% NoServerForRegionException&1&35&11,686&169&11,517&11,512&5&0&169&8&4&0&33&1,921&9,596&75&94&2&9&9&24\\
% \cellcolor{gray!6}{UnknownHostException}&\cellcolor{gray!6}{9}&\cellcolor{gray!6}{234}&\cellcolor{gray!6}{9,942}&\cellcolor{gray!6}{319}&\cellcolor{gray!6}{9,623}&\cellcolor{gray!6}{9,620}&\cellcolor{gray!6}{3}&\cellcolor{gray!6}{0}&\cellcolor{gray!6}{319}&\cellcolor{gray!6}{133}&\cellcolor{gray!6}{3}&\cellcolor{gray!6}{0}&\cellcolor{gray!6}{98}&\cellcolor{gray!6}{9,620}&\cellcolor{gray!6}{3}&\cellcolor{gray!6}{0}&\cellcolor{gray!6}{319}&\cellcolor{gray!6}{133}&\cellcolor{gray!6}{3}&\cellcolor{gray!6}{0}&\cellcolor{gray!6}{98}\\
% ActivitiException&1&30&9,893&9,821&72&0&72&614&9,207&0&9&7&29&0&72&3,094&6,727&0&9&8&21\\
% \cellcolor{gray!6}{IllegalArgumentException}&\cellcolor{gray!6}{17}&\cellcolor{gray!6}{401}&\cellcolor{gray!6}{9,052}&\cellcolor{gray!6}{9,049}&\cellcolor{gray!6}{3}&\cellcolor{gray!6}{0}&\cellcolor{gray!6}{3}&\cellcolor{gray!6}{190}&\cellcolor{gray!6}{8,859}&\cellcolor{gray!6}{0}&\cellcolor{gray!6}{3}&\cellcolor{gray!6}{3}&\cellcolor{gray!6}{401}&\cellcolor{gray!6}{0}&\cellcolor{gray!6}{3}&\cellcolor{gray!6}{212}&\cellcolor{gray!6}{8,837}&\cellcolor{gray!6}{0}&\cellcolor{gray!6}{3}&\cellcolor{gray!6}{3}&\cellcolor{gray!6}{398}\\
% AssertionFailedError&7&98&8,832&7,054&1,778&66&1,712&1,648&5,406&1&21&21&94&66&1,712&4,150&2,904&1&21&21&76\\
% \cellcolor{gray!6}{PersistenceException}&\cellcolor{gray!6}{2}&\cellcolor{gray!6}{30}&\cellcolor{gray!6}{8,581}&\cellcolor{gray!6}{8,580}&\cellcolor{gray!6}{1}&\cellcolor{gray!6}{0}&\cellcolor{gray!6}{1}&\cellcolor{gray!6}{164}&\cellcolor{gray!6}{8,416}&\cellcolor{gray!6}{0}&\cellcolor{gray!6}{1}&\cellcolor{gray!6}{1}&\cellcolor{gray!6}{30}&\cellcolor{gray!6}{0}&\cellcolor{gray!6}{1}&\cellcolor{gray!6}{398}&\cellcolor{gray!6}{8,182}&\cellcolor{gray!6}{0}&\cellcolor{gray!6}{1}&\cellcolor{gray!6}{1}&\cellcolor{gray!6}{29}\\

\bottomrule


\end{tabular}
\label{table:exceptions}
\vspace{-10pt}
\end{table*}




% \begin{table}[t]
%   \setlength{\tabcolsep}{2.0pt}

% % \jon{Resized table to try to fit bigger column headers. Not clear what unique means. Unique by stack trace?}
% \caption{List of top 10 Exceptions in Flaky Failures. \\
% \textnormal{\emph{Total Failure} indicates the count of each exception's occurrences and the total projects where these exceptions appear. \emph{With S} and \emph{Without S} represent the presence or absence of stacktrace lines during the matching using the \syntax, respectively. \emph{TP} refers to failures not matching any non-flaky failures, while \emph{FN} refers flaky failures that match with at least one non-flaky failure.}}
% \label{table:exceptions}
% \vspace{-4pt}
% %\resizebox{\textwidth}{!}{
% % \scriptsize
% \footnotesize
% \begin{tabular}{l|rr|rr|rr}

% \toprule
%       & \multicolumn{2}{c|}{\textbf{Total Failures}} & \multicolumn{2}{c|}{\textbf{With S}} & \multicolumn{2}{c}{\textbf{Without S}}\\
      

% \textbf{Exceptions} & \textbf{F} & \textbf{P}  & \textbf{TP} & \textbf{FN}  & \textbf{TP} & \textbf{FN} \\
% \midrule
% % AssertionError&186&15&88&98&8&178\\
% % \cellcolor{gray!6}{NullPointerException}&\cellcolor{gray!6}{185}&\cellcolor{gray!6}{5}&\cellcolor{gray!6}{39}&\cellcolor{gray!6}{146}&\cellcolor{gray!6}{9}&\cellcolor{gray!6}{176}\\
% % UnknownHostException&136&6&136&0&136&0\\
% % \cellcolor{gray!6}{IOException}&\cellcolor{gray!6}{70}&\cellcolor{gray!6}{4}&\cellcolor{gray!6}{57}&\cellcolor{gray!6}{13}&\cellcolor{gray!6}{36}&\cellcolor{gray!6}{34}\\
% % ProvisionException&49&1&49&0&0&49\\
% % \cellcolor{gray!6}{HCassandraInternalException}&\cellcolor{gray!6}{31}&\cellcolor{gray!6}{1}&\cellcolor{gray!6}{31}&\cellcolor{gray!6}{0}&\cellcolor{gray!6}{31}&\cellcolor{gray!6}{0}\\
% % SocketException&31&1&29&2&1&30\\
% % \cellcolor{gray!6}{Exception}&\cellcolor{gray!6}{30}&\cellcolor{gray!6}{5}&\cellcolor{gray!6}{30}&\cellcolor{gray!6}{0}&\cellcolor{gray!6}{28}&\cellcolor{gray!6}{2}\\
% % RuntimeException&23&3&21&2&9&14\\
% % \cellcolor{gray!6}{AssertionFailedError}&\cellcolor{gray!6}{22}&\cellcolor{gray!6}{5}&\cellcolor{gray!6}{1}&\cellcolor{gray!6}{21}&\cellcolor{gray!6}{1}&\cellcolor{gray!6}{21}\\


% AssertionError&30946&19&6149&24797&399&30547\\
% \cellcolor{gray!6}{NoServerForRegionException}&\cellcolor{gray!6}{11517}&\cellcolor{gray!6}{1}&\cellcolor{gray!6}{11517}&\cellcolor{gray!6}{0}&\cellcolor{gray!6}{1921}&\cellcolor{gray!6}{9596}\\
% UnknownHostException&9623&6&9623&0&9623&0\\
% \cellcolor{gray!6}{NoSuchMethodError}&\cellcolor{gray!6}{8539}&\cellcolor{gray!6}{1}&\cellcolor{gray!6}{8539}&\cellcolor{gray!6}{0}&\cellcolor{gray!6}{8539}&\cellcolor{gray!6}{0}\\
% NullPointerException&8197&5&1647&6550&34&8163\\
% \cellcolor{gray!6}{WroRuntimeException}&\cellcolor{gray!6}{6487}&\cellcolor{gray!6}{1}&\cellcolor{gray!6}{0}&\cellcolor{gray!6}{6487}&\cellcolor{gray!6}{0}&\cellcolor{gray!6}{6487}\\
% SocketException&4547&1&4492&55&3&4544\\
% \cellcolor{gray!6}{IOException}&\cellcolor{gray!6}{4134}&\cellcolor{gray!6}{4}&\cellcolor{gray!6}{3620}&\cellcolor{gray!6}{514}&\cellcolor{gray!6}{998}&\cellcolor{gray!6}{3136}\\
% ExecutionException&3465&1&0&3465&0&3465\\
% \cellcolor{gray!6}{ProvisionException}&\cellcolor{gray!6}{3055}&\cellcolor{gray!6}{1}&\cellcolor{gray!6}{3055}&\cellcolor{gray!6}{0}&\cellcolor{gray!6}{0}&\cellcolor{gray!6}{3055}\\
% \bottomrule
% \end{tabular}
% \vspace{-10pt}
% \end{table}




I summarize the finding of using the \syntax as a flaky failure detection approach. 
Table~\ref{nonunique} shows the confusion matrix of using the approach as described in Section~\ref{sec:matchingEvaluation}.
The performance of the approach varies across projects. For example, there are projects with at least 95\% precision (10 out of 22) while some projects with 0\% (5 out 10 projects). 


% when it is not working and when it works ... 
In projects where the \syntax approach struggles to differentiate between flaky and true failures, a common thread emerges: these failures are typically presented as \emph{assertion} exceptions. 
Example of these failures are \emph{all} of the \textbf{FN} flaky failures in \websocket, \emph{all} of the \textbf{FN} flaky failures in \orbit, and 98\% of the \textbf{FN} in \http. Another type of exceptions is the \emph{NullPointerException} as it is shown that about 90\% of \textbf{FN} failures in \alluxio. Even with the availability of stacktraces in these failures, these exceptions remain challenging to be used in finding the differences between flaky and true failures. 
On the other side, the projects which have reasonable precision and recall scores (or at least precision scores) like the case in \hbase, there are a verity of different exceptions like \emph{UnknownHostException} and \emph{IOException}, and less likely to have general exceptions such as \emph{assertion} and \emph{NullPointerException}.

% 




I conducted a qualitative analysis to see if certain factors influence the results of \textbf{RQ2}. I aim to validate the \syntax's performance and establish the applicability to other datasets. This examination involves multiple factors that could affect the efficacy of \syntax in distinguishing flaky failures from true ones. Key considerations include the proportion of true failures and the number of times that a test flakes. Thegoal is to show whether the approach's success in a specific project (over others) comes from its capability to differentiate failures or if external factors play a role. 
% \sout{Additionally, while I recognize the origin of true failures (comes from mutation) as a possible influencing factor, I discuss this aspect for a separate section, detailed in Section~\ref{threats}.} 
% \abdul{Will I discuss the d4j?} Nope, I don't think that it's going to make the cut



% It is not always that the number of true failures affect the performance. 
I found that the ratio of true failures does not affect the performance of the approach such that the approach under-perform if there are high number of true failures as an example in the project \http when there is no failure labeled as TP. However, in \wildfly, there are 100\% precision and SP scores even there are 4,364 true failures. Even the 12 FN are actually a single flake failures. I found also in\http project, there are 72\% of the flaky failure were labeled as \textbf{FN} even there are few number of the true failures (3,501 flaky vs 387 true failures). 


% The best project ... 
For projects which have more than one flaky tests, I found the project \ambari has a significant result in \textbf{precision}, \textbf{recall}, and \textbf{SP}. I found that the majority of the flaky failures with the exception \emph{ProvisionException}. As discussed in \textbf{RQ1}, the majority of flaky tests in this project failed together. Considering this case could be differ from the true failures because the cause of flakiness sounds to affect many tests ( as most of these tests match each other by their \failures. 


From the Table~\ref{nonunique}, I show \textbf{TP} and \textbf{FN} values by tests. There are two projects (namely \activiti~and \hector) where the all the failures labeled as \textbf{TP} and \textbf{FN}, respectively, come from one test. With this observation, it is hard to say that all failures from one test which used to have its failures belong to one group intent to have all new failure with in the same already known labeled. There for, dealing with each of the failures have to be not influnced by the test name. For example, all flaky tests in \alluxio has at least one flaky failure labeled as \textbf{TP} and 102 of these tests have also failures with \textbf{FN} failures. 


To gain insight into the value of matching stack traces (in addition to exceptions), I conducted an analysis to evaluate if the \syntax approach is able to find the differences between flaky and true failures when only the exception type is considered, \emph{without} the stacktraces. To analyze this, I extracted the most frequently failure exceptions types and evaluated how many of these could be helpful alone to find differences in various projects.
% as shown in Table~\ref{table:exceptions}. 
% In order to see if a failure by only the exception type (without the stacktraces) can be enough to find the differences between flaky and non-flaky failures. 

Table~\ref{table:exceptions} presents the top ten most frequently occurring exceptions observed in the analyzed flaky failures. In some cases, the exception by itself cannot determine the differences of matching compared the case when I consider the stacktraces such as the case of \emph{NullPointerException}. However, One of the reported exception, namely \emph{UnknownHostException}, is still be able by the exception alone to find the differences between the flaky and true failures. It is important to clarify that the presence of an exception like \emph{UnknownHostException} does not necessarily indicate a direct association with flaky failures. I have identified a few of the non-flaky failures reported with this exception in the \emph{okhttp} project. Interestingly, none of the flaky failures in that project have been reported with an \emph{UnknownHostException}. That means some exceptions could be linked to flakiness, but is not likely possible to draw a main rule across all projects. 


In the experiment, I discovered some failures where exceptions match both flaky and true failures, as indicated in Table~\ref{table:exceptions}. In the context of the experiment, the most frequently occurring exception is the \emph{AssertionError} which roughly 20\% of these failures appear in \textbf{TP}. However, when considering only the exception type and excluding stacktrace lines, the proportion drops to less than 2\%.
The reason behind this observation is the generality of the \emph{AssertionError} exception. For example, a test may have multiple assertion statements, and if they fail for different reasons, they match the exception but differ in the stacktrace. Therefore, it becomes challenging to attribute this type of exception to a specific type of failure. 



% Abdul 
\textbf{Summary}. I found that using the de-duplication approach to find flaky and true failures effective in some projects especially when their failures logs more informative than just assertion failures. For most of the project, relying on the stacktraces in addition to the exception type is helpful as most failure exceptions could be seen in both flaky and true failures.










% This section below before the de-duplication methodology I follow.. 










% \jon{Here is a new idea for evaluating syntax-based approach:
% - I assume a use-case where developers have historical failure information available
% - In this use-case, the "same" (by stack-trace and exception) failure may occur many times from the same test - depending on whether or not the flaky test can be flaky in one or multiple ways.
% - Hence, I care about the following metrics:
%     1) How often does a failure match only other flaky failures? This is the number of failures that a developer might be able to quickly infer are flaky
%     2) How often does a failure match both flaky and non-flaky failures? This is the number of failures that I can't use the syntax-based approach on
%     3) How often does a failure match only other non-flaky failures? I'm not sure if I care that much about this statistic?
% - I think that I can show these results at one of two levels:
%     1) Per-failure (I don't think that "per-unique failure" works here - it should be weighted by how often each failure can occur)
%     2) Per-test: How many tests ONLY have flaky failures that match ONLY other flaky failures? This analysis would support a claim that, for some tests, developers might always be able to see the same flaky failure, but for others, require more rigorous analysis.
% }


% In the \emph{Flaky VS Non-Flaky} column of Tables~\ref{nonunique} and~\ref{uniqueFailures}, the \syntax approach's results are presented, highlighting the possibility to find the differences between flaky and non-flaky failures. Taking the \okhttp project from Table~\ref{nonunique} as an illustration: out of its 100 flaky tests with 28,264 flaky failures (forming 121 unique types), 16,517 do not correspond to any non-flaky failures (distributed across 58 flaky tests). Meanwhile, the remaining 11,707 from 16 flaky tests match at least one non-flaky failure, resulting in a 59\% Sensitivity (\emph{Sens.}). When categorized by exception types, the \emph{UnknownHostException} emerges mostly in the non-matching category, accounting for 84\% of 9,615 instances, whereas the \emph{NullPointerException} represents 90\% of 7,232 failures that match with at least one non-flaky failure.


% The results generated by the \syntax approach reveal distinct patterns across the analyzed projects. For instance, the \emph{Sens} scores in the category of failures that flake more than once (Table~\ref{nonunique}) display significant result: projects like \hbase and \ambari have high scores, with at least 93\% \emph{Sens}, making a clear distinction between flaky and non-flaky failures. In contrast, a project such as \httpcore has a 0\% \emph{Sens}, implying all its flaky failures match at least one of the non-flaky ones. However, in the context of failures that flake just once as shown in Table~\ref{unique}, there are 6 distinct projects feature failures that do not match any non-flaky failures. Yet, projects like \alluxio and \activiti exhibit a contrasting behavior. These variances underscore the necessity for a more detailed analysis.



% In projects where the \syntax approach struggles to differentiate between flaky and non-flaky failures, a common thread emerges: these failures are typically presented as assertion exceptions. For example, in the \websocket and \http projects, all failures that have matched non-flaky failures are characterized by the \textit{AssertionError} exception. Also, a significant proportion of failures in projects such as \logback and \httpcore also being categorized under the \textit{AssertionError} exception.
% On the other hand, in projects where flaky failures are distinct, a range of other exceptions emerge, rather than \textit{AssertionError}, like \emph{UnknownHostException} and \emph{IOException}.
% In on the \alluxio project, it is shown that about 90\% of \textbf{FN} failures can be attributed to a singular exception type, the \emph{NullPointerException}. Even with the availability of stacktrace lines, these exceptions remain challenging in terms of separating flaky from non-flaky failures.


% A flaky test can exhibit varied behavior, with some failures matching non-flaky failures and others not. To illustrate, consider the \alluxio project where every flaky test within this project has at least one failures labeled as \textbf{TP}. However, 102 of these tests also present at least one flaky failure categorized as \textbf{FN}. This highlights the nature of flaky tests where they can match and do not match with non-flaky failures.




% In order to see if the failure by only the exception type can be enough to find the differences between flaky and non-flaky failures. Table~\ref{table:exceptions} presents the top ten most frequently occurring exceptions observed in the analyzed flaky failures. Among them, two exceptions, namely \emph{UnknownHostException} and \emph{NoSuchMethodError}, consistently distinguish the failures from the non-flaky failures. The uniqueness of failures based on exceptions can be influenced by the project domain. For instance, the \emph{NoSuchMethodError} has been observed in just a single project. This could be related to the particular domain in which the project operates. However, the \emph{UnknownHostException} exception is encountered in six projects, with roughly the majority of these exceptions originating from the \emph{Alluxio} project. It is important to clarify that the presence of an exception like \emph{UnknownHostException} does not necessarily indicate a direct association with flaky failures. I have identified a few of the non-flaky failures reported with this exception in the \emph{okhttp} project. Interestingly, none of the flaky failures in that project have been reported with an \emph{UnknownHostException}. That means some exceptions could be linked to flakiness, but is not likely possible to draw a main rule across all projects.


% In our experiment, I discovered some failures where failure exceptions belong to both the unique and non-unique failure, as indicated in Table~\ref{table:exceptions}. In the context of our experiment, the most frequently occurring exception is the \emph{AssertionError} which roughly 20\% of these failures appear in \textbf{TP}. However, when considering only the exception type and excluding stacktrace lines, the proportion drops to less than 2\%.
% The reason behind this observation is the generality of the \emph{AssertionError} exception. For example, a test may have multiple assertion statements, and if they fail for different reasons, they match the exception but differ in the stacktrace. Therefore, it becomes challenging to attribute this type of exception to a specific type of failure.
% In the cases where the \textbf{TP} failures involve the \emph{AssertionError} exceptions especially in \okhttp, I observe the the test name does not appear in \abdul{check}\% of the stacktrace lines for these failures. This suggests that these failures occur outside the test itself, e.g. in setup methods.




% % Based on the data presented the column Flaky Categories in Table~\ref{table:classifier_table}, it is evident that the distribution of \emph{flaky} buckets does not directly correlate with the number of flaky failures detected in each project. 
% % This observation is clear when comparing projects with a similar number of flaky failures.
% % For instance, among the three projects that have over 100 flaky failures, two of them exhibit a high number of \emph{flaky} buckets, accounting for at least 86\% of the total failures. 
% % However, the third project, named \alluxio, has less than half of its failures labeled as \emph{flaky}.
% % % This indicates that having \emph{flaky} buckets is not determined by the quantity of flaky failures in a project.
% % Furthermore, even in projects with a low number of detected flaky failures, such as \hector and \activiti with 33 and 32 flaky failures respectively, the rate of unique failures varies significantly. 
% % The project \hector has a high rate of \emph{flaky} buckets (32 out of 33), while \activiti has only 4 out of 32 buckets. \abdul{Done} \jon{I don't understand: doesn't table 2 show failures (not buckets)?}


% I explored whether the \syntax is effective by the case when tests that have a high count of \textbf{TP} but a low of non-flaky failures. Our analysis indicated varying outcomes across projects. For instance, all \textbf{TP} failures in \spring, 96 from \alluxio, and 98 from \okhttp originated from tests with over 100 non-flaky failures. On the other hand, in \alluxio, 71 out of \textbf{FN} failures arose from tests recording fewer than 100 non-flaky failures. The data does not present a clear trend to infer that \textbf{TP} failures emerge more in tests with a low number of non-flaky failures.


% % In the dataset used for the experiment, the distribution of the number of failures per flaky tests during the total number of runs was calculated and categorized into four groups based on the range of runs~\cite{alshammari2021flakeflagger}. I analyze the result if there is a correlation between the flaky failures in \emph{flaky} buckets and the reported flake rate of these failures. \abdul{Assumption: flaky tests less likely flakes are mostly fall in flaky buckets}. \abdul{See Table 5 to summarize this point.}



% \textbf{Summary}. \abdul{To Do ... }





% In the cases where the \textbf{TP} failures involve the \emph{AssertionError} exceptions especially in \okhttp, I observe the the test name does not appear in \abdul{check}\% of the stacktrace lines for these failures. This suggests that these failures occur outside the test itself, e.g. in setup methods.


% What does it mean to have high recall and less precision etc .. 
% Based on the \emph{Evaluation} column in Table~\ref{nonunique}, it is not always having high the scores of precision and recall are aligned to each other. 



% \begin{table*}[t]
%     \caption{caption X}
%     % by Test -> How many flaky tests in a certain project? Show number of tests that flake exactly once, then the minimum number of flakes per-test, max, and a sparkline
%     % by Build -> How many builds have at least one flaky test? How many builds iwht exactly one flaky test, then min, max, and distribution of flaky tests 
% \label{table:classifier_table}
% \vspace{-5pt}
% \setlength{\tabcolsep}{2.5pt}
% \newcommand{\failureRateWidth}{2.5in}
% \newcommand{\failureRateHeight}{4em}
% \scriptsize
% \centering
%     \begin{tabular}{l|ccc|cc|rrrrrrr|rrrrrrr}
%     \toprule
%       & \multicolumn{3}{c}{\textbf{Text-Match-Approach}} & \multicolumn{2}{c}{\textbf{Text-Match-Approach}} & \multicolumn{7}{c}{\textbf{Failure Log Classifier}} & \multicolumn{7}{c}{\textbf{TF-IDF}}  \\ 
     
%      \textbf{Project}&\textbf{Failures}&\textbf{Flaky}&\textbf{Non-Flaky}&\textbf{UNMATCH}&\textbf{MATCH}&\textbf{TP}&\textbf{FN}&\textbf{FP}&\textbf{TN}&\textbf{P}&\textbf{R}&\textbf{F1}&\textbf{TP}&\textbf{FN}&\textbf{FP}&\textbf{TN}&\textbf{P}&\textbf{R}&\textbf{F1}\\
%         \midrule

% java-webSocket&3276&1977&1299&42&3&1942&35&854&445&69\%&98\%&81\%&1855&122&1158&141&62\%&94\%&74\%\\
% assertj-core&30&13&17&1&0&9&4&1&16&90\%&69\%&78\%&10&3&1&16&91\%&77\%&83\%\\
% % ninja&319&110&209&NA&NA&110&0&90&119&55\%&100\%&71\%&110&0&92&117&54\%&100\%&71\%\\
% orbit&1035&213&822&2&5&202&11&69&753&75\%&95\%&83\%&108&105&236&586&31\%&51\%&39\%\\
% % handlebars.java&183&36&147&NA&NA&36&0&16&131&69\%&100\%&82\%&36&0&16&131&69\%&100\%&82\%\\
% achilles&499&57&442&2&3&54&3&61&381&47\%&95\%&63\%&39&18&105&337&27\%&68\%&39\%\\
% logback&2948&334&2614&7&14&298&34&496&2118&38\%&90\%&53\%&186&146&34&2580&85\%&56\%&67\%\\
% okhttp&36685&2419&34266&105&16&2261&158&3019&31247&43\%&93\%&59\%&717&1702&9968&24298&7\%&30\%&11\%\\
% wro4j&576&36&540&12&9&26&8&45&495&37\%&76\%&50\%&23&11&14&526&62\%&68\%&65\%\\
% activiti&48271&2171&46100&5&28&2136&35&7856&38244&21\%&98\%&35\%&480&1691&5544&40556&8\%&22\%&12\%\\
% http-request&405&18&387&7&11&14&4&113&274&11\%&78\%&19\%&18&0&0&387&100\%&100\%&100\%\\
% hbase&11895&571&11324&130&4&425&26&478&10846&47\%&94\%&63\%&269&182&2146&9178&11\%&60\%&19\%\\
% alluxio&33317&709&32608&138&172&630&74&6996&25612&8\%&89\%&15\%&385&319&5649&26959&6\%&55\%&11\%\\
% httpcore&8426&93&8333&1&21&52&41&1789&6544&3\%&56\%&5\%&39&54&2049&6284&2\%&42\%&4\%\\
% hector&3638&34&3604&31&2&31&3&0&3604&100\%&91\%&95\%&33&1&0&3604&100\%&97\%&99\%\\
% % io-undertow&2325&21&2304&NA&NA&17&4&1388&916&1\%&81\%&2\%&19&2&1719&585&1\%&90\%&2\%\\
% spring-boot&2503&353&2150&13&0&5&9&516&1634&1\%&36\%&2\%&12&2&0&2150&100\%&86\%&92\%\\
% ambari&11106&57&11049&52&2&49&7&439&10610&10\%&88\%&18\%&51&5&119&10930&30\%&91\%&45\%\\
% wildfly&3863&16&3847&18&0&16&0&116&3731&12\%&100\%&22\%&16&0&0&3847&100\%&100\%&100\%\\
    

% \bottomrule 
% \end{tabular}
% \vspace{-10pt}
% \end{table*}


\begin{table*}[t]
% \jon{New columns: Flaky tests, Flaky Failures, Non-Flaky Failures, Group of 3 columns with header "Synatx-based approach": (Flaky only, Both, Non-Flaky Only), Failure log classifier, TF-IDF}
% \jon{Remove "Killed Mutant statistics" and instead do some analysis to determine how many flaky failures have fewer than N (N=1,2,3?) mutants to match against, and comment on these (and whether or not they match)}
\caption[The Prediction of \classifier and TF-IDF of Flaky and True Failures]{The Result of \classifier and TF-IDF of Flaky and True Failures Prediction?\\
\textnormal{The \classifier and TF-IDF show (per project) the confusion matrix, precision (P), recall (R), and F1 score of the overall prediction result. 
% Compared to Table\ref{nonunique}, projects with fewer than 10 flaky failures have been opted out.
}}
    % by Test -> How many flaky tests in a certain project? Show number of tests that flake exactly once, then the minimum number of flakes per-test, max, and a sparkline
    % by Build -> How many builds have at least one flaky test? How many builds iwht exactly one flaky test, then min, max, and distribution of flaky tests 
\label{table:classifier_table}
\vspace{-5pt}
\setlength{\tabcolsep}{1.0pt}
\newcommand{\failureRateWidth}{2.5in}
\newcommand{\failureRateHeight}{4em}
\scriptsize
\centering
    \begin{tabular}{l|rrrr|rrrrrrr|rrrrrrr}
    \toprule
      & \multicolumn{4}{c}{\textbf{Total Flaky Tests and Failures}} & \multicolumn{7}{c}{\textbf{Failure Log Classifier}} & \multicolumn{7}{c}{\textbf{TF-IDF}}\\ 
     
     \textbf{Project}&\textbf{Test}&\textbf{Failures}&\textbf{Flaky}&\textbf{True}&\textbf{TP}&\textbf{FN}&\textbf{FP}&\textbf{TN}&\textbf{P}&\textbf{R}&\textbf{F1}&\textbf{TP}&\textbf{FN}&\textbf{FP}&\textbf{TN}&\textbf{P}&\textbf{R}&\textbf{F1}\\
        \midrule

Alluxio-alluxio&114&49,466&16,858&32,608&16,014&844&1,104&31,504&93\%&94\%&94\%&16,580&278&394&32,214&97\%&98\%&98\%\\
\cellcolor{gray!6}{square-okhttp}&\cellcolor{gray!6}{100}&\cellcolor{gray!6}{62,530}&\cellcolor{gray!6}{28,264}&\cellcolor{gray!6}{34,266}&\cellcolor{gray!6}{28,123}&\cellcolor{gray!6}{141}&\cellcolor{gray!6}{1,585}&\cellcolor{gray!6}{32,681}&\cellcolor{gray!6}{94\%}&\cellcolor{gray!6}{99\%}&\cellcolor{gray!6}{97\%}&\cellcolor{gray!6}{28,238}&\cellcolor{gray!6}{26}&\cellcolor{gray!6}{108}&\cellcolor{gray!6}{34,158}&\cellcolor{gray!6}{99\%}&\cellcolor{gray!6}{99\%}&\cellcolor{gray!6}{99\%}\\
apache-hbase&62&31,146&19,822&11,324&19,782&40&369&10,955&98\%&99\%&98\%&19,676&146&19&11,305&99\%&99\%&99\%\\
\cellcolor{gray!6}{apache-ambari}&\cellcolor{gray!6}{51}&\cellcolor{gray!6}{15,112}&\cellcolor{gray!6}{4,063}&\cellcolor{gray!6}{11,049}&\cellcolor{gray!6}{4,055}&\cellcolor{gray!6}{8}&\cellcolor{gray!6}{482}&\cellcolor{gray!6}{10,567}&\cellcolor{gray!6}{89\%}&\cellcolor{gray!6}{99\%}&\cellcolor{gray!6}{94\%}&\cellcolor{gray!6}{4,063}&\cellcolor{gray!6}{0}&\cellcolor{gray!6}{5}&\cellcolor{gray!6}{11,044}&\cellcolor{gray!6}{99\%}&\cellcolor{gray!6}{100\%}&\cellcolor{gray!6}{99\%}\\
Hector&33&10,133&6,529&3,604&6,529&0&405&3,199&94\%&100\%&96\%&6,529&0&13&3,591&99\%&100\%&99\%\\
\cellcolor{gray!6}{activiti-activiti}&\cellcolor{gray!6}{31}&\cellcolor{gray!6}{47,478}&\cellcolor{gray!6}{1,378}&\cellcolor{gray!6}{46,100}&\cellcolor{gray!6}{947}&\cellcolor{gray!6}{431}&\cellcolor{gray!6}{311}&\cellcolor{gray!6}{45,789}&\cellcolor{gray!6}{75\%}&\cellcolor{gray!6}{68\%}&\cellcolor{gray!6}{71\%}&\cellcolor{gray!6}{1,013}&\cellcolor{gray!6}{365}&\cellcolor{gray!6}{60}&\cellcolor{gray!6}{46,040}&\cellcolor{gray!6}{94\%}&\cellcolor{gray!6}{73\%}&\cellcolor{gray!6}{82\%}\\
apache-httpcore&22&8,687&354&8,333&315&39&110&8,223&74\%&88\%&80\%&314&40&16&8,317&95\%&88\%&91\%\\
\cellcolor{gray!6}{Java-websocket}&\cellcolor{gray!6}{22}&\cellcolor{gray!6}{3,394}&\cellcolor{gray!6}{2,095}&\cellcolor{gray!6}{1,299}&\cellcolor{gray!6}{2,082}&\cellcolor{gray!6}{13}&\cellcolor{gray!6}{721}&\cellcolor{gray!6}{578}&\cellcolor{gray!6}{74\%}&\cellcolor{gray!6}{99\%}&\cellcolor{gray!6}{85\%}&\cellcolor{gray!6}{2,082}&\cellcolor{gray!6}{13}&\cellcolor{gray!6}{722}&\cellcolor{gray!6}{577}&\cellcolor{gray!6}{74\%}&\cellcolor{gray!6}{99\%}&\cellcolor{gray!6}{84\%}\\
qos-ch-logback&20&3,052&438&2,614&172&266&104&2,510&62\%&39\%&48\%&239&199&41&2,573&85\%&54\%&66\%\\
\cellcolor{gray!6}{Http-request}&\cellcolor{gray!6}{18}&\cellcolor{gray!6}{3,888}&\cellcolor{gray!6}{3,501}&\cellcolor{gray!6}{387}&\cellcolor{gray!6}{3,498}&\cellcolor{gray!6}{3}&\cellcolor{gray!6}{124}&\cellcolor{gray!6}{263}&\cellcolor{gray!6}{96\%}&\cellcolor{gray!6}{99\%}&\cellcolor{gray!6}{98\%}&\cellcolor{gray!6}{3,498}&\cellcolor{gray!6}{3}&\cellcolor{gray!6}{54}&\cellcolor{gray!6}{333}&\cellcolor{gray!6}{98\%}&\cellcolor{gray!6}{99\%}&\cellcolor{gray!6}{99\%}\\
wildfly-wildfly&18&3,895&48&3,847&0&48&0&3,847&0\%&0\%&0\%&48&0&0&3,847&100\%&100\%&100\%\\
\cellcolor{gray!6}{wro4j-wro4j}&\cellcolor{gray!6}{14}&\cellcolor{gray!6}{11,373}&\cellcolor{gray!6}{10,833}&\cellcolor{gray!6}{540}&\cellcolor{gray!6}{10,833}&\cellcolor{gray!6}{0}&\cellcolor{gray!6}{65}&\cellcolor{gray!6}{475}&\cellcolor{gray!6}{99\%}&\cellcolor{gray!6}{100\%}&\cellcolor{gray!6}{99\%}&\cellcolor{gray!6}{10,833}&\cellcolor{gray!6}{0}&\cellcolor{gray!6}{29}&\cellcolor{gray!6}{511}&\cellcolor{gray!6}{99\%}&\cellcolor{gray!6}{100\%}&\cellcolor{gray!6}{99\%}\\
Spring-boot&12&2,164&14&2,150&6&8&0&2,150&100\%&42\%&60\%&10&4&1&2,149&90\%&71\%&80\%\\
\cellcolor{gray!6}{orbit-orbit}&\cellcolor{gray!6}{7}&\cellcolor{gray!6}{3,765}&\cellcolor{gray!6}{2,943}&\cellcolor{gray!6}{822}&\cellcolor{gray!6}{2,943}&\cellcolor{gray!6}{0}&\cellcolor{gray!6}{69}&\cellcolor{gray!6}{753}&\cellcolor{gray!6}{97\%}&\cellcolor{gray!6}{100\%}&\cellcolor{gray!6}{98\%}&\cellcolor{gray!6}{2,943}&\cellcolor{gray!6}{0}&\cellcolor{gray!6}{59}&\cellcolor{gray!6}{763}&\cellcolor{gray!6}{98\%}&\cellcolor{gray!6}{100\%}&\cellcolor{gray!6}{99\%}\\
Undertow&7&2,396&92&2,304&3&89&0&2,304&100\%&3\%&6\%&5&87&0&2,304&100\%&5\%&10\%\\
\cellcolor{gray!6}{Achilles}&\cellcolor{gray!6}{4}&\cellcolor{gray!6}{607}&\cellcolor{gray!6}{165}&\cellcolor{gray!6}{442}&\cellcolor{gray!6}{120}&\cellcolor{gray!6}{45}&\cellcolor{gray!6}{0}&\cellcolor{gray!6}{442}&\cellcolor{gray!6}{100\%}&\cellcolor{gray!6}{72\%}&\cellcolor{gray!6}{84\%}&\cellcolor{gray!6}{148}&\cellcolor{gray!6}{17}&\cellcolor{gray!6}{26}&\cellcolor{gray!6}{416}&\cellcolor{gray!6}{85\%}&\cellcolor{gray!6}{89\%}&\cellcolor{gray!6}{87\%}\\
% elasticjob-elastic-job-lite&3&0&0&0&0&0&0&0&0\%&0\%&0\%&0&0&0&0&0\%&0\%&0\%\\
Commons-exec&1&92&33&59&0&33&0&59&0\%&0\%&0\%&33&0&2&57&94\%&100\%&97\%\\
\cellcolor{gray!6}{zxing-zxing}&\cellcolor{gray!6}{1}&\cellcolor{gray!6}{398}&\cellcolor{gray!6}{322}&\cellcolor{gray!6}{76}&\cellcolor{gray!6}{322}&\cellcolor{gray!6}{0}&\cellcolor{gray!6}{0}&\cellcolor{gray!6}{76}&\cellcolor{gray!6}{100\%}&\cellcolor{gray!6}{100\%}&\cellcolor{gray!6}{100\%}&\cellcolor{gray!6}{322}&\cellcolor{gray!6}{0}&\cellcolor{gray!6}{0}&\cellcolor{gray!6}{76}&\cellcolor{gray!6}{100\%}&\cellcolor{gray!6}{100\%}&\cellcolor{gray!6}{100\%}\\
handlebars.java&1&558&411&147&411&0&16&131&96\%&100\%&98\%&411&0&16&131&96\%&100\%&98\%\\
\cellcolor{gray!6}{assertj-core}&\cellcolor{gray!6}{1}&\cellcolor{gray!6}{991}&\cellcolor{gray!6}{974}&\cellcolor{gray!6}{17}&\cellcolor{gray!6}{974}&\cellcolor{gray!6}{0}&\cellcolor{gray!6}{1}&\cellcolor{gray!6}{16}&\cellcolor{gray!6}{99\%}&\cellcolor{gray!6}{100\%}&\cellcolor{gray!6}{99\%}&\cellcolor{gray!6}{974}&\cellcolor{gray!6}{0}&\cellcolor{gray!6}{0}&\cellcolor{gray!6}{17}&\cellcolor{gray!6}{100\%}&\cellcolor{gray!6}{100\%}&\cellcolor{gray!6}{100\%}\\
ninja-ninja&1&685&476&209&476&0&90&119&84\%&100\%&91\%&476&0&90&119&84\%&100\%&91\%\\
\midrule
21 Projects Total &540&261,810&99,613&162,197&97,605&2,008&5,556&156,641&&&&98,435&1,178&1,655&160,542&&&\\

\bottomrule


% Alluxio-alluxio&114&49,466&16,858&32,608&16,014&844&1,104&31,504&93\%&94\%&94\%&16,565&293&389&32,219&97\%&98\%&97\%\\
% \cellcolor{gray!6}{square-okhttp}&\cellcolor{gray!6}{100}&\cellcolor{gray!6}{62,530}&\cellcolor{gray!6}{28,264}&\cellcolor{gray!6}{34,266}&\cellcolor{gray!6}{28,123}&\cellcolor{gray!6}{141}&\cellcolor{gray!6}{1,585}&\cellcolor{gray!6}{32,681}&\cellcolor{gray!6}{94\%}&\cellcolor{gray!6}{99\%}&\cellcolor{gray!6}{97\%}&\cellcolor{gray!6}{28,239}&\cellcolor{gray!6}{25}&\cellcolor{gray!6}{108}&\cellcolor{gray!6}{34,158}&\cellcolor{gray!6}{99\%}&\cellcolor{gray!6}{99\%}&\cellcolor{gray!6}{99\%}\\
% apache-hbase&62&31,174&19,850&11,324&19,809&41&369&10,955&98\%&99\%&98\%&19,676&146&19&11,305&99\%&99\%&99\%\\
% \cellcolor{gray!6}{apache-ambari}&\cellcolor{gray!6}{51}&\cellcolor{gray!6}{15,112}&\cellcolor{gray!6}{4,063}&\cellcolor{gray!6}{11,049}&\cellcolor{gray!6}{4,055}&\cellcolor{gray!6}{8}&\cellcolor{gray!6}{481}&\cellcolor{gray!6}{10,568}&\cellcolor{gray!6}{89\%}&\cellcolor{gray!6}{99\%}&\cellcolor{gray!6}{94\%}&\cellcolor{gray!6}{4,063}&\cellcolor{gray!6}{0}&\cellcolor{gray!6}{5}&\cellcolor{gray!6}{11,044}&\cellcolor{gray!6}{99\%}&\cellcolor{gray!6}{100\%}&\cellcolor{gray!6}{99\%}\\
% hector-client-hector&33&10,133&6,529&3,604&6,529&0&405&3,199&94\%&100\%&96\%&6,529&0&12&3,592&99\%&100\%&99\%\\
% \cellcolor{gray!6}{activiti-activiti}&\cellcolor{gray!6}{31}&\cellcolor{gray!6}{47,478}&\cellcolor{gray!6}{1,378}&\cellcolor{gray!6}{46,100}&\cellcolor{gray!6}{1,303}&\cellcolor{gray!6}{75}&\cellcolor{gray!6}{5,421}&\cellcolor{gray!6}{40,679}&\cellcolor{gray!6}{19\%}&\cellcolor{gray!6}{94\%}&\cellcolor{gray!6}{32\%}&\cellcolor{gray!6}{1,038}&\cellcolor{gray!6}{340}&\cellcolor{gray!6}{70}&\cellcolor{gray!6}{46,030}&\cellcolor{gray!6}{93\%}&\cellcolor{gray!6}{75\%}&\cellcolor{gray!6}{83\%}\\
% apache-httpcore&22&8,687&354&8,333&321&33&115&8,218&73\%&90\%&81\%&314&40&17&8,316&94\%&88\%&91\%\\
% \cellcolor{gray!6}{tootallnate-java-websocket}&\cellcolor{gray!6}{22}&\cellcolor{gray!6}{3,394}&\cellcolor{gray!6}{2,095}&\cellcolor{gray!6}{1,299}&\cellcolor{gray!6}{2,082}&\cellcolor{gray!6}{13}&\cellcolor{gray!6}{721}&\cellcolor{gray!6}{578}&\cellcolor{gray!6}{74\%}&\cellcolor{gray!6}{99\%}&\cellcolor{gray!6}{85\%}&\cellcolor{gray!6}{2,082}&\cellcolor{gray!6}{13}&\cellcolor{gray!6}{722}&\cellcolor{gray!6}{577}&\cellcolor{gray!6}{74\%}&\cellcolor{gray!6}{99\%}&\cellcolor{gray!6}{84\%}\\
% qos-ch-logback&20&3,052&438&2,614&172&266&104&2,510&62\%&39\%&48\%&245&193&49&2,565&83\%&55\%&66\%\\
% \cellcolor{gray!6}{kevinsawicki-http-request}&\cellcolor{gray!6}{18}&\cellcolor{gray!6}{3,888}&\cellcolor{gray!6}{3,501}&\cellcolor{gray!6}{387}&\cellcolor{gray!6}{3,498}&\cellcolor{gray!6}{3}&\cellcolor{gray!6}{124}&\cellcolor{gray!6}{263}&\cellcolor{gray!6}{96\%}&\cellcolor{gray!6}{99\%}&\cellcolor{gray!6}{98\%}&\cellcolor{gray!6}{3,498}&\cellcolor{gray!6}{3}&\cellcolor{gray!6}{54}&\cellcolor{gray!6}{333}&\cellcolor{gray!6}{98\%}&\cellcolor{gray!6}{99\%}&\cellcolor{gray!6}{99\%}\\
% wildfly-wildfly&18&3,895&48&3,847&48&0&116&3,731&29\%&100\%&45\%&48&0&0&3,847&100\%&100\%&100\%\\
% \cellcolor{gray!6}{wro4j-wro4j}&\cellcolor{gray!6}{14}&\cellcolor{gray!6}{11,373}&\cellcolor{gray!6}{10,833}&\cellcolor{gray!6}{540}&\cellcolor{gray!6}{10,474}&\cellcolor{gray!6}{359}&\cellcolor{gray!6}{42}&\cellcolor{gray!6}{498}&\cellcolor{gray!6}{99\%}&\cellcolor{gray!6}{96\%}&\cellcolor{gray!6}{98\%}&\cellcolor{gray!6}{10,833}&\cellcolor{gray!6}{0}&\cellcolor{gray!6}{29}&\cellcolor{gray!6}{511}&\cellcolor{gray!6}{99\%}&\cellcolor{gray!6}{100\%}&\cellcolor{gray!6}{99\%}\\
% spring-projects-spring-boot&12&2,164&14&2,150&13&1&850&1,300&1\%&92\%&2\%&10&4&2&2,148&83\%&71\%&76\%\\
% \cellcolor{gray!6}{orbit-orbit}&\cellcolor{gray!6}{7}&\cellcolor{gray!6}{3,765}&\cellcolor{gray!6}{2,943}&\cellcolor{gray!6}{822}&\cellcolor{gray!6}{2,943}&\cellcolor{gray!6}{0}&\cellcolor{gray!6}{69}&\cellcolor{gray!6}{753}&\cellcolor{gray!6}{97\%}&\cellcolor{gray!6}{100\%}&\cellcolor{gray!6}{98\%}&\cellcolor{gray!6}{2,940}&\cellcolor{gray!6}{3}&\cellcolor{gray!6}{59}&\cellcolor{gray!6}{763}&\cellcolor{gray!6}{98\%}&\cellcolor{gray!6}{99\%}&\cellcolor{gray!6}{98\%}\\
% undertow-io-undertow&7&2,396&92&2,304&57&35&148&2,156&27\%&61\%&38\%&5&87&0&2,304&100\%&5\%&10\%\\
% \cellcolor{gray!6}{doanduyhai-Achilles}&\cellcolor{gray!6}{4}&\cellcolor{gray!6}{607}&\cellcolor{gray!6}{165}&\cellcolor{gray!6}{442}&\cellcolor{gray!6}{120}&\cellcolor{gray!6}{45}&\cellcolor{gray!6}{0}&\cellcolor{gray!6}{442}&\cellcolor{gray!6}{100\%}&\cellcolor{gray!6}{72\%}&\cellcolor{gray!6}{84\%}&\cellcolor{gray!6}{148}&\cellcolor{gray!6}{17}&\cellcolor{gray!6}{28}&\cellcolor{gray!6}{414}&\cellcolor{gray!6}{84\%}&\cellcolor{gray!6}{89\%}&\cellcolor{gray!6}{86\%}\\
% \cellcolor{gray!6}{zxing-zxing}&\cellcolor{gray!6}{1}&\cellcolor{gray!6}{398}&\cellcolor{gray!6}{322}&\cellcolor{gray!6}{76}&\cellcolor{gray!6}{322}&\cellcolor{gray!6}{0}&\cellcolor{gray!6}{0}&\cellcolor{gray!6}{76}&\cellcolor{gray!6}{100\%}&\cellcolor{gray!6}{100\%}&\cellcolor{gray!6}{100\%}&\cellcolor{gray!6}{322}&\cellcolor{gray!6}{0}&\cellcolor{gray!6}{0}&\cellcolor{gray!6}{76}&\cellcolor{gray!6}{100\%}&\cellcolor{gray!6}{100\%}&\cellcolor{gray!6}{100\%}\\
% assertj-core&1&991&974&17&974&0&1&16&99\%&100\%&99\%&974&0&0&17&100\%&100\%&100\%\\
% \cellcolor{gray!6}{apache-commons-exec}&\cellcolor{gray!6}{1}&\cellcolor{gray!6}{92}&\cellcolor{gray!6}{33}&\cellcolor{gray!6}{59}&\cellcolor{gray!6}{0}&\cellcolor{gray!6}{33}&\cellcolor{gray!6}{0}&\cellcolor{gray!6}{59}&\cellcolor{gray!6}{0\%}&\cellcolor{gray!6}{0\%}&\cellcolor{gray!6}{0\%}&\cellcolor{gray!6}{33}&\cellcolor{gray!6}{0}&\cellcolor{gray!6}{2}&\cellcolor{gray!6}{57}&\cellcolor{gray!6}{94\%}&\cellcolor{gray!6}{100\%}&\cellcolor{gray!6}{97\%}\\
% ninja-ninja&1&685&476&209&476&0&90&119&84\%&100\%&91\%&476&0&90&119&84\%&100\%&91\%\\
% \cellcolor{gray!6}{handlebars.java}&\cellcolor{gray!6}{1}&\cellcolor{gray!6}{558}&\cellcolor{gray!6}{411}&\cellcolor{gray!6}{147}&\cellcolor{gray!6}{411}&\cellcolor{gray!6}{0}&\cellcolor{gray!6}{16}&\cellcolor{gray!6}{131}&\cellcolor{gray!6}{96\%}&\cellcolor{gray!6}{100\%}&\cellcolor{gray!6}{98\%}&\cellcolor{gray!6}{411}&\cellcolor{gray!6}{0}&\cellcolor{gray!6}{16}&\cellcolor{gray!6}{131}&\cellcolor{gray!6}{96\%}&\cellcolor{gray!6}{100\%}&\cellcolor{gray!6}{98\%}\\
% \midrule
% Total&543&261,838&99,641&162,197&97,744&1,897&11,761&150,436&&&&98,449&1,164&1,671&160,526&&&\\

% \bottomrule



% \bottomrule
\end{tabular}
\vspace{-10pt}
\end{table*}









\subsubsection{RQ3: How far utilizing machine learning being helpful in finding the differences between flaky and true failures?}
\label{matchingRQ3}

% Key findings :
% - features per project could be improved for projects th
% - 


% I design our evaluation on the confusion matrix (different from the one discussed in \textbf{RQ2}). The confusion matrix of the classifier will be as follow: True Positive (TP) when a flaky failure is correctly predicted as flaky, False Negative (FN) for flaky failures predicted as true failures, False Positive (FP) for true failures predicted as flaky, and True Negative (TN) when true failures are correctly identified. From this matrix, I then calculate the precision, recall, and F1-score.
I design the evaluation on the confusion matrix of the classifier as follow: True Positive (TP) when a flaky failure is correctly predicted as flaky, False Negative (FN) for flaky failures predicted as true failures, False Positive (FP) for true failures predicted as flaky, and True Negative (TN) when true failures are correctly identified. From this matrix, I then calculate the precision, recall, and F1-score.
% \jon{How is this different from RQ2?}\abdul{It could be confused especially in \textbf{FN}, so I just add this to clarify. I fixed it.}

I employ the \classifier using two classifiers (decision tree and Naive bayes) and two ways of dealing with imbalance dataset. In terms of balancing the dataset, I use the SMOTE technique if the ratio of one type of failures is less than 10$\%$ of the total number of failures of the other type. I also consider using the dataset as it is without balancing. I use stratified cross-validation and leave one fold for testing purposes. Due to the limitation of showing all result, I picked the best performance. 
I opted out projects that had fewer than 10 total flaky failures to ensure that I have at least one flaky failure in each testing fold. Also, I came across a few tests from which I could not extract features, resulting in missing values. Given their minimal occurrence (two failure in \wildfly), I exclude these failures. 

% Initially, for each project, I use stratified cross-validation and leave one fold for testing purposes. As the distribution between flaky and true failures in the training data is not balanced, I employ the SMOTE technique.
% In the second way which aligns with the discussion in Section~\ref{motivation}, I split the data based on the flaky failures recurrence. Failures that do not match any other flaky failures within the same test are utilized for testing, while others form the training set.

To further understand the efficacy of machine learning in this context, I looked for a state-of-the-art classifier based on the failure logs. Existing methods to detect flaky failures, like the work of Lampel et al.~\cite{lampel2021life}, do not align with the dataset, which is based on the failure logs. Given this and the discussed features, I considered an alternative baseline approach. I utilized TF-IDF to contrast the classifier's predictions. Furthermore, I investigated whether TF-IDF could serve as an alternative method, especially since the features of the \classifier are directly from the syntax of the failure logs without involving dynamic features.



Table~\ref{table:classifier_table} shows the result of using the \classifier and the \tfidf in predicting a failure if it is flaky or not.
While I considered two different classification algorithms (Decision Tree and Naive Bayes), I find that decisions trees (without any dataset balancing) performed the best.
% With multi results of the \classifier by using two classification algorithms and two way to deal with imbalance dataset, I show the best result of using the decision tree and use the dataset as it is without balancing. 
The relative performance of the \classifier and the \tfidf varies as in some projects they have at least 90\% F1 scores with zero \textbf{FN} failures while in few projects it is worse than being randomly guessing. The performance of the two classifiers close to each others (97,632 \textbf{TP} in the \classifier VS 98,426 in the \tfidf \textbf{TP}). Both classifiers have less False Positive rates (5,553 in the \classifier and 1,657 in \tfidf) than the rate of using the \syntax (10,153). 


% 
The main explanation of having better performance in terms of the total number of \textbf{TP} compared to the \syntax in most projects is the ability to learn from other flaky tests in the same test suite especially with projects where flaky failures share mostly the same exceptions as the case in \alluxio where the majority of the flaky failures exception with \emph{NullPointerException}. 
This is because, in my implementation of \syntax, I do \emph{not} remove test-specific lines from the stack trace.
Future work might extend my approaches to abstract these elements out of the stack trace, making matches between tests more likely~\cite{An23JustInTime}.
% \jon{I don't think that it is clear enough throughout the article that the syntax-based approach only performs matching to failures from the same test (and why this is the case)}
% \abdul{Stacktraces of failures typically involve a line reference to the test and by performing the \syntax on two different failures from two different tests, the result expected to be unmatch due to that line. Classifiers not judge on only the line that reference to the test, but consider high level abstraction of the failure in the case of the \classifier, and consider many tokens in the \tfidf}


% Classifer better than tf-idf ( nothing ) 
As the \tfidf approach is motivated to be used as comparable approach to the \classifier, it is roughly better than the \classifier (in term of the number of false positive rates) but both classifiers outperform the \syntax result reported in Table~\ref{nonunique}. The main explanation of having less false positives in the \tfidf is the ability to have more information (e.g. line numbers), {as discussed early in Section \ref{sec:tfidf}. %}\jon{Where is it described WHY TFIDF gets more information? Is it just about what features are included? I would suggest including some discusison of this either here or earlier in the article (and referencing back to it)}. 
% \sout{It most likely the number are removed during the preprocessing of the tokens (similar to stop words)}
% \jon{very confused: tfidf has a stopword step, which does remove line numbers?}. 
I found including the stacktrace lines numbers added more values as reflecting different stacktraces. 
On the other side, the generality of the features that the \classifier could be a reason that, even with high performance in most projects, still not outperform the the \tfidf. 




% tf-idf better than the classifier  ( exec + undertow + wildfly ) 
There are three projects where the \classifier is completely under-perform the \tfidf (as in \wildfly~and \exec).  In the \wildfly, I found all flaky failures with the the exception \emph{RuntimeException} with very low repetitive rate ( each test at most 7 times of failures) while the same exception mostly appear in all true failures. This project performs well in the \tfidf and even in the \syntax. The main observation in these failures is that the line numbers in the tests differ, which is not captured from the features I proposed to train the \classifier. In the \exec, I have a similar situation with another exception type named \emph{AssertionFailedError}. In a project where the classifiers and the \syntax have very low performance (even with the \tfidf) like in \emph{undertow-io-undertow}, I found the majority of flaky failures \emph{and} the true failures are forming with the exception \emph{AssertionError}. 


The usability of different machine learning approaches varies based on the specific use case and objectives. 
If the main goal is to maximize the number of true positives (\textbf{TP}) without being overly concerned about the rate of false positives, the approach with \textbf{TP} is the better. In this scenario, the model is more focused on correctly identifying as many flaky failures as possible, even if it means accepting a higher number of false positives.
One of the main advantages of the \classifier is its flexibility in extending the learned features. The model can be easily augmented with additional static and dynamic features extracted from each failure. The proposed features shown in Table~\ref{table:Features} are not final but serve as a starting point, particularly utilizing information available within the failure log. By leveraging these additional features, the failure log classifier can potentially enhance its performance identifying flaky failures.



% Abdul
\textbf{Summary.} I found that both the \classifier and \tfidf are able to predict flaky and true failures in most the projects. I found \tfidf is slightly better in terms of the total number of false positives and negatives failures compare to the \classifier result.



\subsection{Summary}
I find that flaky test failures can be extremely repetitive --- when a test fails due to flakiness, it is likely to match other flaky failures from the same or other tests.
I apply approaches based on failure de-duplication~\cite{Podgurski03Automated,Jiang17WhatCauses}, text-based matching, and simple machine learning classifiers.
I find that, for some tests, these approaches can be extremely effective (with no false negatives or false positives), yet for other tests, these approaches are entirely ineffective.
By examining attributes of tests and failures, I provide insights for future research on generalized approaches for detecting flaky failures.













% --> This is the previous version of the paper submitted 


% Flaky tests continue to be a part of test suite, regardless of the work in their detection. A major concern is whether a specific failure is flaky, especially if flaky tests continue to exist in the test suite. In response, I am studying the possibility of detecting flaky failures. First, I investigate if a failure can be detected as a flaky based on historical flaky failures, as detailed in Section~\ref{sec:failureLogsStudy}. Second, I am analyzing the failures logs to identify the differences between flaky and non-flaky failures using their failure logs. Using data analysis and machine learning algorithms, these differences could serve as a basis for distinguishing between the two types of failures, so enabling flaky failure detection. This approach is further discussed in Section~\ref{sec:failureLogsApproach}.



% \subsection{Flaky Failure Logs: Study}
% \label{sec:failureLogsStudy}

% % This is a summary of my study section in the paper currently under preparation.
% % \jon{i would write "under preparation" until it is submitted}. \abdul{Done}


% The detection of test flakiness can be achieved through various methods, and one such method is to debug failure logs, as highlighted in a study by Habchi et al. \cite{habchi2022qualitative}. Some developers might prefer this technique over others due to the fact that rerunning or using other detection methods could require significant resources and expertise. However, manual debugging can be a challenging approach, particularly when attempting to determine if a particular failure is due to flakiness or not. This is because the process can be time-consuming and demands a high level of skill and attention to detail in order to accurately identify if the failure is flaky or not. Failure logs can be particularly complex, especially in large and distributed systems, and require a deep understanding of the system's architecture, programming language, and all other dependencies to effectively analyze the logs.


% Experienced developers may sometimes be able to identify whether a failure is flaky or not by examining the failure message and stacktrace, as discussed in a Gradle blog post on preventing flaky tests~\cite{gradlePreventingFlaky}.
% This implies that developers can recognize flaky failures because they have encountered similar failures before, which means that flaky failures could be not unique in their failure exceptions. To gain a deeper understanding of this technique, I am currently analyzing the failure logs of flaky tests documented in Section~\ref{sec:flakeFlaggerStudy}. Specifically, my aim is to collect the failure logs from each reported flaky test in order to determine if each flaky failure, as indicated by its log, has been previously reported. The primary research questions I am addressing are as follows:


% \begin{description}
%   \item[\textbf{RQ:}] Do flaky failures match for the same test across different executions of the test suite?
%   \item[\textbf{RQ:}] Do flaky failures match for different tests within the same execution of a test suite?

%  \end{description}
%  % \jon{Before going into the study design, I would suggest a paragraph or two right here describing what the implications of this study could be: if the answer to RQ1 and RQ2 is "yes", then how does that contribute to your overall goal?}\abdul{How about the following paragraph?}\jon{Looks great!}

% By providing answers to the two research questions (RQs), developers can enhance their decision making process when it comes to utilizing logs to compare newly discovered failures with known flaky failures, in order to assess the potential flakiness of the new failures. The first RQ investigates whether a failure can be identified as flaky if its log matches previous flaky failures based on the failure exception and stacktrace lines. 
% However, it should be noted that this does not necessarily imply that unique failures (failures that do not match any previously detected flaky failures) identified by their failure exception and stacktrace lines are not also flaky; they may simply not have been detected before. To address this aspect, the study aims to determine the proportion of unique failures in each flaky test that contains more than one flaky failure. The second RQ is formulated and discussed due to the occurrence of certain flakiness root causes that can lead to multiple tests failing simultaneously. For instance, if numerous tests share resources that may intermittently become unavailable, all dependent tests could fail. The study intends to assess the likelihood of flaky tests that fail concurrently sharing the same failure exception and stacktrace lines.

% \input{Listings/flakyFailure}
% % \jon{Should line 3 end tag be </E>?}\abdul{Done}

%  \subsubsection{Study Design}

% To initiate this study, I collected the failure logs from the rerun experiment that was described in Section~\ref{sec:flakeFlaggerStudy}. For each flaky test, there was a collection of failure logs, with each log representing a specific failure that occurred during a particular run. I extracted the primary pieces of information that developers typically rely on when manually debugging from each failure log, which were the failure message and stacktrace lines. While the failure message in a log provides an overview of what went wrong in a test, examining the stacktrace lines is providing more details for identifying the root cause of a failure, as it provides a list of method calls that led to the failure. Both of them, in each failure, are used to generate an XML file for each flaky test, which contained all of its flaky failures. Each element in each test XML file corresponded to a specific type of failure based on its failure message and stacktrace lines.

% % \jon{I suggest introducing 1-2 examples of failures here (before describing E/M/S), including the project name, test name, and complete exception/message/stacktrace. The following paragraphs can then reference back to those examples to explain what a "match" would be} \abdul{Next two paragraphs.}
% % \jon{Looks great!}

% Listing~\ref{lst:flakyFailures} shows two \emph{Failure} tags that correspond to two flaky failures reported in one of the tests of the \emph{alluxio} project. Each \emph{Failure} tag includes four primary sub-tags that describe each failure: test name (\textbf{T}), exception type (\textbf{E}), exception message (\textbf{M}), and stacktrace lines (\textbf{S}). The \textbf{T} tag provides the project to which the test belongs. The original failure message in the log follows the format of \textbf{E}:\textbf{M}, where \textbf{E} specifies the exception type (e.g., java.net.UnknownHostException), and \textbf{M} includes any text that comes after the exception. When analyzing the stacktrace lines (\textbf{S}), only the initial lines leading up to the test name are considered. This is because the top lines in a stacktrace represent the most recent method calls that were executed before the exception occurred and are often the ones most directly related to the root cause of the exception. The lower lines may still provide valuable information for debugging, but they are generally less relevant to the root cause of the exception. If the test name is not present in the stacktrace (such as failures occurring in the start or before methods in Listing~\ref{lst:flakyFailures}), the last line of the test class is taken as the final line from the stacktrace.


% The failure message might contain distinct information associated with the specific time when the test fails. Consequently, when comparing two test failures using the elements (\textbf{E}), (\textbf{M}), and (\textbf{S}) as criteria, they may differ due to the unique information provided by the failure. For instance, the two failures shown in Listing\ref{lst:flakyFailures} could initially be perceived as separate failures, despite appearing identical except for the IP address mentioned in \textbf{M}. To compare two failures, I exclude the failure message \textbf{M} to prevent mismatched failures arising from this factor. While considering only \textbf{S} may not alter anything when comparing two failures, incorporating \textbf{E} provides additional information during the comparison of failures.





% % \jon{I think that these two paragraphs clearly describes what you did, but does not completely capture why. Consider a reader who has no familiarity with flaky tests and matching failure logs. Why trim lines from the stacktrace? Why consider stacktraces at all? What would happen if you made different choices here?}\abdul{few updates in the previous paragraphs that hopefully capture Jon notes}\jon{Better now}



% Various factors such as network or shared resources can cause test flakiness that impacts more than a single test, leading to multiple tests failing at the same time and producing similar failure logs. During manual debugging, developers may categorize a failure as flaky if it resembles a previous failure, even if it is not necessarily from the same test. To detect common failure logs, it is essential to compare failures from tests within the same test suite run, as well as failures from the same test. 
% % Although it is possible to compare failure logs from two different projects, the domain of the two projects could impact the accuracy of matching. 
% For the purpose of this study, I focus only on the two methods of comparing failures mentioned above. In the second method, when comparing failures, the stacktrace line that includes the failed test name is excluded to prevent mismatches due to differences only in the test name line.
% % \jon{You are the one proposing these two methods here (not prior work) - can you provide some deeper discusion (at least a few sentences) of why it makes sense to consider these two methods, and whether there might be altenratives that you did not implement?} \abdul{Updated, but not sure if I captured the whole idea} \jon{Better now}


% When dealing with a flaky test that exhibits multiple failures, I \emph{cluster} all failures based on the type of exception and stack trace lines. If there are two failures which have exactly the same exception type and stack trace lines, the expected to be in the same cluster. If a cluster consists only one failure, it is referred to as a unique cluster. Conversely, a non-unique cluster contains multiple failures. The uniqueness of a cluster indicates that the corresponding failure has been encountered only once, which implies challenges for developers when relying on previous knowledge. Equation~\ref{ScoreEq} quantifies the proportion of non-unique failures, providing a measure of the likelihood of similarities between these failures and other flaky failures.



% % clustor instead of set 
% \begin{equation}
%     \label{ScoreEq}
%   \text{NU score} = \frac{\text{Total number of non-unique clusters of failures}}{\text{Total number of failures clusters}}
% \end{equation}




% \subsubsection{Initial Study Result}

% Table~\ref{tab:matchTable} presents a comprehensive overview of the study findings, containing three primary sections. The first column, labeled \emph{Total Flaky Tests and Failure}, provides fundamental statistical information derived from the collected failures including the number of studied flaky tests and the minimum, average, and maximum number of failures per test. 
% The second column, titled \emph{Same Test Different Builds}, focuses on the matching of each test failure with other observed failures of the same test, in different test suite runs. 
% I classify tests into six groups based on the number of flaky failures that I observed when running them 10,000 times.
% For instance, tests that had flaky failures only twice belong to the group labeled \textbf{[2,3)}.
% \sout{Then, for each of these groups, I provide the value \textbf{F}, which represents the total number of failure clusters from the tests in that group. Additionally, I calculate \textbf{NU} for each group of failure clusters.}
% Then, I computer the value \textbf{S} exclusively for the first group, and \textbf{F} and \textbf{NU} for the remaining groups. \textbf{S} pertains to the total count of failure clusters arising from tests that experienced only one failure each. On the other hand, the term \textbf{F} represents the overall count of failure clusters resulting from the tests in each group, regardless to the number of failures within each cluster. The value \textbf{NU} corresponds to the percentage of clusters that contain at least two failures relative to the total number of clusters in the group.
% The third column, titled as \emph{Different Tests Same Test Suite Run}, follows a similar computation method, but this time the grouping is based on the total number of failed tests in a single test suite run. For instance, if a test suite run encounters 3 failed tests, all the failures within that run are categorized into the group labeled \textbf{[3,10)}. As for the value of \textbf{F} in this column, it refers the clusters of failures within the grouped test suite runs.
% \jon{I don't think that this explanation of the table clearly explains what the intervals mean in the header}\abdul{How about now? I feel that the description I want to deliver is too long to fit within the table's caption.}
% \jon{This helps with the intervals. However: I think that the part that is still unclear is: are F and NU averages across all tests within that group? Or: are they totals across all of the tests?}\abdul{I rephrase it in order to be more clear. It is a total failure clusters from all tests within a group.}

% 
\begin{table*}
\centering
\caption{How often do flaky test failures match other flaky failures.}
% \textnormal{ 1) Total Flaky Tests and Failures shows the number of flaky tests, and the failure frequency of those flaky test.\\
% 2) Matching each test failure against other observed failures of the same test (in different test suite run). We bucket each test by the number of test suite run that it failed in, and show the number of tests that failed that many times (F), and the percentage of failures of those tests that matched at least one other failure of the same test (called Non-Unique (NU)). \textbf{S} refers to a failure that has one single occurrence because a test only fails once.\\
% 3) Matching each test failure against a DIFFERENT test failure in the same test suite run as that failure. We bucket each test suite run by the total number of flaky test failures observed in that build, showing the total number of flaky tests that failed in test suite runs of that bucket (F), and the percentage of failures that matched at least one other failure in the same test suite run (NU). \textbf{S} refers to a test suite run that has one single flaky failure.\\ 
% 4) We chose to opt out the projects that have one flaky test reported in studied dataset. }}
% \jon{Generated by MatchingRatesPerTest.ipynb}
\footnotesize
\setlength{\tabcolsep}{2pt}
\resizebox{\textwidth}{!}{%
\begin{tabular}{lrrrr|rrrrrrrrrrr|rrrrrrrrr}
\toprule
\multicolumn{1}{c}{ } & \multicolumn{4}{c}{ Total Flaky Tests and Failures} & \multicolumn{11}{c}{Same Test Different Builds} & \multicolumn{9}{c}{Different Tests Same Test Suite Run} \\
\cmidrule(l{3pt}r{3pt}){2-5} \cmidrule(l{3pt}r{3pt}){6-16} \cmidrule(l{3pt}r{3pt}){17-25}
\multicolumn{2}{c}{ } & \multicolumn{3}{c}{Failures Per-Test} & [1] & \multicolumn{2}{c}{[2,3)}& \multicolumn{2}{c}{[3,10)} & \multicolumn{2}{c}{[10,100)} & \multicolumn{2}{c}{[100,1000)} & \multicolumn{2}{c}{[1000,10000)} & [1] &\multicolumn{2}{c}{[2,3)} & \multicolumn{2}{c}{[3,10)} & \multicolumn{2}{c}{[10,20)} & \multicolumn{2}{c}{[20,200)} \\
% \cmidrule(l{3pt}r{3pt}){4-6} \cmidrule(l{3pt}r{3pt}){7-8}  \cmidrule(l{3pt}r{3pt}){9-10} \cmidrule(l{3pt}r{3pt}){11-12} \cmidrule(l{3pt}r{3pt}){13-14} \cmidrule(l{3pt}r{3pt}){15-16} \cmidrule(l{3pt}r{3pt}){17-18} \cmidrule(l{3pt}r{3pt}){19-20} \cmidrule(l{3pt}r{3pt}){21-22}
 & Tests & Min & Avg & Max & S &F & NU &F & NU & F & NU & F & NU & F & NU & S & F & NU & F & NU & F & NU & F & NU \\
\midrule
spring-projects-spring-boot&163&1&1753&5525&16& & &19&74\%&32&91\%& & &299&100\%&4441&36&0\%& & &13&46\%&49844&78\%\\
\cellcolor{gray!6}{apache-hbase}&\cellcolor{gray!6}{145}&\cellcolor{gray!6}{1}&\cellcolor{gray!6}{716}&\cellcolor{gray!6}{2011}&\cellcolor{gray!6}{2}&\cellcolor{gray!6}{21}&\cellcolor{gray!6}{90\%}&\cellcolor{gray!6}{16}&\cellcolor{gray!6}{100\%}&\cellcolor{gray!6}{33}&\cellcolor{gray!6}{88\%}&\cellcolor{gray!6}{37}&\cellcolor{gray!6}{70\%}&\cellcolor{gray!6}{109}&\cellcolor{gray!6}{92\%}&\cellcolor{gray!6}{1263}&\cellcolor{gray!6}{252}&\cellcolor{gray!6}{17\%}&\cellcolor{gray!6}{3624}&\cellcolor{gray!6}{5\%}&\cellcolor{gray!6}{39654}&\cellcolor{gray!6}{71\%}&\cellcolor{gray!6}{ }&\cellcolor{gray!6}{ }\\
Alluxio-alluxio&116&1&51&168&0& & & & &166&94\%&17&94\%& & &721&13&8\%&3&67\%&15&73\%&1168&88\%\\
\cellcolor{gray!6}{square-okhttp}&\cellcolor{gray!6}{100}&\cellcolor{gray!6}{1}&\cellcolor{gray!6}{234}&\cellcolor{gray!6}{8539}&\cellcolor{gray!6}{31}&\cellcolor{gray!6}{3}&\cellcolor{gray!6}{33\%}&\cellcolor{gray!6}{57}&\cellcolor{gray!6}{89\%}&\cellcolor{gray!6}{4}&\cellcolor{gray!6}{100\%}&\cellcolor{gray!6}{9}&\cellcolor{gray!6}{100\%}&\cellcolor{gray!6}{17}&\cellcolor{gray!6}{94\%}&\cellcolor{gray!6}{1152}&\cellcolor{gray!6}{5910}&\cellcolor{gray!6}{3\%}&\cellcolor{gray!6}{15835}&\cellcolor{gray!6}{25\%}&\cellcolor{gray!6}{39}&\cellcolor{gray!6}{51\%}&\cellcolor{gray!6}{ }&\cellcolor{gray!6}{ }\\
apache-ambari&52&1&77&875&1& & &2&100\%&48&100\%&2&100\%& & &921&16&0\%& & &202&64\%& & \\
\cellcolor{gray!6}{hector-client-hector}&\cellcolor{gray!6}{33}&\cellcolor{gray!6}{43}&\cellcolor{gray!6}{198}&\cellcolor{gray!6}{5147}&\cellcolor{gray!6}{0}&\cellcolor{gray!6}{ }&\cellcolor{gray!6}{ }&\cellcolor{gray!6}{ }&\cellcolor{gray!6}{ }&\cellcolor{gray!6}{32}&\cellcolor{gray!6}{100\%}&\cellcolor{gray!6}{ }&\cellcolor{gray!6}{ }&\cellcolor{gray!6}{1}&\cellcolor{gray!6}{100\%}&\cellcolor{gray!6}{5145}&\cellcolor{gray!6}{50}&\cellcolor{gray!6}{0\%}&\cellcolor{gray!6}{ }&\cellcolor{gray!6}{ }&\cellcolor{gray!6}{87}&\cellcolor{gray!6}{99\%}&\cellcolor{gray!6}{ }&\cellcolor{gray!6}{ }\\
activiti-activiti&32&1&42&932&12&2&100\%&4&100\%&14&93\%&1&100\%& & &1281&98&2\%&3&0\%& & & & \\
\cellcolor{gray!6}{tootallnate-java-websocket}&\cellcolor{gray!6}{23}&\cellcolor{gray!6}{1}&\cellcolor{gray!6}{48}&\cellcolor{gray!6}{215}&\cellcolor{gray!6}{1}&\cellcolor{gray!6}{ }&\cellcolor{gray!6}{ }&\cellcolor{gray!6}{2}&\cellcolor{gray!6}{50\%}&\cellcolor{gray!6}{29}&\cellcolor{gray!6}{100\%}&\cellcolor{gray!6}{13}&\cellcolor{gray!6}{100\%}&\cellcolor{gray!6}{ }&\cellcolor{gray!6}{ }&\cellcolor{gray!6}{1139}&\cellcolor{gray!6}{435}&\cellcolor{gray!6}{26\%}&\cellcolor{gray!6}{265}&\cellcolor{gray!6}{54\%}&\cellcolor{gray!6}{ }&\cellcolor{gray!6}{ }&\cellcolor{gray!6}{ }&\cellcolor{gray!6}{ }\\
wildfly-wildfly&23&1&5&41&13& & &9&100\%&1&100\%& & & & &47&8&0\%&28&50\%& & & & \\
\cellcolor{gray!6}{qos-ch-logback}&\cellcolor{gray!6}{22}&\cellcolor{gray!6}{1}&\cellcolor{gray!6}{185}&\cellcolor{gray!6}{3824}&\cellcolor{gray!6}{6}&\cellcolor{gray!6}{5}&\cellcolor{gray!6}{60\%}&\cellcolor{gray!6}{2}&\cellcolor{gray!6}{100\%}&\cellcolor{gray!6}{8}&\cellcolor{gray!6}{100\%}&\cellcolor{gray!6}{1}&\cellcolor{gray!6}{100\%}&\cellcolor{gray!6}{1}&\cellcolor{gray!6}{100\%}&\cellcolor{gray!6}{3952}&\cellcolor{gray!6}{308}&\cellcolor{gray!6}{0\%}&\cellcolor{gray!6}{6}&\cellcolor{gray!6}{0\%}&\cellcolor{gray!6}{ }&\cellcolor{gray!6}{ }&\cellcolor{gray!6}{ }&\cellcolor{gray!6}{ }\\
apache-httpcore&22&1&16&162&9&5&100\%&6&100\%& & &2&100\%& & &346&8&0\%& & & & & & \\
\cellcolor{gray!6}{apache-incubator-dubbo}&\cellcolor{gray!6}{19}&\cellcolor{gray!6}{1}&\cellcolor{gray!6}{462}&\cellcolor{gray!6}{8849}&\cellcolor{gray!6}{2}&\cellcolor{gray!6}{4}&\cellcolor{gray!6}{100\%}&\cellcolor{gray!6}{9}&\cellcolor{gray!6}{100\%}&\cellcolor{gray!6}{3}&\cellcolor{gray!6}{100\%}&\cellcolor{gray!6}{1}&\cellcolor{gray!6}{100\%}&\cellcolor{gray!6}{1}&\cellcolor{gray!6}{100\%}&\cellcolor{gray!6}{8563}&\cellcolor{gray!6}{654}&\cellcolor{gray!6}{0\%}&\cellcolor{gray!6}{9}&\cellcolor{gray!6}{22\%}&\cellcolor{gray!6}{ }&\cellcolor{gray!6}{ }&\cellcolor{gray!6}{ }&\cellcolor{gray!6}{ }\\
kevinsawicki-http-request&18&1&194&246&3& & & & & & &15&100\%& & &0& & &3&67\%&735&67\%& & \\
\cellcolor{gray!6}{wro4j-wro4j}&\cellcolor{gray!6}{16}&\cellcolor{gray!6}{1}&\cellcolor{gray!6}{474}&\cellcolor{gray!6}{1803}&\cellcolor{gray!6}{1}&\cellcolor{gray!6}{ }&\cellcolor{gray!6}{ }&\cellcolor{gray!6}{2}&\cellcolor{gray!6}{100\%}&\cellcolor{gray!6}{7}&\cellcolor{gray!6}{71\%}&\cellcolor{gray!6}{1}&\cellcolor{gray!6}{100\%}&\cellcolor{gray!6}{12}&\cellcolor{gray!6}{100\%}&\cellcolor{gray!6}{382}&\cellcolor{gray!6}{227}&\cellcolor{gray!6}{0\%}&\cellcolor{gray!6}{6285}&\cellcolor{gray!6}{44\%}&\cellcolor{gray!6}{44}&\cellcolor{gray!6}{27\%}&\cellcolor{gray!6}{ }&\cellcolor{gray!6}{ }\\
undertow-io-undertow&7&1&8&54&0&2&0\%&7&86\%&3&100\%& & & & &92& & & & & & & & \\
\cellcolor{gray!6}{orbit-orbit}&\cellcolor{gray!6}{7}&\cellcolor{gray!6}{7}&\cellcolor{gray!6}{420}&\cellcolor{gray!6}{2546}&\cellcolor{gray!6}{0}&\cellcolor{gray!6}{ }&\cellcolor{gray!6}{ }&\cellcolor{gray!6}{1}&\cellcolor{gray!6}{100\%}&\cellcolor{gray!6}{3}&\cellcolor{gray!6}{100\%}&\cellcolor{gray!6}{2}&\cellcolor{gray!6}{100\%}&\cellcolor{gray!6}{1}&\cellcolor{gray!6}{100\%}&\cellcolor{gray!6}{2721}&\cellcolor{gray!6}{214}&\cellcolor{gray!6}{1\%}&\cellcolor{gray!6}{5}&\cellcolor{gray!6}{20\%}&\cellcolor{gray!6}{ }&\cellcolor{gray!6}{ }&\cellcolor{gray!6}{ }&\cellcolor{gray!6}{ }\\
doanduyhai-Achilles&4&1&26&60&1& & & & &3&100\%& & & & &103&2&0\%& & & & & & \\
\cellcolor{gray!6}{elasticjob-elastic-job-lite}&\cellcolor{gray!6}{3}&\cellcolor{gray!6}{1}&\cellcolor{gray!6}{2}&\cellcolor{gray!6}{4}&\cellcolor{gray!6}{2}&\cellcolor{gray!6}{ }&\cellcolor{gray!6}{ }&\cellcolor{gray!6}{2}&\cellcolor{gray!6}{50\%}&\cellcolor{gray!6}{ }&\cellcolor{gray!6}{ }&\cellcolor{gray!6}{ }&\cellcolor{gray!6}{ }&\cellcolor{gray!6}{ }&\cellcolor{gray!6}{ }&\cellcolor{gray!6}{5}&\cellcolor{gray!6}{1}&\cellcolor{gray!6}{100\%}&\cellcolor{gray!6}{ }&\cellcolor{gray!6}{ }&\cellcolor{gray!6}{ }&\cellcolor{gray!6}{ }&\cellcolor{gray!6}{ }&\cellcolor{gray!6}{ }\\
alibaba-fastjson&3&4&49&121&0& & &2&100\%& & &2&100\%& & &191&4&0\%& & & & & & \\
\cellcolor{gray!6}{zxing-zxing}&\cellcolor{gray!6}{2}&\cellcolor{gray!6}{322}&\cellcolor{gray!6}{352}&\cellcolor{gray!6}{382}&\cellcolor{gray!6}{0}&\cellcolor{gray!6}{ }&\cellcolor{gray!6}{ }&\cellcolor{gray!6}{ }&\cellcolor{gray!6}{ }&\cellcolor{gray!6}{ }&\cellcolor{gray!6}{ }&\cellcolor{gray!6}{2}&\cellcolor{gray!6}{100\%}&\cellcolor{gray!6}{ }&\cellcolor{gray!6}{ }&\cellcolor{gray!6}{694}&\cellcolor{gray!6}{10}&\cellcolor{gray!6}{0\%}&\cellcolor{gray!6}{ }&\cellcolor{gray!6}{ }&\cellcolor{gray!6}{ }&\cellcolor{gray!6}{ }&\cellcolor{gray!6}{ }&\cellcolor{gray!6}{ }\\
\bottomrule
\label{tab:matchTable}
\end{tabular}
}
\end{table*}




% % RQ1 
% \textbf{RQ: Do flaky failures match for the same test across different executions of the test suite?} In Table~\ref{tab:matchTable}, the result per each project has been computed. For instance, the first row is a summary of the project \spring. Among the 10,000 trials, there are 163 flaky tests that have failed an average of 1753 times. 
% These flaky tests produced 366 different types \abdul{not showing in the table}of failures are categorized as follow: 16 failures from tests that failed only once, 19 from tests that failed between 3 to 10 times (\textbf{[3-10)}), 32 from tests that failed between 10 to 100 times (\textbf{[10-100)}), and 299 from tests that failed more than 1,000 times.
% Out the 19 failures in the group \textbf{[3-10)}, 74\% of them appear at least twice and the remaining only occurs once, called \emph{unique}. 
% In terms of the \emph{Different Tests Same Test Suite Run} column in the first row, I found 4,441 test suite runs that include only one failed test. Additionally, there are 36 failure clusters from test suite runs with two failed tests, and none of these clusters have more than one failure (0\%).
% % \jon{Given the complexity of the table, I would suggest quickly providing one or two example findings in prose. Example:
% % For example, in the first row, I can see that there were 163 flaky tests, failing an average of 1,753 times (out of the 10,000 trials).
% % Of those failures, there are 16 unique failure clusters, 19 failure clusters that match 3-10...}\abdul{How about now?}
% % \jon{this is better, thanks.}


% % Generally, there is an increase in non-unique failures across multiple projects, by finding many projects with 100\% scores.
% In numerous projects, a consistent primary observation is that the non-unique proportion (\textbf{NU}) reaches 100\% in the \emph{Same Test Different Builds} results, implying that flaky failures frequently occur. This finding supports the idea that developers often depend on their past experiences with flaky failures to help them identifying these failures flakiness.
% \jon{I don't understand this sentence. Is the meaning: results for NU across projects are quite similar?}
% \abdul{I meant that the concept of NU is not detected only in one project indicating that this could be usefull approach regardless to the project domain.}
% \jon{Here is some proposed new text:
% A key finding is that, for many projects, the non-unique proportion (NU) is 100\% in the ``Same Test Different Builds'' configuration, implying that flaky failures often repeat.
% This is important because...}\abdul{How about now}
% Another notable finding is that tests experiencing frequent flakiness do not always exhibit similarities among their failures consistently. For example, approximately 30\% of the flaky failures that occur more than 100 times in \emph{hbase} are classified as unique failure clusters. On the other hand, within the same project, there are no unique clusters among failures that exhibit flakiness less than 10 times. This prompts me to conduct a comprehensive analysis of the factors contributing to the occurrence of unique failure clusters.

% According to the study findings, there are a total of 101 failure clusters that are classified as unique across all projects. These unique failures contribute to a decrease in UN scores. Among these clusters, 82 of them are considered unique because they differ either in the exception type, the line from the stack trace that begins with the test name, or both. These cases indicate entirely distinct failures, as each one is associated with a different line of test code. In comparison to the non-unique failures, I am investigating whether specific factors, such as test complexity or the type of exception, could be connected to the presence of unique failure clusters. Alternatively, it is possible that uniqueness occurs independently and may be related to the underlying causes of flakiness in general.

% I have examined three factors, namely test length, the number of lines in the stack traces, and the total number of assertion statements, to assess the complexity of each flaky test. Using these factors, I conducted an analysis of both unique and non-unique failures to investigate potential connections with specific clusters.
% For instance, in the \emph{java-websocket} project, the only unique failure cluster exhibited a significantly larger number of lines in the test body compared to all tests associated with non-unique failure clusters (79 lines versus a maximum of 3 lines). In the \emph{spring-boot} project, it was observed that all unique failure clusters had a median of 73 stack trace lines, while considering both unique and non-unique failure clusters resulted in a median of 4 stack trace lines. Additionally, in the \emph{undertwo} project, there were two distinct tests with unique failure clusters, and these tests shared the highest number of assertion statements when compared to other tests.
% \sout{Overall, there is no single factor among these that consistently correlates with unique failure clusters across all projects. However, the impact of these factors varies from one project to another. This analysis provides valuable insights into the relationship between test complexity factors and the occurrence of unique failure clusters.}
% In general, no single factor consistently correlates with unique failure clusters across all projects, as the impact of these factors varies from one project to another. It is crucial to avoid relying on a single factor to determine the uniqueness of a failure. Instead, it is essential to examine multiple factors when necessary. By doing so, a more comprehensive and accurate assessment of failure uniqueness can be achieved, taking into account the specific characteristics of each individual project.
% \jon{The valuable insights are not clear to me - it seemed more like there was no clear pattern/finding?}\abdul{I may not explain it well, I try to see if there is a specific pattern between having unique failure with some discussed factors.}
% \jon{Can you add a sentence or two explaining why it is a valuable insight to see that there is no pattern between these factors? The implications of this are not otherwise motivated in the preceeding text.}\abdul{How clear is it now?}


% Regarding exceptions, I discovered that the \emph{java.lang.IllegalArgumentException} is detecting in 5 unique clusters in two different projects (\emph{Alluxio} and \emph{hbase}, while in \emph{hbase}, it is associated with also a non-unique cluster in only one case. Interestingly, all these tests with unique clusters also have non-failure clusters linked to the \emph{UnknownHostException}. This suggests that considering the correlation among failures per test can help establish connections between the project domain and failures. It is possible that some exceptions occur due to infrequent causes of flakiness. For instance, in the \emph{wro4j} project, there are only two unique failure exceptions, originating from separate tests, but sharing the same type of failure exception (\emph{java.net.SocketTimeoutException}). Upon conducting a comprehensive analysis, I noticed that these two unique failures occurred consecutively during test suite runs and exhibited similar stack trace lines, except for the lines specific to the respective tests involved. 

% To enhance the analysis of exceptions in relation to their occurrence in unique and non-unique failures, I gathered data on the various exception types present in all unique failures. For each exception type, I then determined the frequency of its appearance in non-unique failures, as indicated in Table~\ref{table:uniqueExceptions}. This investigation aimed to identify exceptions that might be closely associated with causing unique failures.
% From a total of 72 unique clusters, I identified 16 different exceptions. Among these exceptions, only two, namely \emph{java.lang.IllegalArgumentException} and \emph{java.net.SocketTimeoutException}, were more observed in unique clusters than in non-unique ones. However, it is also occurred once in non-unique failures even for these two exceptions. 
% Despite examining these exceptions, it remains challenging to establish a direct link between the uniqueness of a failure and the specific type of exception. Furthermore, I observed that even in exceptions that are specific to a particular project, like \emph{org.apache.hadoop.hbase.client.NoServerForRegionException}, cannot be definitively linked to the cause of uniqueness.  
% \jon{What about including the analysis of exception types in this proposal, or, describing it as future work?}\abdul{I have not discussed it here. I discuss only the exception type when I compare flaky with non flaky? If you think it is good to be discussed here, I can provide 1-2 paragraphs here. }
% \jon{I think that it is interesting and relevant to discuss here, too.} \abdul{I added this paragraph to briefly discuss the relationship between the uniqueness and exception type. Any further comments here?}

% 


% This is similar to the Table~\ref{nonunique}, but for the failures that flakes once (which have no match with other flaky failures). For the columns \textit{Failures Flake =1}: it shows the number of failures that flakes once (\textbf{f}), follow by the total number of tests where these failures belong (\textbf{t}), followed by the number of tests where the whole test only flakes once (\textbf{t_{1}})

\begin{table*}[t]
\caption[Top 10 Most Occurrence Exception in Flaky and True Failures]{Top 10 Most Occurrence Exception in Flaky and True Failures \\ 
\textnormal{ The \textit{Exception Occurrence} column details the frequency of a specific exception, indicating in how many projects, tests, and failures this exception has been observed. The \textit{Match Result (with Stacktraces)} column displays the match distributions, considering stacktraces and the related test count while the, \textit{Match Result (without Stacktraces)} column indicates match results based on exception types, excluding stacktraces.}}

\vspace{-5pt}
\setlength{\tabcolsep}{2.5pt}
\newcommand{\failureRateWidth}{2.5in}
\newcommand{\failureRateHeight}{4em}
\scriptsize
\centering

% -- > Version 2: With Tests .. 

    \begin{tabular}{l|rrr|rr|rrrr|rrrr}
    \toprule
      & \multicolumn{5}{c|}{\textbf{Exception Occurrence}} & \multicolumn{4}{c|}{\textbf{Match Result by Failures}} & \multicolumn{4}{c}{\textbf{Match Result by Failures}} \\ 

      & \multicolumn{5}{c|}{\textbf{}} & \multicolumn{4}{c|}{\textbf{(with Stacktraces)}}  &\multicolumn{4}{c}{\textbf{(without Stacktraces)}} \\ 
     
     \textbf{Exception Name}&\textbf{Projects}&\textbf{Tests}&\textbf{Failures}&\textbf{True}&\textbf{Flaky}&\textbf{TP}& \textbf{FN}&\textbf{FP}& \textbf{TN}&\textbf{TP}& \textbf{FN}&\textbf{FP}& \textbf{TN}\\
        \midrule
AssertionError&21&407&51,453&20,507&30,946&6,120&24,826&4,850&15,657&64&30,550&13,968&6,539\\

\cellcolor{gray!6}{NullPointerException}&\cellcolor{gray!6}{22}&\cellcolor{gray!6}{498}&\cellcolor{gray!6}{49,906}&\cellcolor{gray!6}{41,709}&\cellcolor{gray!6}{8,197}&\cellcolor{gray!6}{1,644}&\cellcolor{gray!6}{6,553}&\cellcolor{gray!6}{449}&\cellcolor{gray!6}{41,260}&\cellcolor{gray!6}{34}&\cellcolor{gray!6}{8,163}&\cellcolor{gray!6}{7,913}&\cellcolor{gray!6}{33,796}\\
IOException&7&257&20,097&15,963&4,134&3,614&520&519&15,444&28&3,141&3,717&12,246\\
\cellcolor{gray!6}{RuntimeException}&\cellcolor{gray!6}{17}&\cellcolor{gray!6}{420}&\cellcolor{gray!6}{13,810}&\cellcolor{gray!6}{13,676}&\cellcolor{gray!6}{134}&\cellcolor{gray!6}{43}&\cellcolor{gray!6}{91}&\cellcolor{gray!6}{1,011}&\cellcolor{gray!6}{12,665}&\cellcolor{gray!6}{31}&\cellcolor{gray!6}{103}&\cellcolor{gray!6}{1,141}&\cellcolor{gray!6}{12,535}\\
NoServerForRegionException&1&35&11,686&169&11,517&11,512&5&0&169&1,921&9,596&75&94\\
\cellcolor{gray!6}{UnknownHostException}&\cellcolor{gray!6}{9}&\cellcolor{gray!6}{234}&\cellcolor{gray!6}{9,942}&\cellcolor{gray!6}{319}&\cellcolor{gray!6}{9,623}&\cellcolor{gray!6}{9,620}&\cellcolor{gray!6}{3}&\cellcolor{gray!6}{0}&\cellcolor{gray!6}{319}&\cellcolor{gray!6}{9,620}&\cellcolor{gray!6}{3}&\cellcolor{gray!6}{0}&\cellcolor{gray!6}{319}\\
ActivitiException&1&30&9,893&9,821&72&0&72&614&9,207&0&72&3,094&6,727\\
\cellcolor{gray!6}{IllegalArgumentException}&\cellcolor{gray!6}{17}&\cellcolor{gray!6}{401}&\cellcolor{gray!6}{9,052}&\cellcolor{gray!6}{9,049}&\cellcolor{gray!6}{3}&\cellcolor{gray!6}{0}&\cellcolor{gray!6}{3}&\cellcolor{gray!6}{190}&\cellcolor{gray!6}{8,859}&\cellcolor{gray!6}{0}&\cellcolor{gray!6}{3}&\cellcolor{gray!6}{212}&\cellcolor{gray!6}{8,837}\\
AssertionFailedError&7&98&8,832&7,054&1,778&66&1,712&1,648&5,406&66&1,712&4,150&2,904\\
\cellcolor{gray!6}{PersistenceException}&\cellcolor{gray!6}{2}&\cellcolor{gray!6}{30}&\cellcolor{gray!6}{8,581}&\cellcolor{gray!6}{8,580}&\cellcolor{gray!6}{1}&\cellcolor{gray!6}{0}&\cellcolor{gray!6}{1}&\cellcolor{gray!6}{164}&\cellcolor{gray!6}{8,416}&\cellcolor{gray!6}{0}&\cellcolor{gray!6}{1}&\cellcolor{gray!6}{398}&\cellcolor{gray!6}{8,182}\\



%     \begin{tabular}{l|rrr|rr|rrrr|rrrr|rrrr|rrrr}
%     \toprule
%       & \multicolumn{5}{c|}{\textbf{Exception Occurrence}} & \multicolumn{8}{c|}{\textbf{Match Result (with Stacktraces)}} & \multicolumn{8}{c}{\textbf{Match Result (without Stacktraces)}} \\ 

%       & \multicolumn{5}{c|}{\textbf{}} & \multicolumn{4}{c|}{\textbf{By Failures}} & \multicolumn{4}{c|}{\textbf{By Tests}} & \multicolumn{4}{c}{\textbf{By Tests}} & \multicolumn{4}{c}{\textbf{By Failures}} \\ 
     
%      \textbf{Exception Name}&\textbf{Projects}&\textbf{Tests}&\textbf{Failures}&\textbf{True}&\textbf{Flaky}&\textbf{TP}& \textbf{FN}&\textbf{FP}& \textbf{TN}&\textbf{TP}& \textbf{FN}&\textbf{FP}& \textbf{TN}&\textbf{TP}& \textbf{FN}&\textbf{FP}& \textbf{TN}&\textbf{TP}& \textbf{FN}&\textbf{FP}& \textbf{TN}\\
%         \midrule
% AssertionError&21&407&51,453&20,507&30,946&6,120&24,826&4,850&15,657&64&122&96&371&396&30,550&13,968&6,539&5&176&173&226\\
% \cellcolor{gray!6}{NullPointerException}&\cellcolor{gray!6}{22}&\cellcolor{gray!6}{498}&\cellcolor{gray!6}{49,906}&\cellcolor{gray!6}{41,709}&\cellcolor{gray!6}{8,197}&\cellcolor{gray!6}{1,644}&\cellcolor{gray!6}{6,553}&\cellcolor{gray!6}{449}&\cellcolor{gray!6}{41,260}&\cellcolor{gray!6}{32}&\cellcolor{gray!6}{100}&\cellcolor{gray!6}{99}&\cellcolor{gray!6}{489}&\cellcolor{gray!6}{34}&\cellcolor{gray!6}{8,163}&\cellcolor{gray!6}{7,913}&\cellcolor{gray!6}{33,796}&\cellcolor{gray!6}{9}&\cellcolor{gray!6}{120}&\cellcolor{gray!6}{120}&\cellcolor{gray!6}{369}\\
% IOException&7&257&20,097&15,963&4,134&3,614&520&519&15,444&28&19&13&234&993&3,141&3,717&12,246&19&27&22&212\\
% \cellcolor{gray!6}{RuntimeException}&\cellcolor{gray!6}{17}&\cellcolor{gray!6}{420}&\cellcolor{gray!6}{13,810}&\cellcolor{gray!6}{13,676}&\cellcolor{gray!6}{134}&\cellcolor{gray!6}{43}&\cellcolor{gray!6}{91}&\cellcolor{gray!6}{1,011}&\cellcolor{gray!6}{12,665}&\cellcolor{gray!6}{7}&\cellcolor{gray!6}{14}&\cellcolor{gray!6}{2}&\cellcolor{gray!6}{409}&\cellcolor{gray!6}{31}&\cellcolor{gray!6}{103}&\cellcolor{gray!6}{1,141}&\cellcolor{gray!6}{12,535}&\cellcolor{gray!6}{5}&\cellcolor{gray!6}{16}&\cellcolor{gray!6}{12}&\cellcolor{gray!6}{399}\\
% NoServerForRegionException&1&35&11,686&169&11,517&11,512&5&0&169&8&4&0&33&1,921&9,596&75&94&2&9&9&24\\
% \cellcolor{gray!6}{UnknownHostException}&\cellcolor{gray!6}{9}&\cellcolor{gray!6}{234}&\cellcolor{gray!6}{9,942}&\cellcolor{gray!6}{319}&\cellcolor{gray!6}{9,623}&\cellcolor{gray!6}{9,620}&\cellcolor{gray!6}{3}&\cellcolor{gray!6}{0}&\cellcolor{gray!6}{319}&\cellcolor{gray!6}{133}&\cellcolor{gray!6}{3}&\cellcolor{gray!6}{0}&\cellcolor{gray!6}{98}&\cellcolor{gray!6}{9,620}&\cellcolor{gray!6}{3}&\cellcolor{gray!6}{0}&\cellcolor{gray!6}{319}&\cellcolor{gray!6}{133}&\cellcolor{gray!6}{3}&\cellcolor{gray!6}{0}&\cellcolor{gray!6}{98}\\
% ActivitiException&1&30&9,893&9,821&72&0&72&614&9,207&0&9&7&29&0&72&3,094&6,727&0&9&8&21\\
% \cellcolor{gray!6}{IllegalArgumentException}&\cellcolor{gray!6}{17}&\cellcolor{gray!6}{401}&\cellcolor{gray!6}{9,052}&\cellcolor{gray!6}{9,049}&\cellcolor{gray!6}{3}&\cellcolor{gray!6}{0}&\cellcolor{gray!6}{3}&\cellcolor{gray!6}{190}&\cellcolor{gray!6}{8,859}&\cellcolor{gray!6}{0}&\cellcolor{gray!6}{3}&\cellcolor{gray!6}{3}&\cellcolor{gray!6}{401}&\cellcolor{gray!6}{0}&\cellcolor{gray!6}{3}&\cellcolor{gray!6}{212}&\cellcolor{gray!6}{8,837}&\cellcolor{gray!6}{0}&\cellcolor{gray!6}{3}&\cellcolor{gray!6}{3}&\cellcolor{gray!6}{398}\\
% AssertionFailedError&7&98&8,832&7,054&1,778&66&1,712&1,648&5,406&1&21&21&94&66&1,712&4,150&2,904&1&21&21&76\\
% \cellcolor{gray!6}{PersistenceException}&\cellcolor{gray!6}{2}&\cellcolor{gray!6}{30}&\cellcolor{gray!6}{8,581}&\cellcolor{gray!6}{8,580}&\cellcolor{gray!6}{1}&\cellcolor{gray!6}{0}&\cellcolor{gray!6}{1}&\cellcolor{gray!6}{164}&\cellcolor{gray!6}{8,416}&\cellcolor{gray!6}{0}&\cellcolor{gray!6}{1}&\cellcolor{gray!6}{1}&\cellcolor{gray!6}{30}&\cellcolor{gray!6}{0}&\cellcolor{gray!6}{1}&\cellcolor{gray!6}{398}&\cellcolor{gray!6}{8,182}&\cellcolor{gray!6}{0}&\cellcolor{gray!6}{1}&\cellcolor{gray!6}{1}&\cellcolor{gray!6}{29}\\

\bottomrule


\end{tabular}
\label{table:exceptions}
\vspace{-10pt}
\end{table*}




% \begin{table}[t]
%   \setlength{\tabcolsep}{2.0pt}

% % \jon{Resized table to try to fit bigger column headers. Not clear what unique means. Unique by stack trace?}
% \caption{List of top 10 Exceptions in Flaky Failures. \\
% \textnormal{\emph{Total Failure} indicates the count of each exception's occurrences and the total projects where these exceptions appear. \emph{With S} and \emph{Without S} represent the presence or absence of stacktrace lines during the matching using the \syntax, respectively. \emph{TP} refers to failures not matching any non-flaky failures, while \emph{FN} refers flaky failures that match with at least one non-flaky failure.}}
% \label{table:exceptions}
% \vspace{-4pt}
% %\resizebox{\textwidth}{!}{
% % \scriptsize
% \footnotesize
% \begin{tabular}{l|rr|rr|rr}

% \toprule
%       & \multicolumn{2}{c|}{\textbf{Total Failures}} & \multicolumn{2}{c|}{\textbf{With S}} & \multicolumn{2}{c}{\textbf{Without S}}\\
      

% \textbf{Exceptions} & \textbf{F} & \textbf{P}  & \textbf{TP} & \textbf{FN}  & \textbf{TP} & \textbf{FN} \\
% \midrule
% % AssertionError&186&15&88&98&8&178\\
% % \cellcolor{gray!6}{NullPointerException}&\cellcolor{gray!6}{185}&\cellcolor{gray!6}{5}&\cellcolor{gray!6}{39}&\cellcolor{gray!6}{146}&\cellcolor{gray!6}{9}&\cellcolor{gray!6}{176}\\
% % UnknownHostException&136&6&136&0&136&0\\
% % \cellcolor{gray!6}{IOException}&\cellcolor{gray!6}{70}&\cellcolor{gray!6}{4}&\cellcolor{gray!6}{57}&\cellcolor{gray!6}{13}&\cellcolor{gray!6}{36}&\cellcolor{gray!6}{34}\\
% % ProvisionException&49&1&49&0&0&49\\
% % \cellcolor{gray!6}{HCassandraInternalException}&\cellcolor{gray!6}{31}&\cellcolor{gray!6}{1}&\cellcolor{gray!6}{31}&\cellcolor{gray!6}{0}&\cellcolor{gray!6}{31}&\cellcolor{gray!6}{0}\\
% % SocketException&31&1&29&2&1&30\\
% % \cellcolor{gray!6}{Exception}&\cellcolor{gray!6}{30}&\cellcolor{gray!6}{5}&\cellcolor{gray!6}{30}&\cellcolor{gray!6}{0}&\cellcolor{gray!6}{28}&\cellcolor{gray!6}{2}\\
% % RuntimeException&23&3&21&2&9&14\\
% % \cellcolor{gray!6}{AssertionFailedError}&\cellcolor{gray!6}{22}&\cellcolor{gray!6}{5}&\cellcolor{gray!6}{1}&\cellcolor{gray!6}{21}&\cellcolor{gray!6}{1}&\cellcolor{gray!6}{21}\\


% AssertionError&30946&19&6149&24797&399&30547\\
% \cellcolor{gray!6}{NoServerForRegionException}&\cellcolor{gray!6}{11517}&\cellcolor{gray!6}{1}&\cellcolor{gray!6}{11517}&\cellcolor{gray!6}{0}&\cellcolor{gray!6}{1921}&\cellcolor{gray!6}{9596}\\
% UnknownHostException&9623&6&9623&0&9623&0\\
% \cellcolor{gray!6}{NoSuchMethodError}&\cellcolor{gray!6}{8539}&\cellcolor{gray!6}{1}&\cellcolor{gray!6}{8539}&\cellcolor{gray!6}{0}&\cellcolor{gray!6}{8539}&\cellcolor{gray!6}{0}\\
% NullPointerException&8197&5&1647&6550&34&8163\\
% \cellcolor{gray!6}{WroRuntimeException}&\cellcolor{gray!6}{6487}&\cellcolor{gray!6}{1}&\cellcolor{gray!6}{0}&\cellcolor{gray!6}{6487}&\cellcolor{gray!6}{0}&\cellcolor{gray!6}{6487}\\
% SocketException&4547&1&4492&55&3&4544\\
% \cellcolor{gray!6}{IOException}&\cellcolor{gray!6}{4134}&\cellcolor{gray!6}{4}&\cellcolor{gray!6}{3620}&\cellcolor{gray!6}{514}&\cellcolor{gray!6}{998}&\cellcolor{gray!6}{3136}\\
% ExecutionException&3465&1&0&3465&0&3465\\
% \cellcolor{gray!6}{ProvisionException}&\cellcolor{gray!6}{3055}&\cellcolor{gray!6}{1}&\cellcolor{gray!6}{3055}&\cellcolor{gray!6}{0}&\cellcolor{gray!6}{0}&\cellcolor{gray!6}{3055}\\
% \bottomrule
% \end{tabular}
% \vspace{-10pt}
% \end{table}

 
% In general, it is common for newly detected flaky failures to have matches with previously known detected failures. This practice helps in identifying recurring patterns and establishing a connection between similar failures. However, there are cases where a failure does not have a previous match. Cases where flaky failures have no common cause or previous match require additional investigation to understand their underlying causes. The study findings suggest that these unique cases do not have a shared or identifiable cause among them.

% % One of the observed factors is when the failed test is complex e.g. contains multiple assertion statements, which is subject to have many failures with different stack trace lines. 
% % For example, a flaky test in the "okhttp" project demonstrates this behavior by having three different assertion statements. \jon{I'm not sure what this is trying to say - that the reason why it doesn't match is because there are many different assertions that could fail?} \abdul{Yes, this is what I want to say, but the example sounds weak. I am thinking to study the correlation between the test size ( Test Lines of Code numbers from FlakeFlagger dataset)}
% % \sout{I have not identified any particular projects that exhibit a higher occurrence of non-unique failures compared to others.}  \jon{Is apache-hbase different than the others? Or wro4j? The numbers seem the most different for these projects compard to the others, and maybe I should mention this along with a 1-2 sentence explanation for why that might be.} \abdul{I am working on finding why apache-hbase has many unique failure clusters.}
% % \sout{This leads us to the conclusion that non-unique failures are not specifically tied to the nature of the project itself. However, I have observed that there are four projects with over 10 tests that exhibit flakiness, occurring once. In these projects, a significant portion of the flaky tests, even those that exhibit flakiness multiple times, have unique failures.}
% % I noticed that most of the flaky tests in Spring-boot failed more than half of the total number of runs and none of these failures have been reported as a unique failure. With a close look, I have founds that around third-quarter of the failures comes from the \emph{parameterized} tests. 






% % DQ2
% \textbf{RQ: Do flaky failures match for different tests within the same execution of a test suite?} The main observation from the \emph{Different Tests Same Test Suite Run} result that when a test suite has more failed tests, it is more likely to have similar failures. This could be because the root cause of the failures affect many tests to be failed with the same failures. For example, all flaky failures in Alluxio projects that are under the category [20,200) come always from 116 tests (the total number of flaky tests in the project). That means the same root cause that forces all 116 tests to fail together and \emph{all} failures due to the \emph{UnknownHostException}, similar to the example shown in listing~\ref{lst:flakyFailures}. 
% The reason behind the uniqueness of the remaining failures within this category lies in the fact that each failure is specific to its corresponding test class.
% Similarly to the project spring-boot, I have found around third quarters of failures that are not unique and most of the failures shares similar exceptions. 
% In some instances, there are test suites with a small number of failed tests where multiple failures exhibit similar logs. 
% For instance, approximately one quarter of the test suites in the "java-websocket" project that have two failed tests fall into this scenario. 
% In each of these test suites, the two failures share an identical stack trace. Furthermore, it is interesting to note that neither of the test names is included in these stack traces, suggesting that the failures stem from a common underlying cause.




% \subsection{Flaky Failure Logs Based Approaches}
% \label{sec:failureLogsApproach}


% Chapter \ref{sec:failureLogsStudy} explored the accuracy of identifying new flaky failures as similar to previous confirmed flaky failures, both within and across different tests in the same project. Assuming that developers label a failure as flaky due to specific signals present in the stack trace, I hypothesize that these signals are present in flaky failure logs and absent in non-flaky ones. Otherwise, this could lead to misleading results. To clarify, if two failure logs (one flaky and one non-flaky) are identical, this could indicate that one of them was misclassified, or that the failure exception and stack trace lines are not sufficient for distinguishing between the two types of failures. Through this approach, my goal is to assess the likelihood of detecting the signals that differentiate flaky failures from non-flaky ones based on the failure exception and stack trace lines.

% To conduct this experiment, it is necessary to have access to logs of both flaky and non-flaky failures for the same tests. However, in datasets such as Deflaker \cite{bell2018deflaker} and iDFlakies \cite{lam2019idflakies}, there are no accompanying logs of non-flaky failures for the same flaky tests. Additionally, I am not aware of any available datasets that provide both types of failure logs for the same set of tests. In the previous section, I utilized my FlakeFlagger dataset, and one possible solution for obtaining non-flaky failures is to examine defects in the projects from this dataset. However, this approach may not be practical due to uncertainty regarding the number of collected failures and their non-flakiness status. Given the difficulty in obtaining all deterministic failures for a given test (as it is hard to anticipate all developer mistakes), a reasonable amount of deterministic failures per flaky test would suffice.

% It is possible to obtain alternative sources of non-flaky failures by utilizing mutation testing to gather data on the failures of killed mutants. Just et al. have explored the idea of replacing real test failures with the failures of killed mutants \cite{just2014mutants}. The use of killed mutant failures can increase the likelihood of non-flaky failures, although it should be noted that not every killed mutant failure is necessarily non-flaky, as recent studies have shown that mutants can also exhibit flakiness \cite{shi2019mitigating}. To mitigate this issue, the approach of
% Shi et al. has been used to filter out flaky mutants~\cite{shi2019mitigating}. I began by gathering mutants for each flaky test from FlakeFlagger dataset. For each test, mutants have been collected and executed 20 times in order to identify any potential flakiness. The failure messages and stack traces lines were recorded in a similar manner to the process outlined in Section~\ref{sec:failureLogsStudy}. I then updated the XML result file per test to include a list of mutant blocks. Each mutant block contains the failure exception, message, and stacktrace lines. 

% To compare flaky failures with non-flaky ones for each test, a \syntax will be employed, similar to the approach used in chapter \ref{sec:failureLogsStudy} when comparing two flaky failures. This approach aims to capture any differences, such as in the stack trace line number, as different lines of code being executed can result in different stack traces. Along with the \syntax, I explore the possibility of using a machine learning approach to develop a classifier that can learn from flaky failure logs and predict the status of others. While the syntax approach works better for comparing failures at the test level (within the test), the generality of the machine learning approach motivated me to apply it as well.
% % \jon{Before going into the two approaches, I suggest a 1-paragraph overview describing the different kinds of approaches that could be considered and their relative merits} \abdul{how does this sound?}\jon{Great!}

% \begin{table*}[t]
    \caption{A list of features used to train the \classifier}
\label{table:Features}
\vspace{-5pt}
% \setlength{\tabcolsep}{2.5pt}
\newcommand{\failureRateWidth}{2.5in}
\newcommand{\failureRateHeight}{4em}
\scriptsize
\centering
    \begin{tabular}{l|c|l}
    \toprule     
     \textbf{Feature Name}&\textbf{Type}&\textbf{Description}\\
        \midrule
        Exception Type & Str & The name of the exception e.g. UnknownHostException \\
        Test name in Stacktrace & Boolean & \textit{True} if one of Stacktrace lines starts with the test name else \textit{False} \\
        Test Class name in Stacktrace & Boolean & \textit{True} if one of Stacktrace lines contains the test class name else \textit{False} \\
        Other Tests in Stacktrace & Boolean & \textit{True} if one of Stacktrace lines starts with other tests names else \textit{False} \\
        JUnit in Stacktrace & Boolean & \textit{True} if one of Stacktrace lines starts with any Junit Lines else \textit{False} \\
        CUT in Stacktrace & Boolean & \textit{True} if one of Stacktrace lines contains any lines from Code Under Test else \textit{False} \\
\bottomrule 
\end{tabular}
\vspace{-10pt}
\end{table*}






% \subsubsection{Syntax Based Approach}

% I utilize a syntax-matching method to compare the exception and stack trace lines of two failures. The approach is simply detecting any differences (including line numbers) between two given failures logs. I will be relying on the exception type and the stack trace lines, similar to the discussed technique in Section \ref{sec:failureLogsStudy} of the study. The first step involves clustering failures per test based on their exception and stack trace lines. Each cluster will then be labeled as either flaky (consisting of failures that are exclusively flaky), non-flaky (clusters containing only non-flaky failures), or a combination of both (clusters that include both flaky and non-flaky failures). 

% The presence of flaky clusters indicates that the failures within those clusters do not match with any non-flaky failures. When a project has a higher number of flaky failure clusters, this approach becomes valuable in distinguishing between flaky and non-flaky failures.
% For instance, if a new failure falls into a flaky cluster, it is more likely to be flaky because it matches the patterns of other flaky failures and does not match those of non-flaky ones.
% Moreover, applying this approach and analyzing the resulting clusters can provide insights into the characteristics that differentiate flaky failures from non-flaky ones. When a cluster is labeled as a \emph{flaky cluster}, it means that the failure contains \emph{at least} one stack trace line that has never been observed in any of the non-flaky failures. This discovery raises suspicion and points to a possible link to the root cause of flakiness.
% \abdul{Do you think this should be well explained e.g. with more details and examples.}
% \jon{Yes, I think that a few ``For example" sentences here would be helpful.}\abdul{How about now?}


% \subsubsection{Failure Log Classifier}
% The earlier approach involves comparing the syntax of two failures, including the line number in the stack trace, which may not be suitable for comparing failures from different tests. Assuming that flaky failures exhibit differences compared to non-flaky ones, could a classifier be developed to predict if a failure is likely to be flaky based on other flaky failures from various tests?

% My proposal involves a failure log classifier, which employs a machine learning approach to learn from both flaky and non-flaky failure logs, enabling it to predict the status of a given failure log as either flaky or not. The classifier gathers specific features from the failure logs, which are outlined in Table~\ref{table:Features}. To ensure that the classifier can learn from various tests, the selected features should encompass all tests and not be influenced by the content of a particular test, such as whether the stack trace lines cover any line in the test suite rather than the test itself. Although the initial set of features is not final, I believe they are sufficient to begin training the classifier.

% To begin, I processed the data by extracting features from the failure exception and stack trace lines. These features require knowledge of all test names in the test suite, test names throughout the entire project, and all source code file names of the code under test to facilitate determining the feature values for each test failure. Next, I employed a simple \emph{Decision Tree} (\textbf{DT}) as the supervised learning algorithm and utilized stratified cross-validation to train on a portion of the data and predict the remaining. Additionally, I used SMOTE to balance the data due to its imbalanced nature. To evaluate the performance of the log classifier, I applied \emph{TF-IDF} (Term Frequency-Inverse Document Frequency~\cite{tfidf}), as a baseline for prediction results.


% \subsubsection{Failure logs Approaches: Initial Result}

% I am looking to emphasize the main findings of using the proposed approaches by answering the following questions:

% \begin{description}
%   \item[\textbf{RQ1:}] Is the \syntax able to discriminate the flaky failures?
%   \item[\textbf{RQ2:}] Are some exceptions related to flakiness more than non flaky failures?
%   \item[\textbf{RQ3:}] Can machine learning be utilized to predict flaky failures using the failure logs?

%  \end{description}

% % \begin{table*}[t]
%     \caption{caption X}
%     % by Test -> How many flaky tests in a certain project? Show number of tests that flake exactly once, then the minimum number of flakes per-test, max, and a sparkline
%     % by Build -> How many builds have at least one flaky test? How many builds iwht exactly one flaky test, then min, max, and distribution of flaky tests 
% \label{table:classifier_table}
% \vspace{-5pt}
% \setlength{\tabcolsep}{2.5pt}
% \newcommand{\failureRateWidth}{2.5in}
% \newcommand{\failureRateHeight}{4em}
% \scriptsize
% \centering
%     \begin{tabular}{l|ccc|cc|rrrrrrr|rrrrrrr}
%     \toprule
%       & \multicolumn{3}{c}{\textbf{Text-Match-Approach}} & \multicolumn{2}{c}{\textbf{Text-Match-Approach}} & \multicolumn{7}{c}{\textbf{Failure Log Classifier}} & \multicolumn{7}{c}{\textbf{TF-IDF}}  \\ 
     
%      \textbf{Project}&\textbf{Failures}&\textbf{Flaky}&\textbf{Non-Flaky}&\textbf{UNMATCH}&\textbf{MATCH}&\textbf{TP}&\textbf{FN}&\textbf{FP}&\textbf{TN}&\textbf{P}&\textbf{R}&\textbf{F1}&\textbf{TP}&\textbf{FN}&\textbf{FP}&\textbf{TN}&\textbf{P}&\textbf{R}&\textbf{F1}\\
%         \midrule

% java-webSocket&3276&1977&1299&42&3&1942&35&854&445&69\%&98\%&81\%&1855&122&1158&141&62\%&94\%&74\%\\
% assertj-core&30&13&17&1&0&9&4&1&16&90\%&69\%&78\%&10&3&1&16&91\%&77\%&83\%\\
% % ninja&319&110&209&NA&NA&110&0&90&119&55\%&100\%&71\%&110&0&92&117&54\%&100\%&71\%\\
% orbit&1035&213&822&2&5&202&11&69&753&75\%&95\%&83\%&108&105&236&586&31\%&51\%&39\%\\
% % handlebars.java&183&36&147&NA&NA&36&0&16&131&69\%&100\%&82\%&36&0&16&131&69\%&100\%&82\%\\
% achilles&499&57&442&2&3&54&3&61&381&47\%&95\%&63\%&39&18&105&337&27\%&68\%&39\%\\
% logback&2948&334&2614&7&14&298&34&496&2118&38\%&90\%&53\%&186&146&34&2580&85\%&56\%&67\%\\
% okhttp&36685&2419&34266&105&16&2261&158&3019&31247&43\%&93\%&59\%&717&1702&9968&24298&7\%&30\%&11\%\\
% wro4j&576&36&540&12&9&26&8&45&495&37\%&76\%&50\%&23&11&14&526&62\%&68\%&65\%\\
% activiti&48271&2171&46100&5&28&2136&35&7856&38244&21\%&98\%&35\%&480&1691&5544&40556&8\%&22\%&12\%\\
% http-request&405&18&387&7&11&14&4&113&274&11\%&78\%&19\%&18&0&0&387&100\%&100\%&100\%\\
% hbase&11895&571&11324&130&4&425&26&478&10846&47\%&94\%&63\%&269&182&2146&9178&11\%&60\%&19\%\\
% alluxio&33317&709&32608&138&172&630&74&6996&25612&8\%&89\%&15\%&385&319&5649&26959&6\%&55\%&11\%\\
% httpcore&8426&93&8333&1&21&52&41&1789&6544&3\%&56\%&5\%&39&54&2049&6284&2\%&42\%&4\%\\
% hector&3638&34&3604&31&2&31&3&0&3604&100\%&91\%&95\%&33&1&0&3604&100\%&97\%&99\%\\
% % io-undertow&2325&21&2304&NA&NA&17&4&1388&916&1\%&81\%&2\%&19&2&1719&585&1\%&90\%&2\%\\
% spring-boot&2503&353&2150&13&0&5&9&516&1634&1\%&36\%&2\%&12&2&0&2150&100\%&86\%&92\%\\
% ambari&11106&57&11049&52&2&49&7&439&10610&10\%&88\%&18\%&51&5&119&10930&30\%&91\%&45\%\\
% wildfly&3863&16&3847&18&0&16&0&116&3731&12\%&100\%&22\%&16&0&0&3847&100\%&100\%&100\%\\
    

% \bottomrule 
% \end{tabular}
% \vspace{-10pt}
% \end{table*}


\begin{table*}[t]
% \jon{New columns: Flaky tests, Flaky Failures, Non-Flaky Failures, Group of 3 columns with header "Synatx-based approach": (Flaky only, Both, Non-Flaky Only), Failure log classifier, TF-IDF}
% \jon{Remove "Killed Mutant statistics" and instead do some analysis to determine how many flaky failures have fewer than N (N=1,2,3?) mutants to match against, and comment on these (and whether or not they match)}
\caption[The Prediction of \classifier and TF-IDF of Flaky and True Failures]{The Result of \classifier and TF-IDF of Flaky and True Failures Prediction?\\
\textnormal{The \classifier and TF-IDF show (per project) the confusion matrix, precision (P), recall (R), and F1 score of the overall prediction result. 
% Compared to Table\ref{nonunique}, projects with fewer than 10 flaky failures have been opted out.
}}
    % by Test -> How many flaky tests in a certain project? Show number of tests that flake exactly once, then the minimum number of flakes per-test, max, and a sparkline
    % by Build -> How many builds have at least one flaky test? How many builds iwht exactly one flaky test, then min, max, and distribution of flaky tests 
\label{table:classifier_table}
\vspace{-5pt}
\setlength{\tabcolsep}{1.0pt}
\newcommand{\failureRateWidth}{2.5in}
\newcommand{\failureRateHeight}{4em}
\scriptsize
\centering
    \begin{tabular}{l|rrrr|rrrrrrr|rrrrrrr}
    \toprule
      & \multicolumn{4}{c}{\textbf{Total Flaky Tests and Failures}} & \multicolumn{7}{c}{\textbf{Failure Log Classifier}} & \multicolumn{7}{c}{\textbf{TF-IDF}}\\ 
     
     \textbf{Project}&\textbf{Test}&\textbf{Failures}&\textbf{Flaky}&\textbf{True}&\textbf{TP}&\textbf{FN}&\textbf{FP}&\textbf{TN}&\textbf{P}&\textbf{R}&\textbf{F1}&\textbf{TP}&\textbf{FN}&\textbf{FP}&\textbf{TN}&\textbf{P}&\textbf{R}&\textbf{F1}\\
        \midrule

Alluxio-alluxio&114&49,466&16,858&32,608&16,014&844&1,104&31,504&93\%&94\%&94\%&16,580&278&394&32,214&97\%&98\%&98\%\\
\cellcolor{gray!6}{square-okhttp}&\cellcolor{gray!6}{100}&\cellcolor{gray!6}{62,530}&\cellcolor{gray!6}{28,264}&\cellcolor{gray!6}{34,266}&\cellcolor{gray!6}{28,123}&\cellcolor{gray!6}{141}&\cellcolor{gray!6}{1,585}&\cellcolor{gray!6}{32,681}&\cellcolor{gray!6}{94\%}&\cellcolor{gray!6}{99\%}&\cellcolor{gray!6}{97\%}&\cellcolor{gray!6}{28,238}&\cellcolor{gray!6}{26}&\cellcolor{gray!6}{108}&\cellcolor{gray!6}{34,158}&\cellcolor{gray!6}{99\%}&\cellcolor{gray!6}{99\%}&\cellcolor{gray!6}{99\%}\\
apache-hbase&62&31,146&19,822&11,324&19,782&40&369&10,955&98\%&99\%&98\%&19,676&146&19&11,305&99\%&99\%&99\%\\
\cellcolor{gray!6}{apache-ambari}&\cellcolor{gray!6}{51}&\cellcolor{gray!6}{15,112}&\cellcolor{gray!6}{4,063}&\cellcolor{gray!6}{11,049}&\cellcolor{gray!6}{4,055}&\cellcolor{gray!6}{8}&\cellcolor{gray!6}{482}&\cellcolor{gray!6}{10,567}&\cellcolor{gray!6}{89\%}&\cellcolor{gray!6}{99\%}&\cellcolor{gray!6}{94\%}&\cellcolor{gray!6}{4,063}&\cellcolor{gray!6}{0}&\cellcolor{gray!6}{5}&\cellcolor{gray!6}{11,044}&\cellcolor{gray!6}{99\%}&\cellcolor{gray!6}{100\%}&\cellcolor{gray!6}{99\%}\\
Hector&33&10,133&6,529&3,604&6,529&0&405&3,199&94\%&100\%&96\%&6,529&0&13&3,591&99\%&100\%&99\%\\
\cellcolor{gray!6}{activiti-activiti}&\cellcolor{gray!6}{31}&\cellcolor{gray!6}{47,478}&\cellcolor{gray!6}{1,378}&\cellcolor{gray!6}{46,100}&\cellcolor{gray!6}{947}&\cellcolor{gray!6}{431}&\cellcolor{gray!6}{311}&\cellcolor{gray!6}{45,789}&\cellcolor{gray!6}{75\%}&\cellcolor{gray!6}{68\%}&\cellcolor{gray!6}{71\%}&\cellcolor{gray!6}{1,013}&\cellcolor{gray!6}{365}&\cellcolor{gray!6}{60}&\cellcolor{gray!6}{46,040}&\cellcolor{gray!6}{94\%}&\cellcolor{gray!6}{73\%}&\cellcolor{gray!6}{82\%}\\
apache-httpcore&22&8,687&354&8,333&315&39&110&8,223&74\%&88\%&80\%&314&40&16&8,317&95\%&88\%&91\%\\
\cellcolor{gray!6}{Java-websocket}&\cellcolor{gray!6}{22}&\cellcolor{gray!6}{3,394}&\cellcolor{gray!6}{2,095}&\cellcolor{gray!6}{1,299}&\cellcolor{gray!6}{2,082}&\cellcolor{gray!6}{13}&\cellcolor{gray!6}{721}&\cellcolor{gray!6}{578}&\cellcolor{gray!6}{74\%}&\cellcolor{gray!6}{99\%}&\cellcolor{gray!6}{85\%}&\cellcolor{gray!6}{2,082}&\cellcolor{gray!6}{13}&\cellcolor{gray!6}{722}&\cellcolor{gray!6}{577}&\cellcolor{gray!6}{74\%}&\cellcolor{gray!6}{99\%}&\cellcolor{gray!6}{84\%}\\
qos-ch-logback&20&3,052&438&2,614&172&266&104&2,510&62\%&39\%&48\%&239&199&41&2,573&85\%&54\%&66\%\\
\cellcolor{gray!6}{Http-request}&\cellcolor{gray!6}{18}&\cellcolor{gray!6}{3,888}&\cellcolor{gray!6}{3,501}&\cellcolor{gray!6}{387}&\cellcolor{gray!6}{3,498}&\cellcolor{gray!6}{3}&\cellcolor{gray!6}{124}&\cellcolor{gray!6}{263}&\cellcolor{gray!6}{96\%}&\cellcolor{gray!6}{99\%}&\cellcolor{gray!6}{98\%}&\cellcolor{gray!6}{3,498}&\cellcolor{gray!6}{3}&\cellcolor{gray!6}{54}&\cellcolor{gray!6}{333}&\cellcolor{gray!6}{98\%}&\cellcolor{gray!6}{99\%}&\cellcolor{gray!6}{99\%}\\
wildfly-wildfly&18&3,895&48&3,847&0&48&0&3,847&0\%&0\%&0\%&48&0&0&3,847&100\%&100\%&100\%\\
\cellcolor{gray!6}{wro4j-wro4j}&\cellcolor{gray!6}{14}&\cellcolor{gray!6}{11,373}&\cellcolor{gray!6}{10,833}&\cellcolor{gray!6}{540}&\cellcolor{gray!6}{10,833}&\cellcolor{gray!6}{0}&\cellcolor{gray!6}{65}&\cellcolor{gray!6}{475}&\cellcolor{gray!6}{99\%}&\cellcolor{gray!6}{100\%}&\cellcolor{gray!6}{99\%}&\cellcolor{gray!6}{10,833}&\cellcolor{gray!6}{0}&\cellcolor{gray!6}{29}&\cellcolor{gray!6}{511}&\cellcolor{gray!6}{99\%}&\cellcolor{gray!6}{100\%}&\cellcolor{gray!6}{99\%}\\
Spring-boot&12&2,164&14&2,150&6&8&0&2,150&100\%&42\%&60\%&10&4&1&2,149&90\%&71\%&80\%\\
\cellcolor{gray!6}{orbit-orbit}&\cellcolor{gray!6}{7}&\cellcolor{gray!6}{3,765}&\cellcolor{gray!6}{2,943}&\cellcolor{gray!6}{822}&\cellcolor{gray!6}{2,943}&\cellcolor{gray!6}{0}&\cellcolor{gray!6}{69}&\cellcolor{gray!6}{753}&\cellcolor{gray!6}{97\%}&\cellcolor{gray!6}{100\%}&\cellcolor{gray!6}{98\%}&\cellcolor{gray!6}{2,943}&\cellcolor{gray!6}{0}&\cellcolor{gray!6}{59}&\cellcolor{gray!6}{763}&\cellcolor{gray!6}{98\%}&\cellcolor{gray!6}{100\%}&\cellcolor{gray!6}{99\%}\\
Undertow&7&2,396&92&2,304&3&89&0&2,304&100\%&3\%&6\%&5&87&0&2,304&100\%&5\%&10\%\\
\cellcolor{gray!6}{Achilles}&\cellcolor{gray!6}{4}&\cellcolor{gray!6}{607}&\cellcolor{gray!6}{165}&\cellcolor{gray!6}{442}&\cellcolor{gray!6}{120}&\cellcolor{gray!6}{45}&\cellcolor{gray!6}{0}&\cellcolor{gray!6}{442}&\cellcolor{gray!6}{100\%}&\cellcolor{gray!6}{72\%}&\cellcolor{gray!6}{84\%}&\cellcolor{gray!6}{148}&\cellcolor{gray!6}{17}&\cellcolor{gray!6}{26}&\cellcolor{gray!6}{416}&\cellcolor{gray!6}{85\%}&\cellcolor{gray!6}{89\%}&\cellcolor{gray!6}{87\%}\\
% elasticjob-elastic-job-lite&3&0&0&0&0&0&0&0&0\%&0\%&0\%&0&0&0&0&0\%&0\%&0\%\\
Commons-exec&1&92&33&59&0&33&0&59&0\%&0\%&0\%&33&0&2&57&94\%&100\%&97\%\\
\cellcolor{gray!6}{zxing-zxing}&\cellcolor{gray!6}{1}&\cellcolor{gray!6}{398}&\cellcolor{gray!6}{322}&\cellcolor{gray!6}{76}&\cellcolor{gray!6}{322}&\cellcolor{gray!6}{0}&\cellcolor{gray!6}{0}&\cellcolor{gray!6}{76}&\cellcolor{gray!6}{100\%}&\cellcolor{gray!6}{100\%}&\cellcolor{gray!6}{100\%}&\cellcolor{gray!6}{322}&\cellcolor{gray!6}{0}&\cellcolor{gray!6}{0}&\cellcolor{gray!6}{76}&\cellcolor{gray!6}{100\%}&\cellcolor{gray!6}{100\%}&\cellcolor{gray!6}{100\%}\\
handlebars.java&1&558&411&147&411&0&16&131&96\%&100\%&98\%&411&0&16&131&96\%&100\%&98\%\\
\cellcolor{gray!6}{assertj-core}&\cellcolor{gray!6}{1}&\cellcolor{gray!6}{991}&\cellcolor{gray!6}{974}&\cellcolor{gray!6}{17}&\cellcolor{gray!6}{974}&\cellcolor{gray!6}{0}&\cellcolor{gray!6}{1}&\cellcolor{gray!6}{16}&\cellcolor{gray!6}{99\%}&\cellcolor{gray!6}{100\%}&\cellcolor{gray!6}{99\%}&\cellcolor{gray!6}{974}&\cellcolor{gray!6}{0}&\cellcolor{gray!6}{0}&\cellcolor{gray!6}{17}&\cellcolor{gray!6}{100\%}&\cellcolor{gray!6}{100\%}&\cellcolor{gray!6}{100\%}\\
ninja-ninja&1&685&476&209&476&0&90&119&84\%&100\%&91\%&476&0&90&119&84\%&100\%&91\%\\
\midrule
21 Projects Total &540&261,810&99,613&162,197&97,605&2,008&5,556&156,641&&&&98,435&1,178&1,655&160,542&&&\\

\bottomrule


% Alluxio-alluxio&114&49,466&16,858&32,608&16,014&844&1,104&31,504&93\%&94\%&94\%&16,565&293&389&32,219&97\%&98\%&97\%\\
% \cellcolor{gray!6}{square-okhttp}&\cellcolor{gray!6}{100}&\cellcolor{gray!6}{62,530}&\cellcolor{gray!6}{28,264}&\cellcolor{gray!6}{34,266}&\cellcolor{gray!6}{28,123}&\cellcolor{gray!6}{141}&\cellcolor{gray!6}{1,585}&\cellcolor{gray!6}{32,681}&\cellcolor{gray!6}{94\%}&\cellcolor{gray!6}{99\%}&\cellcolor{gray!6}{97\%}&\cellcolor{gray!6}{28,239}&\cellcolor{gray!6}{25}&\cellcolor{gray!6}{108}&\cellcolor{gray!6}{34,158}&\cellcolor{gray!6}{99\%}&\cellcolor{gray!6}{99\%}&\cellcolor{gray!6}{99\%}\\
% apache-hbase&62&31,174&19,850&11,324&19,809&41&369&10,955&98\%&99\%&98\%&19,676&146&19&11,305&99\%&99\%&99\%\\
% \cellcolor{gray!6}{apache-ambari}&\cellcolor{gray!6}{51}&\cellcolor{gray!6}{15,112}&\cellcolor{gray!6}{4,063}&\cellcolor{gray!6}{11,049}&\cellcolor{gray!6}{4,055}&\cellcolor{gray!6}{8}&\cellcolor{gray!6}{481}&\cellcolor{gray!6}{10,568}&\cellcolor{gray!6}{89\%}&\cellcolor{gray!6}{99\%}&\cellcolor{gray!6}{94\%}&\cellcolor{gray!6}{4,063}&\cellcolor{gray!6}{0}&\cellcolor{gray!6}{5}&\cellcolor{gray!6}{11,044}&\cellcolor{gray!6}{99\%}&\cellcolor{gray!6}{100\%}&\cellcolor{gray!6}{99\%}\\
% hector-client-hector&33&10,133&6,529&3,604&6,529&0&405&3,199&94\%&100\%&96\%&6,529&0&12&3,592&99\%&100\%&99\%\\
% \cellcolor{gray!6}{activiti-activiti}&\cellcolor{gray!6}{31}&\cellcolor{gray!6}{47,478}&\cellcolor{gray!6}{1,378}&\cellcolor{gray!6}{46,100}&\cellcolor{gray!6}{1,303}&\cellcolor{gray!6}{75}&\cellcolor{gray!6}{5,421}&\cellcolor{gray!6}{40,679}&\cellcolor{gray!6}{19\%}&\cellcolor{gray!6}{94\%}&\cellcolor{gray!6}{32\%}&\cellcolor{gray!6}{1,038}&\cellcolor{gray!6}{340}&\cellcolor{gray!6}{70}&\cellcolor{gray!6}{46,030}&\cellcolor{gray!6}{93\%}&\cellcolor{gray!6}{75\%}&\cellcolor{gray!6}{83\%}\\
% apache-httpcore&22&8,687&354&8,333&321&33&115&8,218&73\%&90\%&81\%&314&40&17&8,316&94\%&88\%&91\%\\
% \cellcolor{gray!6}{tootallnate-java-websocket}&\cellcolor{gray!6}{22}&\cellcolor{gray!6}{3,394}&\cellcolor{gray!6}{2,095}&\cellcolor{gray!6}{1,299}&\cellcolor{gray!6}{2,082}&\cellcolor{gray!6}{13}&\cellcolor{gray!6}{721}&\cellcolor{gray!6}{578}&\cellcolor{gray!6}{74\%}&\cellcolor{gray!6}{99\%}&\cellcolor{gray!6}{85\%}&\cellcolor{gray!6}{2,082}&\cellcolor{gray!6}{13}&\cellcolor{gray!6}{722}&\cellcolor{gray!6}{577}&\cellcolor{gray!6}{74\%}&\cellcolor{gray!6}{99\%}&\cellcolor{gray!6}{84\%}\\
% qos-ch-logback&20&3,052&438&2,614&172&266&104&2,510&62\%&39\%&48\%&245&193&49&2,565&83\%&55\%&66\%\\
% \cellcolor{gray!6}{kevinsawicki-http-request}&\cellcolor{gray!6}{18}&\cellcolor{gray!6}{3,888}&\cellcolor{gray!6}{3,501}&\cellcolor{gray!6}{387}&\cellcolor{gray!6}{3,498}&\cellcolor{gray!6}{3}&\cellcolor{gray!6}{124}&\cellcolor{gray!6}{263}&\cellcolor{gray!6}{96\%}&\cellcolor{gray!6}{99\%}&\cellcolor{gray!6}{98\%}&\cellcolor{gray!6}{3,498}&\cellcolor{gray!6}{3}&\cellcolor{gray!6}{54}&\cellcolor{gray!6}{333}&\cellcolor{gray!6}{98\%}&\cellcolor{gray!6}{99\%}&\cellcolor{gray!6}{99\%}\\
% wildfly-wildfly&18&3,895&48&3,847&48&0&116&3,731&29\%&100\%&45\%&48&0&0&3,847&100\%&100\%&100\%\\
% \cellcolor{gray!6}{wro4j-wro4j}&\cellcolor{gray!6}{14}&\cellcolor{gray!6}{11,373}&\cellcolor{gray!6}{10,833}&\cellcolor{gray!6}{540}&\cellcolor{gray!6}{10,474}&\cellcolor{gray!6}{359}&\cellcolor{gray!6}{42}&\cellcolor{gray!6}{498}&\cellcolor{gray!6}{99\%}&\cellcolor{gray!6}{96\%}&\cellcolor{gray!6}{98\%}&\cellcolor{gray!6}{10,833}&\cellcolor{gray!6}{0}&\cellcolor{gray!6}{29}&\cellcolor{gray!6}{511}&\cellcolor{gray!6}{99\%}&\cellcolor{gray!6}{100\%}&\cellcolor{gray!6}{99\%}\\
% spring-projects-spring-boot&12&2,164&14&2,150&13&1&850&1,300&1\%&92\%&2\%&10&4&2&2,148&83\%&71\%&76\%\\
% \cellcolor{gray!6}{orbit-orbit}&\cellcolor{gray!6}{7}&\cellcolor{gray!6}{3,765}&\cellcolor{gray!6}{2,943}&\cellcolor{gray!6}{822}&\cellcolor{gray!6}{2,943}&\cellcolor{gray!6}{0}&\cellcolor{gray!6}{69}&\cellcolor{gray!6}{753}&\cellcolor{gray!6}{97\%}&\cellcolor{gray!6}{100\%}&\cellcolor{gray!6}{98\%}&\cellcolor{gray!6}{2,940}&\cellcolor{gray!6}{3}&\cellcolor{gray!6}{59}&\cellcolor{gray!6}{763}&\cellcolor{gray!6}{98\%}&\cellcolor{gray!6}{99\%}&\cellcolor{gray!6}{98\%}\\
% undertow-io-undertow&7&2,396&92&2,304&57&35&148&2,156&27\%&61\%&38\%&5&87&0&2,304&100\%&5\%&10\%\\
% \cellcolor{gray!6}{doanduyhai-Achilles}&\cellcolor{gray!6}{4}&\cellcolor{gray!6}{607}&\cellcolor{gray!6}{165}&\cellcolor{gray!6}{442}&\cellcolor{gray!6}{120}&\cellcolor{gray!6}{45}&\cellcolor{gray!6}{0}&\cellcolor{gray!6}{442}&\cellcolor{gray!6}{100\%}&\cellcolor{gray!6}{72\%}&\cellcolor{gray!6}{84\%}&\cellcolor{gray!6}{148}&\cellcolor{gray!6}{17}&\cellcolor{gray!6}{28}&\cellcolor{gray!6}{414}&\cellcolor{gray!6}{84\%}&\cellcolor{gray!6}{89\%}&\cellcolor{gray!6}{86\%}\\
% \cellcolor{gray!6}{zxing-zxing}&\cellcolor{gray!6}{1}&\cellcolor{gray!6}{398}&\cellcolor{gray!6}{322}&\cellcolor{gray!6}{76}&\cellcolor{gray!6}{322}&\cellcolor{gray!6}{0}&\cellcolor{gray!6}{0}&\cellcolor{gray!6}{76}&\cellcolor{gray!6}{100\%}&\cellcolor{gray!6}{100\%}&\cellcolor{gray!6}{100\%}&\cellcolor{gray!6}{322}&\cellcolor{gray!6}{0}&\cellcolor{gray!6}{0}&\cellcolor{gray!6}{76}&\cellcolor{gray!6}{100\%}&\cellcolor{gray!6}{100\%}&\cellcolor{gray!6}{100\%}\\
% assertj-core&1&991&974&17&974&0&1&16&99\%&100\%&99\%&974&0&0&17&100\%&100\%&100\%\\
% \cellcolor{gray!6}{apache-commons-exec}&\cellcolor{gray!6}{1}&\cellcolor{gray!6}{92}&\cellcolor{gray!6}{33}&\cellcolor{gray!6}{59}&\cellcolor{gray!6}{0}&\cellcolor{gray!6}{33}&\cellcolor{gray!6}{0}&\cellcolor{gray!6}{59}&\cellcolor{gray!6}{0\%}&\cellcolor{gray!6}{0\%}&\cellcolor{gray!6}{0\%}&\cellcolor{gray!6}{33}&\cellcolor{gray!6}{0}&\cellcolor{gray!6}{2}&\cellcolor{gray!6}{57}&\cellcolor{gray!6}{94\%}&\cellcolor{gray!6}{100\%}&\cellcolor{gray!6}{97\%}\\
% ninja-ninja&1&685&476&209&476&0&90&119&84\%&100\%&91\%&476&0&90&119&84\%&100\%&91\%\\
% \cellcolor{gray!6}{handlebars.java}&\cellcolor{gray!6}{1}&\cellcolor{gray!6}{558}&\cellcolor{gray!6}{411}&\cellcolor{gray!6}{147}&\cellcolor{gray!6}{411}&\cellcolor{gray!6}{0}&\cellcolor{gray!6}{16}&\cellcolor{gray!6}{131}&\cellcolor{gray!6}{96\%}&\cellcolor{gray!6}{100\%}&\cellcolor{gray!6}{98\%}&\cellcolor{gray!6}{411}&\cellcolor{gray!6}{0}&\cellcolor{gray!6}{16}&\cellcolor{gray!6}{131}&\cellcolor{gray!6}{96\%}&\cellcolor{gray!6}{100\%}&\cellcolor{gray!6}{98\%}\\
% \midrule
% Total&543&261,838&99,641&162,197&97,744&1,897&11,761&150,436&&&&98,449&1,164&1,671&160,526&&&\\

% \bottomrule



% \bottomrule
\end{tabular}
\vspace{-10pt}
\end{table*}








% I am looking to evaluate the effectiveness of the \syntax as a method for detecting test flakiness. By distinguishing flaky failures based on their failure exception and stack trace lines, developers can employ this approach to determine if a failure is flaky or not by comparing it with non-flaky failures. Additionally, I am interested in exploring how machine learning can leverage failure logs to construct a classifier that predicts the likelihood of a failure being flaky. 


% \textbf{RQ1: Is the \syntax able to discriminate the flaky failures?}Table~\ref{table:classifier_table} illustrates the outcomes of applying the \syntax to flaky and non-flaky failures in different projects. The column labeled \textit{Failure by \syntax} is divided into three parts: \textit{OnlyFlaky}, \textit{Only Non-Flaky}, and \textit{Both}. The \textit{OnlyFlaky} column represents failure logs that are exclusively observed in flaky failures, while the \textit{Only Non-Flaky} column represents failures that are only present in non-flaky failures. The \textit{Both} column captures failure logs that are common to both flaky and non-flaky failures. If the number of occurrences in the \textit{Both} column is minimized, it suggests that the \syntax is more likely to be considered effective.

% The outcomes of the \syntax exhibit variations among the studied projects. Among the 15 projects analyzed, there are six projects in which at least 86\% of their flaky failures exclusively appear in the \emph{OnlyFlaky} category. Conversely, there are projects where this approach did not perform effectively, such as in the case of the \httpcore project, where it successfully distinguishes only one flaky failures. It is notable that the number of flaky failures is not a common key of these projects, as this approach perform well regardless to the number of flaky failures. These findings motivate further analysis of the failure exceptions and identification of common patterns across these projects.

% The primary observation in projects where it is challenging to distinguish flaky failures using the \syntax is that these failures often manifest as assertion exceptions. For instance, in both the \emph{java-webscoket} and \emph{http-request} projects, all the failures in the column \emph{Both} share the \textit{AssertionError} exception. Similarly, the majority of failures in projects like \emph{qos-ch-logback} and \emph{activiti} also fall under the \textit{AssertionError} exception.
% Conversely, in projects where flaky failures are exclusively present, the majority of failures are characterized by different exceptions, such as \emph{UnknownHostException} and \emph{IOException}. 
% In the context of the \emph{Alluxio} project, there are 164 failure clusters (out of a total of 174 clusters containing both flaky and non-flaky failures) that are attributed to a specific exception type: \emph{NullPointerException}. Despite having access to the stack trace lines, these exceptions prove to be challenging to distinguish, highlighting the difficulty of utilizing failure logs as a criteria for such exception types. It is worth noting that this exception is not detected in failure logs of other projects where it is not reported as an exception in only flaky failures.

% In general, leveraging the exception type and stack trace lines is a promising approach for developers and researchers to analyze failure logs and distinguish between flaky and non-flaky failures. When comparing flaky failures to non-flaky failures, focusing on the stack traces alone can often be effective without explicitly relying on the exception type. This is evident from the discovery that all failures identified as unique based on the exception types are also unique based on their stack trace lines. However, incorporating the exception types can enhance the analysis and provide supplementary information to aid in the classification of failure logs. The use of exception types will be further explored and discussed in detail in RQ3.

% \textbf{RQ2: Are some exceptions related to flakiness more than non flaky failures?} The column \emph{OnlyFlaky} in Table~\ref{table:classifier_table} illustrates the existence of clusters of failures that are only appear in flaky side and never been detected in non-flaky failures.
% Among these failures, there are 284 failures clusters where the exceptions alone never been reported in the non flaky failures of the same test, regardless of using stack trace lines in the comparison, which accounts for 55\% of the these failures. In other words, these exceptions have never been detected in the non-flaky failures obtained from the corresponding tests associated with these flaky failures.


% Table~\ref{table:exceptions} presents the top ten most frequently occurring exceptions observed in the analyzed flaky failures. Among them, two exceptions, namely \emph{UnknownHostException} and \emph{HCassandraInternalException}, consistently distinguish the failures from the non-flaky failures. Approximately 93\% of the failures involve the \emph{IOException} exception, while around 63\% of the failures involve the \emph{Exception} exception.
% The uniqueness of failures based on exceptions can be influenced by the project domain. For example, the \emph{HCassandraInternalException} exception is specific to one project, potentially affecting its uniqueness. The \emph{UnknownHostException} exception is encountered in six projects, with roughly 75\% of these exceptions originating from the \alluxio project.
% However, it is important to clarify that the presence of an exception like \emph{UnknownHostException} does not necessarily indicate a direct association with flaky failures. I have identified a few of the non-flaky failures reported with this exception in the \emph{okhttp} project. Interestingly, none of the flaky failures in that project have been reported with an \emph{UnknownHostException}.


% \begin{table}[t]
  \setlength{\tabcolsep}{2.5pt}


\caption[List of top 10 Exceptions in flaky failures.]{
\textnormal{The Occurrence column denotes the count of failures associated with these exceptions and indicates the number of projects in which these failures originate from.
The With Stacktrace columns provide the count of each exception (including the stacktrace) that occur exclusively in flaky failures, as well as when they occur in non-flaky failures.
The Without Stacktrace (w/o) column corresponds to the With Stacktrace column, but excludes the consideration of stacktraces.}}
\label{table:exceptions}
\centering
\vspace{-4pt}
%\resizebox{\textwidth}{!}{
% \scriptsize
\footnotesize
\begin{tabular}{l|rr|rr|rr}

\toprule
      & \multicolumn{2}{c|}{\textbf{Occurrence}} & \multicolumn{2}{c|}{\textbf{With Stacktrace}} & \multicolumn{2}{c}{\textbf{w/o Stacktrace}}\\
      

\textbf{Exceptions} & \textbf{Failures} & \textbf{Project}  & \textbf{onlyFlaky} & \textbf{Both}  & \textbf{onlyFlaky} & \textbf{Both} \\
\midrule
AssertionError&186&15&88&98&8&178\\
\cellcolor{gray!6}{NullPointerException}&\cellcolor{gray!6}{185}&\cellcolor{gray!6}{5}&\cellcolor{gray!6}{39}&\cellcolor{gray!6}{146}&\cellcolor{gray!6}{9}&\cellcolor{gray!6}{176}\\
UnknownHostException&136&6&136&0&136&0\\
\cellcolor{gray!6}{IOException}&\cellcolor{gray!6}{70}&\cellcolor{gray!6}{4}&\cellcolor{gray!6}{57}&\cellcolor{gray!6}{13}&\cellcolor{gray!6}{36}&\cellcolor{gray!6}{34}\\
ProvisionException&49&1&49&0&0&49\\
\cellcolor{gray!6}{HCassandraInternalException}&\cellcolor{gray!6}{31}&\cellcolor{gray!6}{1}&\cellcolor{gray!6}{31}&\cellcolor{gray!6}{0}&\cellcolor{gray!6}{31}&\cellcolor{gray!6}{0}\\
SocketException&31&1&29&2&1&30\\
\cellcolor{gray!6}{Exception}&\cellcolor{gray!6}{30}&\cellcolor{gray!6}{5}&\cellcolor{gray!6}{30}&\cellcolor{gray!6}{0}&\cellcolor{gray!6}{28}&\cellcolor{gray!6}{2}\\
RuntimeException&23&3&21&2&9&14\\
\cellcolor{gray!6}{AssertionFailedError}&\cellcolor{gray!6}{22}&\cellcolor{gray!6}{5}&\cellcolor{gray!6}{1}&\cellcolor{gray!6}{21}&\cellcolor{gray!6}{1}&\cellcolor{gray!6}{21}\\


\bottomrule
\end{tabular}
\vspace{-10pt}
\end{table}


% It is crucial to emphasize that not all \emph{OnlyFlaky} failure clusters can be identified by a unique exception type. I found some failures where failure exceptions belong to both the unique and non-unique failure categories, as indicated in Table~\ref{table:exceptions}. In the context of the experiment, the most frequently occurring exception is the \emph{AssertionError} which is almost evenly split between the two categories, with 58\% of occurrences in unique failures. However, when considering only the exception type and excluding stacktrace lines, the proportion of unique failures with this exception drops to less than 5\%.
% The reason behind this observation is the generality of the \emph{AssertionError} exception. For example, a test may have multiple assertion statements, and if they fail for different reasons, they match the exception but differ in the stack trace. Therefore, it becomes challenging to attribute this type of exception to a specific type of failure.



% I conducted an analysis of failures involving the \emph{AssertionError} exceptions and identified them as \emph{OnlyFlaky} failures when compared to other non-flaky failures, including stacktrace lines. One important observation I made is that the test name does not appear in 70\% of the stack traces for these failures. This suggests that these failures occur in setup methods, as observed in the majority of flaky failures in projects like \okhttp and \emph{tootallnate-java-websocket}. On the other hand, all of the \emph{Both} failures clusters that contain an \emph{AssertionError} in their exception have the test name present in the stack trace lines.

% I have noticed that approximately half of the \emph{OnlyFlaky} failures clusters can be identified by considering the exception type without examining the stack trace lines. This suggests that certain exceptions may be more likely to be associated with flaky failures under circumstances such as project domains.Also, I found that linking the \emph{AssertionError} exception to one of the two types of failures is challenging due to the general nature of this exception. 


% \textbf{RQ3: Can machine learning be utilized to predict flaky failures using the failure logs?} I utilized the dataset employed in the syntax-based approach to train and evaluate the \classifier. As part of the dataset preparation, I represent each failure with  a set of values, based on the features outlined in Table~\ref{table:Features}. For features that required accessing the code under test, I extracted all the \emph{java} file names from each project. During this process, I encountered missing values, such as two flaky tests in \emph{wildfly-wildfly}. Since the total number of missing values was minimal, I exclude all flaky tests associated with these failures.

% Next, I developed a decision tree model that employed stratified cross-validation per project. Since the proportion of flaky failures was quite low compared to the total number of failures, I balanced the training data using SMOTE. The \classifier column in Table~\ref{table:classifier_table} demonstrates the prediction results per project. For instance, out of 310 flaky failures in \emph{Alluxio-alluxio}, the classifier correctly identified 281 as flaky (\textbf{TP}), while the remaining 29 failures were incorrectly predicted as non-flaky. To evaluate the performance of the classifier per project, I utilized precision, recall, and F1 score as evaluation metrics. To ensure that each testing dataset fold contains at least one flaky failure, I excluded projects with fewer than 10 instances of flaky failures, as compared to the projects analyzed in Section~\ref{sec:failureLogsStudy}.

% In order to obtain a deeper understanding of the effectiveness of applying machine learning concepts, I conducted a thorough investigation to identify a state-of-the-art classifier specifically designed for failure logs. However, to the best of my knowledge, there is no currently publicly available and accessible machine learning approach for predicting test flakiness based on failure logs. As an alternative, I employed TF-IDF as a baseline to compare the prediction results of the classifier. The motivation behind using this approach is that the features of the classifier are directly extracted from the syntax of the failure log, without any dynamic features.




% In the \classifier, there is a considerable number of true positives (714 out of 818) when predicting flaky failures. However, this also leads to a high number of false positives, where non-flaky failures are incorrectly identified as flaky. As a result, the classifier exhibits high recall across most of the project but has low precision rates.
% Upon conducting a thorough analysis, I discovered that the type of exceptions has a significant impact on the overall performance of the classifier. In projects with a substantial number of false positives (e.g., more than 100 cases, as seen in project \emph{activiti-activiti}), general exceptions such as \emph{AssertionError} or \emph{NullPointerException} make up the majority of the failures classified as both flaky and non-flaky.
% On the other hand, in projects with \textbf{no} false positives, like \emph{hector-client-hector}, almost all flaky failures are distinct from non-flaky failures in terms of exception types.
 

% The \classifier exhibits a notable occurrence of false positives. In this experiment, these false positives originated from mutation testing. Despite multiple runs to exclude flaky mutants, there remains a possibility of misclassifying non-flaky failures that happen to cluster with flaky failures ones. It's essential to recognize that real-world scenarios may differ from these experimental results.

% In the \tfidf approach, I noticed that the number of true positives (\textbf{TP}) is relatively low compared to the \classifier. However, there is a significant reduction in the false positive rates. This approach performs well in projects that have a reasonable number of flaky failures, except for the \alluxio project, where a large number of flaky failures are associated with NullPointerExceptions, leading to challenges in classification.
% This \tfidf approach exhibits poor performance in projects where flaky failures are presented in an assertion format, such as in the \websocket. The model struggles to accurately identify flaky failures in such scenarios.
% Overall, the \tfidf approach shows promise in projects with a moderate number of flaky failures but requires careful consideration and adjustments to handle specific patterns, such as NullPointer and Assertion exceptions to achieve better performance. 



% The usability of different machine learning approaches varies based on the specific use case and objectives. If the main goal is to maximize the number of true positives (\textbf{TP}) without being overly concerned about the rate of false positives, the \classifier could be a good fit. In this scenario, the model is more focused on correctly identifying as many flaky failures as possible, even if it means accepting a higher number of false positives.
% One of the main advantages of the \classifier is its flexibility in extending the learned features. The model can be easily augmented with additional static and dynamic features extracted from each failure. The proposed features shown in Table~\ref{table:Features} are not exhaustive but serve as a starting point, particularly utilizing information available within the failure log, such as the log syntax.
% By leveraging these additional features, the failure log classifier can potentially enhance its performance and accuracy in identifying flaky failures.


% \subsection{Summary}

% The primary insights from utilizing the \syntax reveal that it is feasible to distinguish between flaky and non-flaky failures by analyzing the failure logs. A main discovery is that these distinctions can be made by only looking at the exception type, as shown in \textbf{RQ2}. Utilizing machine learning for detecting flaky failures seems to be a viable approach as well. Observations from \textbf{RQ3} indicate that certain projects performs well with machine learning techniques. However, I noticed specific failure exceptions, like assertion failure exceptions, are challenging to leverage in identifying flaky failures due to their generality. These insights could be helpful in the analysis of failure logs and enhancing tools designed to utilize failure logs for flakiness detection.









% 
\begin{table*}
\centering
\caption{How often do flaky test failures match other flaky failures.}
% \textnormal{ 1) Total Flaky Tests and Failures shows the number of flaky tests, and the failure frequency of those flaky test.\\
% 2) Matching each test failure against other observed failures of the same test (in different test suite run). We bucket each test by the number of test suite run that it failed in, and show the number of tests that failed that many times (F), and the percentage of failures of those tests that matched at least one other failure of the same test (called Non-Unique (NU)). \textbf{S} refers to a failure that has one single occurrence because a test only fails once.\\
% 3) Matching each test failure against a DIFFERENT test failure in the same test suite run as that failure. We bucket each test suite run by the total number of flaky test failures observed in that build, showing the total number of flaky tests that failed in test suite runs of that bucket (F), and the percentage of failures that matched at least one other failure in the same test suite run (NU). \textbf{S} refers to a test suite run that has one single flaky failure.\\ 
% 4) We chose to opt out the projects that have one flaky test reported in studied dataset. }}
% \jon{Generated by MatchingRatesPerTest.ipynb}
\footnotesize
\setlength{\tabcolsep}{2pt}
\resizebox{\textwidth}{!}{%
\begin{tabular}{lrrrr|rrrrrrrrrrr|rrrrrrrrr}
\toprule
\multicolumn{1}{c}{ } & \multicolumn{4}{c}{ Total Flaky Tests and Failures} & \multicolumn{11}{c}{Same Test Different Builds} & \multicolumn{9}{c}{Different Tests Same Test Suite Run} \\
\cmidrule(l{3pt}r{3pt}){2-5} \cmidrule(l{3pt}r{3pt}){6-16} \cmidrule(l{3pt}r{3pt}){17-25}
\multicolumn{2}{c}{ } & \multicolumn{3}{c}{Failures Per-Test} & [1] & \multicolumn{2}{c}{[2,3)}& \multicolumn{2}{c}{[3,10)} & \multicolumn{2}{c}{[10,100)} & \multicolumn{2}{c}{[100,1000)} & \multicolumn{2}{c}{[1000,10000)} & [1] &\multicolumn{2}{c}{[2,3)} & \multicolumn{2}{c}{[3,10)} & \multicolumn{2}{c}{[10,20)} & \multicolumn{2}{c}{[20,200)} \\
% \cmidrule(l{3pt}r{3pt}){4-6} \cmidrule(l{3pt}r{3pt}){7-8}  \cmidrule(l{3pt}r{3pt}){9-10} \cmidrule(l{3pt}r{3pt}){11-12} \cmidrule(l{3pt}r{3pt}){13-14} \cmidrule(l{3pt}r{3pt}){15-16} \cmidrule(l{3pt}r{3pt}){17-18} \cmidrule(l{3pt}r{3pt}){19-20} \cmidrule(l{3pt}r{3pt}){21-22}
 & Tests & Min & Avg & Max & S &F & NU &F & NU & F & NU & F & NU & F & NU & S & F & NU & F & NU & F & NU & F & NU \\
\midrule
spring-projects-spring-boot&163&1&1753&5525&16& & &19&74\%&32&91\%& & &299&100\%&4441&36&0\%& & &13&46\%&49844&78\%\\
\cellcolor{gray!6}{apache-hbase}&\cellcolor{gray!6}{145}&\cellcolor{gray!6}{1}&\cellcolor{gray!6}{716}&\cellcolor{gray!6}{2011}&\cellcolor{gray!6}{2}&\cellcolor{gray!6}{21}&\cellcolor{gray!6}{90\%}&\cellcolor{gray!6}{16}&\cellcolor{gray!6}{100\%}&\cellcolor{gray!6}{33}&\cellcolor{gray!6}{88\%}&\cellcolor{gray!6}{37}&\cellcolor{gray!6}{70\%}&\cellcolor{gray!6}{109}&\cellcolor{gray!6}{92\%}&\cellcolor{gray!6}{1263}&\cellcolor{gray!6}{252}&\cellcolor{gray!6}{17\%}&\cellcolor{gray!6}{3624}&\cellcolor{gray!6}{5\%}&\cellcolor{gray!6}{39654}&\cellcolor{gray!6}{71\%}&\cellcolor{gray!6}{ }&\cellcolor{gray!6}{ }\\
Alluxio-alluxio&116&1&51&168&0& & & & &166&94\%&17&94\%& & &721&13&8\%&3&67\%&15&73\%&1168&88\%\\
\cellcolor{gray!6}{square-okhttp}&\cellcolor{gray!6}{100}&\cellcolor{gray!6}{1}&\cellcolor{gray!6}{234}&\cellcolor{gray!6}{8539}&\cellcolor{gray!6}{31}&\cellcolor{gray!6}{3}&\cellcolor{gray!6}{33\%}&\cellcolor{gray!6}{57}&\cellcolor{gray!6}{89\%}&\cellcolor{gray!6}{4}&\cellcolor{gray!6}{100\%}&\cellcolor{gray!6}{9}&\cellcolor{gray!6}{100\%}&\cellcolor{gray!6}{17}&\cellcolor{gray!6}{94\%}&\cellcolor{gray!6}{1152}&\cellcolor{gray!6}{5910}&\cellcolor{gray!6}{3\%}&\cellcolor{gray!6}{15835}&\cellcolor{gray!6}{25\%}&\cellcolor{gray!6}{39}&\cellcolor{gray!6}{51\%}&\cellcolor{gray!6}{ }&\cellcolor{gray!6}{ }\\
apache-ambari&52&1&77&875&1& & &2&100\%&48&100\%&2&100\%& & &921&16&0\%& & &202&64\%& & \\
\cellcolor{gray!6}{hector-client-hector}&\cellcolor{gray!6}{33}&\cellcolor{gray!6}{43}&\cellcolor{gray!6}{198}&\cellcolor{gray!6}{5147}&\cellcolor{gray!6}{0}&\cellcolor{gray!6}{ }&\cellcolor{gray!6}{ }&\cellcolor{gray!6}{ }&\cellcolor{gray!6}{ }&\cellcolor{gray!6}{32}&\cellcolor{gray!6}{100\%}&\cellcolor{gray!6}{ }&\cellcolor{gray!6}{ }&\cellcolor{gray!6}{1}&\cellcolor{gray!6}{100\%}&\cellcolor{gray!6}{5145}&\cellcolor{gray!6}{50}&\cellcolor{gray!6}{0\%}&\cellcolor{gray!6}{ }&\cellcolor{gray!6}{ }&\cellcolor{gray!6}{87}&\cellcolor{gray!6}{99\%}&\cellcolor{gray!6}{ }&\cellcolor{gray!6}{ }\\
activiti-activiti&32&1&42&932&12&2&100\%&4&100\%&14&93\%&1&100\%& & &1281&98&2\%&3&0\%& & & & \\
\cellcolor{gray!6}{tootallnate-java-websocket}&\cellcolor{gray!6}{23}&\cellcolor{gray!6}{1}&\cellcolor{gray!6}{48}&\cellcolor{gray!6}{215}&\cellcolor{gray!6}{1}&\cellcolor{gray!6}{ }&\cellcolor{gray!6}{ }&\cellcolor{gray!6}{2}&\cellcolor{gray!6}{50\%}&\cellcolor{gray!6}{29}&\cellcolor{gray!6}{100\%}&\cellcolor{gray!6}{13}&\cellcolor{gray!6}{100\%}&\cellcolor{gray!6}{ }&\cellcolor{gray!6}{ }&\cellcolor{gray!6}{1139}&\cellcolor{gray!6}{435}&\cellcolor{gray!6}{26\%}&\cellcolor{gray!6}{265}&\cellcolor{gray!6}{54\%}&\cellcolor{gray!6}{ }&\cellcolor{gray!6}{ }&\cellcolor{gray!6}{ }&\cellcolor{gray!6}{ }\\
wildfly-wildfly&23&1&5&41&13& & &9&100\%&1&100\%& & & & &47&8&0\%&28&50\%& & & & \\
\cellcolor{gray!6}{qos-ch-logback}&\cellcolor{gray!6}{22}&\cellcolor{gray!6}{1}&\cellcolor{gray!6}{185}&\cellcolor{gray!6}{3824}&\cellcolor{gray!6}{6}&\cellcolor{gray!6}{5}&\cellcolor{gray!6}{60\%}&\cellcolor{gray!6}{2}&\cellcolor{gray!6}{100\%}&\cellcolor{gray!6}{8}&\cellcolor{gray!6}{100\%}&\cellcolor{gray!6}{1}&\cellcolor{gray!6}{100\%}&\cellcolor{gray!6}{1}&\cellcolor{gray!6}{100\%}&\cellcolor{gray!6}{3952}&\cellcolor{gray!6}{308}&\cellcolor{gray!6}{0\%}&\cellcolor{gray!6}{6}&\cellcolor{gray!6}{0\%}&\cellcolor{gray!6}{ }&\cellcolor{gray!6}{ }&\cellcolor{gray!6}{ }&\cellcolor{gray!6}{ }\\
apache-httpcore&22&1&16&162&9&5&100\%&6&100\%& & &2&100\%& & &346&8&0\%& & & & & & \\
\cellcolor{gray!6}{apache-incubator-dubbo}&\cellcolor{gray!6}{19}&\cellcolor{gray!6}{1}&\cellcolor{gray!6}{462}&\cellcolor{gray!6}{8849}&\cellcolor{gray!6}{2}&\cellcolor{gray!6}{4}&\cellcolor{gray!6}{100\%}&\cellcolor{gray!6}{9}&\cellcolor{gray!6}{100\%}&\cellcolor{gray!6}{3}&\cellcolor{gray!6}{100\%}&\cellcolor{gray!6}{1}&\cellcolor{gray!6}{100\%}&\cellcolor{gray!6}{1}&\cellcolor{gray!6}{100\%}&\cellcolor{gray!6}{8563}&\cellcolor{gray!6}{654}&\cellcolor{gray!6}{0\%}&\cellcolor{gray!6}{9}&\cellcolor{gray!6}{22\%}&\cellcolor{gray!6}{ }&\cellcolor{gray!6}{ }&\cellcolor{gray!6}{ }&\cellcolor{gray!6}{ }\\
kevinsawicki-http-request&18&1&194&246&3& & & & & & &15&100\%& & &0& & &3&67\%&735&67\%& & \\
\cellcolor{gray!6}{wro4j-wro4j}&\cellcolor{gray!6}{16}&\cellcolor{gray!6}{1}&\cellcolor{gray!6}{474}&\cellcolor{gray!6}{1803}&\cellcolor{gray!6}{1}&\cellcolor{gray!6}{ }&\cellcolor{gray!6}{ }&\cellcolor{gray!6}{2}&\cellcolor{gray!6}{100\%}&\cellcolor{gray!6}{7}&\cellcolor{gray!6}{71\%}&\cellcolor{gray!6}{1}&\cellcolor{gray!6}{100\%}&\cellcolor{gray!6}{12}&\cellcolor{gray!6}{100\%}&\cellcolor{gray!6}{382}&\cellcolor{gray!6}{227}&\cellcolor{gray!6}{0\%}&\cellcolor{gray!6}{6285}&\cellcolor{gray!6}{44\%}&\cellcolor{gray!6}{44}&\cellcolor{gray!6}{27\%}&\cellcolor{gray!6}{ }&\cellcolor{gray!6}{ }\\
undertow-io-undertow&7&1&8&54&0&2&0\%&7&86\%&3&100\%& & & & &92& & & & & & & & \\
\cellcolor{gray!6}{orbit-orbit}&\cellcolor{gray!6}{7}&\cellcolor{gray!6}{7}&\cellcolor{gray!6}{420}&\cellcolor{gray!6}{2546}&\cellcolor{gray!6}{0}&\cellcolor{gray!6}{ }&\cellcolor{gray!6}{ }&\cellcolor{gray!6}{1}&\cellcolor{gray!6}{100\%}&\cellcolor{gray!6}{3}&\cellcolor{gray!6}{100\%}&\cellcolor{gray!6}{2}&\cellcolor{gray!6}{100\%}&\cellcolor{gray!6}{1}&\cellcolor{gray!6}{100\%}&\cellcolor{gray!6}{2721}&\cellcolor{gray!6}{214}&\cellcolor{gray!6}{1\%}&\cellcolor{gray!6}{5}&\cellcolor{gray!6}{20\%}&\cellcolor{gray!6}{ }&\cellcolor{gray!6}{ }&\cellcolor{gray!6}{ }&\cellcolor{gray!6}{ }\\
doanduyhai-Achilles&4&1&26&60&1& & & & &3&100\%& & & & &103&2&0\%& & & & & & \\
\cellcolor{gray!6}{elasticjob-elastic-job-lite}&\cellcolor{gray!6}{3}&\cellcolor{gray!6}{1}&\cellcolor{gray!6}{2}&\cellcolor{gray!6}{4}&\cellcolor{gray!6}{2}&\cellcolor{gray!6}{ }&\cellcolor{gray!6}{ }&\cellcolor{gray!6}{2}&\cellcolor{gray!6}{50\%}&\cellcolor{gray!6}{ }&\cellcolor{gray!6}{ }&\cellcolor{gray!6}{ }&\cellcolor{gray!6}{ }&\cellcolor{gray!6}{ }&\cellcolor{gray!6}{ }&\cellcolor{gray!6}{5}&\cellcolor{gray!6}{1}&\cellcolor{gray!6}{100\%}&\cellcolor{gray!6}{ }&\cellcolor{gray!6}{ }&\cellcolor{gray!6}{ }&\cellcolor{gray!6}{ }&\cellcolor{gray!6}{ }&\cellcolor{gray!6}{ }\\
alibaba-fastjson&3&4&49&121&0& & &2&100\%& & &2&100\%& & &191&4&0\%& & & & & & \\
\cellcolor{gray!6}{zxing-zxing}&\cellcolor{gray!6}{2}&\cellcolor{gray!6}{322}&\cellcolor{gray!6}{352}&\cellcolor{gray!6}{382}&\cellcolor{gray!6}{0}&\cellcolor{gray!6}{ }&\cellcolor{gray!6}{ }&\cellcolor{gray!6}{ }&\cellcolor{gray!6}{ }&\cellcolor{gray!6}{ }&\cellcolor{gray!6}{ }&\cellcolor{gray!6}{2}&\cellcolor{gray!6}{100\%}&\cellcolor{gray!6}{ }&\cellcolor{gray!6}{ }&\cellcolor{gray!6}{694}&\cellcolor{gray!6}{10}&\cellcolor{gray!6}{0\%}&\cellcolor{gray!6}{ }&\cellcolor{gray!6}{ }&\cellcolor{gray!6}{ }&\cellcolor{gray!6}{ }&\cellcolor{gray!6}{ }&\cellcolor{gray!6}{ }\\
\bottomrule
\label{tab:matchTable}
\end{tabular}
}
\end{table*}



% 
\section{Flaky Failure Logs Based Approaches}
\label{sec:failureLogsApproach}


Chapter \ref{sec:failureLogsStudy} explored the accuracy of identifying new flaky failures as similar to previous confirmed flaky failures, both within and across different tests in the same project. Assuming that developers label a failure as flaky due to specific signals present in the stack trace, I hypothesize that these signals are present in flaky failure logs and absent in non-flaky ones. Otherwise, this could lead to misleading results. To clarify, if two failure logs (one flaky and one non-flaky) are identical, this could indicate that one of them was misclassified, or that the failure exception and stack trace lines are not sufficient for distinguishing between the two types of failures. Through this approach, my goal is to assess the likelihood of detecting the signals that differentiate flaky failures from non-flaky ones based on the failure exception and stack trace lines.

To conduct this experiment, it is necessary to have access to logs of both flaky and non-flaky failures for the same tests. However, in datasets such as Deflaker \cite{bell2018deflaker} and iDFlakies \cite{lam2019idflakies}, there are no accompanying logs of non-flaky failures for the same flaky tests. Additionally, I am not aware of any available datasets that provide both types of failure logs for the same set of tests. In the previous section, I utilized my FlakeFlagger dataset, and one possible solution for obtaining non-flaky failures is to examine defects in the projects from this dataset. However, this approach may not be practical due to uncertainty regarding the number of collected failures and their non-flakiness status. Given the difficulty in obtaining all deterministic failures for a given test (as it is hard to anticipate all developer mistakes), a reasonable amount of deterministic failures per flaky test would suffice.

It is possible to obtain alternative sources of non-flaky failures by utilizing mutation testing to gather data on the failures of killed mutants. Just et al. have explored the idea of replacing real test failures with the failures of killed mutants \cite{just2014mutants}. The use of killed mutant failures can increase the likelihood of non-flaky failures, although it should be noted that not every killed mutant failure is necessarily non-flaky, as recent studies have shown that mutants can also exhibit flakiness \cite{shi2019mitigating}. To mitigate this issue, the approach of
Shi et al. has been used to filter out flaky mutants~\cite{shi2019mitigating}. I began by gathering mutants for each flaky test from FlakeFlagger dataset. For each test, mutants have been collected and executed 20 times in order to identify any potential flakiness. The failure messages and stack traces lines were recorded in a similar manner to the process outlined in Section~\ref{sec:failureLogsStudy}. I then updated the XML result file per test to include a list of mutant blocks. Each mutant block contains the failure exception, message, and stacktrace lines. 

To compare flaky failures with non-flaky ones for each test, a \syntax will be employed, similar to the approach used in chapter \ref{sec:failureLogsStudy} when comparing two flaky failures. This approach aims to capture any differences, such as in the stack trace line number, as different lines of code being executed can result in different stack traces. Along with the \syntax, I explore the possibility of using a machine learning approach to develop a classifier that can learn from flaky failure logs and predict the status of others. While the syntax approach works better for comparing failures at the test level (within the test), the generality of the machine learning approach motivated me to apply it as well.
% \jon{Before going into the two approaches, I suggest a 1-paragraph overview describing the different kinds of approaches that could be considered and their relative merits} \abdul{how does this sound?}\jon{Great!}

\begin{table*}[t]
    \caption{A list of features used to train the \classifier}
\label{table:Features}
\vspace{-5pt}
% \setlength{\tabcolsep}{2.5pt}
\newcommand{\failureRateWidth}{2.5in}
\newcommand{\failureRateHeight}{4em}
\scriptsize
\centering
    \begin{tabular}{l|c|l}
    \toprule     
     \textbf{Feature Name}&\textbf{Type}&\textbf{Description}\\
        \midrule
        Exception Type & Str & The name of the exception e.g. UnknownHostException \\
        Test name in Stacktrace & Boolean & \textit{True} if one of Stacktrace lines starts with the test name else \textit{False} \\
        Test Class name in Stacktrace & Boolean & \textit{True} if one of Stacktrace lines contains the test class name else \textit{False} \\
        Other Tests in Stacktrace & Boolean & \textit{True} if one of Stacktrace lines starts with other tests names else \textit{False} \\
        JUnit in Stacktrace & Boolean & \textit{True} if one of Stacktrace lines starts with any Junit Lines else \textit{False} \\
        CUT in Stacktrace & Boolean & \textit{True} if one of Stacktrace lines contains any lines from Code Under Test else \textit{False} \\
\bottomrule 
\end{tabular}
\vspace{-10pt}
\end{table*}






\subsection{Syntax Based Approach}

I utilize a syntax-matching method to compare the exception and stack trace lines of two failures. The approach is simply detecting any differences (including line numbers) between two given failures logs. I will be relying on the exception type and the stack trace lines, similar to the discussed technique in Section \ref{sec:failureLogsStudy} of the study. The first step involves grouping failures per test based on their exception and stack trace lines. Each group will then be labeled as either flaky (consisting of failures that are exclusively flaky), non-flaky (groups containing only non-flaky failures), or a combination of both (groups that include both flaky and non-flaky failures). 

The presence of flaky groups indicates that the failures within those groups do not match with any non-flaky failures. When a project has a higher number of flaky failure groups, this approach becomes valuable in distinguishing between flaky and non-flaky failures.
For instance, if a new failure falls into a flaky group, it is more likely to be flaky because it matches the patterns of other flaky failures and does not match those of non-flaky ones.
Moreover, applying this approach and analyzing the resulting groups can provide insights into the characteristics that differentiate flaky failures from non-flaky ones. When a group is labeled as a \emph{flaky group}, it means that the failure contains \emph{at least} one stack trace line that has never been observed in any of the non-flaky failures. This discovery raises suspicion and points to a possible link to the root cause of flakiness.
\abdul{Do you think this should be well explained e.g. with more details and examples.}
\jon{Yes, I think that a few ``For example" sentences here would be helpful.}\abdul{How about now?}


\subsection{Failure Log Classifier}
The earlier approach involves comparing the syntax of two failures, including the line number in the stack trace, which may not be suitable for comparing failures from different tests. Assuming that flaky failures exhibit differences compared to non-flaky ones, could a classifier be developed to predict if a failure is likely to be flaky based on other flaky failures from various tests?

My proposal involves a failure log classifier, which employs a machine learning approach to learn from both flaky and non-flaky failure logs, enabling it to predict the status of a given failure log as either flaky or not. The classifier gathers specific features from the failure logs, which are outlined in Table~\ref{table:Features}. To ensure that the classifier can learn from various tests, the selected features should encompass all tests and not be influenced by the content of a particular test, such as whether the stack trace lines cover any line in the test suite rather than the test itself. Although the initial set of features is not final, I believe they are sufficient to begin training the classifier.

To begin, I processed the data by extracting features from the failure exception and stack trace lines. These features require knowledge of all test names in the test suite, test names throughout the entire project, and all source code file names of the code under test to facilitate determining the feature values for each test failure. Next, I employed a simple \emph{Decision Tree} (\textbf{DT}) as the supervised learning algorithm and utilized stratified cross-validation to train on a portion of the data and predict the remaining. Additionally, I used SMOTE to balance the data due to its imbalanced nature. To evaluate the performance of the log classifier, I applied \emph{TF-IDF} (Term Frequency-Inverse Document Frequency~\cite{tfidf}), as a baseline for prediction results.


\subsection{Failure logs Approaches: Initial Result}

I am looking to emphasize the main findings of using the proposed approaches by answering the following questions:

\begin{description}
  \item[\textbf{RQ1:}] Is the \syntax able to discriminate the flaky failures?
  \item[\textbf{RQ2:}] Are some exceptions related to flakiness more than non flaky failures?
  \item[\textbf{RQ3:}] Can machine learning be utilized to predict flaky failures using the failure logs?

 \end{description}

% \begin{table*}[t]
%     \caption{caption X}
%     % by Test -> How many flaky tests in a certain project? Show number of tests that flake exactly once, then the minimum number of flakes per-test, max, and a sparkline
%     % by Build -> How many builds have at least one flaky test? How many builds iwht exactly one flaky test, then min, max, and distribution of flaky tests 
% \label{table:classifier_table}
% \vspace{-5pt}
% \setlength{\tabcolsep}{2.5pt}
% \newcommand{\failureRateWidth}{2.5in}
% \newcommand{\failureRateHeight}{4em}
% \scriptsize
% \centering
%     \begin{tabular}{l|ccc|cc|rrrrrrr|rrrrrrr}
%     \toprule
%       & \multicolumn{3}{c}{\textbf{Text-Match-Approach}} & \multicolumn{2}{c}{\textbf{Text-Match-Approach}} & \multicolumn{7}{c}{\textbf{Failure Log Classifier}} & \multicolumn{7}{c}{\textbf{TF-IDF}}  \\ 
     
%      \textbf{Project}&\textbf{Failures}&\textbf{Flaky}&\textbf{Non-Flaky}&\textbf{UNMATCH}&\textbf{MATCH}&\textbf{TP}&\textbf{FN}&\textbf{FP}&\textbf{TN}&\textbf{P}&\textbf{R}&\textbf{F1}&\textbf{TP}&\textbf{FN}&\textbf{FP}&\textbf{TN}&\textbf{P}&\textbf{R}&\textbf{F1}\\
%         \midrule

% java-webSocket&3276&1977&1299&42&3&1942&35&854&445&69\%&98\%&81\%&1855&122&1158&141&62\%&94\%&74\%\\
% assertj-core&30&13&17&1&0&9&4&1&16&90\%&69\%&78\%&10&3&1&16&91\%&77\%&83\%\\
% % ninja&319&110&209&NA&NA&110&0&90&119&55\%&100\%&71\%&110&0&92&117&54\%&100\%&71\%\\
% orbit&1035&213&822&2&5&202&11&69&753&75\%&95\%&83\%&108&105&236&586&31\%&51\%&39\%\\
% % handlebars.java&183&36&147&NA&NA&36&0&16&131&69\%&100\%&82\%&36&0&16&131&69\%&100\%&82\%\\
% achilles&499&57&442&2&3&54&3&61&381&47\%&95\%&63\%&39&18&105&337&27\%&68\%&39\%\\
% logback&2948&334&2614&7&14&298&34&496&2118&38\%&90\%&53\%&186&146&34&2580&85\%&56\%&67\%\\
% okhttp&36685&2419&34266&105&16&2261&158&3019&31247&43\%&93\%&59\%&717&1702&9968&24298&7\%&30\%&11\%\\
% wro4j&576&36&540&12&9&26&8&45&495&37\%&76\%&50\%&23&11&14&526&62\%&68\%&65\%\\
% activiti&48271&2171&46100&5&28&2136&35&7856&38244&21\%&98\%&35\%&480&1691&5544&40556&8\%&22\%&12\%\\
% http-request&405&18&387&7&11&14&4&113&274&11\%&78\%&19\%&18&0&0&387&100\%&100\%&100\%\\
% hbase&11895&571&11324&130&4&425&26&478&10846&47\%&94\%&63\%&269&182&2146&9178&11\%&60\%&19\%\\
% alluxio&33317&709&32608&138&172&630&74&6996&25612&8\%&89\%&15\%&385&319&5649&26959&6\%&55\%&11\%\\
% httpcore&8426&93&8333&1&21&52&41&1789&6544&3\%&56\%&5\%&39&54&2049&6284&2\%&42\%&4\%\\
% hector&3638&34&3604&31&2&31&3&0&3604&100\%&91\%&95\%&33&1&0&3604&100\%&97\%&99\%\\
% % io-undertow&2325&21&2304&NA&NA&17&4&1388&916&1\%&81\%&2\%&19&2&1719&585&1\%&90\%&2\%\\
% spring-boot&2503&353&2150&13&0&5&9&516&1634&1\%&36\%&2\%&12&2&0&2150&100\%&86\%&92\%\\
% ambari&11106&57&11049&52&2&49&7&439&10610&10\%&88\%&18\%&51&5&119&10930&30\%&91\%&45\%\\
% wildfly&3863&16&3847&18&0&16&0&116&3731&12\%&100\%&22\%&16&0&0&3847&100\%&100\%&100\%\\
    

% \bottomrule 
% \end{tabular}
% \vspace{-10pt}
% \end{table*}


\begin{table*}[t]
% \jon{New columns: Flaky tests, Flaky Failures, Non-Flaky Failures, Group of 3 columns with header "Synatx-based approach": (Flaky only, Both, Non-Flaky Only), Failure log classifier, TF-IDF}
% \jon{Remove "Killed Mutant statistics" and instead do some analysis to determine how many flaky failures have fewer than N (N=1,2,3?) mutants to match against, and comment on these (and whether or not they match)}
\caption[The Prediction of \classifier and TF-IDF of Flaky and True Failures]{The Result of \classifier and TF-IDF of Flaky and True Failures Prediction?\\
\textnormal{The \classifier and TF-IDF show (per project) the confusion matrix, precision (P), recall (R), and F1 score of the overall prediction result. 
% Compared to Table\ref{nonunique}, projects with fewer than 10 flaky failures have been opted out.
}}
    % by Test -> How many flaky tests in a certain project? Show number of tests that flake exactly once, then the minimum number of flakes per-test, max, and a sparkline
    % by Build -> How many builds have at least one flaky test? How many builds iwht exactly one flaky test, then min, max, and distribution of flaky tests 
\label{table:classifier_table}
\vspace{-5pt}
\setlength{\tabcolsep}{1.0pt}
\newcommand{\failureRateWidth}{2.5in}
\newcommand{\failureRateHeight}{4em}
\scriptsize
\centering
    \begin{tabular}{l|rrrr|rrrrrrr|rrrrrrr}
    \toprule
      & \multicolumn{4}{c}{\textbf{Total Flaky Tests and Failures}} & \multicolumn{7}{c}{\textbf{Failure Log Classifier}} & \multicolumn{7}{c}{\textbf{TF-IDF}}\\ 
     
     \textbf{Project}&\textbf{Test}&\textbf{Failures}&\textbf{Flaky}&\textbf{True}&\textbf{TP}&\textbf{FN}&\textbf{FP}&\textbf{TN}&\textbf{P}&\textbf{R}&\textbf{F1}&\textbf{TP}&\textbf{FN}&\textbf{FP}&\textbf{TN}&\textbf{P}&\textbf{R}&\textbf{F1}\\
        \midrule

Alluxio-alluxio&114&49,466&16,858&32,608&16,014&844&1,104&31,504&93\%&94\%&94\%&16,580&278&394&32,214&97\%&98\%&98\%\\
\cellcolor{gray!6}{square-okhttp}&\cellcolor{gray!6}{100}&\cellcolor{gray!6}{62,530}&\cellcolor{gray!6}{28,264}&\cellcolor{gray!6}{34,266}&\cellcolor{gray!6}{28,123}&\cellcolor{gray!6}{141}&\cellcolor{gray!6}{1,585}&\cellcolor{gray!6}{32,681}&\cellcolor{gray!6}{94\%}&\cellcolor{gray!6}{99\%}&\cellcolor{gray!6}{97\%}&\cellcolor{gray!6}{28,238}&\cellcolor{gray!6}{26}&\cellcolor{gray!6}{108}&\cellcolor{gray!6}{34,158}&\cellcolor{gray!6}{99\%}&\cellcolor{gray!6}{99\%}&\cellcolor{gray!6}{99\%}\\
apache-hbase&62&31,146&19,822&11,324&19,782&40&369&10,955&98\%&99\%&98\%&19,676&146&19&11,305&99\%&99\%&99\%\\
\cellcolor{gray!6}{apache-ambari}&\cellcolor{gray!6}{51}&\cellcolor{gray!6}{15,112}&\cellcolor{gray!6}{4,063}&\cellcolor{gray!6}{11,049}&\cellcolor{gray!6}{4,055}&\cellcolor{gray!6}{8}&\cellcolor{gray!6}{482}&\cellcolor{gray!6}{10,567}&\cellcolor{gray!6}{89\%}&\cellcolor{gray!6}{99\%}&\cellcolor{gray!6}{94\%}&\cellcolor{gray!6}{4,063}&\cellcolor{gray!6}{0}&\cellcolor{gray!6}{5}&\cellcolor{gray!6}{11,044}&\cellcolor{gray!6}{99\%}&\cellcolor{gray!6}{100\%}&\cellcolor{gray!6}{99\%}\\
Hector&33&10,133&6,529&3,604&6,529&0&405&3,199&94\%&100\%&96\%&6,529&0&13&3,591&99\%&100\%&99\%\\
\cellcolor{gray!6}{activiti-activiti}&\cellcolor{gray!6}{31}&\cellcolor{gray!6}{47,478}&\cellcolor{gray!6}{1,378}&\cellcolor{gray!6}{46,100}&\cellcolor{gray!6}{947}&\cellcolor{gray!6}{431}&\cellcolor{gray!6}{311}&\cellcolor{gray!6}{45,789}&\cellcolor{gray!6}{75\%}&\cellcolor{gray!6}{68\%}&\cellcolor{gray!6}{71\%}&\cellcolor{gray!6}{1,013}&\cellcolor{gray!6}{365}&\cellcolor{gray!6}{60}&\cellcolor{gray!6}{46,040}&\cellcolor{gray!6}{94\%}&\cellcolor{gray!6}{73\%}&\cellcolor{gray!6}{82\%}\\
apache-httpcore&22&8,687&354&8,333&315&39&110&8,223&74\%&88\%&80\%&314&40&16&8,317&95\%&88\%&91\%\\
\cellcolor{gray!6}{Java-websocket}&\cellcolor{gray!6}{22}&\cellcolor{gray!6}{3,394}&\cellcolor{gray!6}{2,095}&\cellcolor{gray!6}{1,299}&\cellcolor{gray!6}{2,082}&\cellcolor{gray!6}{13}&\cellcolor{gray!6}{721}&\cellcolor{gray!6}{578}&\cellcolor{gray!6}{74\%}&\cellcolor{gray!6}{99\%}&\cellcolor{gray!6}{85\%}&\cellcolor{gray!6}{2,082}&\cellcolor{gray!6}{13}&\cellcolor{gray!6}{722}&\cellcolor{gray!6}{577}&\cellcolor{gray!6}{74\%}&\cellcolor{gray!6}{99\%}&\cellcolor{gray!6}{84\%}\\
qos-ch-logback&20&3,052&438&2,614&172&266&104&2,510&62\%&39\%&48\%&239&199&41&2,573&85\%&54\%&66\%\\
\cellcolor{gray!6}{Http-request}&\cellcolor{gray!6}{18}&\cellcolor{gray!6}{3,888}&\cellcolor{gray!6}{3,501}&\cellcolor{gray!6}{387}&\cellcolor{gray!6}{3,498}&\cellcolor{gray!6}{3}&\cellcolor{gray!6}{124}&\cellcolor{gray!6}{263}&\cellcolor{gray!6}{96\%}&\cellcolor{gray!6}{99\%}&\cellcolor{gray!6}{98\%}&\cellcolor{gray!6}{3,498}&\cellcolor{gray!6}{3}&\cellcolor{gray!6}{54}&\cellcolor{gray!6}{333}&\cellcolor{gray!6}{98\%}&\cellcolor{gray!6}{99\%}&\cellcolor{gray!6}{99\%}\\
wildfly-wildfly&18&3,895&48&3,847&0&48&0&3,847&0\%&0\%&0\%&48&0&0&3,847&100\%&100\%&100\%\\
\cellcolor{gray!6}{wro4j-wro4j}&\cellcolor{gray!6}{14}&\cellcolor{gray!6}{11,373}&\cellcolor{gray!6}{10,833}&\cellcolor{gray!6}{540}&\cellcolor{gray!6}{10,833}&\cellcolor{gray!6}{0}&\cellcolor{gray!6}{65}&\cellcolor{gray!6}{475}&\cellcolor{gray!6}{99\%}&\cellcolor{gray!6}{100\%}&\cellcolor{gray!6}{99\%}&\cellcolor{gray!6}{10,833}&\cellcolor{gray!6}{0}&\cellcolor{gray!6}{29}&\cellcolor{gray!6}{511}&\cellcolor{gray!6}{99\%}&\cellcolor{gray!6}{100\%}&\cellcolor{gray!6}{99\%}\\
Spring-boot&12&2,164&14&2,150&6&8&0&2,150&100\%&42\%&60\%&10&4&1&2,149&90\%&71\%&80\%\\
\cellcolor{gray!6}{orbit-orbit}&\cellcolor{gray!6}{7}&\cellcolor{gray!6}{3,765}&\cellcolor{gray!6}{2,943}&\cellcolor{gray!6}{822}&\cellcolor{gray!6}{2,943}&\cellcolor{gray!6}{0}&\cellcolor{gray!6}{69}&\cellcolor{gray!6}{753}&\cellcolor{gray!6}{97\%}&\cellcolor{gray!6}{100\%}&\cellcolor{gray!6}{98\%}&\cellcolor{gray!6}{2,943}&\cellcolor{gray!6}{0}&\cellcolor{gray!6}{59}&\cellcolor{gray!6}{763}&\cellcolor{gray!6}{98\%}&\cellcolor{gray!6}{100\%}&\cellcolor{gray!6}{99\%}\\
Undertow&7&2,396&92&2,304&3&89&0&2,304&100\%&3\%&6\%&5&87&0&2,304&100\%&5\%&10\%\\
\cellcolor{gray!6}{Achilles}&\cellcolor{gray!6}{4}&\cellcolor{gray!6}{607}&\cellcolor{gray!6}{165}&\cellcolor{gray!6}{442}&\cellcolor{gray!6}{120}&\cellcolor{gray!6}{45}&\cellcolor{gray!6}{0}&\cellcolor{gray!6}{442}&\cellcolor{gray!6}{100\%}&\cellcolor{gray!6}{72\%}&\cellcolor{gray!6}{84\%}&\cellcolor{gray!6}{148}&\cellcolor{gray!6}{17}&\cellcolor{gray!6}{26}&\cellcolor{gray!6}{416}&\cellcolor{gray!6}{85\%}&\cellcolor{gray!6}{89\%}&\cellcolor{gray!6}{87\%}\\
% elasticjob-elastic-job-lite&3&0&0&0&0&0&0&0&0\%&0\%&0\%&0&0&0&0&0\%&0\%&0\%\\
Commons-exec&1&92&33&59&0&33&0&59&0\%&0\%&0\%&33&0&2&57&94\%&100\%&97\%\\
\cellcolor{gray!6}{zxing-zxing}&\cellcolor{gray!6}{1}&\cellcolor{gray!6}{398}&\cellcolor{gray!6}{322}&\cellcolor{gray!6}{76}&\cellcolor{gray!6}{322}&\cellcolor{gray!6}{0}&\cellcolor{gray!6}{0}&\cellcolor{gray!6}{76}&\cellcolor{gray!6}{100\%}&\cellcolor{gray!6}{100\%}&\cellcolor{gray!6}{100\%}&\cellcolor{gray!6}{322}&\cellcolor{gray!6}{0}&\cellcolor{gray!6}{0}&\cellcolor{gray!6}{76}&\cellcolor{gray!6}{100\%}&\cellcolor{gray!6}{100\%}&\cellcolor{gray!6}{100\%}\\
handlebars.java&1&558&411&147&411&0&16&131&96\%&100\%&98\%&411&0&16&131&96\%&100\%&98\%\\
\cellcolor{gray!6}{assertj-core}&\cellcolor{gray!6}{1}&\cellcolor{gray!6}{991}&\cellcolor{gray!6}{974}&\cellcolor{gray!6}{17}&\cellcolor{gray!6}{974}&\cellcolor{gray!6}{0}&\cellcolor{gray!6}{1}&\cellcolor{gray!6}{16}&\cellcolor{gray!6}{99\%}&\cellcolor{gray!6}{100\%}&\cellcolor{gray!6}{99\%}&\cellcolor{gray!6}{974}&\cellcolor{gray!6}{0}&\cellcolor{gray!6}{0}&\cellcolor{gray!6}{17}&\cellcolor{gray!6}{100\%}&\cellcolor{gray!6}{100\%}&\cellcolor{gray!6}{100\%}\\
ninja-ninja&1&685&476&209&476&0&90&119&84\%&100\%&91\%&476&0&90&119&84\%&100\%&91\%\\
\midrule
21 Projects Total &540&261,810&99,613&162,197&97,605&2,008&5,556&156,641&&&&98,435&1,178&1,655&160,542&&&\\

\bottomrule


% Alluxio-alluxio&114&49,466&16,858&32,608&16,014&844&1,104&31,504&93\%&94\%&94\%&16,565&293&389&32,219&97\%&98\%&97\%\\
% \cellcolor{gray!6}{square-okhttp}&\cellcolor{gray!6}{100}&\cellcolor{gray!6}{62,530}&\cellcolor{gray!6}{28,264}&\cellcolor{gray!6}{34,266}&\cellcolor{gray!6}{28,123}&\cellcolor{gray!6}{141}&\cellcolor{gray!6}{1,585}&\cellcolor{gray!6}{32,681}&\cellcolor{gray!6}{94\%}&\cellcolor{gray!6}{99\%}&\cellcolor{gray!6}{97\%}&\cellcolor{gray!6}{28,239}&\cellcolor{gray!6}{25}&\cellcolor{gray!6}{108}&\cellcolor{gray!6}{34,158}&\cellcolor{gray!6}{99\%}&\cellcolor{gray!6}{99\%}&\cellcolor{gray!6}{99\%}\\
% apache-hbase&62&31,174&19,850&11,324&19,809&41&369&10,955&98\%&99\%&98\%&19,676&146&19&11,305&99\%&99\%&99\%\\
% \cellcolor{gray!6}{apache-ambari}&\cellcolor{gray!6}{51}&\cellcolor{gray!6}{15,112}&\cellcolor{gray!6}{4,063}&\cellcolor{gray!6}{11,049}&\cellcolor{gray!6}{4,055}&\cellcolor{gray!6}{8}&\cellcolor{gray!6}{481}&\cellcolor{gray!6}{10,568}&\cellcolor{gray!6}{89\%}&\cellcolor{gray!6}{99\%}&\cellcolor{gray!6}{94\%}&\cellcolor{gray!6}{4,063}&\cellcolor{gray!6}{0}&\cellcolor{gray!6}{5}&\cellcolor{gray!6}{11,044}&\cellcolor{gray!6}{99\%}&\cellcolor{gray!6}{100\%}&\cellcolor{gray!6}{99\%}\\
% hector-client-hector&33&10,133&6,529&3,604&6,529&0&405&3,199&94\%&100\%&96\%&6,529&0&12&3,592&99\%&100\%&99\%\\
% \cellcolor{gray!6}{activiti-activiti}&\cellcolor{gray!6}{31}&\cellcolor{gray!6}{47,478}&\cellcolor{gray!6}{1,378}&\cellcolor{gray!6}{46,100}&\cellcolor{gray!6}{1,303}&\cellcolor{gray!6}{75}&\cellcolor{gray!6}{5,421}&\cellcolor{gray!6}{40,679}&\cellcolor{gray!6}{19\%}&\cellcolor{gray!6}{94\%}&\cellcolor{gray!6}{32\%}&\cellcolor{gray!6}{1,038}&\cellcolor{gray!6}{340}&\cellcolor{gray!6}{70}&\cellcolor{gray!6}{46,030}&\cellcolor{gray!6}{93\%}&\cellcolor{gray!6}{75\%}&\cellcolor{gray!6}{83\%}\\
% apache-httpcore&22&8,687&354&8,333&321&33&115&8,218&73\%&90\%&81\%&314&40&17&8,316&94\%&88\%&91\%\\
% \cellcolor{gray!6}{tootallnate-java-websocket}&\cellcolor{gray!6}{22}&\cellcolor{gray!6}{3,394}&\cellcolor{gray!6}{2,095}&\cellcolor{gray!6}{1,299}&\cellcolor{gray!6}{2,082}&\cellcolor{gray!6}{13}&\cellcolor{gray!6}{721}&\cellcolor{gray!6}{578}&\cellcolor{gray!6}{74\%}&\cellcolor{gray!6}{99\%}&\cellcolor{gray!6}{85\%}&\cellcolor{gray!6}{2,082}&\cellcolor{gray!6}{13}&\cellcolor{gray!6}{722}&\cellcolor{gray!6}{577}&\cellcolor{gray!6}{74\%}&\cellcolor{gray!6}{99\%}&\cellcolor{gray!6}{84\%}\\
% qos-ch-logback&20&3,052&438&2,614&172&266&104&2,510&62\%&39\%&48\%&245&193&49&2,565&83\%&55\%&66\%\\
% \cellcolor{gray!6}{kevinsawicki-http-request}&\cellcolor{gray!6}{18}&\cellcolor{gray!6}{3,888}&\cellcolor{gray!6}{3,501}&\cellcolor{gray!6}{387}&\cellcolor{gray!6}{3,498}&\cellcolor{gray!6}{3}&\cellcolor{gray!6}{124}&\cellcolor{gray!6}{263}&\cellcolor{gray!6}{96\%}&\cellcolor{gray!6}{99\%}&\cellcolor{gray!6}{98\%}&\cellcolor{gray!6}{3,498}&\cellcolor{gray!6}{3}&\cellcolor{gray!6}{54}&\cellcolor{gray!6}{333}&\cellcolor{gray!6}{98\%}&\cellcolor{gray!6}{99\%}&\cellcolor{gray!6}{99\%}\\
% wildfly-wildfly&18&3,895&48&3,847&48&0&116&3,731&29\%&100\%&45\%&48&0&0&3,847&100\%&100\%&100\%\\
% \cellcolor{gray!6}{wro4j-wro4j}&\cellcolor{gray!6}{14}&\cellcolor{gray!6}{11,373}&\cellcolor{gray!6}{10,833}&\cellcolor{gray!6}{540}&\cellcolor{gray!6}{10,474}&\cellcolor{gray!6}{359}&\cellcolor{gray!6}{42}&\cellcolor{gray!6}{498}&\cellcolor{gray!6}{99\%}&\cellcolor{gray!6}{96\%}&\cellcolor{gray!6}{98\%}&\cellcolor{gray!6}{10,833}&\cellcolor{gray!6}{0}&\cellcolor{gray!6}{29}&\cellcolor{gray!6}{511}&\cellcolor{gray!6}{99\%}&\cellcolor{gray!6}{100\%}&\cellcolor{gray!6}{99\%}\\
% spring-projects-spring-boot&12&2,164&14&2,150&13&1&850&1,300&1\%&92\%&2\%&10&4&2&2,148&83\%&71\%&76\%\\
% \cellcolor{gray!6}{orbit-orbit}&\cellcolor{gray!6}{7}&\cellcolor{gray!6}{3,765}&\cellcolor{gray!6}{2,943}&\cellcolor{gray!6}{822}&\cellcolor{gray!6}{2,943}&\cellcolor{gray!6}{0}&\cellcolor{gray!6}{69}&\cellcolor{gray!6}{753}&\cellcolor{gray!6}{97\%}&\cellcolor{gray!6}{100\%}&\cellcolor{gray!6}{98\%}&\cellcolor{gray!6}{2,940}&\cellcolor{gray!6}{3}&\cellcolor{gray!6}{59}&\cellcolor{gray!6}{763}&\cellcolor{gray!6}{98\%}&\cellcolor{gray!6}{99\%}&\cellcolor{gray!6}{98\%}\\
% undertow-io-undertow&7&2,396&92&2,304&57&35&148&2,156&27\%&61\%&38\%&5&87&0&2,304&100\%&5\%&10\%\\
% \cellcolor{gray!6}{doanduyhai-Achilles}&\cellcolor{gray!6}{4}&\cellcolor{gray!6}{607}&\cellcolor{gray!6}{165}&\cellcolor{gray!6}{442}&\cellcolor{gray!6}{120}&\cellcolor{gray!6}{45}&\cellcolor{gray!6}{0}&\cellcolor{gray!6}{442}&\cellcolor{gray!6}{100\%}&\cellcolor{gray!6}{72\%}&\cellcolor{gray!6}{84\%}&\cellcolor{gray!6}{148}&\cellcolor{gray!6}{17}&\cellcolor{gray!6}{28}&\cellcolor{gray!6}{414}&\cellcolor{gray!6}{84\%}&\cellcolor{gray!6}{89\%}&\cellcolor{gray!6}{86\%}\\
% \cellcolor{gray!6}{zxing-zxing}&\cellcolor{gray!6}{1}&\cellcolor{gray!6}{398}&\cellcolor{gray!6}{322}&\cellcolor{gray!6}{76}&\cellcolor{gray!6}{322}&\cellcolor{gray!6}{0}&\cellcolor{gray!6}{0}&\cellcolor{gray!6}{76}&\cellcolor{gray!6}{100\%}&\cellcolor{gray!6}{100\%}&\cellcolor{gray!6}{100\%}&\cellcolor{gray!6}{322}&\cellcolor{gray!6}{0}&\cellcolor{gray!6}{0}&\cellcolor{gray!6}{76}&\cellcolor{gray!6}{100\%}&\cellcolor{gray!6}{100\%}&\cellcolor{gray!6}{100\%}\\
% assertj-core&1&991&974&17&974&0&1&16&99\%&100\%&99\%&974&0&0&17&100\%&100\%&100\%\\
% \cellcolor{gray!6}{apache-commons-exec}&\cellcolor{gray!6}{1}&\cellcolor{gray!6}{92}&\cellcolor{gray!6}{33}&\cellcolor{gray!6}{59}&\cellcolor{gray!6}{0}&\cellcolor{gray!6}{33}&\cellcolor{gray!6}{0}&\cellcolor{gray!6}{59}&\cellcolor{gray!6}{0\%}&\cellcolor{gray!6}{0\%}&\cellcolor{gray!6}{0\%}&\cellcolor{gray!6}{33}&\cellcolor{gray!6}{0}&\cellcolor{gray!6}{2}&\cellcolor{gray!6}{57}&\cellcolor{gray!6}{94\%}&\cellcolor{gray!6}{100\%}&\cellcolor{gray!6}{97\%}\\
% ninja-ninja&1&685&476&209&476&0&90&119&84\%&100\%&91\%&476&0&90&119&84\%&100\%&91\%\\
% \cellcolor{gray!6}{handlebars.java}&\cellcolor{gray!6}{1}&\cellcolor{gray!6}{558}&\cellcolor{gray!6}{411}&\cellcolor{gray!6}{147}&\cellcolor{gray!6}{411}&\cellcolor{gray!6}{0}&\cellcolor{gray!6}{16}&\cellcolor{gray!6}{131}&\cellcolor{gray!6}{96\%}&\cellcolor{gray!6}{100\%}&\cellcolor{gray!6}{98\%}&\cellcolor{gray!6}{411}&\cellcolor{gray!6}{0}&\cellcolor{gray!6}{16}&\cellcolor{gray!6}{131}&\cellcolor{gray!6}{96\%}&\cellcolor{gray!6}{100\%}&\cellcolor{gray!6}{98\%}\\
% \midrule
% Total&543&261,838&99,641&162,197&97,744&1,897&11,761&150,436&&&&98,449&1,164&1,671&160,526&&&\\

% \bottomrule



% \bottomrule
\end{tabular}
\vspace{-10pt}
\end{table*}








I am looking to evaluate the effectiveness of the \syntax as a method for detecting test flakiness. By distinguishing flaky failures based on their failure exception and stack trace lines, developers can employ this approach to determine if a failure is flaky or not by comparing it with non-flaky failures. Additionally, I am interested in exploring how machine learning can leverage failure logs to construct a classifier that predicts the likelihood of a failure being flaky. 

\subsubsection{RQ1: Is the \syntax able to discriminate the flaky failures?}

Table~\ref{table:classifier_table} illustrates the outcomes of applying the \syntax to flaky and non-flaky failures in different projects. The column labeled \textit{Failure by \syntax} is divided into three parts: \textit{OnlyFlaky}, \textit{Only Non-Flaky}, and \textit{Both}. The \textit{OnlyFlaky} column represents failure logs that are exclusively observed in flaky failures, while the \textit{Only Non-Flaky} column represents failures that are only present in non-flaky failures. The \textit{Both} column captures failure logs that are common to both flaky and non-flaky failures. If the number of occurrences in the \textit{Both} column is minimized, it suggests that the \syntax is more likely to be considered effective.

The outcomes of the \syntax exhibit variations among the studied projects. Among the 15 projects analyzed, there are six projects in which at least 86\% of their flaky failures exclusively appear in the \emph{OnlyFlaky} category. Conversely, there are projects where this approach did not perform effectively, such as in the case of the \httpcore project, where it successfully distinguishes only one flaky failures. It is notable that the number of flaky failures is not a common key of these projects, as this approach perform well regardless to the number of flaky failures. These findings motivate further analysis of the failure exceptions and identification of common patterns across these projects.

The primary observation in projects where it is challenging to distinguish flaky failures using the \syntax is that these failures often manifest as assertion exceptions. For instance, in both the \emph{java-webscoket} and \emph{http-request} projects, all the failures in the column \emph{Both} share the \textit{AssertionError} exception. Similarly, the majority of failures in projects like \emph{qos-ch-logback} and \emph{activiti} also fall under the \textit{AssertionError} exception.
Conversely, in projects where flaky failures are exclusively present, the majority of failures are characterized by different exceptions, such as \emph{UnknownHostException} and \emph{IOException}. 
In the context of the \emph{Alluxio} project, there are 164 failure groups (out of a total of 174 groups containing both flaky and non-flaky failures) that are attributed to a specific exception type: \emph{NullPointerException}. Despite having access to the stack trace lines, these exceptions prove to be challenging to distinguish, highlighting the difficulty of utilizing failure logs as a criteria for such exception types. It is worth noting that this exception is not detected in failure logs of other projects where it is not reported as an exception in only flaky failures.

In general, leveraging the exception type and stack trace lines is a promising approach for developers and researchers to analyze failure logs and distinguish between flaky and non-flaky failures. When comparing flaky failures to non-flaky failures, focusing on the stack traces alone can often be effective without explicitly relying on the exception type. This is evident from the discovery that all failures identified as unique based on the exception types are also unique based on their stack trace lines. However, incorporating the exception types can enhance the analysis and provide supplementary information to aid in the classification of failure logs. The use of exception types will be further explored and discussed in detail in RQ3.

\subsubsection{RQ2: Are some exceptions related to flakiness more than non flaky failures?}

The column \emph{OnlyFlaky} in Table~\ref{table:classifier_table} illustrates the existence of groups of failures that are only appear in flaky side and never been detected in non-flaky failures.
Among these failures, there are 284 failures groups where the exceptions alone never been reported in the non flaky failures of the same test, regardless of using stack trace lines in the comparison, which accounts for 55\% of the these failures. In other words, these exceptions have never been detected in the non-flaky failures obtained from the corresponding tests associated with these flaky failures.


Table~\ref{table:exceptions} presents the top ten most frequently occurring exceptions observed in the analyzed flaky failures. Among them, two exceptions, namely \emph{UnknownHostException} and \emph{HCassandraInternalException}, consistently distinguish the failures from the non-flaky failures. Approximately 93\% of the failures involve the \emph{IOException} exception, while around 63\% of the failures involve the \emph{Exception} exception.
The uniqueness of failures based on exceptions can be influenced by the project domain. For example, the \emph{HCassandraInternalException} exception is specific to one project, potentially affecting its uniqueness. The \emph{UnknownHostException} exception is encountered in six projects, with roughly 75\% of these exceptions originating from the \alluxio project.
However, it is important to clarify that the presence of an exception like \emph{UnknownHostException} does not necessarily indicate a direct association with flaky failures. I have identified a few of the non-flaky failures reported with this exception in the \emph{okhttp} project. Interestingly, none of the flaky failures in that project have been reported with an \emph{UnknownHostException}.


\begin{table}[t]
  \setlength{\tabcolsep}{2.5pt}


\caption[List of top 10 Exceptions in flaky failures.]{
\textnormal{The Occurrence column denotes the count of failures associated with these exceptions and indicates the number of projects in which these failures originate from.
The With Stacktrace columns provide the count of each exception (including the stacktrace) that occur exclusively in flaky failures, as well as when they occur in non-flaky failures.
The Without Stacktrace (w/o) column corresponds to the With Stacktrace column, but excludes the consideration of stacktraces.}}
\label{table:exceptions}
\centering
\vspace{-4pt}
%\resizebox{\textwidth}{!}{
% \scriptsize
\footnotesize
\begin{tabular}{l|rr|rr|rr}

\toprule
      & \multicolumn{2}{c|}{\textbf{Occurrence}} & \multicolumn{2}{c|}{\textbf{With Stacktrace}} & \multicolumn{2}{c}{\textbf{w/o Stacktrace}}\\
      

\textbf{Exceptions} & \textbf{Failures} & \textbf{Project}  & \textbf{onlyFlaky} & \textbf{Both}  & \textbf{onlyFlaky} & \textbf{Both} \\
\midrule
AssertionError&186&15&88&98&8&178\\
\cellcolor{gray!6}{NullPointerException}&\cellcolor{gray!6}{185}&\cellcolor{gray!6}{5}&\cellcolor{gray!6}{39}&\cellcolor{gray!6}{146}&\cellcolor{gray!6}{9}&\cellcolor{gray!6}{176}\\
UnknownHostException&136&6&136&0&136&0\\
\cellcolor{gray!6}{IOException}&\cellcolor{gray!6}{70}&\cellcolor{gray!6}{4}&\cellcolor{gray!6}{57}&\cellcolor{gray!6}{13}&\cellcolor{gray!6}{36}&\cellcolor{gray!6}{34}\\
ProvisionException&49&1&49&0&0&49\\
\cellcolor{gray!6}{HCassandraInternalException}&\cellcolor{gray!6}{31}&\cellcolor{gray!6}{1}&\cellcolor{gray!6}{31}&\cellcolor{gray!6}{0}&\cellcolor{gray!6}{31}&\cellcolor{gray!6}{0}\\
SocketException&31&1&29&2&1&30\\
\cellcolor{gray!6}{Exception}&\cellcolor{gray!6}{30}&\cellcolor{gray!6}{5}&\cellcolor{gray!6}{30}&\cellcolor{gray!6}{0}&\cellcolor{gray!6}{28}&\cellcolor{gray!6}{2}\\
RuntimeException&23&3&21&2&9&14\\
\cellcolor{gray!6}{AssertionFailedError}&\cellcolor{gray!6}{22}&\cellcolor{gray!6}{5}&\cellcolor{gray!6}{1}&\cellcolor{gray!6}{21}&\cellcolor{gray!6}{1}&\cellcolor{gray!6}{21}\\


\bottomrule
\end{tabular}
\vspace{-10pt}
\end{table}


It is crucial to emphasize that not all \emph{OnlyFlaky} failure groups can be identified by a unique exception type. I found some failures where failure exceptions belong to both the unique and non-unique failure categories, as indicated in Table~\ref{table:exceptions}. In the context of our experiment, the most frequently occurring exception is the \emph{AssertionError} which is almost evenly split between the two categories, with 58\% of occurrences in unique failures. However, when considering only the exception type and excluding stacktrace lines, the proportion of unique failures with this exception drops to less than 5\%.
The reason behind this observation is the generality of the \emph{AssertionError} exception. For example, a test may have multiple assertion statements, and if they fail for different reasons, they match the exception but differ in the stack trace. Therefore, it becomes challenging to attribute this type of exception to a specific type of failure.



I conducted an analysis of failures involving the \emph{AssertionError} exceptions and identified them as \emph{OnlyFlaky} failures when compared to other non-flaky failures, including stacktrace lines. One important observation I made is that the test name does not appear in 70\% of the stack traces for these failures. This suggests that these failures occur in setup methods, as observed in the majority of flaky failures in projects like \okhttp and \emph{tootallnate-java-websocket}. On the other hand, all of the \emph{Both} failures groups that contain an \emph{AssertionError} in their exception have the test name present in the stack trace lines.

I have noticed that approximately half of the \emph{OnlyFlaky} failures groups can be identified by considering the exception type without examining the stack trace lines. This suggests that certain exceptions may be more likely to be associated with flaky failures under circumstances such as project domains.Also, I found that linking the \emph{AssertionError} exception to one of the two types of failures is challenging due to the general nature of this exception. 


\subsubsection{RQ3: Can machine learning be utilized to predict flaky failures using the failure logs?}

I utilized the dataset employed in the syntax-based approach to train and evaluate the \classifier. As part of the dataset preparation, I represent each failure with  a set of values, based on the features outlined in Table~\ref{table:Features}. For features that required accessing the code under test, I extracted all the \emph{java} file names from each project. During this process, I encountered missing values, such as two flaky tests in \emph{wildfly-wildfly}. Since the total number of missing values was minimal, I exclude all flaky tests associated with these failures.

Next, I developed a decision tree model that employed stratified cross-validation per project. Since the proportion of flaky failures was quite low compared to the total number of failures, I balanced the training data using SMOTE. The \classifier column in Table~\ref{table:classifier_table} demonstrates the prediction results per project. For instance, out of 310 flaky failures in \emph{Alluxio-alluxio}, the classifier correctly identified 281 as flaky (\textbf{TP}), while the remaining 29 failures were incorrectly predicted as non-flaky. To evaluate the performance of the classifier per project, I utilized precision, recall, and F1 score as evaluation metrics. To ensure that each testing dataset fold contains at least one flaky failure, I excluded projects with fewer than 10 instances of flaky failures, as compared to the projects analyzed in Section~\ref{sec:failureLogsStudy}.

In order to obtain a deeper understanding of the effectiveness of applying machine learning concepts, I conducted a thorough investigation to identify a state-of-the-art classifier specifically designed for failure logs. However, to the best of my knowledge, there is no currently publicly available and accessible machine learning approach for predicting test flakiness based on failure logs. As an alternative, I employed TF-IDF as a baseline to compare the prediction results of the classifier. The motivation behind using this approach is that the features of the classifier are directly extracted from the syntax of the failure log, without any dynamic features.




In the \classifier, there is a considerable number of true positives (714 out of 818) when predicting flaky failures. However, this also leads to a high number of false positives, where non-flaky failures are incorrectly identified as flaky. As a result, the classifier exhibits high recall across most of the project but has low precision rates.
Upon conducting a thorough analysis, I discovered that the type of exceptions has a significant impact on the overall performance of the classifier. In projects with a substantial number of false positives (e.g., more than 100 cases, as seen in project \emph{activiti-activiti}), general exceptions such as \emph{AssertionError} or \emph{NullPointerException} make up the majority of the failures classified as both flaky and non-flaky.
On the other hand, in projects with \textbf{no} false positives, like \emph{hector-client-hector}, almost all flaky failures are distinct from non-flaky failures in terms of exception types.
 

The \classifier exhibits a notable occurrence of false positives. In this experiment, these false positives originated from mutation testing. Despite multiple runs to exclude flaky mutants, there remains a possibility of misclassifying non-flaky failures that happen to group with flaky failures ones. It's essential to recognize that real-world scenarios may differ from these experimental results.

In the \tfidf approach, I noticed that the number of true positives (\textbf{TP}) is relatively low compared to the \classifier. However, there is a significant reduction in the false positive rates. This approach performs well in projects that have a reasonable number of flaky failures, except for the \alluxio project, where a large number of flaky failures are associated with NullPointerExceptions, leading to challenges in classification.
This \tfidf approach exhibits poor performance in projects where flaky failures are presented in an assertion format, such as in the \websocket. The model struggles to accurately identify flaky failures in such scenarios.
Overall, the \tfidf approach shows promise in projects with a moderate number of flaky failures but requires careful consideration and adjustments to handle specific patterns, such as NullPointer and Assertion exceptions to achieve better performance. 



The usability of different machine learning approaches varies based on the specific use case and objectives. If the main goal is to maximize the number of true positives (\textbf{TP}) without being overly concerned about the rate of false positives, the \classifier could be a good fit. In this scenario, the model is more focused on correctly identifying as many flaky failures as possible, even if it means accepting a higher number of false positives.
One of the main advantages of the \classifier is its flexibility in extending the learned features. The model can be easily augmented with additional static and dynamic features extracted from each failure. The proposed features shown in Table~\ref{table:Features} are not exhaustive but serve as a starting point, particularly utilizing information available within the failure log, such as the log syntax.
By leveraging these additional features, the failure log classifier can potentially enhance its performance and accuracy in identifying flaky failures.









\newpage
\section{Accessible UI Search}
\label{sec:SearchAcces}
We take the work done in the previous two projects and combine them in SeachAccess. 
In UI design, initial UI mockup sketches serve as the starting point for developers, offering an image that can be easily modified. However, addressing accessibility concerns within these designs often proves challenging. To bridge this gap, we introduce SearchAccess, a UI search engine designed to identify accessibility issues in mockups and find similar screens which are more accessible.

SearchAccess goes beyond traditional search functionalities, providing developers with a visual interface to identify and address accessibility issues within their UI mockups. I leverage computer vision techniques to enable developers to locate accessibility issues within the mockup screens. I use an image embedding representation of the screen to facilitate search between other screens.  

SearchAccess is built with 6 custom detectors. Each detector is built using computer vision techniques that are able to detect accessibility violations given only a screenshot. This makes it easy for developers to check for accessibility and make changes as necessary. Additionally, the search functionality is built to be multimodal, offering developers the opportunity to find similar screens with either the mockup or a text description and mockup. The screens displayed as most similar to the input screens are screens that are both similar in style, but more accessible than the input screen. The similar, more accessible, screens can provide an insight to developers on how to improve the accessibility within their own screens. 

This project aims to give developers the tools so they can make informed accessibility driven decisions early in the design process. 

\subsection{Current Progress}

Currently, the search engine has been implemented along with the search functionality. SearchAccess is fully functioning locally and we aim to host the tool soon. We ideally intend to perform a mixed evaluation of qualitative and quantitative measures. 
These are the proposed research questions. These questions are subject to change: 
\begin{description}
  
    \item \textbf{RQ$_1$}: \textit{How does SearchAccess perform in screen retrieval tasks?}
    \item \textbf{RQ$_2$}: \textit{How accurate are the detectors in SeachAccess?}
    \item \textbf{RQ$_3$}: \textit{Are developers able to identify accessibility issues within their UI designs?}
    \item \textbf{RQ$_4$}: \textit{Do developers benefit from UI search when looking to make their apps more accessible?}

 \end{description}

We intend to evaluate RQ1 and RQ2 quantitatively using retrieval metrics and accuracy metrics respectively. RQ3 and RQ4 will examine the qualitative impact this tool will have on developers. This study will be a performance study to analyze how well SearchAccess is able to supplement the accessibility focused design process and how it can expedite the process of making UIs more accessible. 
The timeline for SearchAccess involves conducting a user study and quantitative evaluation of the tool. Currently we are making efforts to manually create datasets to provide a quantitative evaluation while beginning work on initial user study ideas. We hope to complete the quantitative evaluation by May 10, 2024 and the user study by the middle of June. This will allow us to submit to the ICSE second deadline. 













\newpage
\section{Literature Review}
\label{sec:SLR}

In order to fully capture the current landscape of accessibility guidelines that may impact various populations of users, and to aid in selecting the most impactful guidelines that aim to assist motor impaired users we conducted a systematic literature reviews on research at the intersection of software engineering, human-computer interaction, and accessibility. To conduct this review, we followed the methodology set forth by Kitchenham~ et al. ~\cite{kitchenham2007guidelines}. We defined a single research question that asked \textit{``What accessibility guidelines have been identified and discussed in prior research?''}. We used the relatively simple search string of "accessibility" to search DBLP, the ACM Digital Library, and IEEE Xplore, for work at the intersection of accessibility and software engineering for the date range of January 2010 - December 2023. The purpose of using such a simple search string was to "cast a wide net" and ensure that we did not miss important work. We defined inclusion criteria as follows: (i) must have been published in our studied date range, (ii) must have been published at one of 16 conference venues (ICSE, FSE, ASE, ICSME, MSR, ICPC, ISSTA, ICST, SANER, UIST, CHI, SPLASH, OOPSLA, PLDI, CSCW, ASSETS) or 5 journal venues (TSE, TOSEM, EMSE, JSS, ASE) that cross cut software engineering, HCI, and accessibility, (iii) the paper must describe a study or developer tool directly related to an accessibility issue that impacts end-users. The scope of our search was limited to these venues and digital libraries as they provide the highest quality of research in all matters including accessibility. Our search results returned 2948 papers from our selected conferences within our given date range. Then, two authors manually checked  each paper for adherence to the final inclusion criteria, resulting in 20 papers that intersect our desired research areas \textit{and} discuss developer guidelines for addressing accessibility issues. In addition to these 20 identified primary studies, we also examined Apple's and Google's design guidelines related to accessibility~\cite{AppleAccess,GoogleAccess}, as several of our primary studies referenced these sources. The results of this search are shown in in \ref{tab:guidelines}. The resulting guidelines have been used in my ongoing work to present a comprehensive list of guidelines that have been mentioned in accessibility focused literature. 

\begin{table}[h]
	\footnotesize
	\vspace{-0.0em}
	\caption{Accessibility guidelines extracted from our systematic literature review of accessibility guidelines -- includes recent research and Google's~\cite{GoogleAccess} and Apple's~\cite{AppleAccess} accessibility guidelines. (LV = low vision users, DHH = deaf and hard of hearing users)}
	\vspace{-1em}
	\begin{tabular}{>{\centering\arraybackslash}p{2in}|>{\centering\arraybackslash}p{1.8in}|>{\centering\arraybackslash}p{.82in}|>{\centering\arraybackslash}p{.82in}}
	
		\textbf{Accessibility Guideline} & \textbf{Primary Affected User Demographic} & \textbf{Guideline Source}  & \textbf{Previous Implementation} \\
		\hline
		Visual Touch Target Size & \footnotesize {Motor, LV} & \cite{Kong21, Parhi06} &  \\ 
		\rowcolor{gray!30!} Touch Target Size & \footnotesize {Motor, LV} & \cite{AppleAccess, GoogleAccess, HarvardAccess, WebGuide, Nunes15, Calvo16, Alshayban20, Abascal11, Kane11, Kong21} & X \\ 
		Persistent Element Location & \footnotesize {Motor} &\cite{AppleAccess, HarvardAccess, WebGuide, GoogleAccess} &   \\
		\rowcolor{gray!30!} Clickable Span & \footnotesize {Motor, LV} & \cite{Alshayban20} & X \\
		Duplicate Clickable Bounds & \footnotesize {Motor} & \cite{Alshayban20} & X \\
		\rowcolor{gray!30!} Editable Item Descriptions & \footnotesize {LV} & \cite{Alshayban20, Eler18} & X \\
		Expanding Section Closure & \footnotesize {Motor, LV} & \cite{AppleAccess, GoogleAccess, HarvardAccess, WebGuide} &   \\
		\rowcolor{gray!30!} Non-Native Elements & \footnotesize {Motor, LV} & \cite{GoogleAccess, Calvo16}& X \\
		Motion Activation & \footnotesize {Motor, LV} & \cite{AppleAccess, GoogleAccess, HarvardAccess} &  \\
		\rowcolor{gray!30!} Labeled Elements & \footnotesize {Visual, DHH} & \cite{Alshayban20, FlrezAristizbal19, Li21, Eler18} & X \\
		Screen Captioning & \footnotesize {LV, DHH} & \cite{AppleAccess, GoogleAccess, HarvardAccess, WebGuide, ADAWeb, AccessGov, Ross18, Li21, Pavel20, Kane11}& X \\
		\rowcolor{gray!30!}Keyboard Navigation & \footnotesize {Motor, LV} & \cite{ADAWeb, AccessGov, FlrezAristizbal19, Li21, Chiou21} & X \\
		Traversal Order & \footnotesize {Motor} & \cite{AccessGov, Alshayban20, FlrezAristizbal19} & X \\
		\rowcolor{gray!30!}Adjacent Visual Icon Distance & \footnotesize {Motor} & \cite{AppleAccess, GoogleAccess, WebGuide, Yan19, Abascal11, Nunes15} &  X \\
		
		 Proper Information Organization & \footnotesize {Motor, LV} & \cite{Calvo16} & X \\
		\rowcolor{gray!30!} Facial Recognition & \footnotesize {Motor} & \cite{Calvo16, Astler11} & X \\
		 Single Tap Navigation & \footnotesize {Motor} & \cite{AppleAccess, GoogleAccess, HarvardAccess, WebGuide, FlrezAristizbal19, Milne18} &  \\
		\rowcolor{gray!30!} Poor form design/instructions & \footnotesize {Motor, LV} & \cite{ADAWeb, AccessGov} &  
		\end{tabular}
\vspace{-1em}
	\label{tab:guidelines}
\end{table}

\subsection{Current Progress}

Currently I have found almost 2,948 different papers and and narrowed it down to about 100 papers to begin the data extraction process. I intend to extract the data from these papers present a comprehensive study about the current state of developer tools focused on accessibility. Below are a list of initial research questions that may be subject to change barring further discussion. 



\begin{description}
  
    \item \textbf{RQ$_1$}: \textit{Is the research targeted at automating developer activities, enhancing existing software, or creating guidelines for developers?}
    \item \textbf{RQ$_2$}: \textit{What software domains does research on accessibility typically target?}
    \item \textbf{RQ$_3$}: \textit{Which populations of users with accessibility needs has software engineering targeted?}
    \item \textbf{RQ$_4$}: \textit{What type of data do studies use and how can the quality of data and its collection suggest an impact on the research in the field?}
    \item \textbf{RQ$_5$}: \textit{What are the primary means of evaluation for research that targets users with accessibility needs.?}

 \end{description}


The timeline for the SLR involves data extraction and analysis. We are at the halfway point with data extraction and hope to finish it by May 3, 2024. Given that data consolidation and analysis takes time, we hope to complete that process within two weeks following the completion of the extraction. This paper is intended for journal publication. As of now we do not have any specific journal in mind, but hope to submit this work in June of 2024. 













































% \section{Proposed Work}
\label{sec:proposedWork}


For the remainder of my PhD program, I intend to concentrate on two primary areas. The first area will involve expanding my research on failure logs to address issues that have arisen during my present work. The second area will involve utilizing machine learning techniques to predict the underlying cause of test flakiness.


\subsection{Failure logs in Test Flakiness}

Failure logs are the main source that describe how the test fails. Also, failure logs are the source that mostly being accessible from the developers side. As discussed in Section \ref{sec:failureLogsStudy}, developers use failure logs manually to tell if failure due to flakiness or not. However, there is light efforts on how proposing tools to leverage the details of failure logs to analyze flaky failures. 

There are some interesting findings during my work in analyzing failure logs for flakiness detection. There are a numerous amount of failure logs that comes as assertion error e.g. expected X but got Y. I am planning to see if there are hard-to-detect differences from flaky failures with these messages in the failure logs with non flaky failures that contain the same failure structure. I am planning to study the values of X and Y in the failure message and how close, for example, the X value in flaky failure from the X values in non flaky failures. I want to reach if having failure logs with these types of failures messages are enough to stop continuing debugging the failure logs and use alternative techniques. Similar to these failure exception are failures that comes with NullPointer exceptions.

Another finding is to analyze the failures logs of tests that fail together and how these logs differ in terms of the root causes of this type of failures with the root causes of tests which flakes randomly and, in some cases, separately. This includes comparing the failure logs of tests that have been previously reported as order dependent flaky tests. I would like to conclude that the failure logs of order and non order dependent differ and how the root cause make these failure logs are different. I am planning to collect the failure logs of of the same flaky tests detected in Section \ref{sec:flakeFlaggerStudy}, but in previous studied \cite{lam2019idflakies}. 


I am planning to investigate more on how mutation variants linked to revealing flaky failures. In Section \ref{sec:failureLogsApproach}, I use killed mutants failure exceptions as a substitute for non flaky failures for the dataset I am studying currently. As mutants could be flaky as well which also claimed in Section \ref{sec:failureLogsApproach} by running the killed mutants many times, I am trying to investigate between the link of mutants being flaky compared to the test flaky failure with the mutant variants. 

Another interesting direction in this area is to study the reproducibility of flaky failures and whether certain flaky failures logs can not be detected in certain running environments. One of the main problem that developers encounter is how to reproduce the same failures to be further analyzed by the testing team, which it is sometime is hard to run the same tests on the same environment due to various reasons such as accessibility and authentication. I am planning to see how flaky failure logs may differ if I rerun the same experiment in Section \ref{sec:flakeFlaggerStudy}, but with different environments setting.


\subsection{Machine learning}

Machine learning has been used in test flakiness to leverage the effectiveness to detect flaky tests. Large amount of the Machine learning current work aims to detect flaky tests, I would like to participate in this field, in addition to my first work, by propose a tool that aim to help developers when they analyze the root cause of the test flakiness. This is important to help developers how to fix the cause and and assign the task. I am planning to take the advantages of code coverage, libraries usages and failure logs exceptions to predict flakiness cause. With some cases where developers have no prior knowledge about other flaky test causes, and instead of debugging every failure log to know the cause, the tool should cluster the failure logs based on the similarity of the expected causes. This will reduce the overhead to go though every failure log to analyze a cause that could be common across different tests. 

Another interesting topic I plan to investigate to expect which test in the test suite that could be flake by knowing that one test in the same test has been detected as a flaky. Giving a flaky test, the main questions I am interested in is what is the possability percnetage for each test in the same test suite that could flake in future runs. I am planning to take into consideration the static and dynamic features that have been proposed in Table X. My end goal is to have some dynamic tool that monitor the failed test and order not yet detected flaky tests in a specific order where thwe first test means it is most likely to flake in future runs. 

Another work is to extend the failure log classifer presented in Section \ref{sec:failureLogsApproach}, by representing the failure messages after being preprocessed and extract tokens that can be used in addition to the already studied features. I aim to reevaluate the classifier again with the extended list of features. I am planning to use the Association Rules technique as a main ML approach that will take only the failure excpeiton and message. My goal is to build a dataset of a rules that commonly seen in flaky failures even in different projects. The proposed tool will take any failure excpetion and message and predict if this failure could be flaky because of the co-occurance of certain tokens in the previous known flaky failures. 



% One of the common failures logs is when it comes as a format of assertion e.g. expect true but was false. This is a common type even in flaky failures. X\% of flaky tests in FlakeFlagger dataset have at least one failure within the assertion format. It is obviously that this type of failures contains information might be hard to linked to a specific type of failure e.g. flaky or not flaky. I am going to study if the these failures types may contain information to tell if a failure is more likely to be flaky or not.



\section{Research Plan}
\label{sec:researchPlan}

During the Fall 2023, I will continue working on the proposed work and measure how likely the findings are motivated to investigate into new areas. In terms of writing my dissertation, I intend to defend it during 2024. 





% \subsection{Flaky Tests and Their Impacts}
% \subsection{Techniques in Detecting Flaky Tests}

% \section{Detecting Flaky Tests Without Rerun}
% \subsection{Rerun Study}
% \subsection{Flaky Tests Classifier}
% \subsubsection{Features Collections}
% \subsubsection{Dataset}
% \subsubsection{Evaluation of Flaky Tests Classifier}

% \section{Detecting Flaky Failures Using Logs}
% \subsection{Study Part}
% \subsection{Approach Part}
% \subsection{Evaluation of Detecting Using Failure Logs}

% \section{Proposed Work}
% \subsection{NA}
% \begin{itemize}
%     \item start with our already collected dataset ... 
%     \item How many s a test flake with different failures. 
%     \item OD NOD ( maybe both)
%     \item Root causes based on the logs.. 
%     \item fix a bug may eliminate the flakiness cause.. 
%     \item 
% \end{itemize}

% \section{Research Plan}




\newpage


\begin{footnotesize}
\bibliographystyle{plain}
% \bibliography{string,itu,rfc,i-d}
\bibliography{ProposalBib.bib}
\end{footnotesize}

\newpage



\end{document}


