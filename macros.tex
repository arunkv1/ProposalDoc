\usepackage{booktabs}
\usepackage{xspace}
\usepackage{enumitem}
\usepackage[table]{xcolor}  
\newcommand{\sysName}{FlakeFlagger\xspace}
\newcommand{\vocabName}{vocabulary-based approach\xspace}
\usepackage{flushend}

\newcommand{\testName}[1]{%
  \begingroup
  \ttfamily
  \begingroup\lccode`~=`/\lowercase{\endgroup\def~}{/\discretionary{}{}{}}%
  \begingroup\lccode`~=`[\lowercase{\endgroup\def~}{[\discretionary{}{}{}}%
  \begingroup\lccode`~=`.\lowercase{\endgroup\def~}{.\discretionary{}{}{}}%
  \catcode`/=\active\catcode`[=\active\catcode`.=\active
  \scantokens{#1\noexpand}%
  \endgroup
}
\definecolor{c1}{HTML}{AD302E}
\definecolor{c2}{HTML}{E9923E}
\definecolor{c3}{HTML}{F8CC47}
\definecolor{c4}{HTML}{428F4D}
\newcommand{\rowHighlight}{\rowcolor{gray!15}}


% Terms used :
\newcommand{\stack}{stacktrace lines\xspace}
\newcommand{\exception}{Exception\xspace}

\newcommand{\failure}{failure message and stacktraces\xspace}
\newcommand{\failures}{failure messages and stacktraces\xspace}


% Table 1 


% Names
\newcommand{\syntax}{text-based matching\xspace}
\newcommand{\classifier}{Failure Log Classifier\xspace}
\newcommand{\tfidf}{{TF-IDF}\xspace}
\newcommand{\flaky}{\emph{OnlyFlaky}\xspace}
\newcommand{\nonflaky}{\emph{Only Non-Flaky}\xspace}
\newcommand{\both}{\emph{Both}\xspace}


% Project names
\newcommand{\alluxio}{\emph{Alluxio}\xspace}
\newcommand{\okhttp}{\emph{Okhttp}\xspace}
\newcommand{\hbase}{\emph{Hbase}\xspace}
\newcommand{\ambari}{\emph{Ambari}\xspace}
\newcommand{\hector}{\emph{Hector}\xspace}
\newcommand{\activiti}{\emph{Activiti}\xspace}
\newcommand{\httpcore}{\emph{Httpcore}\xspace}
\newcommand{\websocket}{\emph{Java-websocket}\xspace}
\newcommand{\logback}{\emph{logback}\xspace}
\newcommand{\wildfly}{\emph{Wildfly}\xspace}
\newcommand{\http}{\emph{Http-request}\xspace}
\newcommand{\spring}{\emph{Spring-boot}\xspace}
\newcommand{\undertow}{\emph{Undertow}\xspace}
\newcommand{\elastic}{\emph{Elastic-job-lite}\xspace}
\newcommand{\orbit}{\emph{Orbit}\xspace}
\newcommand{\exec}{\emph{Commons-exec}\xspace}





\newcommand{\Space}[1]{}
% list of general macros .. ~ removed just to manage when numbers between parenthesis .. ~ will be followed each command instead. 

\newcommand{\numruns}{10,000} % number of runs ..
\newcommand{\numtests}{22,244} % number of all tests ..
\newcommand{\numflakyruns}{811} % number of observed flaky tests .. 
\newcommand{\numprojects}{24} % number of projects in rerun .. 

% after inspection phase
\newcommand{\numtestsinspected}{22,244} % number of tests ( more than 10 flaky + no missing values) .. 
\newcommand{\numflakyinspected}{808}%number of actual flaky tests ..
\newcommand{\numflakypredict}{599}% number of predicted flaky tests ..
\newcommand{\bestfscore}{67\%} % best F1 score
\newcommand{\misstests}{298} % Tests ( flaky + non flaky) with missing values
\newcommand{\missflaky}{three} % flaky with missing values .. 

% number of projects in/out classification and others .. 
\newcommand{\projectsin}{14}% projects have more than or(=) 10 flaky tests .. 
\newcommand{\projectsout}{10}% projects have less than 10 flaky tests ..
\newcommand{\projectsmaxflaky}{163} % a project has maximum number of flaky tests ..
\newcommand{\projectsminflaky}{16} % a project has minimum number of flaky tests ..
\newcommand{\projectsoneflaky}{4} % projects which have only one flaky test ..
\newcommand{\projectshundredsflaky}{4} %  projects which have more than or(=) 100 flaky test ..

% list of flaky tests by projects
\newcommand{\springbootFlaky}{163}
\newcommand{\hbaseFlaky}{145}
\newcommand{\alluxioFlaky}{116}
\newcommand{\okhttpFlaky}{100}
\newcommand{\ambariFlaky}{52}
\newcommand{\hectorFlaky}{33}
\newcommand{\activitiFlaky}{32}
\newcommand{\javawebsocketFlaky}{23}
\newcommand{\wildflyFlaky}{23}
\newcommand{\logbackFlaky}{22}
\newcommand{\httpcoreFlaky}{22}
\newcommand{\incubatordubboFlaky}{19}
\newcommand{\httprequestFlaky}{18}
\newcommand{\wrojFlaky}{16}
\newcommand{\orbitFlaky}{7}
\newcommand{\undertowFlaky}{7}
\newcommand{\achillesFlaky}{4}
\newcommand{\elasticjobliteFlaky}{3}
\newcommand{\zxingFlaky}{2}
\newcommand{\handlebarsFlaky}{1}
\newcommand{\ninjaFlaky}{1}
\newcommand{\assertjcoreFlaky}{1}
\newcommand{\commonexecFlaky}{1}


% FP results .. 
\newcommand{\flaggerfp}{406} % total FP in FlakeFlagger .. 
\newcommand{\msrfp}{4,683} % total FP in MSR .. 
\newcommand{\mergefp}{314} % total FP in combined models .. 


% related Works stats
\newcommand{\idflakiesTotalFlaky}{422} % total flaky tests in idflakies .. 
\newcommand{\idflakiesTotalprojects}{82} % total projects in idflakies ..
\newcommand{\idflakiesNODFlaky}{191} % total NOD flaky tests in idflakies ..
\newcommand{\idflakiesCommonProjects}{3} % Common project with rerun (consider SHA)
\newcommand{\idflakiesRerunFlaky}{70} % flaky tests based on common projects
\newcommand{\idflakiesCommonFlaky}{28} % common flaky tests based on rerun
\newcommand{\idflakiesMissedFlaky}{42} % number of flaky tests that not observed by rerun

\newcommand{\deflakerTotalFlaky}{96} % total flaky tests in deflaker ..
\newcommand{\deflakerTotalProjects}{26} % total projects in deflaker ..
\newcommand{\deflakerCommonProjects}{12} % Common project with rerun (consider SHA)
\newcommand{\deflakerRerunFlaky}{20} % flaky tests based on common projects
\newcommand{\deflakerCommonFlaky}{10} % common flaky tests based on rerun
\newcommand{\deflakerMissedFlaky}{10} % number of flaky tests that not observed by rerun


% FLAST part ..
\newcommand{\FLASTdefalker}{57}
\newcommand{\FLASTidflakies}{258}
\newcommand{\FLASTsmells}{3,424}

% Percentages lists ... 
\newcommand{\missingtestsrate}{1.3} % ratio of tests with missing values to total tests.. 
\newcommand{\flakytestsrate}{3.6} % ratio of flaky tests to total tests..
\newcommand{\highestflakyrate}{20} % ratio of highest project (spring-boot) to total number of flaky tests ..
\newcommand{\redbarsratio}{8} % how many projects where red bar appears as the majority ..
\newcommand{\alluxiofailrate}{90}% ratio of flaky test failure in alluxio project .. 
\newcommand{\NumFailingRunsTen}{33}% 268 flaky tests fail <=10 times .. 
\newcommand{\NumFailingRunsHundred}{22}% 175 flaky tests fail (10:100] times .. 
\newcommand{\NumFailingRunsThousand}{19}% 157 flaky tests fail (100:1000] times .. 
\newcommand{\NumFailingRunsOthers}{26}% 210 flaky tests fail >1000 times .. 


% missclassified result .. 
\newcommand{\missclassifiedrate}{26\%}% percentage of FN to flaky tests
\newcommand{\FNokhttp}{63}% FN in okhttp
\newcommand{\FNfourprojects}{74}% FN in activiti, logback, httpcore and wro4j
\newcommand{\Flakyfourprojects}{92}% flaky tests in activiti, logback, httpcore and wro4j


% smells part .. 
\newcommand{\smellsflaky}{80\%}% percentage of flaky tests which have at least one smell
\newcommand{\smellsnonflaky}{85\%}% percentage of non flaky tests which have at least one smell


\usepackage{textcomp}

{\makeatletter
 \gdef\jonmark{%
   \expandafter\ifx\csname @mpargs\endcsname\relax % in minipage?
     \expandafter\ifx\csname @captype\endcsname\relax % in figure/caption?
       \marginpar{\textcolor{blue}{jon~}}% not in a caption or minipage, can use marginpar
     \else
       \textcolor{blue}{jon~}% notice trailing space
     \fi
   \else
     \textcolor{blue}{jon~}% notice trailing space
   \fi}
 \gdef\jon{\@ifnextchar[\jon@lab\jon@nolab}
 \long\gdef\jon@lab[#1]#2{{\bf [\jonmark \textcolor{blue}{#2} ---{\sc #1}]}}
 \long\gdef\jon@nolab#1{{\bf [\jonmark \textcolor{blue}{#1}]}}
  % This turns them off:
%  \long\gdef\jon@lab[#1]#2{}\long\gdef\jon@nolab#1{}%
}

{\makeatletter
 \gdef\checkmark{%
  \expandafter\ifx\csname @mpargs\endcsname\relax % in minipage?
     \expandafter\ifx\csname @captype\endcsname\relax % in figure/caption?
      \marginpar{\textcolor{red}{CHECK~}}% not in a caption or minipage, can use marginpar
     \else
      \textcolor{red}{CHECK~}% notice trailing space
     \fi
  \else
     \textcolor{red}{CHECK~}% notice trailing space
  \fi}
 \gdef\check{\@ifnextchar[\check@lab\check@nolab}
 \long\gdef\check@lab[#1]#2{{\bf [\checkmark \textcolor{red}{#2} ---{\sc #1}]}}
 \long\gdef\check@nolab#1{{\bf [\checkmark \textcolor{red}{#1}]}}
  % This turns them off:
%  \long\gdef\check@lab[#1]#2{}\long\gdef\check@nolab#1{}%
}

{\makeatletter
 \gdef\michaelmark{%
   \expandafter\ifx\csname @mpargs\endcsname\relax % in minipage?
     \expandafter\ifx\csname @captype\endcsname\relax % in figure/caption?
       \marginpar{\textcolor{blue}{michael~}}% not in a caption or minipage, can use marginpar
     \else
       \textcolor{blue}{michael~}% notice trailing space
     \fi
     \textcolor{blue}{michael~}% notice trailing space
   \fi}
 \gdef\michael{\@ifnextchar[\michael@lab\michael@nolab}
 \long\gdef\michael@lab[#1]#2{{\bf [\michaelmark \textcolor{blue}{#2} ---{\sc #1}]}}
 \long\gdef\michael@nolab#1{{\bf [\michaelmark \textcolor{blue}{#1}]}}
  % This turns them off:
%  \long\gdef\michael@lab[#1]#2{}\long\gdef\michael@nolab#1{}%
}


\definecolor{paulcolor}{rgb}{0.44, 0.26, 0.08}
{\makeatletter
 \gdef\paulmark{%
   \expandafter\ifx\csname @mpargs\endcsname\relax % in minipage?
     \expandafter\ifx\csname @captype\endcsname\relax % in figure/caption?
       \marginpar{\textcolor{paulcolor}{paul~}}% not in a caption or minipage, can use marginpar
     \else
       \textcolor{paulcolor}{paul~}% notice trailing space
     \fi
   \else
     \textcolor{paulcolor}{paul~}% notice trailing space
   \fi}
 \gdef\paul{\@ifnextchar[\paul@lab\paul@nolab}
 \long\gdef\paul@lab[#1]#2{{\bf [\paulmark \textcolor{paulcolor}{#2} ---{\sc #1}]}}
 \long\gdef\paul@nolab#1{{\bf [\paulmark \textcolor{paulcolor}{#1}]}}
  % This turns them off:
%  \long\gdef\paul@lab[#1]#2{}\long\gdef\paul@nolab#1{}%
}

\definecolor{abdulcolor}{rgb}{0.0, 0.37, 0.23}
{\makeatletter
 \gdef\abdulmark{%
   \expandafter\ifx\csname @mpargs\endcsname\relax % in minipage?
     \expandafter\ifx\csname @captype\endcsname\relax % in figure/caption?
       \marginpar{\textcolor{abdulcolor}{abdul~}}% not in a caption or minipage, can use marginpar
     \else
       \textcolor{abdulcolor}{abdul~}% notice trailing space
     \fi
   \else
     \textcolor{abdulcolor}{abdul~}% notice trailing space
   \fi}
 \gdef\abdul{\@ifnextchar[\abdul@lab\abdul@nolab}
 \long\gdef\abdul@lab[#1]#2{{\bf [\abdulmark \textcolor{abdulcolor}{#2} ---{\sc #1}]}}
 \long\gdef\abdul@nolab#1{{\bf [\abdulmark \textcolor{abdulcolor}{#1}]}}
  % This turns them off:
%  \long\gdef\abdul@lab[#1]#2{}\long\gdef\abdul@nolab#1{}%
}


\definecolor{dkgreen}{rgb}{0,0.6,0}
\definecolor{gray}{rgb}{0.5,0.5,0.5}
\definecolor{verylightgray}{rgb}{0.94,0.94,0.94}

\definecolor{mauve}{rgb}{0.58,0,0.82}
\definecolor{lightblue}{rgb}{0.61,0.78,0.91}
\definecolor{lightpurple}{rgb}{0.85,0.83,0.91}

\lstdefinelanguage{HTML5}{
        language=html,
        sensitive=true, 
        alsoletter={<>=-},
        otherkeywords={
        % HTML tags
        <html>, <head>, <title>, </title>, <meta, />, </head>, <body>,
        <canvas, \/canvas>, <script>, </script>, </body>, </html>, <!, html>, <style>, </style>, ><
        },  
        ndkeywords={
        % General
        =,
        % HTML attributes
        charset=, id=, width=, height=,
        % CSS properties
        border:, transform:, -moz-transform:, transition-duration:, transition-property:, transition-timing-function:
        },  
        morecomment=[s]{<!--}{-->},
        tag=[s]
}


\lstdefinestyle{inlinecode}{basicstyle=\ttfamily,language=Java, backgroundcolor = \color{lightgray}, breaklines=true}
\lstdefinestyle{inlinesql}{basicstyle=\ttfamily,language=SQL, backgroundcolor = \color{lightgrey}, showspaces=false,                
        showstringspaces=false,
        showtabs=false, breaklines=true}
\lstdefinestyle{inlinehtml}{basicstyle=\ttfamily,language=Java, backgroundcolor = \color{lightgrey}, 
        showtabs=false, breaklines=true} %I actually DONT think we should have code colors in this so this is set to "java" to not highlight HTML tags.
\lstset{
	language=Java,
        basicstyle=\small,
	breaklines=true,
	showspaces=false,
	showstringspaces=false,
	commentstyle=\color{dkgreen},
	stringstyle=\color{mauve},
	abovecaptionskip=2pt,
	captionpos=b,
	numbers=left,
	frame=single,
	xleftmargin=10pt,
	xrightmargin=5pt,
	numbersep=2pt,
	framexleftmargin=7pt,
	framexrightmargin=0pt,
	numberstyle=\color{gray},
	keywordstyle=\color{blue},
	tabsize=2,
	keepspaces=true,
	moredelim=[is][\underbar]{ZZ}{ZZ},
    escapeinside={/*@}{@*/},
	upquote=true
}

\lstdefinestyle{smalljava}{
	language=Java,
        basicstyle=\footnotesize,
	breaklines=true,
	showspaces=false,
	showstringspaces=false,
	commentstyle=\color{dkgreen},
	stringstyle=\color{mauve},
	abovecaptionskip=2pt,
	captionpos=b,
	numbers=none,
	frame=single,
	xleftmargin=4pt,
	xrightmargin=4pt,
	numbersep=0pt,
	framexleftmargin=0pt,
	framexrightmargin=0pt,
	numberstyle=\color{gray},
	keywordstyle=\color{blue},
	tabsize=2,
	keepspaces=true,
	moredelim=[is][\underbar]{ZZ}{ZZ},
    escapeinside={/*@}{@*/},
	upquote=true
}
\lstdefinestyle{smalljavanumbered}{
	language=Java,
        basicstyle=\footnotesize,
	breaklines=true,
	showspaces=false,
	showstringspaces=false,
	commentstyle=\color{dkgreen},
	stringstyle=\color{mauve},
	abovecaptionskip=2pt,
	captionpos=b,
	numbers=left,
	frame=single,
	xleftmargin=4pt,
	xrightmargin=4pt,
	numbersep=0pt,
	framexleftmargin=3pt,
	framexrightmargin=0pt,
	numberstyle=\color{gray},
	keywordstyle=\color{blue},
	tabsize=2,
	keepspaces=true,
	moredelim=[is][\underbar]{ZZ}{ZZ},
    escapeinside={/*@}{@*/},
	upquote=true
}


% \newcommand{\code}{\lstinline[style=inlinecode]}
\newcommand{\inlinesql}{\lstinline[style=inlinesql]}
\newcommand{\inlinehtml}{\lstinline[style=inlinehtml]}