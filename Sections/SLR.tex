\newpage
\section{Literature Review}
\label{sec:SLR}

In order to fully capture the current landscape of accessibility guidelines that may impact various populations of users, and to aid in selecting the most impactful guidelines that aim to assist motor impaired users we conducted a systematic literature reviews on research at the intersection of software engineering, human-computer interaction, and accessibility. To conduct this review, we followed the methodology set forth by Kitchenham~ et al. ~\cite{kitchenham2007guidelines}. We defined a single research question that asked \textit{``What accessibility guidelines have been identified and discussed in prior research?''}. We used the relatively simple search string of "accessibility" to search DBLP, the ACM Digital Library, and IEEE Xplore, for work at the intersection of accessibility and software engineering for the date range of January 2010 - December 2022. The purpose of using such a simple search string was to "cast a wide net" and ensure that we did not miss important work. We defined inclusion criteria as follows: (i) must have been published in our studied date range, (ii) must have been published at one of 16 conference venues (ICSE, FSE, ASE, ICSME, MSR, ICPC, ISSTA, ICST, SANER, UIST, CHI, SPLASH, OOPSLA, PLDI, CSCW, ASSETS) or 5 journal venues (TSE, TOSEM, EMSE, JSS, ASE) that cross cut software engineering, HCI, and accessibility, (iii) the paper must describe a study or developer tool directly related to an accessibility issue that impacts end-users. The scope of our search was limited to these venues and digital libraries as they provide the highest quality of research in all matters including accessibility. Our search results returned 2948 papers from our selected conferences within our given date range. Then, two authors manually checked  each paper for adherence to the final inclusion criteria, resulting in 20 papers that intersect our desired research areas \textit{and} discuss developer guidelines for addressing accessibility issues. In addition to these 20 identified primary studies, we also examined Apple's and Google's design guidelines related to accessibility~\cite{AppleAccess,GoogleAccess}, as several of our primary studies referenced these sources.




\subsection{Current Progress}

Currently I have found almost 2,948 different papers and and narrowed it down to about 100 papers to begin the data extraction process. I intend to extract the data from these papers present a comprehensive study about the current state of developer tools focused on accessibility. Below are a list of initial research questions that may be subject to change barring further discussion. 

\begin{description}
  
    \item \textbf{RQ$_1$}: \textit{Is the research targeted at automating developer activities, enhancing existing software, or creating guidelines for developers?}
    \item \textbf{RQ$_2$}: \textit{What software domains does research on accessibility typically target?}
    \item \textbf{RQ$_3$}: \textit{Which populations of users with accessibility needs has software engineering targeted?}
    \item \textbf{RQ$_4$}: \textit{What type of data do studies use and how can the quality of data and its collection suggest an impact on the research in the field?}
    \item \textbf{RQ$_5$}: \textit{What are the primary means of evaluation for research that targets users with accessibility needs.?}

 \end{description}














































