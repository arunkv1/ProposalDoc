\newpage
\section{Accessible UI Search}
\label{sec:SearchAcces}
We take the work done in the previous two projects and combine them in SeachAccess. 
In UI design, initial UI mockup sketches serve as the starting point for developers, offering an image that can be easily modified. However, addressing accessibility concerns within these designs often proves challenging. To bridge this gap, we introduce SearchAccess, a UI search engine designed to identify accessibility issues in mockups and find similar screens which are more accessible.

SearchAccess goes beyond traditional search functionalities, providing developers with a visual interface to identify and address accessibility issues within their UI mockups. I leverage computer vision techniques to enable developers to locate accessibility issues within the mockup screens. I use an image embedding representation of the screen to facilitate search between other screens.  

SearchAccess is built with 6 custom detectors. Each detector is built using computer vision techniques that are able to detect accessibility violations given only a screenshot. This makes it easy for developers to check for accessibility and make changes as necessary. Additionally, the search functionality is built to be multimodal, offering developers the opportunity to find similar screens with either the mockup or a text description and mockup. The screens displayed as most similar to the input screens are screens that are both similar in style, but more accessible than the input screen. The similar, more accessible, screens can provide an insight to developers on how to improve the accessibility within their own screens. 

This project aims to give developers the tools so they can make informed accessibility driven decisions early in the design process. 

\subsection{Current Progress}

Currently, the search engine has been implemented along with the search functionality. SearchAccess is fully functioning locally and we aim to host the tool soon. We ideally intend to perform a mixed evaluation of qualitative and quantitative measures. 
These are the proposed research questions. These questions are subject to change: 
\begin{description}
  
    \item \textbf{RQ$_1$}: \textit{How does SearchAccess perform in screen retrieval tasks?}
    \item \textbf{RQ$_2$}: \textit{How accurate are the detectors in SeachAccess?}
    \item \textbf{RQ$_3$}: \textit{Are developers able to identify accessibility issues within their UI designs?}
    \item \textbf{RQ$_4$}: \textit{Do developers benefit from UI search when looking to make their apps more accessible?}

 \end{description}

We intend to evaluate RQ1 and RQ2 quantitatively using retrieval metrics and accuracy metrics respectively. RQ3 and RQ4 will examine the qualitative impact this tool will have on developers. This study will be a performance study to analyze how well SearchAccess is able to supplement the accessibility focused design process and how it can expedite the process of making UIs more accessible. 
The timeline for SearchAccess involves conducting a user study and quantitative evaluation of the tool. Currently we are making efforts to manually create datasets to provide a quantitative evaluation while beginning work on initial user study ideas. We hope to complete the quantitative evaluation by May 10, 2024 and the user study by the middle of June. This will allow us to submit to the ICSE second deadline. 












