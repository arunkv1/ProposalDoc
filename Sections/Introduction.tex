
\section{Introduction}
\label{sec:introduction}

Software has become an integral part of everyone's lives and its impact continues to grow. Software takes many forms such as web applications, smartphone applications, and desktop or operating system applications, etc \cite{7}. Our everyday lives depend on the use of critical software applications which allow us to bank, invest, get news, and communicate with others. Touch screen devices such as smartphones and tablets provide a quick and easy means of access to important information and functions within our daily lives. The abundance of critical software introduces the responsibility to ensure people of all abilities and skill level are able to use and access their information. Software is ideally designed as all encompassing, where any user can use it as intended under their own abilities, but that is far from the current situation of software.

Software accessibility has become more important as more users are dependent on smartphones and computers. People of different abilities have found it difficult to use software the way it is currently developed and designed \cite{11}. According to the world health organization (WHO), 15\% of people have some disability \cite{28}, making software accessibility more important to ensure all users are able to use applications as intended.Though software engineers and companies are ethically motivated to create more accessible software, the United States Government is also making efforts to require public websites and services to be accessible \cite{23}. The government in conjunction with the American with Disabilities Act (ADA) introduced legislation which "prohibits discrimination on the basis of disability in the activities of public accommodations" \cite{23,16}. This has lead to a 180\% increase in more accessible software as of 2018 \cite{27}. This change in policy increases a need for more accessible software and tools that will help developers make their applications more accessible.

This effort to make more accessible software, however, has found its limitations. Google and Apple have the largest distribution for applications for the market \cite{17}. Google's Google Play and Apple’s App Store make it convenient for users to both download and create their own applications \cite{17}. Though, these companies have worked to make their own devices more accessible, most of the applications on their app stores are not controlled and developed by them, therefore being largely inaccessible \cite{16}. Large corporations have made their guidelines available to developers to follow \cite{25,26}, but prior work has suggested that, although there is an abundance of accessibility research and guidelines, there has not been much research or work done to educate the large community of developers on accessibility related issues \cite{16, 15}. The current state of software accessibility tools are generally ignored by developers because of a lack of concise warnings and difficulty of use \cite{9,16}.As a result, 95\% of the android applications randomly mined had elements that violated android GUI accessibility guidelines making the applications less accessible \cite{15}. This raises the need for better accessibility focused developer tools so that developers can make more accessible applications. 

Software engineering research is constantly innovating how software is made. Software testing and developer tools are constantly evolving and are making their way into the accessibility space. Software testing has been around for decades, but has recently grown into a more complex field with the use of computer vision and machine learning techniques to automatically generate and run tests. Moran et al. \cite{42}, for example, created a testing framework that took in application screenshots and automatically reported GUI design violations set by Android, therefore, letting developers know that their applications are not in accordance with the guidelines that are suggested. This exciting new way of testing could provide an ample amount of ways to test for accessibility guideline violations without developers needing to explicitly check for violations on their own. An extremely important part of software engineering research is the constant need for new, robust developer tools to assist developers in the process of designing and creating software. These tools allow developers to have quick access to information and functions that make the development process more efficient. Zhang et al. \cite{15} created a tool for developers to automatically label the elements in their code that helps screen readers identify each element on the screen. Simple, but intelligent tools such as the ones Zhang et al. \cite{15} proposed would drastically improve how accessible software is made by providing an effortless way for developers to quickly incorporate accessibility features into their software. 

Given the current state of HCI, accessibility, and software engineering research, we can work towards the intersection of these areas and develop exciting new ways to make accessible software. Using data, accessibility research, and software engineering tools, we can address the lack of software accessibility at its core by providing intelligent tools for developers to create accessible software in the developmental and design stages. 


The thesis proposal is organized as follows: Section~\ref{sec:background} provides an introduction of works towards the problem of test flakiness. The main contributions of the thesis are discussed in Section~\ref{sec:thesis}. Section~\ref{sec:detectFlakyTests} provides a summary of the findings related to the detection of flaky tests. Section~\ref{sec:livingTestFlakiness} emphasizes current research on how to identify flaky \emph{failures} and explores techniques for their detection. The main key points for my future work, which form the remainder of my PhD, are discussed in Section~\ref{sec:categotize}. Lastly, the current plan for the remaining phase of my PhD is outlined in Section~\ref{sec:researchPlan}.
